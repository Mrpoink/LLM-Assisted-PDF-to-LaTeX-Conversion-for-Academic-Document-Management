=====FILE: main.tex=====
\documentclass[11pt]{article}
\usepackage[margin=1in]{geometry}
\usepackage{times}
\usepackage{latexsym}
\usepackage{amsmath,amssymb}
\usepackage{booktabs}
\usepackage{array}
\usepackage{multirow}
\usepackage{graphicx}
\usepackage{url}
\usepackage[hidelinks]{hyperref}

\title{Emotion Granularity from Text:\An Aggregate-Level Indicator of Mental Health}

\author{
Krishnapriya Vishnubhotla$^{1,2}$ \and Daniela Teodorescu$^{3}$ \and Mallory J. Feldman$^{4}$ \and Kristen A. Lindquist$^{4}$ \and Saif M. Mohammad$^{5}$ \
\
$^{1}$Department of Computer Science, University of Toronto\
$^{2}$Vector Institute, Toronto\
$^{3}$Department of Computing Science, University of Alberta\
$^{4}$Department of Psychology and Neuroscience, University of North Carolina at Chapel Hill\
$^{5}$National Research Council Canada
}

\date{}

\begin{document}
\maketitle

\begin{abstract}
We are united in how emotions are central to shaping our experiences; yet, individuals dif-fer greatly in how we each identify, categorize, and express emotions. In psychology, varia-tion in the ability of individuals to differentiate between emotion concepts is called emotion granularity (determined through self-reports of one's emotions). High emotion granularity has been linked with better mental and physical health; whereas low emotion granularity has been linked with maladaptive emotion regula-tion strategies and poor health outcomes. In this work, we propose computational measures of emotion granularity derived from temporally-ordered speaker utterances in social media (in lieu of self-reports that suffer from various bi-ases). We then investigate the effectiveness of such text-derived measures of emotion gran-ularity in functioning as markers of various mental health conditions (MHCs). We estab-lish baseline measures of emotion granularity derived from textual utterances, and show that, at an aggregate level, emotion granularities are significantly lower for people self-reporting as having an MHC than for the control popula-tion. This paves the way towards a better un-derstanding of the MHCs, and specifically the role emotions play in our well-being.
\end{abstract}

\section{Introduction}
Emotions play a central role in how we construct meaning and communicate with those around us. Yet, individuals vary in their understanding and experience of emotions, or ``emotional expertise'' (Hoemann et al., 2021b). Some people are able to recognize, identify, and describe what they feel using precise, context-specific terms like guilt, anger, frustration, or helplessness; others tend to use more broad terms to convey a general sense of feeling bad or feeling low. Emotion granularity (EG), aka emotion differentiatior.t, is defined by psychologists as the ability of an individual to experience and categoize emotions in very specific terms (Barrett et al., 2001). Highly granular individuals have a broad range of highly situated and differentiated emotion concepts, and can reliably describe these concepts using language --- for example, distin-guishing between when they are feeling angry vs.\ when they are feeling sad, or when they are feeling elated from when they are feeling content.

Evidence collected in the last two decades pro-vides consistent support for a link between emo-tional granularity and mental health (Erbas et a1., 2014, 2018; Starr et a1., 2017; Seah et aL., 2020), physical health (Hoemann et a1., 2021b; Bonar et a7., 2023), and adaptive health behavior (Dixon-Gordon et a1., 2014; Kashdan et al., 2015). Note that this is different from other findings that study how the prevalence of specific emotions varies with mental health (for example, people with depres-sion tend to use more sadness-associated words). The link between EG and mental health suggests that there is a fundamental difference in how one perceives an emotion (broadly or speciflcally), and that in turn can impact their mental health.

Typically, granularity is measured across emo-tions with the same valence; one can therefore have a measure of negative emotion granularity, mea-sured as the granularity of negative emotions (such as anger, sadness, and fear) and positive emotion granularity, measured as the granularity of positive emotions (such as joy, excitement, and satisfac-tion). Some works also look at the co-endorsement of emotions that express opposite valence, such as joy and sadness (Lindquist and Barrett, 2008).

In psychology and the affective sciences, emo-tion granularity is often measured using repeated measurements, where individuals are asked to rate the intensity of experiencing certain emotions mul-tiple times over a period of days (e.g., 2--3 times each day for 5 days), i.e, with self-reports of emo-tions felt. An individual's emotional granularity is then operationalized as the extent to which multi-ple emotions are co-endorsed over time, i.e, how similarly the emotions are rated across all measure-ments, using the intraclass correlation coefficient (ICC) (Shrout and Fleiss, 1979), which measures the extent to which the emotions co-vary in reports at the aggregate level. Individuals who tend to fre-quently rate multiple emotions at the same intensity levels are defined as low in granularity --- the fre-quent co-endorsement across time indicates that they are failing to differentiate between these emo-tions in their reports. In contrast, individuals high in emotion granularity co-endorse multiple emo-tions less frequently over time (Tugade et a1., 2004; Hoemann et al., 202la; Lee et a7., 2017; Reitsema et a1., 2022).

While prior work in NLP has studied the link between emotions and mental health, these have largely been limited to measuring the prevalence or intensity of positive and negative emotions. In this work, we, for the first time, propose a way to compute emotion grartilaity from the textual utterqnces of an individual. Ow method uses the temporal sequence of the utterances to first con-struct emotion arcs along multiple emotions, and computes granularity as the correlation of these emotion arcs. We hypothesize that this measure is indicative of the individual consistently expressing the same set of emotions together over a period of time, and can therefore act as a proxy measure of emotional granularity.

We then study the relationship between ag-gregate, population-level measures of emotion granularity in text for eight Mental Health Conditions (MHCs), namely attention-deficit hyperactivity disorder (ADHD), anxiety, bipolar disorder, depression, major depressive disorder (MDD), obsessive-compulsive disorder (OCD), postpartum depression (PPD), and post-traumatic stress disorder (PTSD), and compare them to a control group. We use two social media datasets where users have chosen to self-disclose their mental health diagnosis (Suhavi et al., 2022; Losada et a1., 2017, 2018). We compute emotion granularity metrics for each of these groups to answer the following questions:
\begin{enumerate}
\item Do measures of emotional granularity differ between the MHCs and the control group?
\item Which measures of emotion granularity are the most effective at differentiating between the MHCs and the control group?
\item Which emotion pairs lead to the greatest difference in granularity between an MHC and the control group?
\end{enumerate}

Exploring this line of questions helps us better un-derstand how emotion granularity presents itself in text, whether emotion granularity from text can be a useful tool to study MHCs, and how an MHC impacts our perception of emotions (and perhaps even, how the perception of emotions impacts our mental health).

Our results establish baseline measures of emo-tion granularity from text, and show that these measures function as reliable indicators, at the aggregate-level, for the presence of many of the mental health conditions we study. Our work makes an important contribution to the growing wealth of research on textual measures of emotional expression as biosocial markers of MHCs, and has a broader utility in functioning as an additional indicator of the mental well-being of populations.\footnote{[MISSING]} All our code will be made available through the project webpage.\footnote{\url{[https://github.com/priya22/emotion-granularity-from-text}}](https://github.com/priya22/emotion-granularity-from-text}})

\section{Related Work}

\subsection{Emotions and Mental Health}
Measures of emotional experience and their pat-terns of change over time have been extensively studied as markers of mental and physical well-being (Lewis et al., 2010). The Emotion Dynamics framework in psychology quantifies the patterns with which emotions change over time, allowing researchers to better understand emotional experi-ences and individual variation (Kuppens and Ver-duyn, 2077). The framework includes several mea-sures such as the duration, intensity, variability, and granularity of one's emotional experiences. Nu-merous studies in psychology have shown emotion dynamics correlate with overall well-being, mental health, and psychopathology (the scientific study of mental illness or disorders) (Kuppens and Verduyn, 2017; Houben et al., 2015; Silk et a1., 2011; Sperry et aL, 2020).

Emotion granularity in particular is positively associated with adaptive behaviour in adverse con-ditions --- accurately labeling our emotions can inform us of the right coping strategies to use in different contexts. Individuals with higher emo-tion granularity tend to use a broader range of strategies to deal with negative emotions, and are more successful at doing so (Barrett et al., 2001). Several studies have shown that emotion granu-larity is lower in individuals with mental health conditions like bipolar disorder (Suvak et aL., 2011; Dixon-Gordon et al., 2014), manic depressive dis-order (Demiralp et al., 2012), schizophrenia (Kring et a1., 2003), autism spectrum disorder (Erbas et al., 2013), and affective disorders like anxiety (Seah et a1., 2020) and depression (Stan et a1., 2017 Willroth et a1., 2020). Lower granularity is also associated with increased tendencies to engage in maladaptive behaviour, such as alcohol consump-tion (Kashdan et al., 2015; Emery et a7., 2014) and aggression (Pond Jr et a1., 2072).

Researchers in affective science typically mea-sure emotional granularity through experience sam-pling methodologies (ESMs), or ecological mo-mentary assessments (EMAs), where individuals (participants) are repeatedly asked to report on their emotional states on several occasions throughout the day, for several days. For example, participants may be asked to endorse a series of ten emotion words (e.g., anger, fear, happy, etc.) on a Likert scale across several sampling instances. Emotional granularity would then be computed as the intra-class correlation (ICC) of ratings across sampling instances. A high ICC would suggest that a partic-ipant experiences all of the emotions in a similar way across trials (treating them as synonyms for more general affectual states such as `unpleasant\-ness'' or `pleasantness''), whereas a low ICC would suggest that a participant experienced emotions in a granular and context-specific way.

While emotion granularity is generally measured between emotion categories that are close to each other in the affective space (i.e, express similar valence), the concept of dialecticisrz refers to the co-incidental experience of both negative and pos-itive emotions (Lindquist and Barrett, 2008). Di-alecticism can therefore be operationalized as the co-endorsement of emotion pairs that express posi-tive and negative valence.

\subsection{Language and Mental Health}
Given the limitations of self-report surveys (e.g., limited data coverage and time spans, biases, etc.\ (Kragel et a1., 2022)), another approach to measure well-being indicators is through one's language usage. Some well-known linguistic indicators of mental health include the proportion of pronouns used for those with depression (Koops et a1., 2023), syntax reduction for anorexia nervosa (Cuteri et al., 2022), certain lexical and syntactic features for mild cognitive impairment and dementia (Calzir et al., 2021; Gagliardi and Tamburini, 202l), and semantic connectedness for schizophrenia (Corco-ran et al., 2020).

Recently, another linguistic feature that re-searchers leveraged for insights into overall well-being, are the emotions expressed in language. Largely, only sentiment has been explored and mainly from social media data (a rich source of language data). For example, more negative sen-timent was expressed in text by individuals with depression (De Choudhury et al., 2013; Seabrook et a1., 2018; De Choudhury et al., 2021). Other work has found that suicide watch, anxiety, and self-harm subreddits had markedly lower negative sentiment compared to other mental health subred-dits such as Autism and Asperger's (Gkotsis et al., 2016).

Hipson and Mohammad (2021) and Vishnub-hotla and Mohammad (2022) introduced Utterance Emotion Dynamics (UED), a framework to quan-tify patterns of change of emotional states associ-ated with utterances along a longitudinal (tempo-ral) axis (using data from screenplays and tweets). Teodorescu et al.\ (2023) found that measures of emotion dynamics from text correlate with various mental health diagnoses.

These works overall show that the average emo-tion expressed in text and also the characteristics of individual emotion change over time (e.g., vari-ability) are meaningful indicators of well-being. In this work, we explore whether the degree of co-expression of pairs of emotions in text (emotion granularity) is a meaningful indicator of mental health.

\section{Datasets}
We use the Twitter-STMHD dataset (Suhavi et a1., 2022) for our experiments and also verify our re-sults with a smaller Reddit eRisk (Losada et a1., 2017, 2018) dataset. We describe both of them below.

\paragraph{Twitter-STMHD dataset:}
Suhavi et al.\ (2022) identified tweeters who self-disclosed as having an MHC diagnosis using carefully constructed regular expression patterns and manual verification. We summarize key details on the dataset creation process in Appendix A. The control group consists of users identified from a random sample of tweets (who posted during approximately the same time period as the MHC tweets). These tweeters did not post any tweets meeting the MHC regex de-scribed above. Additionally, users who had any posts about mental health discourse were removed from the control group. Note that this process does not guarantee that users in the control group did not have an MHC diagnosis, but rather the group as a whole may have very few tweeters from these MHC groups. The number of users in the control group was selected to match the size of the de-pression dataset, which had the largest number of users.

For the final set of users, four years of tweets were collected for each user: two years before self-reporting a mental health diagnosis and two years after. For the control group, tweets were randomly sampled from between January 2011 and May 2021 (same date range as the other MHC classes).

\paragraph{Reddit eRisk dataset:}
To further add to our find-ings, we also included the eRisk 2018 dataset (Losada et a1., 2017, 2018) in our experiments. It consists of users who self-disclosed as having depression on Reddit (expressions were manually checked), and a control group (individuals were randomly sampled). The dataset includes several hundred posts per user, over approximately a 500-day period. We combined users and their instances from both the training set (which is from the eRisk 2017 task (Losada et al., 2017)) and the rest set (from the eRisk 2018 task (Losada et al., 2018)).

\subsection{Preprocessing}
We further preprocessed both the Twitter-STMHD dataset and the eRisk dataset for our experiments (Section 4), as we are specifically interested in the relationship between emotion granularity and each disorder. Several users self-reported as being di-agnosed with more than one disorder, referred to as comorbidity. We found a high comorbidity rate between users who self-reported as having anxiety and depression, as is also supported in the literature (Pollack, 2005; Gorman, 1996; Hirschfeld, 2001; Cummings et a1., 2014). Since we wanted to focus on each MHC separately (and not on co-morbidity) we only considered users who self-reported as having one MHC. We also performed the following preprocessing steps:
\begin{itemize}
\item We only considered posts in English.
\item We filtered out posts that contained URLs (the text in such posts is often not self-contained).
\item We removed retweets (identified through tweets containing `RT', `rt'). This is to fo-cus exclusively on texts written by the user.
\item To ensure that we did not include users that post very infrequently or very frequently, we excluded users based on the number of posts per individual. We discarded data from those who either had less than 100 posts (as was similarly done in Vishnubhotla and Moham-mad (2022)) and those who had posted more than 1.5 times the interquartile range above quartile three (75th percentile) of the control group.\footnote{The interquartile range is from the 25th to 75th percentile.}
\end{itemize}

Table~\ref{tab:table1} shows key details of the flltered Twitter-STMHD and Reddit eRisk datasets.

\begin{table}[t]
\centering
\small
\begin{tabular}{l l r r r}
\toprule
Dataset & Group & #people & Av.\ #posts & Av.\ #tokens/post\
\midrule
Twitter & MHC & t9,324 & 2,590.48 & 17.59\
& ADHD & 6,356 & 2,497.43 & 17.46\
& Anxiety & 3,036 & 2,921.05 & l'7.46\
& Bipolar & 1,061 & 2,820.17 & t7.32\
& Depression & 4,855 & 2,526.62 & 16.75\
& MDD & 2t9 & 2,640.69 & 16.40\
& OCD & 1,009 & 2,388.73 & 18.38\
& PPD & t19 & 2,58t.19 & 19.18\
& PTSD & 2,609 & 2,533.85 & 19.41\
& Control & 6,001 & 2,420.50 & 16.16\
\midrule
Reddit & Depression & tt2 & s56.57 & 47.22\
& Control & 90'7 & 665.00 & 41.09\
\bottomrule
\end{tabular}
\caption{The number of users in each MHC, the average number of posts per user, and the average number of tokens per post in the preprocessed version of the Twitter-STMHD and Reddit eRisk datasets. Entries reflect the source PDF text; unclear characters are left as-is.}
\label{tab:table1}
\end{table}

\section{Emotion Granularity from Text}
The core metric that we want to capture from the text utterances of an individual is emotion granularity---what psychologists term the ``co-endorsement'' of pairs of emotions. Analogous to their operationalization of granularity in terms of the Intra-Class Correlation (ICC) of repeated emotion intensity measurements along emotion ad-jectives, we use textual utterances to derive a tem-poral sequence of emotion states, referred to as an emotion arc for the speaker (Section~4.1), and op-erationalize granularity as the correlations of these arcs. We construct emotion arcs for multiple emo-tions, for each user in the MHC groups and the control group.

\paragraph{Emotion Dimensions:}
A key requirement of our computational method is that we must be able to quantify the emotional score of a text along a se-lected emotion dimension. We are therefore limited by the resources and models available to compute such a score for an emotion dimension.

Here, keeping in mind the necessity of includ-ing multiple emotion dimensions that are similarly-valenced, we work with the eight emotions rep-resented in the NRC Emotion Intensity Lexicon: anger, anticipation, dis gus t, fear, joy, sadness, sur-prise, and trust (Mohammad, 2018).

We partition these emotions into three groups based on the valence association: joy and trust are in the positive valence group; anger, sadness, fear, and disgust are in the negative valence group; and anticipation and surprise are in the variable valence group. The distinction for surprise and anticipation is necessary because specific instances of these emotions can have a positive or a negative connotation (e.g., a good or a bad surprise).

\subsection{Constructing Emotion Arcs}
We order the utterances for each user based on timestamp information in the metadata. We construct emotion arcs for the temporal sequence of the utterances of each user, along each of the eight emotions, in two ways pertaining to different window sizes. This is to make sure that the results are largely robust even when varying the window size to some extent.

\paragraph{Utterance-level Window:}
Emotion scores (for each emotion category) are computed for each utter-ance (i.e, tweet or Reddit post). Here, an utterance is assumed to represent the speaker's emotion state at a particular point in time (analogous to sampling instances). The sequence of utterance emotion scores for a user forms their temporal emotion arc.\footnote{The frequency and time of posting often differs between users, but we ignore that for now.}

\paragraph{Word-Count based Window:}
Here, the emotion score at a point in time is computed for a window of words (say, 100 words) that are uttered around that point, and the window is moved forward by a fixed step size (say, 1 word at a time) to obtain the emotion score for the next time step. In prior work on constructing emotion arcs from temporally-ordered text, such sliding windows are usually employed to ensure smoother arcs that more accurately capture the flow of emotions over time.

Teodorescu and Mohammad (2023) conducted extensive quantitative evaluations of several hy-perparameters involved in emotion arc construc-tion, on datasets from diverse domains (including tweets) annotated with emotion scores. We fol-low many of their recommendations to construct emotion arcs for the utterances of each of our users. The texts are tokenized using the \texttt{twokenizers} package\footnote{\url{[https://github.com/myleott/ark-twokenize-py}}](https://github.com/myleott/ark-twokenize-py}}) to obtain a similarly-ordered sequence of words. Emotion scores are computed with window sizes of 100 words and 500 words each, and the window is moved forward by one word at each timepoint to obtain a series of overlapping emotion scores.

\paragraph{Emotion scoring method:}
Keeping in mind the necessity of an interpretable method of emotion scoring, we use word-emotion lexicons to compute the emotion scores of text spans. For each win-dow, the emotion scores of its constituent words are averaged to obtain the window-level score for that emotion. Teodorescu and Mohammad (2023) showed that emotion arcs constructed with lexicon-based scoring methods, when used with sliding window sizes of 100 instances or more, can mimic the ground-truth emotion arcs with an accuracy of 0.9 or more.

Word-emotion scores are obtained from the NRC Emotion Intensity Lexicon, which associates close to 10,000 English words with a real-valued score between 0 and 1 for each dimension. A score of 0 indicates that the word has little to no associa-tion with that particular emotion, and a score of 1 indicates a high association.

\paragraph{Qualitative Checks on Emotion Lexicons:}
Lexicon-based methods for constructing emotion arcs are reliable and interpretable; however, it is good practice to modify the lexicon to the specific domain of use, in order to account for terms that are expected to be used in the target domain in a sense different from the predominant word sense annotated in the lexicons (Mohammad, 2023).

We identify and remove words and bigrams whose usage on Twitter (and sometimes more collo-quially) is markedly different from the predominant word sense annotated in the lexicons, such as \textit{like} and \textit{chaotic evil}. We also remove words and bi-grams that are explicitly associated with mental health, such as \textit{aruxious}, \textit{disorder} and \textit{panic attack}. Though our EG metric does not explicitly rely on the presence of such terms to flnd associations with MHCs, we remove them in order to capture more fundamental differences in emotional expression between users in the MHC groups and the control group. The full list of stopwords is in Appendix B.

\paragraph{Hyperparameters:}
We additionally make the following choices of hyperparameters for constructing and comparing a pair of emotion arcs:
\begin{itemize}
\item For a given pair of emotions, we drop all emo-tion terms that are common to the two lexicons before constructing their emotion arcs. This ensures that we are not using words associ-ated with both emotions, giving us a clearer indication of co-endorsement.
\item An utterance (or window) with no emotion terms from a particular emotion lexicon is assigned a score of 0. An alternative is to assign them a score of \texttt{nan}, in which case they are not considered a part of the emotion arc.\footnote{We do not observe any major changes to our results based on these hyperparameter choices.}
\end{itemize}

A visualization of the emotion arcs obtained us-ing the utterance-level window for a sampled user from the Twitter-STMHD dataset is presented in Appendix E.

\subsection{Quantifying Emotion Granularity}
We compute the emotion granularity metric as the negative of the Spearman correlation between each pair of emotions arcs, for each user.\footnote{We choose Spearman correlation as it is rank-based as compared to Pearson correlation which utilizes the raw-values.} A high corre-lation between two arcs indicates that the speaker is consistently and repeatedly expressing the two emotions concurrently; we hypothesize that this is an indicator of a lower ability to dffirentiate between the two emotions, and therefore a lower emotion granularity.\footnote{We choose to use Spearman correlation over ICC-based metrics because the emotion scores that we extract from tex-tual utterances are a relative indicator of the intensity of the emotion, and not an absolute measure. Further, these scores cannot be directly compared across different emotions in terms of absolute intensity (\emph{a score of 0.9 for anger may not equate to the same level of anger as a score of 0.9 would for joy}) due to differences in how overtly different emotions are expressed via language.}

For each person, we average the correlation scores between emotion pairs in the different va-lence groups to obtain the following measures of emotion granularity (EG):
\begin{itemize}
\item EG(pos): The negative of (i.e., $-1$ times) the average of the correlation scores between each of the pairs of emotions in the positive valence group (joy-trust).
\item EG(neg): The negative of the average of the correlation scores between each of the pairs of emotions in the negative valence group (anger-fear, fear-disgust, etc.).
\item EG(var): The negative of the average of the correlation scores between each of the pairs of emotions in the variable valence group (surprise-anticipation).
\item EG(overall): Overall emotion granularity, measured as the negative of the average of the correlation scores between emotion pairs whose constituents are in the same group. Here, the average is taken across all of the emotion pairs drawn from the positive valence group, the negative valence group, and the variable valence group.
\item EG(cross): Emotion granularity of cross-group emotion pairs. That is, the negative of the average of the correlation scores between emotion pairs whose constituents come from different groups. This measure to some ex-tent quantifies the amount of dialecticism (ex-pressing both positive and negative emotions in a narrow window of time); however, note that EG(cross) also includes emotions that ex-press variable valence (surprise and anticipa-tion), rather than only considering positive-negative valence emotion pairs.
\end{itemize}

We consider EG(overall) to be the bottom line measure of emotion granularity for a user (analo-gous to that used in psychology studies). Note that cross-group pairs are not included in this measure.

\section{Emotion Granularity and Mental Health}
We now test if there are significant differences between the emotion granularities of each of the MHC groups and the control group, using t-tests. We first limit the users in each group by placing thresholds on (a) the number of user tweets with a valid emotion score (set to a minimum of 50), and (b) the number of unique lexicon terms used in their tweets (set to a minimum of 25). These thresh-olds ensure that we are drawing inferences based on users with valid emotion arcs, with sufficient lexicon coverage and temporal information.

We performed independent t-tests to compare emotion granularities between each of the MHCs and the control group, for each emotion group, us-ing the SciPy library (Virtanen et al., 2020). To correct for multiple comparisons (eight tests per-formed for each MHC per emotion granularity group), we used the Benjamini-Hochberg proce-dure in the \texttt{statsmodels} library (Seabold and Perk-told, 2010). Further details on the data assumptions for t-tests are in Appendix C.

\subsection{Term Specificity as a Control}
Lower emotion granularity occurs when, for a per-son, the concepts of the relevant emotions are so broad (and non-specific) that their meanings over-lap substantially. This work is testing the hypoth-esis of whether people who have self-disclosed as having an MHC have lower emotion granularity than those that do not. However, another plausi-ble hypothesis is that people in a particular group (e.g., MHC or the control) tend to use more spe-ciflc words overall. Doing so would imply a higher specificity (i.e, a higher granularity) in their usage of all words, and that the high granularity of emo-tion words is simply a by-product of their general style of speaking (or posting online).

To ensure that the level of word specificity does not differ between MHCs and the control group and act as a confounder for our measure of emo-tion granularity, we performed a control experi-ment. We compute the average information content of the noun and verb terms in the posts of users in each group, and use this as a measure of the specificity of their language. We use the metric proposed in Resnik (1995), and implemented in the NLTK WordNet library,\footnote{\url{[https://github.com/nItk/wordnet}}](https://github.com/nItk/wordnet}}) which combines informa-tion about the depth of the term in the WordNet tree hierarchy and its frequency of occurrence in a large corpus (here, the Brown corpus) to com-pute an information content score (Miller, 1995). We then compute the following measures of term specificity for each user:
\begin{itemize}
\item IC(n): The information content score for all nouns is averaged across all posts of each individual in each group.
\item IC(v): The information content score for all verbs is averaged across all posts of each indi-vidual in each group.
\end{itemize}
Statistical tests for significant differences are simi-larly performed as described above.

\section{Results}
In Table~\ref{tab:table2}, we report the statistical results from the pairwise comparisons between each MHC and the control group, for the control experiment on gen-eral term speciflcity as well as emotion granularity, when scores are computed at the utterance-level. All statistically signifi cant differences between an MHC and the control group are described as either `higher'' or `lower'', and a dash (-) for no statistical difference. A `lower'' value in a cell in\-dicates that the MHC (rows) has lower emotion granularity (or lower term specificity) than the con\-trol group, i,e., higher correlation between emotion pairs in that group (columns); `higher'' indicates the MHC has higher emotion granularity (or higher term specificity) than the control group (i.e., lower correlation between emotion pairs in that group). In Table 12 in the Appendix, we also report the ab-solute Spearman correlation scores for each group. Below we summarize the results for each column.

\begin{table}[t]
\centering
\small
\begin{tabular}{l c c c c c c c}
\toprule
Dataset, MHC-Control & IC(n) & IC(v) & EG(pos) & EG(neg) & EG(var) & EG(cross) & EG(overall)\
\midrule
ADHD-control & [MISSING] & [MISSING] & lower & lower & lower & lower & lower\
Anxiety-control & [MISSING] & [MISSING] & lower & lower & lower & lower & lower\
Bipolar-control & [MISSING] & [MISSING] & lower & lower & lower & lower & lower\
MDD-control & [MISSING] & [MISSING] & lower & [MISSING] & [MISSING] & lower & Iower\
OCD-control & [MISSING] & [MISSING] & lower & lower & lower & lower & lower\
PPD-control & [MISSING] & [MISSING] & lower & [MISSING] & [MISSING] & [MISSING] & [MISSING]\
PTSD-control & [MISSING] & [MISSING] & lower & lower & lower & lower & lower\
Depression-control & [MISSING] & [MISSING] & lower & lower & lower & lower & lower\
\midrule
Reddit eRisk Depression-control & [MISSING] & [MISSING] & lower & lower & lower & [MISSING] & [MISSING]\
\bottomrule
\end{tabular}
\caption{The difference in emotion granularity between each MHC group and the control. A significant difference is indicated by the word `lower'' or `higher'', indicating the direction of the difference in granularity. Unavailable cells are marked as [MISSING].}
\label{tab:table2}
\end{table}

\subsection{Emotion Granularity as an Indicator of MHCs}
IC(n) and IC(v): We do not see any signiflcant differences in the information content of noun and verb terms (IC(n) and IC(v)) between MHCs and the control group. This indicates that no group tends to use more specific or less specific language in general when posting on these platforms. Details on the statistical results are shown in Appendix H.

EG(pos): All MHCs except for PPD had signifl-cantly lower positive emotion granularity than the control group (which had similar granularity com-pared to the control group). That is, tweeters in these MHC groups (ADHD, Anxiety, etc.) consis-tently expressed multiple positive emotions concur-rently, more so than the control group.

EG(neg): All MHCs except MDD had signifl-cantly lower negative emotion granularity than the control group, in both datasets. Thus, tweeters in these MHCs were generally not differentiating be-tween the negative emotions of anger, disgust, fear, and sadness, as well as the control group.

EG(var): Tweeters in the ADHD, Anxiety, Bipolar, OCD, PTSD, and Depression (Twitter-STMHD) had signiflcantly lower variable emotion granular-ity than the control group (i.e., these groups gen-erally differentiated between surprise and anticipa-tion less than the control group).

EG(overall): All MHCs except PPD had signifl-cantly lower emotion granularity for emotion cat-egories that express the same valence (the mixed valence emotions of surprise and anticipation are also included here). Tweeters in these groups are therefore expressing multiple close emotions fre-quently with one another --- more so than the control group.

EG(cross): All MHCs except PPD had signifl-cantly lower granularity between emotion pairs that come from different valence groups. This in-dicates that positive and negative emotions are ex-pressed together more frequently by tweeters in these groups compared to the control, as well as emotions like joy (positive valence) and surprise (variable valence).

Discussion: These results demonstrate that our measures of emotion granularity from text are con-sistently lower for users in the MHC groups com-pared to the control. The term specificity results also tell us that it is the specificity of emotion word usage in particular that is differentiating MHCs from the control group.

Aligning with self-report studies in psychology, the emotion granularity between negative-valence emotions is lower for most (7 out of 8) MHCs in our datasets with utterance-level operationaliza-tions. Positive emotion granularity is also corre-lated with many of the MHCs (7 out of 8 disorders). In general, the granularity of emotional expression between within-group emotion pairs is lower for all MHC groups compared to the control in both datasets, except PPD. This is in line with both the theoretical and conceptual links established in the psychology literature on emotion granularity and mental health: the ability to better differentiate between emotion concepts that are close to one another leads to more adaptive health behaviour.

While emotion pairs from differently-valenced emotion groups are not usually operationalized in affective science experiments, we find that this mea-sure is also significantly lower in many MHCs. Further investigations into what the concurrent ex-pression of positive and negative emotions means, for emotion granularity and emotion dynamics in general, are interesting research directions.

Variation with hyperparameter choices: We ob-serve only minor variations from the results re-ported in Table 2 when the hyperparameters de-scribed in Section 4.7 for emotion arc construction were changed --- less than 10% of the cells differed in their values across all variations. We provide a more detailed report in Appendix F.

\subsection{Additional Window Sizes}
We also examined how the measures of differences in emotion granularity between MHCs and the con-trol change when we compute emotion scores with larger window sizes.

Many of the utterance-level outcomes are repli-cated for negative, positive, and overall emotion granularity with window sizes 100 and 500. Some measures are no longer significantly different be-tween certain MHCs and the control. We also find that EG(cross) is higher for certain MHCs (Anxi-ety, PPD, PTSD, Depression in Twitter-STMHD) when compared to the control, i.e., users in the control group are expressing negative and positive emotions together more frequently than those in the MHCs.

With larger window sizes, we end up capturing emotions expressed by the individual over longer time spans (tweets posted over the span of several hours or days), rather than co-endorsement at the same time. We hypothesize that these effects of dialecticism, where the control group has a higher co-occurrence of cross-valence group emotions, are capturing the extent to which users balance neg-ative emotions with positive emotions (and vice versa). The consistent effects with 100 and 500-word windows, and for several MHCs, makes this a promising area for future work. All emotion gran-ularity measures with window sized 100 and 500 are reported in Appendix F.1.

\subsection{Individual Emotion Pairs}
In order to understand which emotion pairs are expressed together more frequently (resulting in lower emotion granularity), we performed the same significance tests as before between MHCs and the control for correlation scores between all individual emotion pairs. We found that:
\begin{itemize}
\item Seven out of the eight MHCs in the Twitter-STMHD dataset had a lower granularity (a higher correlation) for anger-disgust (except PPD) and anger-sadness (except MDD) in the negative valence group.
\item All eight MHCs had a lower emotion granu-larity (higher correlation) between multiple cross-group emotion pairs, notably those in-volving the mixed-valence emotions of antici-pation and trust.
\item Contrary to trends, the Bipolar group had a higher emotion granularity (i.e., a lower cor-relation of emotion arcs) for the cross-group emotion pairs of anger-joy and fear-joy.
\end{itemize}
Detailed results for each of the emotion pairs and all MHCs are in Appendix G, Table 10.

Discussion: While lower granularity among cer-tain emotion pairs consistently function as indica-tors of all MHCs, we also see a few instances where MHCs (specifically Bipolar disorder) have a higher granularity between the emotions when compared to the control. These findings are of interest to researchers studying the links between how emo-tions are expressed in text, and how they vary with different MHCs.

\section{Conclusion}
In this work, we operationalized for the first time a computational measure of emotion granularity that can be derived from the textual utterances of individuals. We applied this measure to two so-cial media datasets of posts from individuals who have self-disclosed as having an MHC. Our find-ings showed that our measure of negative emotion granularity is significantly lower for 7 out of the 8 MHC groups under consideration when compared to a control group, at an aggregate-level. Also, all MHCs except for PPD had lower overall emo-tion granularity (and lower positive emotion gran-ularity) compared to the control group. Our work makes an important contribution towards deriving aggregate-level indicators of emotional health from the large amounts of utterance data available on social media platforms. We hope this opens up an avenue of future work to explore emotional expres-sion in text and mental health.

\section*{Limitations}
Our work uses the social media utterances of in-dividuals to derive measures of emotional expres-sion that, at an aggregate level, are found to corre-late with multiple mental health conditions. While we use datasets that were compiled by other re-searchers in the fleld, we stress that they may not be representative of the general population. Our methods therefore cannot be directly applied to make inferences on other datasets without a care-ful experimental validation first. The datasets we study rely on self-disclosures made on social me-dia platforms; it is possible that users report only one such MHC but are diagnosed with others, or that they misrepresent their diagnoses. Further, the users in the control groups may include those who have chosen to simply not self-disclose on these platforms. This can occur due to many reasons, like social desirability (Latkin et a1., 2017) or impres-sion management (Tedeschi, 20 1 3). Nonetheless, since we draw inferences at an aggregate level, the methods used can overcome some amount of noisy data.

The set of emotions that we have considered in our measurement of emotion granularity are also limited to those for which we can computationally obtain text-derived emotion scores. These eight emotions do not represent the wide range of emo-tion concepts that exist and are experienced and expressed by us with language, and future research can attempt to expand our operationalization to more emotion concepts. It should be noted though, that past psychology studies on emotion granularity have also tended to explore small sets of emotions, largely because it is cumbersome to ask users about how they feel for a large set of emotions.

The emotion lexicons that we use are some of the largest that exist with wide coverage and large number of annotators (thousands of people as op-posed to just a handful). However, no lexicon can cover the full range of linguistic and cultural diver-sity in emotion expression. The lexicons are largely restricted to words that are most commonly used in Standard American English and they capture emotion associations as judged by American native speakers of English. See Mohammad (2023) for a discussion of the limitations and best-practises in the use of emotion lexicons.

Lastly, further work should explore if the relationships we found hold around various social factors such as age, region, language, etc. As we focus on English text, and the region of users is not known (some information could be extracted from user proflles in the Twitter-STMHD dataset however it is fairly noisy), conclusions should be drawn cautiously across various sociolinguistic factors.

\section*{Ethics Statement}
Our approach, as with all data-driven models of determining indicators of mental health, should be considered as aggregate-level indicators, rather than biomarkers for individuals (Guntuku et a1., 2017). We do not attempt to predict the presence of MHCs for individual users at any stage of the process. These measures should also not be taken as standalone indicators of mental health or mental wellness, even at the population level, but rather as an additional metric that can be used in conjunction with other population-level markers, and with the expertise of clinicians, psychologists, and public health experts.

Individuals vary considerably in how, and how well, they express their internal emotional states using language. Our method of assessing the emo-tional states of users based on their utterances may miss several linguistic cues of emotion expression, and may not account for individual variation or the extent to which these emotions are expressed on social media. The emotionality of one's language may also be conveying information about the emo-tions of the speaker, the listener, or something or someone else mentioned in the utterances. See further discussions of ethical considerations when using computational methods for affective science in Mohamm ad (2023, 2022).

\begin{thebibliography}{99}
\bibitem{refs}
Lisa Feldman Barrett, James Jonathan Gross, Tam-lin Conner Christensen, and Michael Benvenuto. 2001. Knowing what you're feeling and knowing what to do about it: Mapping the relation between emotion differentiation and emotion regulation. Cog-nition and Emotion, l5713 -724.\
Adrienne S Bonar, Jennifer K MacCormack, Mallory J Feldman, and Kristen A Lindquist. 2023. Examining the role of emotion differentiation on emotion and cardiovascular physiological activity during acute stress. Affective Science, pages 1-15.\
LawaCalzd,, Gloria Gagliardi, Rema Rossini Favretti, and Fabio Tamburini. 2021 . Linguistic features and automatic classifiers for identifying mild cognitive impairment and dementia. Computer Speech & Lon-guage, 65:101113.\
Cheryl M. Corcoran, Vijay A. Mittal, Carrie E. Bear-den, Raquel E. Gur, Kasia Hitczenko, Zarina Bil-grami, Aleksandar Savic, Guillermo A. Cecchi, and Phillip Wolff.2020. Language as a biomarker for psychosis: A natural language processing approach. S c hi zo p hre nia Re s e arc h, 226: 1 58-l 66. Biomarkers in the Attenuated Psychosis Syndrome.\
Colleen M. Cummings, Nicole E. Caporino, and Philip C. Kendall. 2014. Comorbidity of anxiety and depression in children and adolescents: 20 years after. P sy c ho I o gic al B ul letin, I 40(3 ): 8 I 6-845.\
Vittoria Cuteri, Giulia Minori, Gloria Gagliardi, Fabio Tamburini, Elisabetta Malaspina, Paola Gualandi, Francesca Rossi, Milena Moscano, Valentina Fran-cia, and Antonia Parmeggiani. 2022. Lrngdstic fea-ture of anorexia nervosa: a prospective case-control pilot study. Eating and weight disorder,s : EWD, 27 (4):1367 -137 5 .\
Munmun De Choudhury, Scott Counts, and Eric Horvitz. 2013. Social media as a measurement tool of de-pression in populations. ln Proceedings of the 5th Annual ACM Web Science Conference, WebSci '13, page 47-56, New York, NY, USA. Association for Computing Machinery.\
Munmun De Choudhury Michael Gamon, Scott Counts, and Eric Horvitz. 2021. Predicting depression via social media. Proceedings of the International AAAI Conference on Web and Social Media,l(l):128-137 .\
Emre Demiralp, Renee J Thompson, Jutta Mata, Su-sanne M Jaeggi, Martin Buschkuehl, Lisa Feldman Barrett, Phoebe C Ellsworth, Metin Demiralp, Luis Hemandez-Garcia, Patricia J Deldin, et al. 2012. Feeling blue or turquoise? emotional differentiation in major depressive disorder. Psychological science, 23(11):1410-1416,\
Katherine L. Dixon-Gordon, Alexander L. Chapman, Nicole H. Weiss, and M. Zachary Rosenthal. 2014. A preliminary examination of the role of emotion differentiation in the relationship between border-line personality and urges for maladaptive behaviors. JoLtrnal of Psychopathology and Behavioral Assess-ment,36:676-625.\
Noah N Emery, Jeffrey S Simons, C Joseph Clarke, and Raluca M Gaher. 2014. Emotion differentiation and alcohol-related problems: The mediating role of urgency. Addic tiv e B e hav iors, 3 9( 1 0) : I 459-1 463.\
Yasemin Erbas, Eva Ceulemans, Johanna Boonen, Ilse Noens, and Peter Kuppens. 2013. Emotion differ-entiation in autism spectrum disorder. Research in Aut i sm S p e c trum D i s o rde rs, 7 (l 0): 122l-1227 .\
Yasemin Erbas, Eva Ceulemans, Elise K Kalokerinos, Marlies Houben, Peter Koval, Madeline L Pe, and Peter Kuppens. 2018. Why i don't always know what i'm feeling: The role of stress in within-person fluctuations in emotion differentiation. Journal of p e r s onality and S o c i al P sy c h o I o gy, I I 5 (2) : 17 9 .\
Yasemin Erbas, Eva Ceulemans, Madeline Lee Pe, Pe-ter Koval, and Peter Kuppens. 2014. Negative emo-tion differentiation: Its personality and well-being correlates and a comparison of different assessment methods. Cognition and Emotion,2S:1196 - 1213.\
Gloria Gagliardi and Fabio Tamburini.202l. Linguistic biomarkers for the detection of mild cognitive im-pairment. Lingue e linguaggio, Rivista semestrale, (112021):3-31.\
George Gkotsis, Anika Oellrich, Tim Hubbard, Richard Dobson, Maria Liakata, Sumithra Velupillai, and Rina Dutta. 2016. The language of mental health problems in social media. In Proceedings of the Third Workshop on Computational Linguistics and Clinical Psychology, pages 63-73, San Diego, CA, USA. Association for Computational Linguistics.\
Jack M. Gorman. 1996. Comorbid depression and anx-iety spectrum disorders. Depression and Anxiety, 4(4):1 60-1 68.\
Sharath Chandra Guntuku, David Bryce Yaden, Mar-garet L. Kern, Lyle H. Ungar, and Johannes C. Eich-staedt. 2017. Detecting depression and mental illness on social media: an integrative review. Curuent Opin-ion in Behavioral Sciences, 18:4349.\
Will E. Hipson and Saif M. Mohammad.2021. Emotion dynamics in movie dialogues. PLOS ONE, 16:l-19.\
Robert M. A. Hirschfeld. 2001. The comorbidity of major depression and anxiety disorders: Recogni-tion and management in primary care. Primary care companion to the Journal of clinical psychiatry, 3(6):244-254.\
Katie Hoemann, Lisa Feldman Barrett, and Karen S. Quigley. 2021a. Emotional granularity increases with intensive ambulatory assessment: Methodological and individual factors influence how much. Frontiers in Psychology, 12.\
Katie Hoemann, Cathy Nielson, Ashley Yuen, Jacob Gurera, Karen S. Quigley, and Lisa Feldman Bar-rctl2021b. Expertise in emotion: A scoping review and unifying framework for individual differences in the mental representation of emotional experience. P sycholo gical bulletin, 147 1 I :l I 59-l I 83.\
Marlies Houben, Wim Van Den Noortgate, and Peter Kuppens. 2015. The relation between shofi-term emotion dynamics and psychological well-being: A meta-analysis.\
Todd B Kashdan, Lisa Feldman Barrett, and Patrick E McKnight. 201 5. Unpacking emotion differentiation: Transforming unpleasant experience by perceiving distinctions in negativity. Current Directions in Psy-cho lo g ic al S c ienc e, 24(l): l0-1 6.\
Sanne Koops, Sanne G Brederoo, Janna N de Boer, Femke G Nadema, Alban E Voppel, and Iris E Sommer. 2023. Speech as a biomarker for depres-sion. CNS&,' neurological disorders drug targets, 22(2):152-160.\
Philip A. Kragel, Ahmad R. Hariri, and Kevin S. LaBar. 2022. The temporal dynamics of spontaneous emo-tional brain states and their implications for mental health. Journal of cognitive neuroscience,34(5):7 15-728. May,2022.\
Ann M Kring, Lisa Feldman Barrett, and David E Gard. 2003. On the broad applicability of the affective circumplex: representations of affective knowledge among schizophrenia patients. Psychological Sci-ence, 14(3):207-214.\
Peter Kuppens and Philippe Verduyn. 2017. Emotion dynamics. Current Optnion in P sycholo gy, 17 :22-26. Emotion.\
Carl A Latkin, Catie Edwards, Melissa A Davey-Rothwell, and Karin E Tobin. 2017. The relation-ship between social desirability bias and self-reports of health, substance use, and social network factors among urban substance users in baltimore, maryland. Addict iv e b e hav i o rs, 7 3 : 133-13 6.\
Ja Y. Lee, Kristen A. Lindquist, and Chang S. Nam. 2017. Emotional granularity effects on event-related brain potentials during affective picture processing. Frontiers in Human Neuroscience, ll.\
Palaniyappan Lena. 2021. More than a biomarker: could language be a biosocial marker of psy-chosis? NPI Schizophrenia, T(l). [ILLEGIBLE] \
Michael Lewis, Jeannette M Haviland-Jones, and Lisa Feldman Barrett. 2010. Handbook of emotions. Guilford Press.\
Kristen A Lindquist and Lisa Feldman Barrett. 2008. Emotional complexity. Handbook of emotions, 4:5 l3-530.\
David E Losada, Fabio Crestani, and Javier Parapar. 2011 . Clef 2017 erisk overview: Early risk prediction on the internet: Experimental foundations. pages 346-360.\
David E Losada, Fabio Crestani, and Javier Parapar. 2018. Overview of erisk: early risk prediction on the internet. In Experimental IR Meets Multilingual-ity, Multimodality, and Interaction: 9th International Conference of the CLEF Association, CLEF 2018, Avignon, France, September 10-14, 201 8, Proceed-ings 9,\
Saif Mohammad.2023. Best practices in the creation and use of emotion lexicons. ln Finding,s o.f the Asso-ciation for Computational Linguistics: EACL 2023, pages I 825-1836, Dubrovnik, Croatia. Association for Computational Linguistics.\
Saif M. Mohammad. 2018. Word affect intensities. In Proceedtngs of the I lth Edition of the Language Re-s o urc e s and Ev al uation C onfe renc e ( LRE C -2 0 I 8 ), Miyazaki, Japan.\
Saif M. Mohammad. 2022. Ethics sheet for automatic emotion recognition and sentiment analysis. Compu' tational Linguistics, 48(2):239-27 8 .\
Mark H Pollack. 2005. Comorbid anxiety and depres-sion. lournal of Clinical Psychiatry,66:22.\
Richard S Pond Jr, Todd B Kashdan, C Nathan DeWall, Antonina Savostyanova, Nathaniel M Lambert, and Frank D Fincham. 2012. Emotion differentiation moderates aggressive tendencies in angry people: A daily diary analysis. Emotion, 12(2):326.\
Anne M Reitsema, Bertus F Jeronimus, Marijn van Dijk, and Peter de Jonge.2022. Emotion dynamics in children and adolescents: A meta-analytic and descriptive review. Emotion, 22(2):37 4.\
Philip Resnik. 1995. Using information content to eval-uate semantic similarity in a taxonomy. In Interna-tional Joint Conference on Artificial Intelligence.\
Skipper Seabold and JosefPerktold. 2010. statsmodels: Econometric and statistical modeling with python. In 9th Python in Science Conference.\
Elizabeth M Seabrook, Margaret L Kem, Ben D Fulcher, and Nikki S Rickard. 2018. Predicting depression from language-based emotion dynamics: Longitudi-nal analysis of facebook and twitter status updates. Med Internet Res, 20(5):e168.\
TH Stanley Seah, Pallavi Aurora, and Karin G Coifman. 2020. Emotion differentiation as a protective factor against the behavioral consequences of rumination: A conceptual replication and extension in the context of social anxiety. B ehavior Therapy, 5 I ( I ): I 35-148.\
Patrick E Shrout and Joseph L Fleiss. 1979. Intraclass correlations: uses in assessing rater reliability. Ps)-cho I o g ic al bull etin, 86(2) :420.\
Jennifer S. Silk, Erika E. Forbes, Diana J. Whalen, Jen-nifer L. Jakubcak, Wesley K. Thompson, Neal D. Ryan, David A. Axelson, Boris Birmaher, and Ronald E. Dahl. 201 l. Daily emotional dynamics in depressed youth: A cell phone ecological momentary assessment sitdy. Journal of Experimental Child P sy cho lo gy, I l0(2) :241 -257.\
Sarah H. Sperry, Molly A. Walsh, and Thomas R. Kwapil. 2020. Emotion dynamics concurrently and prospectively predict mood psychopathology. J our-nal of Affective Disorders, 261 :67 5.\
Lisa R Starr, Rachel Hershenberg, Y Irina Li, and Zoey A Shaw. 2017. When feelings lack precision: Low positive and negative emotion differentiation and depressive symptoms in daily life. Clinical Psy-cholo gic al S c i enc e, 5 (4):613-63 l.\
Suhavi, Asmit Kumar Singh, Udit Arora, Somyadeep Shrivastava, Aryaveer Singh, Rajiv Ratn Shah, and Ponnurangam Kumaraguru. 2022. Twitter-stmhd: An extensive user-level database of multiple men-tal health disorders. Proceedings of the Interna-tional AAAI Conference on Web and Social Media, 16(l): I 182-1 19L\
Michael K Suvak, Brett T Litz, Denise M Sloan, Mary C Zanartni, Lisa Feldman Barrett, and Stefan G Hof-fmann. 2011. Emotional granularity and borderline personality disorder. Journal of abnormal psychol-oey,120(2):414.\
James T Tedeschi. 2073. Impression management the-ory and social psychological research. Academic Press.\
Daniela Teodorescu, Tiffany Cheng, Alona Fyshe, and Saif Mohammad. 2023. Language and mental health: Measures of emotion dynamics from text as linguistic biosocial markers. In Proceedings of the 2023 Con-ference on Empirical Methods in Natural Language Processing, pages 31 17-3 133, Singapore. Associa-tion for Computational Linguistics.\
Daniela Teodorescu and Saif Mohammad. 2023. Eva\-uating emotion arcs across languages: Bridging the global divide in sentiment analysis. ln Findings of the Association for Computational Lingttistics : EMNLP 2023, pages 41244137.\
Michele M. Tugade, Barbara L. Fredrickson, and Lisa Feldman Barrett. 2004. Psychological resilience and positive emotional granularity: examining the benefits of positive emotions on coping and health. Journal of personaliry,72 6:l I 6 l-90.\
Pauli Virtanen, Ralf Gommers, Travis E. Oliphant, Matt Haberland, Tyler Reddy, David Coumapeau, Ev-geni Burovski, Pearu Peterson, Warren Weckesser, Jonathan Bright, St6fan J. van der Walt, Matthew Brett, Joshua Wilson, K. Jarrod Millman, Nikolay Mayorov, Andrew R. J. Nelson, Eric Jones, Robert Cimrman, Ian Henriksen, E. A. Quintero, Charles R. Harris, Anne M. Archibald, An-t6nio H. Ribeiro, Fabian Pedregosa, Paul van Mul-bregt, and SciPy 1.0 Contributors. 2020. SciPy 1.0: Fundamental Algorithms for Scientific Computing in Python. Nature Methods, 17 :261-21 2.\
Krishnapriya Vishnubhotla and Saif M. Mohammad. 2022. Tweet Emotion Dynamics: Emotion word us-age in tweets from US and Canada. In Proceedings of the Thirteenth lnnguage Resources and Evaluation Conference, pages 41624176, Marseille, France. Eu-ropean Language Resources Association.\
Emily C Willroth, Jayde AM Flett, and Iris B Mauss. 2020. Depressive symptoms and deficits in stress-reactive negative, positive, and within-emotion-category differentiation: A daily diary study. Journal of personality, 88(2):17 4-184.
\end{thebibliography}

\appendix

\section{Appendix A}
\textbf{Ttvitter-STMHDDataset}\
Suhavi et al.\ (2022) created a regular expression pattern to identify posts which contained a self-[MISSING]\
Below we describe in more depth the requirements for [MISSING]

\section{Appendix B}
\begin{table}[t]
\centering
\small
\begin{tabular}{l}
\toprule
Twitter-specific words and bigrams removed from the emotion lexicons\
\midrule
love \quad flushot \quad raptor \quad discord\
christmas \quad good day \quad good morning \quad good evening\
birthday \quad good night \quad good afternoon \quad bloody murder\
pretty \quad true crime \quad full time \quad gut punch\
vibe \quad wholesome content \quad slur word \quad life time\
vote \quad jump scares \quad hot chocolate \quad chaotic evil\
trump \quad fever dream \quad chaotic energy \quad chaotic good\
like \quad guilty pleasure \quad chaotic neutral \quad hot mess\
\bottomrule
\end{tabular}
\caption{Table 3: Twitter-specific words and bigrams removed from the emotion lexicons.}
\end{table}

\begin{table}[t]
\centering
\small
\begin{tabular}{l l}
\toprule
\multicolumn{2}{l}{Mental health specific terms removed from the emotion lexicons}\
\midrule
disability & ptsd\
psychosis & adhd\
suicide & depressive\
depressed & disorder\
anxiety & mental health\
anxious & mental illness\
disabled & panic attack\
\bottomrule
\end{tabular}
\caption{Table 4: Mental health specific terms removed from the emotion lexicons.}
\end{table}

\section{Appendix C}
\textbf{Statistical Assumptions}\
Below we describe in more depth the requirements for [MISSING]

\section{Appendix E}
\begin{table}[t]
\centering
\scriptsize
\begin{tabular}{l l r r r r r}
\toprule
emo1 & emo2 & e1-all & e2-all & e1&2-comm & e1-excl & e2-excl\
\midrule
anger & anticipation & 1157 & 782 & 43 & 1114 & 739\
anger & disgust & 1,157 & 886 & 407 & 750 & 479\
anger & fear & tt5'7 & 1343 & 551 & 606 & 792\
anger & jov & tt5'l & 946 & 3 & 1154 & 943\
anger & sadness & 1157 & 1014 & 382 & 775 & 632\
anger & surprise & 1157 & 454 & 102 & 1055 & 352\
anger & trust & 1157 & 1332 & 6 & 1 151 & 1326\
anticipation & disgust & 782 & 886 & ),9 & 763 & 867\
anticipation & fear & '182 & 1343 & 82 & 700 & 1261\
anticipation & jov & '782 & 946 & 283 & 499 & 63\
anticipation & sadness & 782 & 1014 & 3.!- & 750 & 982\
anticipation & surprise & 782 & 454 & 131 & 651 & 323\
anticipation & trust & 782 & 1332 & 283 & 499 & 1M9\
disgust & fear & 886 & 13[ILLEGIBLE] & 43 & 400 & 486\
disgust & jov & 886 & 946 & 1 & 885 & 945\
disgust & sadness & 886 & 1014 & 336 & 550 & 678\
disgust & surprise & 886 & 454 & 56 & 830 & 398\
disgust & trust & 886 & t332 & 2 & 884 & 1 330\
fear & jov & 1343 & 946 & 2 & 1,341, & 944\
fear & sadness & 1343 & 1014 & 545 & 798 & 469\
fear & surprise & 1343 & 454 & 137 & 1,206 & 31,7\
fear & trust & 1343 & 1332 & 9 & 1334 & 1323\
jov & sadness & 946 & 1014 & 0 & 946 & 1014\
jov & surprise & 946 & 454 & 113 & 833 & 341\
jov & trust & 946 & t332 & 308 & 638 & 1,024\
sadness & surprise & 1014 & 454 & 73 & 941 & 381\
sadness & trust & i014 & t332 & 3 & 1011 & t329\
surprise & trust & 454 & 1332 & 56 & 398 & tn6\
\bottomrule
\end{tabular}
\caption{Table 5: Emotion Lexicons: For each emotion pair (emo1, emo2), the number of terms in each lexicon (e1-all, e2-all), the number of emotion terms common to the two lexicons (e1&2-comm), and the number of mutually-exclusive emotion terms (e1-excl, e2-excl).}
\end{table}

\begin{figure}[t]
\centering
\fbox{\parbox{0.9\linewidth}{\centering IMAGE NOT PROVIDED}}
\caption{Emotion arcs com with utterance-level windows [ILLEGIBLE]}
\end{figure}

\section{Appendix F}
\begin{table}[t]
\centering
\small
\begin{tabular}{l c c c c c}
\toprule
MHC-control & EG(pos) & EG(neg) & EG(var) & EG(cross) & EG(overall)\
\midrule
ADHD-control & lower & lower & lower & lower & lower\
Anxiety-control & lower & lower & lower & lower & lower\
Bipolar-control & [MISSING] & lower & lower & lower & lower\
MDD-control & lower & [MISSING] & [MISSING] & lower & Iower\
OCD-control & lower & lower & lower & lower & lower\
PPD-control & [MISSING] & lower & lower & [MISSING] & lower\
PTSD-control & lower & lower & lower & lower & lower\
Depression-control & lower & lower & lower & lower & lower\
Reddit eRisk Depression-control & lower & lower & lower & lower & lower\
\bottomrule
\end{tabular}
\caption{Table 6: Emotion Granularity - hyperparameter variations: Non-lexicon terms and tweets are assigned a score of zero; user tweet and unique term thresholds are both set to 0, and only mutually-exclusive emotion terms are considered.}
\end{table}

\begin{table}[t]
\centering
\small
\begin{tabular}{l c c c c c}
\toprule
MHC-control & EG(pos) & EG(neg) & EG(var) & EG(cross) & EG(overall)\
\midrule
ADHD-control & lower & lower & lower & lower & lower\
Anxiety-control & [MISSING] & lower & lower & lower & lower\
Bipolar-control & lower & lower & lower & lower & lower\
MDD-control & [MISSING] & [MISSING] & [MISSING] & [MISSING] & [MISSING]\
OCD-control & lower & lower & lower & lower & lower\
PPD-control & lower & lower & lower & lower & lower\
PTSD-control & lower & lower & lower & lower & lower\
Depression-control & higher & lower & lower & lower & [MISSING]\
Reddit eRisk Depression-control & [MISSING] & lower & lower & lower & lower\
\bottomrule
\end{tabular}
\caption{Table 7: Emotion Granularity - hyperparameter variations: Non-lexicon terms and tweets are discarded; user tweet and unique term thresholds are set to 50 and 25, and only mutually-exclusive emotion terms are considered.}
\end{table}

\noindent Table 13 shows the Spearman correlation be-tween emotion arcs for all pairs of emotions for the control group. These results indicate that as baselines largely emotions in the same group (e.g., positive, negative, mixed, overall) co-occur more often than emotions across groups.

\section{Appendix G}
\begin{table}[t]
\centering
\small
\begin{tabular}{l c c c c}
\toprule
MHC-control & EG(pos) & EG(neg) & EG(var) & EG(cross)\
\midrule
ADHD-control & lower & lower & [MISSING] & lower\
Anxiety-control & lower & lower & lower & lower\
Bipolar-control & lower & lower & lower & lower\
MDD-control & lower & [MISSING] & [MISSING] & lower\
OCD-control & lower & [MISSING] & [MISSING] & lower\
PPD-control & [MISSING] & lower & higher & [MISSING]\
PTSD-control & [MISSING] & lower & higher & lower\
Depression-control & lower & lower & lower & lower\
Reddit eRisk Depression-control & lower & [MISSING] & lower & lower\
\bottomrule
\end{tabular}
\caption{Table 8: Emotion Granularity - using window 100: The difference in emotion granularity between each MHC group and the control.}
\end{table}

\begin{table}[t]
\centering
\small
\begin{tabular}{l c c c c}
\toprule
MHC-control & EG(pos) & EG(neg) & EG(var) & EG(cross)\
\midrule
ADHD-control & lower & lower & [MISSING] & lower\
Anxiety-control & lower & lower & lower & lower\
Bipolar-control & lower & lower & lower & lower\
MDD-control & [MISSING] & [MISSING] & [MISSING] & lower\
OCD-control & lower & [MISSING] & [MISSING] & [MISSING]\
PPD-control & [MISSING] & [MISSING] & higher & [MISSING]\
PTSD-control & [MISSING] & lower & higher & lower\
Depression-control & [MISSING] & lower & higher & lower\
Reddit eRisk Depression-control & lower & [MISSING] & [MISSING] & [MISSING]\
\bottomrule
\end{tabular}
\caption{Table 9: Emotion Granularity - using window 500: The difference in emotion granularity between each MHC group and the control.}
\end{table}

\begin{table}[t]
\centering
\small
\begin{tabular}{l l}
\toprule
Emotion pair & Difference (Twitter-STMHD, MHC vs.\ control)\
\midrule
anger-disgust & lower lower lower lower lower lower lower\
anger-fear & lower lower lower lower [MISSING] lower lower lower\
anger-joy & lower lower higher lower [MISSING] lower [MISSING]\
anger-sadness & lower lower lower lower [MISSING] lower lower lower\
anger-surprise & lower lower [MISSING] lower [MISSING] lower - lower\
anger-trust & lower lower lower lower lower lower - lower\
anticipation-disgust & lower lower lower lower - lower - lower\
anticipation-fear & lower lower - lower lower lower lower lower\
anticipation-joy & lower lower lower lower lower lower lower lower\
anticipation-sadness & lower lower lower lower - lower - lower\
anticipation-surprise & lower lower lower lower - lower - lower\
anticipation-trust & lower lower lower lower lower lower lower lower\
disgust-fear & lower lower lower lower - lower lower lower\
disgusrjoy & lower lower [MISSING] lower [MISSING] lower lower\
disgust-sadness & lower lower lower lower [MISSING] lower lower\
disgust-surprise & lower lower lower lower [MISSING] lower lower\
fear-sadness & lower lower lower lower [MISSING] lower lower\
fear-surprise & lower lower lower lower [MISSING] lower lower\
fear-trust & lower lower lower lower lower lower lower lower\
joy-sadness & lower lower [MISSING] lower - lower lower\
joy-surprise & lower lower [MISSING] lower - lower lower\
joy-trust & lower lower lower lower lower lower lower\
sadness-surprise & lower lower lower lower - lower lower lower\
sadness-trust & lower lower lower lower lower lower lower lower\
surprise-trust & lower lower [MISSING] lower [MISSING] lower lower lower\
\bottomrule
\end{tabular}
\caption{Table 10: Emotion Granularity - emotion pairs: The difference in emotion granularity between each emotion pair, for each MHC group and the control in the Twitter-STMHD dataset. A significant difference is indicated by the word `lower'' or `higher''.}
\end{table}

\section{Appendix H}
\begin{table}[t]
\centering
\scriptsize
\begin{tabular}{l l l r r r}
\toprule
PoS & Dataset & MHC-Control & df & T-Statistic & P-value\
\midrule
Noun & [MISSING] & ADHD-+ontrol & 2368.64 & -t.94 & 0.144\
Noun & [MISSING] & Anxiety-control & 2233.36 & -0.58 & 0.7i8\
Noun & [MISSING] & Bipolar-control & 1726.26 & -2.40 & 0.131\
Noun & [MISSING] & MDD-control & 226.83 & -0.52 & 0.718\
Noun & [MISSING] & OCD-control & 1817.93 & 1,93 & 0)44\
Noun & [MISSING] & PPD-control & l'18.33 & -0.49 & 0.718\
Noun & [MISSING] & PTSD-control & 2237.82 & -0.54 & 0.718\
Noun & [MISSING] & Depression-control & 2245.64 & -0.27 & 0.'787\
Noun & Reddit eRisk & Depression-conrrol & 128.95 & 0.98 & 0.330\
Noun & [MISSING] & Anxiety{ontrol & 2235.85 & I.l2 & 0.420\
Noun & [MISSING] & Bipolar-control & 1852.44 & -2.10 & 0.096\
Noun & [MISSING] & MDD-control & 213.0 & 1.36 & 0.354\
Noun & [MISSING] & OCD-control & 1645.17 & 2.28 & 0.091\
Noun & [MISSING] & PPD-control & 169.49 & 0.98 & 0.438\
Noun & [MISSING] & PTSD-control & 2351j2 & 2.53 & 0.09 t\
Noun & [MISSING] & Depression-control & 2274.59 & 0.54 & 0.589\
Noun & Reddit eRisk & Depression-conrrol & I 10.40 & I .59 & 0. I l6\
\bottomrule
\end{tabular}
\caption{Table 11: Information Content: The degrees of freedom, t-statistic and p-value for the word specificity experiments described in Section 5.1.}
\end{table}

\begin{table}[t]
\centering
\scriptsize
\begin{tabular}{l l c c c c c}
\toprule
Dataset & MHC & EG(pos) & EG(neg) & EG(var) & EG(cross) & EG(overall)\
\midrule
Twitter-STMHD & Control & 0.02'7 & 0.023 & 0.0t2 & 0.006 & 0.022\
Twitter-STMHD & ADHD & -U.UIZ* & -0.0uE+ & -0.005+ & -0.010+ & -u.0l0E\
Twitter-STMHD & Anxiety & -0.012* & -0.008* & -0.003* & -0.008* & -0.010,F\
Twitter-STMHD & Bipolar & -0.004* & -0.008* & -0.004* & -0.002* & -0.006*\
Twitter-STMHD & MDD & -0.01 1* & -0.002 & -0.001 & -0.005* & -0.005*\
Twitter-STMHD & OCD & -0.013* & -0.009* & -0.006* & -0.009* & -0.009*\
Twitter-STMHD & PPD & -0.005 & -0.009,k & -0.005 & -0.004 & -0.003\
Twitter-STMHD & PTSD & -0.01 3 * & -0.014* & -0.006* & -0.008* & -0.013*\
Twitter-STMHD & Depression & -0.01 1* & -0.005* & -0.003x & -0.005* & -0.008*\
Reddit eRisk & Control & 0.117 & 0.114 & 0.090 & 0.094 & 0.1 l2\
Reddit eRisk & Depression & -0.016x & -0.021 * & -0.0t2 & -0.022* & -0.017*\
\bottomrule
\end{tabular}
\caption{Table 12: Emotion Granularity - Spearman correlations: Spearman correlation values between utterance-level emotion arcs for the Control group, and the delta for each MHC when compared to the Control group. Emotion granularity is defined as the negative of these correlations (i.e., higher correlations imply a lower granularity). Hyperparameters are the same as in Table 2.}
\end{table}

\begin{table}[t]
\centering
\scriptsize
\begin{tabular}{l c c c c c c c c}
\toprule
& anger & anticipation & disgust & fear & joy & sadness & surprise & trust\
\midrule
anger & [MISSING] & -0.003 & 0.020 & 0.027 & -0.010 & 0.024 & 0.009 & 0.007\
anticipation & [MISSING] & [MISSING] & -0.003 & 0.007 & 0.021 & 0.004 & 0.012 & 0.027\
disgust & [MISSING] & [MISSING] & [MISSING] & 0.023 & -0.010 & 0.021 & 0.006 & 0.003\
fear & [MISSING] & [MISSING] & [MISSING] & [MISSING] & -0.003 & 0.021 & 0.012 & 0.013\
joy & [MISSING] & [MISSING] & [MISSING] & [MISSING] & [MISSING] & -0.000 & 0.008 & 0.027\
sadness & [MISSING] & [MISSING] & [MISSING] & [MISSING] & [MISSING] & [MISSING] & 0.0t2 & 0.008\
surprise & [MISSING] & [MISSING] & [MISSING] & [MISSING] & [MISSING] & [MISSING] & [MISSING] & 0.023\
\bottomrule
\end{tabular}
\caption{Table 13: Emotion-Emotion Spearman correlations: Spearman correlation values between pairs of utterance-level emotion arcs for all users in the control group of the Twitter-STMHD dataset. Hyperparameters are the same as in Table 2.}
\end{table}

\end{document}
=====END FILE=====
