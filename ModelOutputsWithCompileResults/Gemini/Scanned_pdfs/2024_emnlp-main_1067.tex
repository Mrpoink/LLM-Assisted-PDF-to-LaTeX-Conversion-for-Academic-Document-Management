=====FILE: main.tex=====
\documentclass[11pt, letterpaper]{article}
\usepackage[utf8]{inputenc}
\usepackage[T1]{fontenc}
\usepackage{geometry}
\geometry{margin=1in}
\usepackage{times}
\usepackage{graphicx}
\usepackage{hyperref}
\usepackage{booktabs}
\usepackage{natbib}
\usepackage{enumitem}
\usepackage{listings}
\usepackage{xcolor}

\title{Comparing a BERT Classifier and a GPT classifier for Detecting Connective Language Across Multiple Social Media}

\author{
Josephine Lukito\textsuperscript{1}, Bin Chen\textsuperscript{2}, Gina M. Masullo\textsuperscript{1}, and Natalie Jomini Stroud\textsuperscript{1} \
\textsuperscript{1}Center for Media Engagement, University of Texas at Austin \
\textsuperscript{2}The University of Hong Kong
}

\date{}

\begin{document}

\maketitle

\begin{abstract}
This study presents an approach for detecting connective language---defined as language that facilitates engagement, understanding, and conversation---from social media discussions. We developed and evaluated two types of classifiers: BERT and GPT-3.5 turbo. Our results demonstrate that the BERT classifier significantly outperforms GPT-3.5 turbo in detecting connective language. Furthermore, our analysis confirms that connective language is distinct from related concepts measuring discourse qualities, such as politeness and toxicity. We also explore the potential of BERT-based classifiers for platform-agnostic tools. This research advances our understanding of the linguistic dimensions of online communication and proposes practical tools for detecting connective language across diverse digital environments.
\end{abstract}

\section{Introduction}

The growth and popularity of social media over the past two decades has created many opportunities for natural language processing and computational social science researchers to study short-form text. During this time, researchers have built a wide variety of text classifiers to understand social media posts, including f[MISSING]

\section*{[MISSING CONTENT: PAGES 2-13]}
\begin{center}
\textit{[The content for sections regarding Related Work, Methodology, Results, Discussion, and References was not provided in the source snippets. This section acts as a placeholder to preserve the logical structure of the document.]}
\end{center}

\appendix

\section{Prompt Details}

\noindent ... as 1 = connective, 0 = non-connective

\paragraph{Second Prompt}
Please perform a text annotation task: I will provide you with the definition of `connective democracy', some human-labeled social media posts, and some posts to be coded. Your task is to classify the unlabeled posts based on whether it demonstrates connective democracy or not.

Here is the definition of `connective democracy':
Connective democracy seeks to build bridges between divided groups so that they can hear each other in a deliberative manner. ``Connectivity'' refers to a willingness to prioritize relationships over competitiveness and engage in conversation with one's political adversaries to genuinely understand their viewpoints.

\section{Examples}
\label{sec:examples}

Examples of messages that were coded as connective but not polite include:

\begin{itemize}
\item wasnt sure wether to comment on this publicly you couldve removed that shit of the tweet my g but i hear you.
\item rt ninoboxer folks this isnt over the shit show is just beginning in my opinion
\item im not going to contest that the anime community isnt toxic because i have no idea but im gonna be honest with you i watch most of the mainstream anime and i have no fuckin clue whats going on in those communities and i really dont care quite frankly you dont have to be apart of the community to enjoy these shows
\end{itemize}

\section{Replication Files}
\label{sec:replication}

The labeled dataset, codebook, and BERT model can be found here: \url{[https://osf.io/lxrkva/](https://www.google.com/search?q=https://osf.io/lxrkva/)}.

\end{document}
=====END FILE=====