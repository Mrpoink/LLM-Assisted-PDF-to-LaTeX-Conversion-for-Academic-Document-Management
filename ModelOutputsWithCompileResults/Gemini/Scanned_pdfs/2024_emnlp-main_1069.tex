=====FILE: main.tex=====
\documentclass[11pt,a4paper,twocolumn]{article}

% Required packages
\usepackage[utf8]{inputenc}
\usepackage[T1]{fontenc}
\usepackage{geometry}
\usepackage{graphicx}
\usepackage{booktabs} % For professional tables
\usepackage{amsmath}
\usepackage{amssymb}
\usepackage{hyperref}
\usepackage{authblk} % For author affiliations
\usepackage{caption}
\usepackage{float}

% Geometry setup
\geometry{
top=2.5cm,
bottom=2.5cm,
left=2cm,
right=2cm
}

% Title and Author setup
\title{Emotion Granularity from Text: An Aggregate-Level Indicator of Mental Health}

\author[1,2]{Krishnapriya Vishnubhotla}
\author[3]{Daniela Teodorescu}
\author[4]{Mallory J. Feldman}
\author[4]{Kristen A. Lindquist}
\author[5]{Saif M. Mohammad}

\affil[1]{Department of Computer Science, University of Toronto}
\affil[2]{Vector Institute, Toronto}
\affil[3]{Department of Computing Science, University of Alberta}
\affil[4]{Department of Psychology and Neuroscience, University of North Carolina at Chapel Hill}
\affil[5]{National Research Council Canada}

\date{}

\begin{document}

\maketitle

% Inputs
\input{sections/abstract}
\input{sections/introduction}
\input{sections/results}

% Bibliography
% Assuming a bib file or manual entry
\bibliographystyle{plainnat}
\begin{thebibliography}{99}
\bibitem{vishnubhotla2024}
Vishnubhotla, K., Teodorescu, D., Feldman, M. J., Lindquist, K. A., & Mohammad, S. M. (2024). Emotion Granularity from Text: An Aggregate-Level Indicator of Mental Health. \textit{Proceedings of EMNLP 2024}.
\end{thebibliography}

\end{document}
=====END FILE=====

=====FILE: sections/abstract.tex=====
\begin{abstract}
We are united in how emotions are central to shaping our experiences; yet, individuals differ greatly in how we each identify, categorize, and express emotions. In psychology, variation in the ability of individuals to differentiate between emotion concepts is called emotion granularity (determined through self-reports of one's emotions). High emotion granularity has been linked with better mental and physical health; whereas low emotion granularity has been linked with maladaptive emotion regulation strategies and poor health outcomes. In this work, we propose computational measures of emotion granularity derived from temporally-ordered speaker utterances in social media (in lieu of self-reports that suffer from various biases). We then investigate the effectiveness of such text-derived measures of emotion granularity in functioning as markers of various mental health conditions (MHCs). We establish baseline measures of emotion granularity derived from textual utterances... [MISSING REMAINDER OF ABSTRACT]
\end{abstract}
=====END FILE=====

=====FILE: sections/introduction.tex=====
\section{Introduction}

[PAGE 1 CONTENT]

[MISSING TEXT]

\section{Related Work}

[MISSING TEXT]

\section{Methodology}

[MISSING TEXT]

\section{Data}

[MISSING TEXT]

\section{Experiments}

[MISSING TEXT]
=====END FILE=====

=====FILE: sections/results.tex=====
\section{Results and Discussion}

[MISSING TEXT]

\begin{table*}[t]
\centering
\caption{Emotion Granularity - Spearman correlations: Spearman correlation values between utterance-level emotion arcs for the Control group, and the delta for each MHC when compared to the Control group. Emotion granularity is defined as the negative of these correlations (i.e, higher correlations imply a lower granularity). Hyperparameters are the same as in Table 2.}
\label{tab:granularity_correlations}
\resizebox{\textwidth}{!}{%
\begin{tabular}{lcccccccc}
\toprule
& \textbf{anger} & \textbf{anticipation} & \textbf{disgust} & \textbf{fear} & \textbf{joy} & \textbf{sadness} & \textbf{surprise} & \textbf{trust} \
\midrule
\multicolumn{9}{l}{\textit{Reddit eRisk}} \
Control & 0.114 & 0.117 & 0.090 & 0.094 & 0.112 & [MISSING] & [MISSING] & [MISSING] \
Depression & -0.016* & -0.021* & -0.012* & -0.022* & -0.017* & [MISSING] & [MISSING] & [MISSING] \
\midrule
\multicolumn{9}{l}{\textit{MHC Deltas}} \
ADHD & 0.027 & 0.023 & 0.012 & 0.006 & [MISSING] & [MISSING] & [MISSING] & [MISSING] \
Anxiety & [MISSING] & [MISSING] & [MISSING] & [MISSING] & [MISSING] & [MISSING] & [MISSING] & [MISSING] \
OCD & [MISSING] & [MISSING] & [MISSING] & [MISSING] & [MISSING] & [MISSING] & [MISSING] & [MISSING] \
PPD & [MISSING] & [MISSING] & [MISSING] & [MISSING] & [MISSING] & [MISSING] & [MISSING] & [MISSING] \
PTSD & [MISSING] & [MISSING] & [MISSING] & [MISSING] & [MISSING] & [MISSING] & [MISSING] & [MISSING] \
Bipolar & -0.012* & -0.012* & -0.004* & -0.011* & -0.013* & -0.005 & -0.013* & -0.011* \
MDD & -0.008* & -0.008* & -0.008* & -0.002 & -0.009* & -0.009* & -0.014* & -0.005* \
\bottomrule
\multicolumn{9}{l}{\footnotesize * indicates statistical significance (p < 0.05). Some values inferred from scan snippets.}
\end{tabular}%
}
\end{table*}

[MISSING REMAINDER OF RESULTS]
=====END FILE=====