=====FILE: main.tex=====
\documentclass[11pt,a4paper,twocolumn]{article}
\usepackage[utf8]{inputenc}
\usepackage[T1]{fontenc}
\usepackage{times}
\usepackage{latexsym}
\usepackage{amsmath}
\usepackage{booktabs}
\usepackage{url}
\usepackage{hyperref}

\title{Automated Essay Scoring: A Reflection on the State of the Art}

\author{Shengjie Li and Vincent Ng \
Human Language Technology Research Institute \
University of Texas at Dallas \
\texttt{{sx1180006, vince}@hlt.utdallas.edu}}

\date{}

\begin{document}
\maketitle

\begin{abstract}
While steady progress has been made on the task of automated essay scoring (AES) in the past decade, much of the recent work in this area has focused on developing models that beat existing models on a standard evaluation dataset. While improving performance numbers remains an important goal in the short term, such a focus is not necessarily beneficial for the long-term development of the field. We reflect on the state of the art in AES research, discussing issues that we believe can encourage researchers to think bigger than improving performance numbers, with the ultimate goal of triggering discussion among AES researchers on how we should move forward.
\end{abstract}

\section{Introduction}
Automated Essay Scoring (AES), the task of automatically assigning a holistic score to an essay that summarizes its overall quality, is arguably one of the most important applications in natural language processing (NLP). As an example of AES, consider the essay in Table 1, which is written in response to the prompt shown at the top of the table. Given the scoring rubric in Table 2, an AES system should assign a score of 3 to this essay for the following reasons. First, its author takes a position but fails to provide adequate support and details. Specifically, the author talks about computers giving people entertainment and lists some general social networking websites, but there is no elaboration on how computers enhance access to entertainment. In terms of organization, while the essay has a basic structure (an introduction, three main body sentences, and a conclusion), the ideas within each body sentence are poorly connected. Finally, the transitions are rather awkward and repetitive. For example, the author uses "so as you can see" three times. While the author exhibits some awareness of the audience as the essay is addressed to a local newspaper, it is not clear whether the essay is intended to urge the editor to write an article on how computers benefit people.

Given the large number of essays written by students from all over the world in both test and classroom settings, being able to automatically score essays could save a tremendous amount of manual grading effort. Despite the fact that AES has been investigated for more than 50 years (Page, 1967), the task is still far from being solved. Nevertheless, AES has been progressing steadily. In recent years, a standard recipe for publishing in this area of research seems to involve proposing a sophisticated neural model that can beat competing models on a standard evaluation dataset such as ASAP (see Section 2.1 for details).

While improving performance numbers remains an important short-term goal for AES research, such a focus is not necessarily beneficial for the long-term development of the field. In this position paper, we reflect on the state of the art in AES research. Specifically, we discuss issues that we believe can encourage researchers, particularly junior researchers, to think bigger than merely improving performance numbers, with the ultimate goal of triggering discussion among AES researchers on how we should move forward. These issues range from evaluation (Sections 3 and 4) to new tasks (Sections 5 and 6) to corpus annotation (Sections 7 and 8) and beyond (Section 9).

\section{Current State of AES Research}
To facilitate our discussion of the future directions of AES research, in this section, we provide an overview of the current state of AES research. While Beigman Klebanov and Madnani's (2020) theme paper focuses more broadly on the emergent uses of automated writing technologies, we focus specifically on research directions that we think are worth pursuing by AES researchers. For a comprehensive overview of AES research, we refer the reader to the books published by Shermis and Burstein (2003), Shermis et al. (2010) and Beigman Klebanov and Madnani (2021), as well as the surveys published by our group (Ke and Ng, 2019; Li and Ng, 2024a).

\begin{table*}[t]
\centering
\small
\begin{tabular}{p{0.95\textwidth}}
\toprule
\textbf{[Prompt]} \
More and more people use computers, but not everyone agrees that this benefits society. Those who support advances in technology believe that computers have a positive effect on people. They teach hand-eye coordination, give people the ability to learn about faraway places and people, and even allow people to talk online with other people. Others have different ideas. Some experts are concerned that people are spending too much time on their computers and less time exercising, enjoying nature, and interacting with family and friends. Write a letter to your local newspaper in which you state your opinion on the effects computers have on people. Persuade the readers to agree with you. \ \midrule
\textbf{[Essay]} \
Dear local newspaper, @CAPS1 opinion about computers iters is that they benifet people. they help people when they need it, it gives people entertainment. So that is why computers benifet people. @CAPS1 first reason is that a computer can help people when they need help like for instace if you have to write a paper and you need to look something up such as reseach you can use the computer or if there is a long main cwayshion then you can look up the formula. So as you can see that is why computer benifts people @CAPS1 second resson is that it gives people entertainment, such as myspace, you twitter. And those are social networks, if you like to watch tv. or listen to music there are websites for them also. So that is why I said computers give you entertainmen too. @CAPS1 third resson is that it helps people such as if you in different contries and you can't go see them you to them through instans or web cam. Or if you are home schooled and you dnt see other kids your age that can talk to the on the computer. So as you can see that why I say you can through the computer. So as you can see that is why I say computer are very benifical be cause they give you entertainment they let you communicate and they help people so. \ \midrule
This essay receives a score of 3 as it takes a position but fails to provide adequate support and details. Specifically, the author talks about computers giving people entertainment and lists some general social networking websites, but there is no elaboration on how computers enhance access to entertainment. In terms of organization, the essay has a basic structure: an introduction, three main body sentences, and a conclusion, but the ideas within each body sentence are poorly connected. Moreover, the transitions are rather awkward and repetitive. For example, the author uses "so as you can see" three times. While the author exhibits some awareness of the audience as the essay is addressed to a local newspaper, it is not clear whether the essay is intended to urge the editor to write an article on how computers benefit people. \
\bottomrule
\end{tabular}
\caption{A sample essay taken from Essay Set 1 of the ASAP corpus. The writing prompt is shown at the top.}
\end{table*}

\begin{table*}[t]
\centering
\small
\begin{tabular}{cp{0.85\textwidth}}
\toprule
\textbf{Score} & \textbf{Description} \ \midrule
1 & An undeveloped response that may take a position but offers no more than very minimal support. Typical elements: (1) Contains few or vague details, (2) Is awkward and fragmented, (3) May be difficult to read and understand, (4) May show no awareness of audience. \ \midrule
2 & An under-developed response that may or may not take a position. Typical elements: (1) Contains only general reasons with unelaborated and/or list-like details, (2) Shows little or no evidence of organization, (3) May be awkward and confused or simplistic, (4) May show little awareness of audience. \ \midrule
3 & A minimally developed response that may take a position, but with inadequate support and details. Typical elements: (1) Has reasons with minimal elaboration and more general than specific details, (2) Shows some organization, (3) May be awkward in parts with few transitions, (4) Shows some awareness of audience. \ \midrule
4 & A somewhat-developed response that takes a position and provides adequate support. Typical elements: (1) Has adequately elaborated reasons with a mix of general and specific details, (2) Shows satisfactory organization, (3) May be somewhat fluent with some transitional language, (4) Shows adequate awareness of audience. \ \midrule
5 & A developed response that takes a clear position and provides reasonably persuasive support. Typical elements: (1) Has moderately well elaborated reasons with mostly specific details, (2) Exhibits generally strong organization, (3) May be moderately fluent with transitional language throughout, (4) May show a consistent awareness of audience. \ \midrule
6 & A well-developed response that takes a clear and thoughtful position and provides persuasive support. Typical elements: (1) Has fully elaborated reasons with specific details, (2) Exhibits strong organization, (3) Is fluent and uses sophisticated transitional language, (4) May show a heightened awareness of audience. \
\bottomrule
\end{tabular}
\caption{Rubric for scoring essays in Essay Set 1 of the ASAP corpus.}
\end{table*}

\subsection{Corpora}
While a number of AES corpora have been developed, ASAP is arguably the most extensively used in AES research. Introduced as part of a 2012 Kaggle competition, the Automated Student Assessment Prize (ASAP) corpus has become a popular dataset for holistic scoring, especially given its vast collection of essays per prompt (up to 1,800 for some prompts). ASAP facilitates the development of high-performing, prompt-specific systems. ASAP++ (Mathias and Bhattacharyya, 2018) is an extension of ASAP where each essay is scored along multiple traits (i.e., dimensions of essay quality such as Coherence). Note that ASAP is composed of three types of essays (namely, narrative/descriptive essays, persuasive essays, and source-dependent essays). Not all traits are applicable to all essay types. For instance, Organization and Conventions are scored for narrative/descriptive essays and persuasive essays only.

Other English corpora include: (1) TOEFL (Blanchard et al., 2013); (2) the Cambridge Learner Corpus-First Certificate in English exam (CLC-FCE) (Yannakoudakis et al., 2011); (3) the International Corpus of Learner English (ICLE) (Granger et al., 2009); and (4) the Argument Annotated Essays (AAE) corpus (Stab and Gurevych, 2014). AES corpora in other languages exist, such as Swedish, German, Portuguese, Japanese, and the MERLIN dataset (German, Italian, and Czech).

\subsection{Evaluation Metric}
The standard metric used to evaluate AES models is Quadratic weighted Kappa .  is an agreement metric that ranges from 0 to 1 but can be negative if there is less agreement than what is expected by chance. More specifically,  is a weighted version of Kappa where each case of disagreement is weighted by the squared difference between the reference score and the predicted score. This allows the metric to distinguish between near misses and far misses.

\subsection{Systems}
AES systems can be divided into three categories:

\subsubsection{Heuristic Approaches}
Early systems typically possess trait-driven holistic scoring and focus on non-content-based traits. The holistic score is typically computed as the weighted sum of trait scores (e.g., e-rater).

\subsubsection{Machine Learning Approaches}
Focus shifted in the 2010s to algorithms like SVM or linear regression. These models focus on feature engineering (low-level length-based features and high-level readability or discourse features) and within-prompt scoring.

\subsubsection{Deep Learning Approaches}
Recent models are deep learning-based, focusing on distributed representation learning and tasks like cross-prompt scoring. Early neural models combined CNNs and RNNs (Taghipour and Ng, 2016), later replaced by Transformer-based models like BERT (, Yang et al., 2020).

\section{AES is more than improving QWK}
We recommend that researchers understand the strengths and weaknesses of the systems they developed by performing a qualitative and quantitative analysis of the system outputs. \textbf{Recommendation #1}: We recommend that researchers understand the strengths and weaknesses of the systems they developed by performing a qualitative and quantitative analysis of the system outputs.

\section{ASAP is not the only essay corpus}
\textbf{Recommendation #2}: To understand whether models trained on ASAP can generalize well when applied to other essay corpora, we recommend that AES researchers evaluate their systems on not only ASAP but at least one other publicly available corpus.

\section{Is it time for cross-prompt AES?}
\textbf{Recommendation #3}: Cross-prompt AES is a challenging task that deserves the attention of AES researchers. We recommend that researchers focus on developing new annotated corpora, new methods for gathering prompt-specific knowledge, and new AES models for cross-prompt AES.

\section{Traits are useful but under-explored}
\textbf{Recommendation #4}: We recommend that a thorough investigation of the impact of traits on holistic scoring be conducted as they could be a viable solution to key problems concerning neural model interpretability and cross-prompt model generalizability.

\section{Corpus development is slow}
\textbf{Recommendation #5}: We recommend that the community devote more effort to corpus development, as progress on model development will eventually be hindered by the arguably slow progress on corpus development.

\section{LLMs can be used to assist AES}
\textbf{Recommendation #6}: We recommend that AES researchers explore novel ways of exploiting LLMs, especially concerning how LLMs can be used to improve human efficiency in completing daunting tasks such as corpus creation and annotation.

\section{Essay grading beyond scoring}
\textbf{Recommendation #7}: We recommend that AES researchers think beyond essay scoring and develop a vision of how AES models can be combined with related AI technologies to develop more complex systems.

\section{Conclusion}
In this position paper, we discussed a range of issues that we believe AES researchers should seriously consider and provided seven recommendations to push the field forward.

\begin{thebibliography}{99}
\bibitem{Page1967} Page, E. B. 1967. The Imminence of Grading Essays by Computer.
\bibitem{Attali2006} Attali, Y., and Burstein, J. 2006. Automated essay scoring with e-rater v. 2.0.
\bibitem{Taghipour2016} Taghipour, H., and Ng, V. 2016. A Neural Approach to Automated Essay Scoring.
\bibitem{Yang2020} Yang, R., et al. 2020. : A AES model obtained by fine-tuning BERT.
\end{thebibliography}

\end{document}
=====END FILE=====