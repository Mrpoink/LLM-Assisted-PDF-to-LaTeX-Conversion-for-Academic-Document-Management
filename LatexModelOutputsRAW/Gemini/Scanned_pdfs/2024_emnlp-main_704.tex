ACK

=====FILE: main.tex=====
\documentclass[11pt,a4paper,twocolumn]{article}
\usepackage[utf8]{inputenc}
\usepackage{times}
\usepackage{latexsym}
\usepackage[T1]{fontenc}
\usepackage[margin=1in]{geometry}
\usepackage{graphicx}
\usepackage{amsmath}
\usepackage{amssymb}
\usepackage{booktabs}
\usepackage{url}
\usepackage{natbib}
\usepackage{float}

% Title and Author
\title{Domain adapted machine translation: What does catastrophic forgetting forget and why?}
\author{Danielle Saunders \and Steve DeNeefe \
RWS Language Weaver \
\texttt{danielle.saunders@cantab.net} \
\texttt{sdeneefe@rws.com}}
\date{}

\begin{document}

\maketitle

\begin{abstract}
\input{sections/00_abstract}
\end{abstract}

\input{sections/01_introduction}
\input{sections/02_what_forgets}
\input{sections/03_why_varies}
\input{sections/04_improving}
\input{sections/05_limitations}

\bibliographystyle{plainnat}
\bibliography{refs}

\appendix
\input{sections/appendix}

\end{document}
=====END FILE=====

=====FILE: sections/00_abstract.tex=====
Neural Machine Translation (NMT) models can be specialized by domain adaptation, often involving fine-tuning on a dataset of interest. This process risks catastrophic forgetting: rapid loss of generic translation quality. Forgetting has been widely observed, with many mitigation methods proposed. However, the causes of forgetting and the relationship between forgetting and adaptation data are under-explored.

This paper takes a novel approach to understanding catastrophic forgetting during NMT adaptation by investigating the impact of the data. We provide a first investigation of what is forgotten, and why. We examine the relationship between forgetting and the in-domain data, and show that the amount and type of forgetting is linked to that data's target vocabulary coverage. Our findings pave the way toward better informed NMT domain adaptation.
=====END FILE=====

=====FILE: sections/01_introduction.tex=====
\section{Introduction}

The specialization of Neural Machine Translation (NMT) models for high performance in a specific domain, such as legal or healthcare, is of strong interest to academia \citep{Barrault2020} and industry \citep{Savenkov2022}. Fine-tuning, sometimes known as transfer learning, is a well-established domain adaptation method that continues training a pre-trained NMT model on some new dataset from the domain of interest \citep{Luong2015}. However, fine-tuning on domain-shifted data can result in catastrophic forgetting \citep{McCloskey1989}. This apparently simple statement has been widely observed in NMT, but rarely examined.

Catastrophic forgetting in NMT is described variously as `degradation of general-domain performance' \citep{Thompson2019} or `[forgetting] previous domain knowledge' \citep{Gu2020}, typically referencing lower scores in some quality metric. However, prior work on forgetting in NMT focuses on mitigation, leaving two important gaps.

Firstly, prior work does not determine what is forgotten in concrete terms. Lower scores in a reference-based quality metric can indicate either poor translation or simply a vocabulary shift towards the new domain. This is especially true for string-based metrics like BLEU \citep{Papineni2002}. Prior work does not distinguish between quality drop and vocabulary shift, and further does not address whether vocabulary shift `forgetting' is beneficial or detrimental to translation of the generic and new domains.

Secondly, prior work almost universally treats forgetting and its mitigation as independent of the adaptation dataset. In fact, the contents of the adaptation dataset will impact forgetting -- consider the amount of forgetting expected if fine-tuning on a 1000-sentence sample of the pre-training dataset, versus 1000 copies of the same sentence pair. Understanding the relationship between adaptation data and forgetting is crucial for predicting how well domain adaptation will work, whether adapted performance is likely to generalise, or whether forgetting mitigation approaches are necessary.

The contributions of this paper lie in addressing these gaps. Specifically:

\begin{itemize}
\item We provide a first exploration of what domain-adapted NMT forgets. This includes quantifying the degree of detrimental vocabulary shift, and demonstrating that this shift is not well-characterised by common MT quality metrics.
\item We show that forgetting can consist of using in-domain vocabulary inappropriately in out-of-vocabulary contexts and, unexpectedly, that this can take place even when the source sentence has no in-domain triggers.
\item We also provide a first investigation into the relationship between forgetting and adaptation dataset, examining the correlation between forgetting and several domain heuristics for eight domains across two language pairs.
\item We find that some commonly used domain heuristics including sentence pair count and vocabulary distribution cannot explain how forgetting varies by domain, but that forgetting does have a strong relationship with generic vocabulary coverage.
\item We support our findings by demonstrating significantly reduced forgetting with minimal, coverage-based mixed fine-tuning. In the process we show that much of the benefit of generic data mix-in comes from a relatively small vocabulary-covering set.
\end{itemize}

\subsection{Related work}
NMT adaptation with the goal of improved in-domain performance sometimes accounts for domain-specific data characteristics. Examples include selecting adaptation data by target domain similarity \citep{Aharoni2020}, gradually emphasising in-domain data during training \citep{Zhang2019}, or determining hyperparameters via meta-learning \citep{Sharaf2020}.

Work focusing on catastrophic forgetting in NMT, by contrast, takes an adaptation-set-agnostic approach depending only on the generic dataset \citep{Saunders2022}. This can include regularizing parameters relative to the generic domain \citep{Barone2017}, training on complementary inputs via generic-trained teacher models \citep{Shao2022} or mixing in generic data during adaptation \citep{Chu2017}. The specific adaptation dataset is not typically considered beyond broad suggestions such as tuning for fewer steps on smaller datasets \citep{Xu2019}.

In this work, by contrast, we aim to understand forgetting based on the characteristics of the domain-specific adaptation dataset.
=====END FILE=====

=====FILE: sections/02_what_forgets.tex=====
\section{What does adapted NMT forget?}

In this section, we explore what is forgotten during adaptation in concrete terms, using two quality metrics and a new measure for analysing vocabulary shift. We adapt pre-trained generic NMT models to eight diverse domains across two language pairs, intentionally triggering catastrophic forgetting, and analyse the degree of quality degradation versus vocabulary shift. In particular, we examine which tokens are forgotten and what replaces them after adaptation. We find that models experiencing forgetting produce in-domain vocabulary incorrectly and in entirely out-of-domain contexts.

\subsection{Measuring vocabulary-shift forgetting}
To determine which tokens are forgotten during adaptation we propose a new forgetting measure. Prior work measures forgetting via a drop in a corpus-level quality metric \citep{Thompson2019,Gu2020}. However, these do not mark which terms are forgotten. To measure vocabulary shift forgetting, a score should highlight terms that are used correctly before but not after adaptation. We focus on unigram terms: these are easily interpretable with respect to the vocabulary, which often signifies domain \citep{vanderWees2015}.

Consider a test set where for each reference translation  we can compare a translation from an original model  and a translation from an adapted model . We are interested in how the adapted model translation changes relative to the original model translation and the reference. For each reference  we find the count of every reference token in original model and adapted model translations,  and , capped at the count in the reference :

\begin{align*}
O[tok]*{T_R} &= \min(#tok_O, #tok_R) \
A[tok]*{T_R} &= \min(#tok_A, #tok_R)
\end{align*}

\begin{equation}
\text{ForgetGenUse}[tok] = \sum_{T_R}^{n} \max(O[tok]*{T_R} - A[tok]*{T_R}, 0)
\end{equation}

High  means we forget generic use of . For example, if the generic model correctly produced   times and the adapted model did not produce it at all, : all generic uses of  are forgotten. If the generic and adapted models both fail to produce  at all, : this is a quality problem but not specific to forgetting.

A normalized corpus-level score over a set of multiple tokens, , is given by:
\begin{equation}
\text{ForgetGenUse}*V = \frac{\sum*{tok \in V} \text{ForgetGenUse}[tok]}{\sum_{T_R} \sum_{tok \in V} #tok_R}
\end{equation}

 could consist of all tokens  -- in which case the denominator is the test set reference token count -- or a subset, for example, out-of-domain (OOD) tokens, in which case the denominator is the count of just those tokens in all reference sentences. We report ForgetGenUse over subword level tokens for brevity and ease of interpretation, but could equally calculate over words or n-grams, if we wished to extend measurement to better reflect style or syntax.

ForgetGenUse is related to change in unigram BLEU, but there are two crucial differences. First, it is defined for all occurrences of given tokens, whereas BLEU is defined on given segments which will include some instances of a token but not others. Secondly, BLEU masks detrimental vocabulary shift with beneficial shift where a token is translated correctly after adaptation but not before. If a score remains unchanged, some (e.g. out-of-domain) tokens may be translated worse, and others better. We are interested only in tokens which are translated worse. For this reason ForgetGenUse minimises reward for beneficial vocabulary shift by only marking no-longer-correctly-output tokens per segment.

\subsection{Intentionally triggering forgetting: Lower quality and detrimental vocabulary shift}
Our first experiments intentionally trigger forgetting to explore what is forgotten. We pre-train one encoder-decoder Transformer model for each of German to English (de-en) and English to Japanese (en-ja) NMT -- all subsequent adaptation experiments are a fine-tuning run of one of these models. Appendix A gives details of model preparation.

Our generic test sets are concatenated WMT News/General test sets\footnote{See \url{[https://machinetranslate.org](https://machinetranslate.org)}.} for 2019-22 for de-en and 2020-22 for en-ja. While WMT news sets are often described as `generic', each may feature quite specific vocabulary -- for example, articles about recent news items. Combining test sets increases the reliability of forgetting evaluation via the increased segment count, as well as being more truly generic in topic coverage.

Our adaptation domains are drawn from widely-used datasets with standard test splits. For de-en, we adapt to the five domains from the OPUS multi-domain split produced by \citet{Aharoni2020}, including test sets. For en-ja we use three target domains: IWSLT \citep{Cettolo2012}, KFTT \citep{Neubig2011} and BSD \citep{Rikters2019}. We use test15 as test data for en-ja IWSLT and the standard test splits for the remainder. The datasets, listed in Table 1, vary in domain and size.

We measure vocabulary shift forgetting via increased ForgetGenUse, and track quality degradation via decreases in a string-based metric, BLEU, and a neural metric, COMET \citep{Rei2020}. ForgetGenUse expresses forgetting in the sense of vocabulary shift. Throughout this paper unless stated otherwise we report a drop in BLEU or COMET relative to the baseline as positive for brevity -- high  meaning more forgetting. For reference, Table 1 gives generic and in-domain absolute BLEU and COMET scores for the pre-trained models, from which all other absolute values can be calculated.

We fine-tune our pre-trained models on domain-specific datasets until catastrophic forgetting is seen in the sense of quality drop on generic test sets. As we wish to understand the impact of dataset on forgetting independent of other variables, all experiments in this paper adapt for 20K steps. We found this caused similar forgetting to that previously described in the literature \citep{Hasler2021}.

Table 2 shows generic forgetting after adaptation to each domain. The different domains exhibit a wide range of forgetting in terms of quality and vocabulary shift. Additionally, although COMET and BLEU are strongly and significantly correlated across the sets of domains (Kendall's ), ForgetGenUseAll does not have a significant correlation with either. This suggests that corpus-level quality metrics like BLEU and COMET do not sufficiently measure detrimental vocabulary shift. To confirm that the vocabulary shift measured by ForgetGenUse is indeed detrimental despite not correlating with BLEU or COMET, we must analyse what replaces forgotten tokens.

\begin{table*}[t]
\centering
\small
\begin{tabular}{llrrrrrr}
\toprule
& & \multicolumn{2}{c}{# Sentences} & \multicolumn{2}{c}{BLEU} & \multicolumn{2}{c}{COMET} \
& Domain & Train & Test & Generic & Domain & Generic & Domain \
\midrule
de-en & Generic & 43.9M & 5769 & 35.3 & & 0.54 & \
& IT & 223K & 2000 & 33.0 & & 0.33 & \
& Kor & 18K & 2000 & 15.9 & & -0.03 & \
& Law & 467K & 2000 & 45.9 & & 0.60 & \
& Med & 248K & 2000 & 44.8 & & 0.55 & \
& Sub & 500K & 2000 & 26.4 & & 0.21 & \
\midrule
en-ja & Generic & 22.4M & 4037 & 22.5 & & 0.43 & \
& IWSLT & 220K & 1194 & 14.0 & & 0.16 & \
& KFTT & 427K & 1160 & 17.6 & & 0.34 & \
& BSD & 20K & 2120 & 13.6 & & 0.46 & \
\bottomrule
\end{tabular}
\caption{Segment counts and absolute generic model BLEU and COMET on the generic domain test sets and on each in-domain test set.}
\label{tab:1}
\end{table*}

\begin{table}[h]
\centering
\small
\begin{tabular}{lrrrrr}
\toprule
& \multicolumn{5}{c}{de-en} \
& IT & Kor & Law & Med & Sub \
\midrule
BLEU & 5.0 & 22.3 & 4.7 & 8.3 & 5.7 \
COMET & 0.11 & 0.78 & 0.08 & 0.16 & 0.09 \
ForgetGenUseAll & 0.09 & 0.25 & 0.07 & 0.11 & 0.09 \
\midrule
& \multicolumn{3}{c}{en-ja} & & \
& IWSLT & KFTT & BSD & & \
\midrule
BLEU & 4.8 & 2.0 & 7.0 & & \
COMET & 0.07 & 0.06 & 0.29 & & \
ForgetGenUseAll & 0.14 & 0.11 & 0.16 & & \
\bottomrule
\end{tabular}
\caption{Measuring forgetting on generic test sets for de-en and en-ja.}
\label{tab:2}
\end{table}

\subsection{Which tokens are forgotten, and what replaces them?}
Vocabulary shift in a domain-adapted NMT system can be beneficial or detrimental. Beneficial vocabulary shift produces in-domain tokens in in-domain contexts, and out-of-domain tokens where more contextually appropriate. Detrimental vocabulary shift produces in-domain vocabulary tokens when it is not contextually appropriate.

To make this distinction and find the token-level replacements after adaptation to each domain, we compare the generic-model and adapted-model translations of the generic test sets. We align the two sets of translations using symmetrized fast align \citep{Dyer2013}, which lets us identify which translation hypothesis tokens change after adaptation. We can also find the frequency of those tokens in the in-domain adaptation dataset.

We find that replacements have at least one token appearing in the in-domain adaptation dataset. Tokens which themselves appear in the adaptation dataset are replaced less frequently, and only by alternatives with far more adaptation set occurrences. The replacements are often semantically similar, judged both by manual inspection and by average FastText embedding cosine similarity \citep{Bojanowski2017} between the original and replacing tokens.

Surprisingly, we find by inspection that the replacements tend to occur in very different contexts in the adaptation data and generic test set. Of the seven IT domain instances of Donald, two refer to computer scientist Knuth and five are subworded Mc_+ Donald -- none have the same referent as Trump. Internet is legitimately replaced by web after adaptation to Kor, but the Kor text only uses web in the sense of spider's web -- including a different source term (Internet vs Netz). The Med domain only uses match as a verb in the context of experiments, not as a noun synonym for game as in Med-adapted test outputs -- not only a different source term but a different source part of speech (Spiel vs e.g. abstimmen). The target vocabulary alone can influence forgetting, without requiring a contextually relevant source.

Focusing on the most-forgotten Kor domain, we perform a deeper analysis for two tokens that are forgotten vs two that are not forgotten. Using FastText embedding cosine similarity, we find the closest Kor-domain English tokens which can have the same part-of-speech. For each, the in-domain training count is in brackets:
\begin{itemize}
\item satisfied (3): happy (29) and pleased (90)
\item water (236): waters (8) and lake (1)
\item England (0): Kingdom (10)
\item genes (0): species (3)
\end{itemize}

When the generic model translates the generic test set, it produces satisfied 11 times. The Kor-adapted model produces pleased for 10 of these, and happy for the remaining. Although all are in-domain, satisfied is far rarer. By contrast both models produce water, with no more frequent in-domain alternative, in the same locations.

By contrast, we consider out-of-domain tokens. The generic model produces England 14 times. The Kor-adapted model replaces 10 of these with United Kingdom, with the remaining four null-aligned indicating undertranslation. Although the phrase United Kingdom does not occur in the Kor data, both words do occur separately. United Kingdom occurs in similar contexts to England during pre-training, making it a plausible, if incorrect, replacement. Interestingly, another out-of-domain term, genes, is not forgotten during adaptation. The closest in-domain alternative, species, is neither common nor a plausible replacement.

The token forget-replace effect can be triggered by individual subwords, not just whole-word tokens. For example, ignoring post-processing, the pre-trained model produces one token October where the Kor model produces two subwords Oc_ +tober. Neither word is in the Kor domain -- but the subword Oc_, making oc_+ tober preferred.

\begin{table*}[t]
\centering
\small
\begin{tabular}{p{3cm}p{3cm}p{3cm}p{3cm}}
\toprule
IT & Kor & Law & Med \
Generic / Adapted & Generic / Adapted & Generic / Adapted & Generic / Adapted \
\midrule
species 2 / types 664 & satisfied 3 / pleased 90 & euros 39 / EUR 4988 & danger 3 / risk 5466 \
citizens 0 / people 292 & week 0 / month 35 & warranty 20 / guarantee 2331 & defeat 0 / loss 1377 \
infections 0 / cases 207 & England 0 / Kingdom 10 & Donald 0 / President 1685 & billion 0 / million 464 \
win 120 / victory 1 & accident 0 / injury 7 & touch 5 / contact 948 & guests 0 / visitors 11 \
Trump 0 / Donald 7 & Internet 0 / web 1 & wants 9 / intends 416 & game 0 / match 5 \
\bottomrule
\end{tabular}
\caption{High ForgetGenUse tokens for de-en domains. Counts are for that token in the in-domain adaptation dataset. Left columns: Output from generic model. Right columns: Most frequent aligned replacements post-adaptation.}
\label{tab:3}
\end{table*}

\subsection{Out-of-domain tokens are forgotten more}
Given possible different requirements for in-domain (ID) and out-of-domain (OOD) tokens, it is interesting to calculate  separately for these token subsets. For  we sum and normalize in Equation 1 over all vocabulary tokens that appear in at least one adaptation set reference sentence for each domain. For  we do so for the complement, again for each domain.

The results in Table 4 show a striking difference in term shift between in-domain and out-of-domain tokens.  has relatively small absolute values, and a small range of values.  values by contrast are higher for every domain, meaning out-of-domain tokens are forgotten at a higher rate.  and  are equal for Law and Sub (de-en) and IWSLT and KFTT (en-ja): for these domains, almost all generic test set tokens are in-domain.

It is not wholly surprising that out-of-domain tokens are forgotten more than in-domain tokens: a goal of adaptation is to use in-domain terminology instead of generic. However, the vocabulary shift reported by ForgetGenUse does not just consist of generic terms being replaced by their in-domain equivalents. Instead, as shown in Table 3, shifts can be technically correct but not domain-relevant (game match, Trump Donald) -- these are unnecessary and can confuse users. While the definition of an NMT domain is an open question \citep{vanderWees2015,Saunders2022}, it is not at all clear that a user or machine translation client would expect a subtitles domain to entail a shift to US-English terms like fall or aircraft, or expect an adapted model to no longer use standard but incidentally out-of-domain terms like accident or species. More serious still are meaning-changing errors (billion million, week month) -- these unambiguously harm translation. Such vocabulary shift is clearly detrimental and lowers quality.

\begin{table*}[t]
\centering
\small
\begin{tabular}{lrrrrrrrr}
\toprule
& \multicolumn{5}{c}{de-en} & \multicolumn{3}{c}{en-ja} \
& IT & Kor & Law & Med & Sub & IWSLT & KFTT & BSD \
\midrule
ForgetGenUseAll & 0.09 & 0.25 & 0.07 & 0.11 & 0.09 & 0.14 & 0.11 & 0.16 \
ForgetGenUseOOD & 0.37 & 0.60 & 0.27 & 0.47 & 0.11 & 0.34 & 0.65 & 0.49 \
ForgetGenUseID & 0.07 & 0.17 & 0.07 & 0.09 & 0.09 & 0.14 & 0.11 & 0.12 \
\bottomrule
\end{tabular}
\caption{Calculating ForgetGenUse over tokens that are out-of-domain (OOD) vs in-domain (ID) for each domain.}
\label{tab:4}
\end{table*}
=====END FILE=====

=====FILE: sections/03_why_varies.tex=====
\section{Why does forgetting vary by domain?}

In this section we aim to understand the relationship between adaptation dataset and forgetting. This relationship is key for real world adaptation scenarios when deciding whether to adapt, how to adjust tuning hyperparameters, and which if any forgetting mitigation steps to take. To investigate, we compare datasets exhibiting varying degrees of forgetting in terms of multiple domain-differentiating heuristics. We find that many domain features do not correlate with forgetting, but that vocabulary coverage does.

\subsection{Controlling for dataset size}
Dataset size is recognized as having an impact on MT adaptation performance \citep{Sharaf2020}, and has been associated in forgetting \citep{Pham2020}. However, its relationship with forgetting across domains is unclear. We can assess correlation of forgetting with number of lines per dataset for our results in Table 2. Surprisingly, Kendall's  is not significant between data size and either of COMET or BLEU. ForgetGenUse does show significant negative correlation with dataset size (), suggesting smaller datasets have a greater likelihood of vocabulary shift, but not necessarily general quality degradation.

We further investigate by controlling for dataset size in terms of tokens. We randomly subsample each de-en dataset except for Kor to the same approximate token count of Kor/BSD, allowing direct comparison.

\begin{table*}[t]
\centering
\small
\begin{tabular}{lrrrrrrrr}
\toprule
& \multicolumn{5}{c}{de-en} & \multicolumn{3}{c}{en-ja} \
& IT-s & Kor & Law-s & Med-s & Sub-s & IWSLT-s & KFTT-s & BSD \
\midrule
BLEU & 10.4 & 22.3 & 12.3 & 14.4 & 11.5 & 7.2 & 6.9 & 7.0 \
COMET & 0.24 & 0.78 & 0.28 & 0.36 & 0.24 & 0.20 & 0.28 & 0.29 \
ForgetGenUseAll & 0.14 & 0.25 & 0.15 & 0.17 & 0.15 & 0.17 & 0.17 & 0.16 \
\bottomrule
\end{tabular}
\caption{Forgetting when adapting on subsampled (-s) domains. All de-en sets except Kor, and all en-ja sets except BSD, are subsampled to match the smallest domain size.}
\label{tab:5}
\end{table*}

\begin{table*}[t]
\centering
\small
\begin{tabular}{lrrrrrrrr}
\toprule
& Law-ss & Law-s & Sub-s & Sub-ss & Sub-ssf & IWSLT-s & IWSLT-ss & KFTT-s \
\midrule
Av. #toks & 66.1 & 17.0 & 23.0 & 12.5 & 13.6 & 44.4 & 15.7 & 60.8 \
BLEU & 12.3 & 12.4 & 11.5 & 30.4 & 15.1 & 7.2 & 10.4 & 6.9 \
\bottomrule
\end{tabular}
\caption{Forgetting on further subsamples to test sentence length (-ss) and filtering (-ssf).}
\label{tab:6}
\end{table*}

\begin{table*}[t]
\centering
\small
\begin{tabular}{lrrrrrrrr}
\toprule
& \multicolumn{5}{c}{de-en} & \multicolumn{3}{c}{en-ja} \
& IT & Kor & Law & Med & Sub & IWSLT & KFTT & BSD \
\midrule
BLEU & 5.0 & 22.3 & 4.7 & 8.3 & 5.7 & 4.8 & 2.0 & 7.0 \
COMET & 0.11 & 0.78 & 0.08 & 0.16 & 0.09 & 0.07 & 0.06 & 0.29 \
ForgetGenUseAll & 0.09 & 0.25 & 0.07 & 0.11 & 0.09 & 0.14 & 0.11 & 0.16 \
\midrule
Generic NLL & -2.1 & -2.4 & -1.4 & -1.8 & -2.6 & & & \
Src-vcb JSD & 0.42 & 0.50 & 0.44 & 0.42 & & & & \
Trg-vcb JSD & 0.39 & 0.46 & 0.39 & 0.40 & & & & \
Src-vcb cover & 0.69 & 0.23 & 0.70 & 0.63 & & & & \
Trg-vcb cover & 0.52 & 0.23 & 0.59 & 0.48 & & & & \
Src-vcb cover (-s) & 0.50 & & 0.43 & 0.44 & & & & \
\bottomrule
\end{tabular}
\caption{Correlations of forgetting metrics with various domain heuristics.}
\label{tab:7}
\end{table*}

\noindent [MISSING CONTENT: Remaining analysis of Sections 3.2 and 3.3 regarding vocabulary coverage and domain similarity heuristics.]
=====END FILE=====

=====FILE: sections/04_improving.tex=====
\section{Improving adaptation with coverage-based data selection}

\noindent [MISSING CONTENT: Section 4.1]

\subsection{Minimal mix-in, better in-domain scores}
A primary goal of adapting NMT is improved in-domain translation. Table 10 gives quality metric deltas on the in-domain test sets: higher values are now better. A 1:1 generic ratio has a negative impact, with noticeable BLEU and COMET drops relative to unmixed fine-tuning. By contrast, Minimal Mix-in scores similarly to unmixed fine-tuning for all except de-en Med. Improvement in terms of COMET shows less variation than under BLEU, possibly because COMET assigns higher scores to paraphrases which may not use domain-specific terminology.

Mixing in large amounts of generic data reduces scores relative to Minimal Mix-in. It is interesting to note that for the smallest domain, Kor, mixing no generic data also leads to reduced in-domain scores.

\begin{table}[h]
\centering
\small
\begin{tabular}{lrr}
\toprule
Domain & Random 1:1 & Minimal Mix-in \
\midrule
IT & 222927 & 18861 \
Kor & 17982 & 22769 \
Law & 467309 & 17485 \
Med & 248099 & 19605 \
Sub & 500000 & 20100 \
IWSLT & 223103 & 14500 \
KFTT & 427218 & 15200 \
BSD & 20000 & 16800 \
\bottomrule
\end{tabular}
\caption{Size of mix-in datasets.}
\label{tab:8}
\end{table}

\begin{table*}[t]
\centering
\small
\begin{tabular}{lrrrrrrrr}
\toprule
Mix-in method & \multicolumn{5}{c}{de-en} & \multicolumn{3}{c}{en-ja} \
& IT & Kor & Law & Med & Sub & IWSLT & KFTT & BSD \
\midrule
BLEU & & & & & & & & \
No mix-in & 8.2 & 5.3 & 6.2 & 5.8 & 3.6 & 5.8 & 3.5 & 10.5 \
Random 1:1 & 6.5 & 6.1 & 3.4 & 4.7 & 2.7 & 4.6 & 3.2 & 9.1 \
Minimal Mix-in & 7.9 & 6.4 & 4.4 & 5.6 & 3.4 & 5.6 & 3.6 & 10.3 \
\midrule
COMET & & & & & & & & \
No mix-in & 0.26 & 0.09 & 0.05 & 0.04 & 0.07 & 0.10 & 0.04 & 0.12 \
Random 1:1 & 0.23 & 0.11 & 0.05 & 0.04 & 0.04 & 0.09 & 0.05 & 0.10 \
Minimal Mix-in & 0.26 & 0.12 & 0.05 & 0.05 & 0.06 & 0.11 & 0.05 & 0.12 \
\bottomrule
\end{tabular}
\caption{Forgetting metrics on generic test sets, varying the mix-in dataset when fine-tuning for 20K iterations in each case. Lower is better for all metrics. Negative scores indicate improvement.}
\label{tab:9}
\end{table*}

\begin{table*}[t]
\centering
\small
\begin{tabular}{lrrrrrrrr}
\toprule
Mix-in method & \multicolumn{5}{c}{de-en} & \multicolumn{3}{c}{en-ja} \
& IT & Kor & Law & Med & Sub & IWSLT & KFTT & BSD \
\midrule
BLEU & & & & & & & & \
No mix-in & 10.5 & 22.3 & 12.3 & 14.4 & 11.5 & 7.2 & 6.9 & 7.0 \
Random 1:1 & 8.4 & 19.8 & 10.1 & 12.1 & 9.2 & 6.1 & 5.8 & 6.2 \
Minimal Mix-in & 10.1 & 21.9 & 11.8 & 13.9 & 11.1 & 7.0 & 6.7 & 6.9 \
\midrule
COMET & & & & & & & & \
No mix-in & 0.24 & 0.78 & 0.28 & 0.36 & 0.24 & 0.20 & 0.28 & 0.29 \
Random 1:1 & 0.19 & 0.65 & 0.22 & 0.30 & 0.18 & 0.15 & 0.22 & 0.23 \
Minimal Mix-in & 0.23 & 0.75 & 0.26 & 0.34 & 0.23 & 0.19 & 0.27 & 0.28 \
\bottomrule
\end{tabular}
\caption{BLEU and COMET on in-domain test sets for the same experiments as in Table 9. Higher is better.}
\label{tab:10}
\end{table*}

As noted in Section 3, ForgetGenUse has less correlation with target coverage. A richer mix-in set may be required to address detrimental vocabulary shift.
=====END FILE=====

=====FILE: sections/05_limitations.tex=====
\section{Limitations}

Another limitation is model variety. In the interests of brevity, time and cost we only conduct our experiments with moderately sized Transformers trained for NMT. There has been much recent interest in machine translation by prompting Large Language Models (LLMs) pre-trained on huge uncurated datasets \citep{Zhang2023}. Work concurrent with ours by \citet{Pang2024} observe that LLMs also struggle with domain-specific translation. Indeed, when fine-tuning on the same de-en OPUS domain-specific datasets as us, they report that LLMs exhibit similar behaviour in terms of `forgetting' domain-specific terminology in preference to tokens appearing in the adaptation set, although they do not attempt to explain or mitigate this. We leave confirming experiments to future work.
=====END FILE=====

=====FILE: sections/appendix.tex=====
\section{Experimental setup}
We pre-train two Transformer models using the Tensorflow T2T toolkit \citep{Vaswani2018}, one for each of German-English (de-en) and English Japanese (en-ja). Both use BPE vocabulary \citep{Sennrich2016}, with details given in Table 11.

Following findings from the most recent WMT shared task \citep{Kocmi2023} on Transformer NMT models, we use deep encoders with relatively shallow decoders for a balance of speed and quality. We found a slightly deeper encoder and smaller, not shared BPE vocabulary gave better results for en-ja in initial testing.

The de-en model is pre-trained on 43.9M lines of parallel data made available via the WMT shared task: Paracrawl v9, Europarl v10, NewsCommentary v14, Tilde and WikiMatrix \citep{Kocmi2022}. The en-ja model is pre-trained on 22.4M lines of JParacrawl v3.0 \citep{Morishita2022}. When calculating BLEU for en-ja, we use Sacrebleu v2.0 \citep{Post2018} with the Mecab tokenizer.

To minimize our computational and energy use, we pre-train each model only once on 4 GPUs for approximately two days.
=====END FILE=====

=====FILE: refs.bib=====
@inproceedings{Aharoni2020,
title = {Unsupervised Domain Clusters in Neural Machine Translation},
author = {Aharoni, Roee and Goldberg, Yoav},
booktitle = {Proceedings of the 58th Annual Meeting of the Association for Computational Linguistics},
year = {2020}
}

@inproceedings{Barrault2020,
title = {Findings of the 2020 Conference on Machine Translation (WMT20)},
author = {Barrault, Loic and others},
booktitle = {Proceedings of the Fifth Conference on Machine Translation},
year = {2020}
}

@article{Savenkov2022,
title = {The state of the machine translation 2022},
author = {Savenkov, Konstantin and Lopez, Michel},
journal = {Intento},
year = {2022}
}

@inproceedings{Luong2015,
title = {Effective Approaches to Attention-based Neural Machine Translation},
author = {Luong, Minh-Thang and Manning, Christopher D.},
booktitle = {Proceedings of the 2015 Conference on Empirical Methods in Natural Language Processing},
year = {2015}
}

@inproceedings{McCloskey1989,
title = {Catastrophic Interference in Connectionist Networks: The Sequential Learning Problem},
author = {McCloskey, Michael and Cohen, Neal J.},
booktitle = {Psychology of Learning and Motivation},
year = {1989}
}

@inproceedings{Thompson2019,
title = {Overcoming catastrophic forgetting during domain adaptation of neural machine translation},
author = {Thompson, Brian and Gwinnup, Jeremy and Khayrallah, Huda and Duh, Kevin and Koehn, Philipp},
booktitle = {Proceedings of the 2019 Conference of the North American Chapter of the Association for Computational Linguistics},
pages = {2062--2068},
year = {2019}
}

@inproceedings{Gu2020,
title = {Investigating Catastrophic Forgetting During Domain Adaptation for Neural Machine Translation},
author = {Gu, Shuhao and Feng, Yang},
booktitle = {Proceedings of COLING},
year = {2020}
}

@inproceedings{Papineni2002,
title = {BLEU: a Method for Automatic Evaluation of Machine Translation},
author = {Papineni, Kishore and Roukos, Salim and Ward, Todd and Zhu, Wei-Jing},
booktitle = {Proceedings of the 40th Annual Meeting of the Association for Computational Linguistics},
year = {2002}
}

@inproceedings{Zhang2019,
title = {Curriculum Learning for Domain Adaptation in Neural Machine Translation},
author = {Zhang, Xuan and others},
booktitle = {Proceedings of NAACL},
year = {2019}
}

@inproceedings{Sharaf2020,
title = {Meta-Learning for Domain Adaptation in Neural Machine Translation},
author = {Sharaf, Amr and others},
booktitle = {Proceedings of the 58th Annual Meeting of the Association for Computational Linguistics},
year = {2020}
}

@article{Saunders2022,
title = {Domain adaptation and multi-domain adaptation for neural machine translation: A survey},
author = {Saunders, Danielle},
journal = {Journal of Artificial Intelligence Research},
volume = {75},
pages = {351--424},
year = {2022}
}

@inproceedings{Barone2017,
title = {Regularization techniques for fine-tuning in neural machine translation},
author = {Barone, Antonio Valerio Miceli and others},
booktitle = {Proceedings of EMNLP},
year = {2017}
}

@inproceedings{Shao2022,
title = {Overcoming Catastrophic Forgetting Beyond Continual Learning: Balanced Training for Neural Machine Translation},
author = {Shao, Chenze and Feng, Yang},
booktitle = {Proceedings of ACL},
year = {2022}
}

@inproceedings{Chu2017,
title = {Empirical Training Strategies for Domain Adaptation in Neural Machine Translation},
author = {Chu, Chenhui and others},
booktitle = {Proceedings of WAT},
year = {2017}
}

@inproceedings{Xu2019,
title = {Lexical Micro-adaptation for Neural Machine Translation},
author = {Xu, Jitao and others},
booktitle = {Proceedings of MT Summit XVII},
year = {2019}
}

@inproceedings{vanderWees2015,
title = {Translation Data Selection for Domain Adaptation},
author = {van der Wees, Marlies and Bisazza, Arianna and Weerkamp, Wouter and Monz, Christof},
booktitle = {Proceedings of the International Workshop on Spoken Language Translation},
year = {2015}
}

@inproceedings{Dyer2013,
title = {A Simple, Fast, and Effective Reparameterization of IBM Model 2},
author = {Dyer, Chris and Chahuneau, Victor and Smith, Noah A.},
booktitle = {Proceedings of NAACL},
year = {2013}
}

@article{Bojanowski2017,
title = {Enriching Word Vectors with Subword Information},
author = {Bojanowski, Piotr and others},
journal = {Transactions of the Association for Computational Linguistics},
year = {2017}
}

@inproceedings{Cettolo2012,
title = {WIT3: Web Inventory of Transcribed and Translated Talks},
author = {Cettolo, Mauro and Girardi, Christian and Federico, Marcello},
booktitle = {Proceedings of EAMT},
year = {2012}
}

@misc{Neubig2011,
title = {The Kyoto free translation task},
author = {Neubig, Graham},
year = {2011},
url = {[http://www.phontron.com/kftt](http://www.phontron.com/kftt)}
}

@inproceedings{Rikters2019,
title = {Designing the business conversation corpus},
author = {Rikters, Matiss and Ri, Ryokan and Li, Tong and Nakazawa, Toshiaki},
booktitle = {Proceedings of the 6th Workshop on Asian Translation},
pages = {54--61},
year = {2019}
}

@inproceedings{Rei2020,
title = {COMET: A neural framework for MT evaluation},
author = {Rei, Ricardo and Stewart, Craig and Farinha, Ana C and Lavie, Alon},
booktitle = {Proceedings of EMNLP},
pages = {2685--2702},
year = {2020}
}

@inproceedings{Hasler2021,
title = {Entangled Losses: Forgetting and Plasticity in Transfer Learning},
author = {Hasler, Eva and others},
booktitle = {Proceedings of EMNLP},
year = {2021}
}

@inproceedings{Pham2020,
title = {Study on Catastrophic Forgetting in Neural Machine Translation},
author = {Pham, Minh-Quang and others},
booktitle = {Proceedings of COLING},
year = {2020}
}

@inproceedings{Zhang2023,
title = {Big Translate: Augmenting Large Language Models with Multilingual Translation Capability},
author = {Zhang, Wen and others},
booktitle = {Proceedings of ACL},
year = {2023}
}

@inproceedings{Pang2024,
title = {Salute the classic: Revisiting challenges of machine translation in the age of large language models},
author = {Pang, Jianhui and others},
booktitle = {Proceedings of ACL},
year = {2024}
}

@inproceedings{Vaswani2018,
title = {Tensor2Tensor for Neural Machine Translation},
author = {Vaswani, Ashish and others},
booktitle = {Proceedings of AMTA},
year = {2018}
}

@inproceedings{Sennrich2016,
title = {Neural Machine Translation of Rare Words with Subword Units},
author = {Sennrich, Rico and others},
booktitle = {Proceedings of ACL},
year = {2016}
}

@inproceedings{Kocmi2023,
title = {Findings of the 2023 Conference on Machine Translation (WMT23)},
author = {Kocmi, Tom and others},
booktitle = {Proceedings of WMT},
year = {2023}
}

@inproceedings{Kocmi2022,
title = {Findings of the 2022 Conference on Machine Translation (WMT22)},
author = {Kocmi, Tom and others},
booktitle = {Proceedings of WMT},
year = {2022}
}

@inproceedings{Morishita2022,
title = {JParaCrawl v3.0: A large-scale English-Japanese parallel corpus},
author = {Morishita, Makoto and Chousa, Katsuki and Suzuki, Jun and Nagata, Masaaki},
booktitle = {Proceedings of LREC},
pages = {6704--6710},
year = {2022}
}

@inproceedings{Post2018,
title = {A Call for Clarity in Reporting BLEU Scores},
author = {Post, Matt},
booktitle = {Proceedings of WMT},
year = {2018}
}
=====END FILE=====

=====FILE: figures/README.txt=====
This project contains tables reconstructed from the text.
No image figures were found in the provided snippets, or they were not legible.
Any missing figures should be placed in this directory and referenced in the LaTeX files.
=====END FILE=====