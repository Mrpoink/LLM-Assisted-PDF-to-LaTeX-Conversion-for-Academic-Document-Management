=====FILE: main.tex=====
\documentclass[11pt,a4paper]{article}
\usepackage[utf8]{inputenc}
\usepackage[T1]{fontenc}
\usepackage{amsmath}
\usepackage{amsfonts}
\usepackage{amssymb}
\usepackage{graphicx}
\usepackage{booktabs}
\usepackage{multirow}
\usepackage{array}
\usepackage{caption}
\usepackage{subcaption}
\usepackage{geometry}
\geometry{left=1in,right=1in,top=1in,bottom=1in}
\usepackage{url}
\usepackage[normalem]{ulem}
\usepackage{xspace}
\sloppy

\title{A Comparison of Language Modeling and Translation as Multilingual Pretraining Objectives}
\author{
Zihao Li, $^{1}$ Shaoxiong Ji, $^{1}$ Timothee Mickus, $^{1}$ Vincent Segonne, $^{2}$ and J\"org Tiedemann $^{1}$ \\
$^{1}$ University of Helsinki \\
$^{2}$ Universit\'e Bretagne Sud \\
\texttt{firstname.lastname@\{helsinki.fi, $^{2}$ univ-ubs.fr\}}
}
\date{}

\begin{document}

\maketitle

\begin{abstract}
Pretrained language models (PLMs) display impressive performances and have captured the attention of the NLP community. Establishing best practices in pretraining has, therefore, become a major focus of NLP research, especially since insights gained from monolingual English models may not necessarily apply to more complex multilingual models. One significant caveat of the current state of the art is that different works are rarely comparable: they often discuss different parameter counts, training data, and evaluation methodology.

This paper proposes a comparison of multilingual pretraining objectives in a controlled methodological environment. We ensure that training data and model architectures are comparable, and discuss the downstream performances across 6 languages that we observe in probing and fine- tuning scenarios. We make two key observations: (1) the architecture dictates which pretraining objective is optimal; (2) multilingual translation is a very effective pretraining objective under the right conditions. We make our code, data, and model weights available at \url{https://github.com/Helsinki- NLP/lm- vs- mt}.
\end{abstract}

\section{Introduction}
The release of BERT (Devlin et al., 2019) has marked a paradigm shift in the NLP landscape and has ushered in a thorough investment of the NLP research community in developing large language models that can readily be adapted to novel situations. The design, training, and evaluation of these models has become a significant enterprise of its own.

In recent years, that sustained interest has shifted also to encompass multilingual models (e.g., Muennighoff et al., 2022; Alves et al., 2024). There is considerable variation as to how such models are trained: For instance, some rely on datasets comprising multiple languages without explicit crosslingual supervision (e.g., Liu et al., 2020), and some use explicit supervision (Xue et al., 2021). One complication that arises from this blossoming field of study is that much of the work being carried out is not directly comparable beyond the raw performances on some well- established benchmark, a procedure which may well be flawed (Gorman and Bedrick, 2019). Avoiding apples- to- oranges comparison requires a methodical approach in strictly comparable circumstances, which is the stance we adopt in this paper.

In short, we focus on two variables- model architecture and pretraining objectives- and set out to train five models in strictly comparable conditions and compare their monolingual performances in three downstream applications: sentiment analysis, named entity recognition, and POS- tagging. The scope of our study spans from encoder- decoder machine translation models, to decoder- only causal language models and encoder- only BERT- like masked language models. We categorize them into double- stacks (encoder- decoder) and single- stacks (encoder- only or decoder- only) models. We intend to answer two research questions:

(i) Does the explicit cross-lingual training signal of translation objectives foster better downstream performances in monolingual tasks? (ii) Is the optimal choice of architecture independent of the training objective?

There are a prima facie reasons to favor either answers to both of these questions. For instance, the success of multilingual pretrained language models (LM) on cross- lingual tasks has been underscored repeatedly (Wu and Dredze, 2019, e.g.,), yet explicit alignments such as linear mapping (Wang et al., 2019) and L2 alignment (Cao et al., 2020) between source and target languages do not necessarily improve the quality of cross-lingual representations (Wu and Dredze, 2020).

Our experiments provide tentative evidence that insofar as a BART denoising autoencoder architecture is concerned, models pretrained with a translation objective consistently outperform those trained with a denoising objective. However, for single- stack transformers, we observe causal language models to perform well in probing and masked language models to generally outperform translation and causal objectives when fine- tuned on downstream tasks. This leads us to conjecture that the optimal pretraining objective depends on the architecture. Furthermore, the best downstream results we observe appear to stem from a machine- translation system, highlighting that MT encoder- decoder systems might constitute an understudied but potentially very impactful type of pretrained model.

\section{Methods and Settings}
We start our inquiry by adopting a principled stance: We train strictly comparable models with MT and LM objectives before contrasting their performances on monolingual tasks.

\subsection{Models and objectives}
To allow a systematic evaluation, we train models with various neural network architectures and learning objectives. All models are based on the transformer architecture (Vaswani et al., 2017) and implemented in fairseq (Ott et al., 2019). We consider both double- stacks (encoder- decoder) and single- stacks (encoder- only or decoder- only) models.

The two double- stack models are variants of the BART architecture of (Lewis et al., 2020); they are trained either on a straightforward machine translation (MT) objective, using language tokens to distinguish the source, or on the original denoising auto- encoder objective of Lewis et al.. We refer to these two models as 2- LM and 2- MT respectively.

We also consider three single- stack models: (i) an encoder- only model trained on the masked language modeling objective (MLM) of Devlin et al. (2019); (ii) an autoregressive causal language model (CLM), similar to Radford et al. (2019); and (iii) an autoregressive model trained to generate a sentence, followed by its translation in the language specified by a given control token, known as a translation language model (TLM) as proposed by Conneau and Lample (2019).\footnote{We provide an example datapoint for each pretraining objective in Table 3, Appendix A.}

\subsection{Pretraining conditions}
Our core focus is on guaranteeing comparable conditions across the different pretraining objectives we consider. This entails that our datasets need to be doubly structured: both in documents for CLM pretraining; and as aligned bitexts for MT pretraining. Two datasets broadly match these criteria: the UNPC (Ziemski et al., 2016) and OpenSubtitles (OpSub; Tiedemann, 2012) corpora. The choice also narrows down the languages considered in this study: we take the set of languages present in both resources, namely the six languages in UNPC: Arabic (AR), Chinese (ZH), English (EN), French (FR), Russian (RU), and Spanish (ES).

To guarantee that models are trained on the same data, whenever a document is available in multiple languages, we greedily assign it to the least represented language pair thus far and discard all other possible language pairs where it could have contributed; we then discard documents which cannot be used as bitexts. This ensures that all documents are used exactly once for both document- level and bitext- level pretraining objectives. Dataset statistics are shown in Table 4, Appendix B.

To ensure a fair comparison, we control key variables, including tokenization (100k BPE pieces; Sennrich et al., 2016), number of transformer layers (12), hidden dimensions (512), attention heads (8), and feedforward layer dimensions (2048). We perform 600k steps of updates, using the largest batch size that fits into the GPU memory, deploy distributed training to make a global batch size of 4096, and apply the Adam optimizer (Kingma and Ba, 2017). Owing to the computational requirements, we only train one seed for each of the five types of models considered.

\subsection{Downstream evaluation}
The evaluations encompassed both sequence- level and token- level classification tasks using datasets tailored for sentiment analysis (SA), named entity recognition (NER), part- of- speech (POS) tagging, and natural language inference (NLI).

For SA, we utilized the Amazon review dataset (Hou et al., 2024) in English, Spanish, French, and Chinese. RuReviews (Smetanin and Komarov, 2019) for Russian, and ar\_res\_reviews (ElSahar and El- Beltagy, 2015) for Arabic. While the datasets for most languages were pre- split, ar\_res\_reviews required manual division into training, validation, and testing sets, using an 8:1:1 ratio.

For NER, we model the problem as an entity span extraction using a BIO scheme. In practice, we classify tokens into three basic categories: Beginning of an entity (B), Inside an entity (I), or Outside any entity (O). We use the MultiCoNER v2 dataset (Fetahu et al., 2023) for English, Spanish, French, and Chinese, MultiCoNER v1 (Malmasi et al., 2022) for Russian and the AQMAR Wikipedia NER corpus (Mohit et al., 2012a) for Arabic. Simplifying the NER task to these fundamental categories allows us to focus more on assessing the basic entity recognition capabilities of the models without the additional complexity of differentiating numerous entity types, which can vary significantly between languages and datasets.

For POS tagging, we utilized the Universal Dependencies (UD) 2.0 datasets (Nivre et al., 2020), selecting specific corpora tailored to each language to ensure both linguistic diversity and relevance. We select multiple UD treebanks per language, such that each language dataset comprises approximately 160,000 tokens, which are then split into training, validation, and testing segments with an 8:1:1 ratio.

For NLI, we employed the XNLI dataset (Conneau et al., 2018) for the six languages. The XNLI dataset consists of sentence pairs translated from the MultiNLI dataset (Williams et al., 2018) into 15 languages, providing consistent annotations across languages. The task focuses on classifying the relationship between pairs of sentences into one of three categories: Entailment, Contradiction, or Neutral. Unlike the original cross- lingual design of XNLI, we conducted monolingual experiments for each language to evaluate the performance of our models individually in each linguistic context.

Supplementary details regarding data preprocessing for downstream experiments are available in Appendix B.

We evaluate the performances of the encoder output representations for the 2- MT and 2- LM models and of the last hidden representation before the vocabulary projection for the single- stack models.

The evaluation of the models involves two distinct experimental approaches to test the performance: probing and fine- tuning. In the probing experiments, only the parameters of the classification heads are adjusted. This method primarily tests the raw capability of the pre- trained models' embeddings to adapt to specific tasks with minimal parameter changes, preserving the underlying pre- trained network structure. Conversely, in the fine- tuning experiments, all parameters of the models are adjusted. This approach allows the entire model to adapt to the specifics of the task, potentially leading to higher performance at the cost of significantly altering the pre- trained weights.

For both experimental approaches, each model is trained for 10 epochs to ensure sufficient learning without overfitting. We optimize parameters with AdamW (Loshchilov and Hutter, 2017), with a constant learning rate of 0.0001 across all tasks and models. This setup was chosen to standardize the training process, providing a fair basis for comparing the performance outcomes across different models and tasks. We reproduce probing and fine- tuning for 5 seeds to ensure stability.

\section{Results}

\subsection{Double- stack models}
We first compare the performance of 2- LM and 2- MT across several key language processing tasks including SA, NER, POS tagging, and NLI. Results are shown in Tables 1a and 1b. The pretraining objectives play a significant role in shaping the models' effectiveness. Specifically, 2- MT, which is pretrained with a machine translation objective, consistently outperforms 2- LM, which utilizes a denoising objective. This pattern is consistent across all languages tested after fine- tuning as well as probing.

\subsection{Single- stack models}
Turning to the single- stack models (CLM, MLM, TLM), we find a somewhat more complex picture. In a probing context (cf. Table 2a), we find the CLM to be almost always the most effective, except for NLI in five languages and NER in Arabic, where it performs slightly less favorably compared to the MLM. As for fine- tuning (Table 2b), while the MLM generally ranks first on all POS, NER, and NLI datasets, the TLM is usually effective for SA.\footnote{3}

\begin{table}[h]
\centering
\begin{tabular}{lcccccc}
\hline
Setup & EN & ES & FR & ZH & RU & AR \\
\hline
2-1LM & 42.86$\pm$0.86 & 42.80$\pm$0.69 & 43.00$\pm$0.60 & 40.41$\pm$1.02 & 65.83$\pm$0.70 & 70.88$\pm$1.62 \\
2-2MT & 46.71$\pm$0.88 & 46.64$\pm$0.55 & 46.10$\pm$0.43 & 43.74$\pm$0.65 & 68.79$\pm$0.42 & 73.77$\pm$0.97 \\
\hline
2-2LM & 82.69$\pm$0.09 & 84.74$\pm$0.07 & 82.80$\pm$0.06 & 78.88$\pm$0.25 & 77.93$\pm$0.15 & 85.28$\pm$0.22 \\
2-2MT & 89.47$\pm$0.06 & 90.54$\pm$0.04 & 89.41$\pm$0.10 & 88.78$\pm$0.09 & 83.39$\pm$0.22 & 89.70$\pm$0.18 \\
\hline
2-2LM & 78.85$\pm$0.29 & 78.12$\pm$0.25 & 81.57$\pm$0.32 & 66.09$\pm$0.25 & 77.93$\pm$0.12 & 47.68$\pm$0.10 \\
2-2MT & 92.22$\pm$0.14 & 90.59$\pm$0.20 & 95.39$\pm$0.10 & 75.87$\pm$0.17 & 93.20$\pm$0.08 & 61.84$\pm$0.24 \\
\hline
2-2LM & 48.56$\pm$0.01 & 49.31$\pm$0.01 & 48.33$\pm$0.01 & 38.81$\pm$0.01 & 48.34$\pm$0.01 & 45.11$\pm$0.01 \\
2-2MT & 60.50$\pm$0.01 & 59.56$\pm$0.01 & 59.00$\pm$0.01 & 59.01$\pm$0.01 & 59.83$\pm$0.01 & 59.58$\pm$0.01 \\
\hline
\end{tabular}
\caption{Accuracy $(\times 100)$ of double- stack models ($\pm \mathrm{s.d.}$ over 5 runs). (a) Probing (b) Fine-tuning}
\label{tab:doublestack}
\end{table}

\begin{table}[h]
\centering
\caption{Single-stack models results.}
\label{tab:singlestack}
\begin{subtable}{\textwidth}
\centering
\begin{tabular}{lcccccc}
\hline
Setup & EN & ES & FR & ZH & RU & AR \\
\hline
CLM & 35.14$\pm$0.92 & 35.66$\pm$1.10 & 34.14$\pm$1.63 & 33.62$\pm$0.83 & 57.57$\pm$1.11 & 67.71$\pm$1.24 \\
MLM & 34.26$\pm$1.34 & 34.82$\pm$1.58 & 33.90$\pm$1.12 & 32.52$\pm$1.65 & 54.55$\pm$1.86 & 65.94$\pm$3.30 \\
TLM & 29.68$\pm$2.22 & 32.20$\pm$3.07 & 32.26$\pm$2.34 & 29.88$\pm$4.17 & 56.45$\pm$1.81 & 64.45$\pm$1.81 \\
\hline
CLM & 55.23$\pm$0.72 & 47.81$\pm$1.55 & 54.84$\pm$0.62 & 51.18$\pm$0.94 & 75.07$\pm$0.21 & 66.18$\pm$1.74 \\
MLM & 55.22$\pm$0.92 & 55.67$\pm$1.17 & 54.08$\pm$2.43 & 51.00$\pm$1.07 & 74.53$\pm$1.36 & 75.00$\pm$3.48 \\
TLM & 55.14$\pm$0.92 & 55.84$\pm$0.59 & 55.22$\pm$0.98 & 51.46$\pm$0.53 & 75.31$\pm$0.57 & 72.75$\pm$2.25 \\
\hline
CLM & 80.27$\pm$0.12 & 82.59$\pm$0.06 & 80.38$\pm$0.12 & 77.92$\pm$0.28 & 76.39$\pm$0.03 & 84.17$\pm$0.08 \\
MLM & 78.77$\pm$0.02 & 81.61$\pm$0.00 & 79.11$\pm$0.01 & 70.67$\pm$0.10 & 76.34$\pm$0.01 & 84.29$\pm$0.00 \\
TLM & 79.10$\pm$0.06 & 81.94$\pm$0.13 & 79.56$\pm$0.14 & 77.26$\pm$0.24 & 76.39$\pm$0.02 & 84.26$\pm$0.02 \\
\hline
CLM & 89.91$\pm$0.33 & 91.42$\pm$0.15 & 90.65$\pm$0.17 & 89.97$\pm$0.14 & 83.20$\pm$0.31 & 87.50$\pm$2.22 \\
MLM & 93.31$\pm$0.57 & 93.93$\pm$0.60 & 93.67$\pm$0.30 & 92.99$\pm$0.99 & 87.49$\pm$0.78 & 85.78$\pm$3.30 \\
TLM & 89.88$\pm$0.06 & 91.45$\pm$0.25 & 90.49$\pm$0.23 & 90.10$\pm$0.14 & 83.76$\pm$0.63 & 84.29$\pm$0.00 \\
\hline
CLM & 69.06$\pm$0.38 & 70.32$\pm$0.50 & 76.67$\pm$0.46 & 51.40$\pm$0.47 & 59.64$\pm$0.62 & 43.49$\pm$0.40 \\
MLM & 37.92$\pm$0.61 & 44.26$\pm$0.11 & 46.89$\pm$0.32 & 31.16$\pm$0.21 & 34.62$\pm$0.16 & 34.71$\pm$0.94 \\
TLM & 62.96$\pm$1.02 & 62.08$\pm$1.99 & 63.89$\pm$1.06 & 50.46$\pm$0.53 & 54.27$\pm$0.87 & 40.94$\pm$1.16 \\
\hline
CLM & 91.72$\pm$0.14 & 90.51$\pm$0.13 & 95.75$\pm$0.10 & 78.61$\pm$0.31 & 85.50$\pm$0.15 & 57.43$\pm$1.63 \\
MLM & 96.00$\pm$0.15 & 94.45$\pm$0.13 & 97.94$\pm$0.20 & 89.96$\pm$0.71 & 96.69$\pm$0.13 & 74.35$\pm$0.53 \\
TLM & 91.68$\pm$0.19 & 90.38$\pm$0.20 & 86.99$\pm$19.40 & 78.50$\pm$0.52 & 85.71$\pm$0.18 & 59.11$\pm$0.50 \\
\hline
CLM & 42.32$\pm$0.02 & 42.99$\pm$0.01 & 43.43$\pm$0.02 & 40.55$\pm$0.02 & 40.06$\pm$0.02 & 41.99$\pm$0.01 \\
MLM & 45.64$\pm$0.02 & 44.49$\pm$0.01 & 43.11$\pm$0.02 & 42.80$\pm$0.01 & 43.16$\pm$0.01 & 43.55$\pm$0.01 \\
TLM & 38.36$\pm$0.02 & 41.95$\pm$0.02 & 41.89$\pm$0.01 & 38.93$\pm$0.04 & 41.20$\pm$0.02 & 39.50$\pm$0.02 \\
\hline
CLM & 48.84$\pm$0.14 & 56.46$\pm$0.03 & 55.45$\pm$0.03 & 49.70$\pm$0.06 & 55.23$\pm$0.02 & 49.02$\pm$0.07 \\
MLM & 59.41$\pm$0.01 & 57.54$\pm$0.04 & 55.04$\pm$0.06 & 47.96$\pm$0.03 & 57.80$\pm$0.01 & 53.60$\pm$0.01 \\
TLM & 49.76$\pm$0.10 & 52.12$\pm$0.11 & 54.20$\pm$0.10 & 49.03$\pm$0.04 & 53.60$\pm$0.04 & 44.39$\pm$0.10 \\
\hline
\end{tabular}
\caption{Probing}
\end{subtable}
\begin{subtable}{\textwidth}
\centering
\begin{tabular}{lcccccc}
\hline
Setup & EN & ES & FR & ZH & RU & AR \\
\hline
[ILLEGIBLE] & [ILLEGIBLE] & [ILLEGIBLE] & [ILLEGIBLE] & [ILLEGIBLE] & [ILLEGIBLE] & [ILLEGIBLE] \\
\hline
\end{tabular}
\caption{Fine-tuning}
\end{subtable}
\end{table}

\subsection{Discussion}
A first global observation that we can make for these results is that single- stack and double- stack models appear to behave differently. While the MT objective yields the highest performances for BART- type models, the downstream performances of the TLM do not really stand out compared to the CLM in probing and the MLM in fine- tuning scenarios. It is important to note that the performances stem at least in part from the architecture itself: 2- MT and 2- LM both consistently outperform all single- stack models in probing. However, it is crucial to acknowledge the limitations of our study, as we only conducted one pretraining round for all the objectives. Hence, this evidence should be interpreted as tentative at best.

Fine- tuning also tends to minimize the difference between single- stack and double- stack models which suggests that the higher quality of double- stack representations could be an artifact of training limitations. Moreover, the relative ranks of the three single- stack models fluctuate much more than what we see for the double- stack models, owing to no little extent to the oftentimes momentous variation across seeds for single- stack models. We therefore conjecture that while a translation objective can yield a clear training signal towards semantically informed representations, this comes with two caveats: first, the signal can only be leveraged with dedicated separate modeling of source and target (viz. double- stack models); second, this advantage is much less consequential when fine- tuning.

\section{Related works}
Multilingual foundation models have flourished in recent years (a.o., Conneau and Lample, 2019; Liu et al., 2020; Xue et al., 2021; Kale et al., 2021; Fang et al., 2021; Chi et al., 2021; Alves et al., 2024; Ustun et al., 2024), and with them so have studies of their representations (Conneau et al., 2020; Siddhant et al., 2020; Choudhury and Deshpande, 2021; Fierro and Sogaard, 2022; Hammerl et al., 2023 a.o.). All of these works, however, fail to control for some of the most crucial factors, such as ensuring that all models are trained on comparable amounts of data.

This work is specifically related to Conneau and Lample (2019), which also compares MLM, CLM, and TLM but does not normalize the training data. Another point of comparison is Ji et al. (2024), which studies the impact of MT continued pretraining in BART on cross- lingual downstream tasks. Monolingual evaluation of multilingual systems has also been broached a.o. by Rust et al. (2021).

free evaluation methods can offer insights into a model's inherent capabilities without the variability introduced by fine- tuning.

\section{Conclusion}
This paper conducts an empirical study of how pretraining conditions of multilingual models impact downstream performances in probing and finetuning scenarios. Despite the inherent limitations that stem from our stringent data requirements, our experiments offer a novel perspective that highlights directions for future inquiry into how multilingual foundation models ought to be pretrained. We observe that double- stack BART- based models fare much better than single- stack models in probing scenarios, but the difference is overall less clear when it comes to fine- tuning. We also find some tentative evidence that translation objectives can be highly effective for model pretraining in precise circumstances: Namely, the most effective model on downstream tasks among those we experimented with is an MT- pretrained BART- like model, which outperforms both a more traditional denoising objective for BART as well as decoder- only CLM and encoder- only MLM models. This would suggest that translation can serve as a powerful pretraining objective, although it is currently under- explored.\footnote{4}

Another crucial aspect of our study is that we present strictly comparable models, trained on comparable data, with comparable parameter counts and unified implementations. While this entails some limitations, especially with regard to the scale of models and data used, we nonetheless believe that a strict comparison can help discriminate between the various factors at play in other works. Here, we find clear evidence that CLM pretraining objectives, such as those used in GPT, outperform MLM- based models, such as BERT, in probing scenarios; we are also able to isolate and highlight how the optimal choice of pretraining objective is contingent on the architecture being employed.

For future work, we recommend exploring multitask learning during pretraining by combining objectives like translation, denoising, and language modeling; in such cases, models could harness the strengths of each task to become more robust and versatile. Additionally, investigating training

\section*{Acknowledgments}
We thank Alessandro Raganato and our colleagues at the Helsinki- NLP group for useful discussions throughout this project, as well as the three anonymous reviewers for their comments.

This project has received funding from the European Union's Horizon Europe research and innovation programme under Grant agreement No 101070350 and from UK Research and Innovation (UKRI) under the UK government's Horizon Europe funding guarantee [grant number 10052546], and partially funded by the French National Research Agency [grant ANR- 23- IAS1- 0001]. The contents of this publication are the sole responsibility of its authors and do not necessarily reflect the opinion of the European Union.

The authors wish to thank CSC- IT Center for Science, Finland, for the generous computational resources on the Puhti supercomputer and LUMI supercomputer through the LUMI extreme scale access (MOOMIN and LumiNMT). Some of the experiments were performed using the Jean Zay and Adastra clusters from GENCI- IDRIS [grant 2022 A0131013801].

\section*{Limitations}
This study employs models that are not large in terms of parameters in the era of large language models. Such a constraint potentially hinders the generalizability of our results to much larger architectures that are capable of handling a broader array of linguistic nuances. Furthermore, our study focuses on a small selected group of languages and specific NLP tasks. This focus might limit the applicability of our findings to other linguistic contexts or more complex real- world applications where diverse language phenomena or different task demands play a crucial role.

Another limitation is our reliance on specific corpora. The datasets utilized, while valuable, represent a potential source of selection bias. They may not fully encompass the vast diversity of global language use, thus skewing the model training and evaluation. Such a bias could affect the robustness and effectiveness of the pretrained models when applied to languages that are not well- represented in the training data.

\section*{References}
\begin{thebibliography}{00}

\bibitem{alves2024}
Duarte M. Alves, Jose Pombal, Nuno M. Guerreiro, Pedro H. Martins, Joao Alves, Amin Farajian, Ben Peters, Ricardo Rei, Patrick Fernandes, Sweta Agrawal, Pierre Colombo, Jose G. C. de Souza, and Andre F. T. Martins.
\newblock 2024.
\newblock Tower: An open multilingual large language model for translation- related tasks.

\bibitem{cao2020}
Steven Cao, Nikita Kitaev, and Dan Klein.
\newblock 2020.
\newblock Multilingual alignment of contextual word representations.
\newblock In {\em International Conference on Learning Representations}.

\bibitem{chi2021}
Zewen Chi, Li Dong, Shuming Ma, Shaohan Huang, Saksham Singhal, Xian- Ling Mao, Heyan Huang, Xia Song, and Furu Wei.
\newblock 2021.
\newblock mT6: Multilingual pretrained text- to- text transformer with translation pairs.
\newblock In {\em Proceedings of the 2021 Conference on Empirical Methods in Natural Language Processing}, pages 1671- 1683, Online and Punta Cana, Dominican Republic. Association for Computational Linguistics.

\bibitem{choudhury2021}
Monojit Choudhury and Amit Deshpande.
\newblock 2021.
\newblock How linguistically fair are multilingual pre- trained language models?
\newblock {\em Proceedings of the AAAI Conference on Artificial Intelligence}, 35(14):12710- 12718.

\bibitem{conneau2019}
Alexis Conneau and Guillaume Lample.
\newblock 2019.
\newblock Crosslingual language model pretraining.
\newblock In {\em Advances in Neural Information Processing Systems}, volume 32. Curran Associates, Inc.

\bibitem{conneau2018}
Alexis Conneau, Ruty Rinott, Guillaume Lample, Adina Williams, Samuel Bowman, Holger Schwenk, and Veselin Stoyanov.
\newblock 2018.
\newblock XNLI: Evaluating crosslingual sentence representations.
\newblock In {\em Proceedings of the 2018 Conference on Empirical Methods in Natural Language Processing}, pages 2475- 2485, Brussels, Belgium. Association for Computational Linguistics.

\bibitem{conneau2020}
Alexis Conneau, Shijie Wu, Haoran Li, Luke Zettlemoyer, and Veselin Stoyanov.
\newblock 2020.
\newblock Emerging crosslingual structure in pretrained language models.
\newblock In {\em Proceedings of the 58th Annual Meeting of the Association for Computational Linguistics}, pages 6022- 6034, Online. Association for Computational Linguistics.

\bibitem{devlin2019}
Jacob Devlin, Ming- Wei Chang, Kenton Lee, and Kristina Toutanova.
\newblock 2019.
\newblock BERT: Pre- training of deep bidirectional transformers for language understanding.
\newblock In {\em Proceedings of the 2019 Conference of the North American Chapter of the Association for Computational Linguistics: Human Language Technologies, Volume 1 (Long and Short Papers)}, pages 4171- 4186, Minneapolis, Minnesota. Association for Computational Linguistics.

\bibitem{elsahar2015}
Hady ElSahar and Samhaa R El- Beltagy.
\newblock 2015.
\newblock Building large arabic multi- domain resources for sentiment analysis.
\newblock In {\em International conference on intelligent text processing and computational linguistics}, pages 23- 34. Springer.

\bibitem{fang2021}
Yuwei Fang, Shuohang Wang, Zhe Gan, Siqi Sun, and Jingjing Liu.
\newblock 2021.
\newblock Filter: An enhanced fusion method for cross- lingual language understanding.
\newblock In {\em Proceedings of the AAAI Conference on Artificial Intelligence}, volume 35, pages 12776- 12784.

\bibitem{fetahu2023}
Besnik Fetahu, Zhiyu Chen, Sudipta Kar, Oleg Rokhlenko, and Shervin Malmasi.
\newblock 2023.
\newblock Multi- CoNER v2: a large multilingual dataset for fine- grained and noisy named entity recognition.
\newblock In {\em Findings of the Association for Computational Linguistics: EMNLP 2023}, pages 2027- 2051, Singapore. Association for Computational Linguistics.

\bibitem{fierro2022}
Constanza Fierro and Anders Sogaard.
\newblock 2022.
\newblock Factual consistency of multilingual pretrained language models.
\newblock In {\em Findings of the Association for Computational Linguistics: ACL 2022}, pages 3046- 3052, Dublin, Ireland. Association for Computational Linguistics.

\bibitem{gorman2019}
Kyle Gorman and Steven Bedrick.
\newblock 2019.
\newblock We need to talk about standard splits.
\newblock In {\em Proceedings of the 57th Annual Meeting of the Association for Computational Linguistics}, pages 2786- 2791, Florence, Italy. Association for Computational Linguistics.

\bibitem{guillaume2019}
Bruno Guillaume, Marie- Catherine de Marneffe, and Guy Perrier.
\newblock 2019.
\newblock Conversion et améliorations de corpus du français annotés en Universal Dependencies [conversion and improvement of Universal Dependencies French corpora].
\newblock {\em Traitement Automatique des Langues}, 60(2):71- 95.

\bibitem{hammerl2023}
Katharina Hammerl, Alina Fastowski, Jindřich Libovický, and Alexander Fraser.
\newblock 2023.
\newblock Exploring anisotropy and outliers in multilingual language models for cross- lingual semantic sentence similarity.
\newblock In {\em Findings of the Association for Computational Linguistics: ACL 2023}, pages 7023- 7037, Toronto, Canada. Association for Computational Linguistics.

\bibitem{hou2024}
Yupeng Hou, Jiacheng Li, Zhankui He, An Yan, Xiusi Chen, and Julian McAuley.
\newblock 2024.
\newblock Bridging language and items for retrieval and recommendation.
\newblock arXiv preprint arXiv:2403.03952.

\bibitem{ji2024}
Shaoxiong Ji, Timothee Mickus, Vincent Segonne, and Jörg Tiedemann.
\newblock 2024.
\newblock Can machine translation bridge multilingual pretraining and cross- lingual transfer learning?
\newblock In {\em Proceedings of the 2024 Joint International Conference on Computational Linguistics, Language Resources and Evaluation (LRECCOLING 2024)}, pages 2809- 2818, Torino, Italia. ELRA and ICCL.

\bibitem{kale2021}
Mihir Kale, Aditya Siddhant, Rami Al- Rfou, Linting Xue, Noah Constant, and Melvin Johnson.
\newblock 2021.
\newblock nmT5 - is parallel data still relevant for pre- training massively multilingual language models?
\newblock In {\em Proceedings of the 59th Annual Meeting of the Association for Computational Linguistics and the 11th International Joint Conference on Natural Language Processing (Volume 2: Short Papers)}, pages 683- 691, Online. Association for Computational Linguistics.

\bibitem{kingma2017}
Diederik P. Kingma and Jimmy Ba.
\newblock 2017.
\newblock Adam: A method for stochastic optimization.

\bibitem{lee2017}
John Lee, Herman Leung, and Keying Li.
\newblock 2017.
\newblock Towards Universal Dependencies for learner Chinese.
\newblock In {\em Proceedings of the NoDaLiDa 2017 Workshop on Universal Dependencies (UDW 2017)}, pages 67- 71, Gothenburg, Sweden. Association for Computational Linguistics.

\bibitem{lewis2020}
Mike Lewis, Yinhan Liu, Naman Goyal, Marjan Ghazvininejad, Abdelrahman Mohamed, Omer Levy, Veselin Stoyanov, and Luke Zettlemoyer.
\newblock 2020.
\newblock BART: Denoising sequence- to- sequence pre- training for natural language generation, translation, and comprehension.
\newblock In {\em Proceedings of the 58th Annual Meeting of the Association for Computational Linguistics}, pages 7871- 7880, Online. Association for Computational Linguistics.

\bibitem{li2022}
Yixuan Li, Gerdes Kim, Guillaume Bruno, and Dan Zeman.
\newblock 2022.
\newblock Ud chinese patentchar.

\bibitem{liu2020}
Yinhan Liu, Jiatao Gu, Naman Goyal, Xian Li, Sergey Edunov, Marjan Ghazvininejad, Mike Lewis, and Luke Zettlemoyer.
\newblock 2020.
\newblock Multilingual denoising pretraining for neural machine translation.
\newblock {\em Transactions of the Association for Computational Linguistics}, 8:726- 742.

\bibitem{loshchilov2017}
Ilya Loshchilov and Frank Hutter.
\newblock 2017.
\newblock Decoupled weight decay regularization.
\newblock arXiv preprint arXiv:1711.05101.

\bibitem{lyashevskaya2018}
Olga Lyashevskaya, Olga Rudina, Natalia Vlasova, and Anna Zhuravleva.
\newblock 2018.
\newblock Ud russian taiga.

\bibitem{malmasi2022}
Shervin Malmasi, Anjie Fang, Besnik Fetahu, Sudipta Kar, and Oleg Rokhlenko.
\newblock 2022.
\newblock MultiCoNER: A large- scale multilingual dataset for complex named entity recognition.
\newblock In {\em Proceedings of the 29th International Conference on Computational Linguistics}, pages 3798- 3809, Gyeongju, Republic of Korea. International Committee on Computational Linguistics.

\bibitem{mcdonald2013}
Ryan McDonald, Joakim Nivre, Yvonne Quirmbach- Brundage, Yoav Goldberg, Dipanjan Das, Kuzman Ganchev, Keith Hall, Slav Petrov, Hao Zhang, Oscar Tackstrom, Claudia Bedini, Nuria Bertomeu Castello, and Jungmee Lee.
\newblock 2013.
\newblock Universal Dependency annotation for multilingual parsing.
\newblock In {\em Proceedings of the 51st Annual Meeting of the Association for Computational Linguistics (Volume 2: Short Papers)}, pages 92- 97, Sofia, Bulgaria. Association for Computational Linguistics.

\bibitem{mohit2012a}
Behrang Mohit, Nathan Schneider, Rishav Bhowmick, Kemal Oflazer, and Noah A. Smith.
\newblock 2012a.
\newblock Recall- oriented learning of named entities in Arabic Wikipedia.
\newblock In {\em Proceedings of the 13th Conference of the European Chapter of the Association for Computational Linguistics}, pages 162- 173, Avignon, France. Association for Computational Linguistics.

\bibitem{mohit2012b}
Behrang Mohit, Nathan Schneider, Rishav Bhowmick, Kemal Oflazer, and Noah A Smith.
\newblock 2012b.
\newblock Recall- oriented learning of named entities in arabic wikipedia.
\newblock In {\em Proceedings of the 13th Conference of the European Chapter of the Association for Computational Linguistics}, pages 162- 173.

\bibitem{muennighoff2022}
Niklas Muennighoff, Thomas Wang, Lintang Sutawika, Adam Roberts, Stella Biderman, Teven Le Scao, M Saiful Bari, Sheng Shen, Zheng- Xin Yong, Hailey Schoelkopf, et al.
\newblock 2022.
\newblock Crosslingual generalization through multitask finetuning.
\newblock arXiv preprint arXiv:2211.01786.

\bibitem{nivre2017}
Joakim Nivre, Zeljko Agic, Lars Ahrenberg, Lene Antonsen, Maria Jesus Aranzabe, Masayuki Asahara, Luma Ateyah, Mohammed Attia, Aitziber Atutxa, Elena Badmaeva, Miguel Ballesteros, Esha Banerjee, Sebastian Bank, John Bauer, Kepa Bengoetxea, Riyaz Ahmad Bhat, Eckhard Bick, Cristina Bosco, Gosse Bouma, Sam Bowman, Aljoscha Burchardt, Marie Candito, Gauthier Caron, Gulsen Cebiroglu Eryigit, Giuseppe G. A. Celano, Savas Cetin, Fabricio Chalub, Jinho Choi, Yongseok Cho, Silvie Cinkova, Cagri Coltekin, Miriam Connor, Marie- Catherine de Marneffe, Valeria de Paiva, Arantza Diaz de Ilarraza, Kaja Dobrovoljc, Timothy Dozat, Kira Droganova, Marhaba Eli, Ali Elkahky, Tomaz Erjavec, Richard Farkas, Hector Fernandez Alcalde, Jennifer Foster, Claudia Freitas, Katarina Gajdoso, Daniel Galbraith, Marcos Garcia, Filip Ginter, Iakes Goenaga, Koldo Gojenola, Memduh Gokirmak, Yoav Goldberg, Xavier Gomez Guinovart, Berta Gonzalez Saavedra, Matias Grioni, Normunds Gruzitis, Bruno Guilllaume, Nizar Habash, Jan Hajic, Jan Hajic jr., Linh Ha My, Kim Harris, Dag Haug, Barbora Hladka, Jaroslava Hlavacova, Petter Hohle, Radu Ion, Elena Irimia, Anders Johannsen, Fredrik Jorgensen, Huner Kasikara, Hiroshi Kanayama, Jenna Kanerva, Tolga Kayadelen, Vaclava Kettnerova, Jesse Kirchner, Natalia Kotsyba, Simon Krek, Sookyoung Kwak, Veronika Laippala, Lorenzo Lambertino, Tatiana Lando, Phoeng Le Hong, Alessandro Lenci, Saran Lertpradit, Herman Leung, Cheuk Ying Li, Josie Li, Nikola Ljubešić, Olga Loginova, Olga Lyashevskaya, Teresa Lynn, Vivien Macketanz, Aibek Makazhanov, Michael Mandl, Christopher Manning, Ruli Manurung, Catilina Maranduc, David Marecek, Katrin Marheinecke, Hector Martinez Alonso, Andre Martins, Jan Mašek, Yuji Matsumoto, Ryan McDonald, Gustavo Mendonca, Anna Missila, Verginica Mititelu, Yusuke Miyao, Simonetta Montemagni, Amir More, Laura Moreno Romero, Shunsuke Mori, Bohdan Moskalevskyi, Kadri Muischnek, Nina Mustafina, Kaili Muiirisep, Pinkey Nainwani, Anna Nedoluzhko, Loeng Nguyen Thi, Huyen Nguyen Thi Minh, Vitaly Nikolaev, Rattima Nitisaroj, Hanna Nurmi, Stina Ojala, Petya Osenova, Lilja Ovreld, Elena Pascual, Marco Passarotti, CenelAugusto Perez, Guy Perrier, Slav Petrov, Jussi Piitulainen, Emily Pitler, Barbara Plank, Martin Popel, Lauma Pretkalnina, Prokopis Prokopidis, Tiina Puolakainen, Sampo Pyysalo, Alexandre Rademaker, Liyu Real, Siva Reddy, Georg Rehm, Larissa Rinaldi, Laura Rituma, Rudolf Rosa, Davide Rovati, Shadi Saleh, Manuela Sanguinetti, Baiba Saulite, Yanin Sawanakunanon, Sebastian Schuster, Djame Seddah, Wolfgang Seeker, Mojgan Seraji, Lena Shakurova, Mo Shen, Atsuko Shimada, Muh Shohibussirri, Natalia Silveira, Maria Simi, Radu Simionescu, Katalin Simko, Maria Simková, Kiril Simov, Aaron Smith, Antonio Stella, Jana Strnadová, Alane Suhr, Umut Sulubacak, Zsolt Szántó, Dima Taji, Takaaki Tanaka, Trond Trosterud, Anna Trukhina, Reut Tsarfaty, Francis Tyers, Sumire Uematsu, Zdenka Uresová, Larraitz Uria, Hans Uszkoreit, Gertjan van Noord, Viktor Varga, Veronika Vincze, Jonathan North Washington, Zhuoran Yu, Zdeněk Žabokrtský, Daniel Zeman, and Hanzhi Zhu.
\newblock 2017.
\newblock Universal dependencies 2.0 - CoNLL 2017 shared task development and test data.
\newblock LINDAT/CLARIAH- CZ digital library at the Institute of Formal and Applied Linguistics (UFAL), Faculty of Mathematics and Physics, Charles University.

\bibitem{nivre2020}
Joakim Nivre, Marie- Catherine De Marneffe, Filip Ginter, Jan Hajic, Christopher D Manning, Sampo Pyysalo, Sebastian Schuster, Francis Tyers, and Daniel Zeman.
\newblock 2020.
\newblock Universal dependencies v2: An evergrowing multilingual treebank collection.
\newblock arXiv preprint arXiv:2004.10643.

\bibitem{ott2019}
Myle Ott, Sergey Edunov, Alexei Baevski, Angela Fan, Sam Gross, Nathan Ng, David Grangier, and Michael Auli.
\newblock 2019.
\newblock fairseq: A fast, extensible toolkit for sequence modeling.
\newblock In {\em Proceedings of the 2019 Conference of the North American Chapter of the Association for Computational Linguistics (Demonstrations)}, pages 48- 53, Minneapolis, Minnesota. Association for Computational Linguistics.

\bibitem{qi2019}
Peng Qi, Koichi Yasuoka, and Dan Zeman.
\newblock 2019.
\newblock Ud chinese gsdsimp.

\bibitem{radford2019}
Alec Radford, Jeffrey Wu, Rewon Child, David Luan, Dario Amodei, Ilya Sutskever, et al.
\newblock 2019.
\newblock Language models are unsupervised multitask learners.
\newblock OpenAI blog, 1(8):9.

\bibitem{rust2021}
Phillip Rust, Jonas Pfeiffer, Ivan Vulić, Sebastian Ruder, and Iryna Gurevych.
\newblock 2021.
\newblock How good is your tokenizer? on the monolingual performance of multilingual language models.
\newblock In {\em Proceedings of the 59th Annual Meeting of the Association for Computational Linguistics and the 11th International Joint Conference on Natural Language Processing (Volume 1: Long Papers)}, pages 3118- 3135, Online. Association for Computational Linguistics.

\bibitem{sennrich2016}
Rico Sennrich, Barry Haddow, and Alexandra Birch.
\newblock 2016.
\newblock Neural machine translation of rare words with subword units.
\newblock In {\em Proceedings of the 54th Annual Meeting of the Association for Computational Linguistics (Volume 1: Long Papers)}, pages 1715- 1725, Berlin, Germany. Association for Computational Linguistics.

\bibitem{siddhant2020}
Aditya Siddhant, Melvin Johnson, Henry Tsai, Naveen Ari, Jason Riesa, Ankur Bapna, Orhan Firat, and Karthik Raman.
\newblock 2020.
\newblock Evaluating the cross- lingual effectiveness of massively multilingual neural machine translation.
\newblock In {\em Proceedings of the AAAI conference on artificial intelligence}, volume 34, pages 8854- 8861.

\bibitem{smetanin2019}
Sergey Smetanin and Michail Komarov.
\newblock 2019.
\newblock Sentiment analysis of product reviews in russian using convolutional neural networks.
\newblock In {\em 2019 IEEE 21st Conference on Business Informatics (CBI)}, volume 01, pages 482- 486.

\bibitem{tiedemann2012}
Jörg Tiedemann.
\newblock 2012.
\newblock Parallel data, tools and interfaces in opus.
\newblock In {\em Proceedings of LREC}, volume 2012, pages 2214- 2218.

\bibitem{vaswani2017}
Ashish Vaswani, Noam Shazeer, Niki Parmar, Jakob Uszkoreit, Llion Jones, Aidan N Gomez, Ł ukasz Kaiser, and Illia Polosukhin.
\newblock 2017.
\newblock Attention is all you need.
\newblock In {\em Advances in Neural Information Processing Systems}, volume 30. Curran Associates, Inc.

\bibitem{wang2019}
Yuxuan Wang, Wanxiang Che, Jiang Guo, Yijia Liu, and Ting Liu.
\newblock 2019.
\newblock Cross- lingual bert transformation for zero- shot dependency parsing.
\newblock In {\em Proceedings of the 2019 Conference on Empirical Methods in Natural Language Processing and the 9th International Joint Conference on Natural Language Processing (EMNLP- IJCNLP)}, pages 5721- 5727.

\bibitem{williams2018}
Adina Williams, Nikita Nangia, and Samuel Bowman.
\newblock 2018.
\newblock A broad- coverage challenge corpus for sentence understanding through inference.
\newblock In {\em Proceedings of the 2018 Conference of the North American Chapter of the Association for Computational Linguistics: Human Language Technologies, Volume 1 (Long Papers)}, pages 1112- 1122, New Orleans, Louisiana. Association for Computational Linguistics.

\bibitem{wong2017}
Tak- sum Wong, Kim Gerdes, Herman Leung, and John Lee.
\newblock 2017.
\newblock Quantitative comparative syntax on the Cantonese- Mandarin parallel dependency treebank.
\newblock In {\em Proceedings of the Fourth International Conference on Dependency Linguistics (Depling 2017)}, pages 266- 275, Pisa, Italy. Linkoping University Electronic Press.

\bibitem{wu2019}
Shijie Wu and Mark Dredze.
\newblock 2019.
\newblock Beto, bentz, becas: The surprising cross- lingual effectiveness of bert.
\newblock In {\em Proceedings of the 2019 Conference on Empirical Methods in Natural Language Processing and the 9th International Joint Conference on Natural Language Processing (EMNLP- IJCNLP)}, pages 833- 844.

\bibitem{wu2020}
Shijie Wu and Mark Dredze.
\newblock 2020.
\newblock Do explicit alignments robustly improve multilingual encoders?
\newblock In {\em Proceedings of the 2020 Conference on Empirical Methods in Natural Language Processing (EMNLP)}, pages 4471- 4482.

\bibitem{xue2021}
Linting Xue, Noah Constant, Adam Roberts, Mihir Kale, Rami Al- Rfou, Aditya Siddhant, Aditya Barua, and Colin Raffel.
\newblock 2021.
\newblock mT5: A massively multilingual pre- trained text- to- text transformer.
\newblock In {\em Proceedings of the 2021 Conference of the North American Chapter of the Association for Computational Linguistics: Human Language Technologies}, pages 483- 498, Online. Association for Computational Linguistics.

\bibitem{zeldes2017}
Amir Zeldes.
\newblock 2017.
\newblock The GUM corpus: Creating multilayer resources in the classroom.
\newblock {\em Language Resources and Evaluation}, 51(3):581- 612.

\bibitem{zeman2023}
Dan Zeman, Kirian Guiller, and Bruno Guillaume.
\newblock 2023.
\newblock Ud chinese beginner.

\bibitem{zemanek2008}
Otakar Smrz Viktor Bielicky Iveta Koufilova Jakub Kracmar Zemanek.
\newblock 2008.
\newblock Dependency treebank : A word on the million words.

\bibitem{ziemski2016}
Michal Ziemski, Marcin Junczys- Dowmunt, and Bruno Pouliquen.
\newblock 2016.
\newblock The United Nations parallel corpus v1.0.
\newblock In {\em Proceedings of the Tenth International Conference on Language Resources and Evaluation (LREC'16)}, pages 3530- 3534, Portoroz, Slovenia. European Language Resources Association (ELRA).

\bibitem{ustun2024}
Ahmet Ustun, Viraat Aryabumi, Zheng- Xin Yong, Wei- Yin Ko, Daniel D'souza, Gbemileke Onilude, Neel Bhandari, Shivalika Singh, Hui- Lee Ooi, Amr Kayid, Freddie Vargus, Phil Blunsom, Shayne Longpre, Niklas Muennighoff, Marzieh Fadaee, Julia Kreutzer, and Sara Hooker.
\newblock 2024.
\newblock Aya model: An instruction finetuned open- access multilingual language model.
\newblock arXiv preprint arXiv:2402.07827.

\end{thebibliography}

\appendix
\section{Overview of pretraining objectives}
Table 3 displays an example data point for all pretraining objectives we consider. In principle, the CLM is a document- level objective, i.e., the full document would be used as an input rather than the two sentences we show here.

\section{Datasets statistics}
An overview of the volume of data available for pretraining is displayed in Table 4. The majority of the data were used for training.

In Table 5, we present an overview of the datasets used for downstream evaluation.

\section{Detailed results}
In Table 6 and Table 7, we present the macro- f1 score of models in the downstream evaluation.

\begin{table}[h]
\centering
\caption{Overview of the different objectives considered in this study. Top two rows: two-stacks (encoder-decoder) models; bottom three rows: single-stack (encoder-only or decoder-only) models.}
\label{tab:objectives}
\begin{tabular}{ll}
\hline
Objective & Example \\
\hline
2-LM & \texttt{\_D’autres \_mesures \_de \_ce \_type \_vont \_être [MASK] [MASK], ...} \\
2-MT & \texttt{<fr> \_D’autres \_mesures \_de \_ce \_type \_vont \_être \_appliqu ées, ...} \\
CLM & \texttt{... \_Divers \_accords \_ad \_hoc \_ont \_été \_conclus ...} \\
TLM & \texttt{\_D’autres \_mesures \_de \_ce \_type \_vont \_être \_appliqu ées, ... <fr2en> \_Other \_similar ...} \\
MLM & \texttt{\_D’autres \_mesures \_de \_ce \_type \_vont \_être [MASK] [MASK], ...} \\
\hline
\end{tabular}
\end{table}

\begin{table}[h]
\centering
\caption{Number of sentences in pretraining corpora.}
\label{tab:pretrain_stats}
\begin{tabular}{lcccc}
\hline
 & Train & Validation & Test & Total \\
\hline
UNPC & 114,376,177 & 76,303 & 40,712 & 114,493,192 \\
OpSub & 81,622,353 & 359,035 & 77,342 & 82,058,730 \\
Total & 195,998,530 & 435,338 & 118,054 & 196,551,922 \\
\hline
\end{tabular}
\end{table}

\begin{table}[h]
\centering
\caption{Statistics of datasets used for downstream evaluation tasks.}
\label{tab:downstream_stats}
\begin{tabular}{llllll}
\hline
Task & Language & Dataset & Class Count & Train & Validation & Test \\
\hline
\multirow{6}{*}{SA} & EN & Amazon Review (Hou et al., 2024) & 5 & 200000 & 5000 & 5000 \\
 & ES & Amazon Review & 5 & 200000 & 5000 & 5000 \\
 & FR & Amazon Review & 5 & 200000 & 5000 & 5000 \\
 & ZH & Amazon Review & 5 & 200000 & 5000 & 5000 \\
 & RU & RuReviews (Smetanin and Komarov, 2019) & 3 & 8560 & 1214 & 3213 \\
 & AR & ar\_res\_reviews (ElSahar and El-Beltagy, 2015) & 2 & 6680 & 835 & 835 \\
\hline
\multirow{6}{*}{NER} & EN & MultiCoNER v2 (Fetahu et al., 2023) & 32 & 530111 & 3323 & 3737 \\
 & ES & MultiCoNER v2 & 32 & 628141 & 3462 & 3925 \\
 & FR & MultiCoNER v2 & 32 & 477431 & 3062 & 3742 \\
 & ZH & MultiCoNER v2 & 32 & 456061 & 2816 & 4896 \\
 & RU & MultiCoNER v1 (Malmasi et al., 2022) & 32 & 423841 & 2787 & 2061 \\
 & AR & AQMAR Wikipedia NER corpus (Mohit et al., 2012b) & 3 & 5705 & 386 & 1581 \\
\hline
\multirow{6}{*}{POS} & EN & UD\_English-GUM (Zeldes, 2017) & 16 & 12839 & 1607 & 1555 \\
 & ES & UD\_Spanish-GSD (McDonald et al., 2013) & 16 & 12745 & 1691 & 564 \\
 & FR & UD\_French-GSD (Guillaume et al., 2019) & 15 & 12763 & 1620 & 167 \\
 & ZH & UD\_Chinese-Beginner (Zeman et al., 2023; AllSet Learning, 2023)+... & 16 & 12893 & 1568 & 1575 \\
 & RU & UD\_Russian-Taiga (Lyashevskaya et al., 2018) & 16 & 12765 & 1660 & 1584 \\
\hline
\multirow{6}{*}{NLI} & EN & XNLI (Conneau et al., 2018) & 3 & 392702 & 2490 & 5010 \\
 & ES & XNLI & 3 & 392702 & 2490 & 5010 \\
 & FR & XNLI & 3 & 392702 & 2490 & 5010 \\
 & ZH & XNLI & 3 & 392702 & 2490 & 5010 \\
 & RU & XNLI & 3 & 392702 & 2490 & 5010 \\
 & AR & XNLI & 3 & 392702 & 2490 & 5010 \\
\hline
\end{tabular}
\end{table}

\begin{table}[h]
\centering
\caption{Macro F1 score after model fine-tuning.}
\label{tab:macro_f1_finetune}
\begin{tabular}{llcccccc}
\hline
Task & Model & EN & ES & FR & ZH & RU & AR \\
\hline
\multirow{5}{*}{SA} & 2-LM & 0.5213$\pm$0.0068 & 0.5254$\pm$0.0083 & 0.5244$\pm$0.0135 & 0.4739$\pm$0.0096 & 0.7421$\pm$0.0059 & 0.7522$\pm$0.0151 \\
 & 2-MT & 0.5407$\pm$0.0086 & 0.5510$\pm$0.0084 & 0.5398$\pm$0.0054 & 0.4956$\pm$0.0093 & 0.7522$\pm$0.0056 & 0.7767$\pm$0.0156 \\
 & CLM & 0.5443$\pm$0.0072 & 0.4446$\pm$0.2115 & 0.5421$\pm$0.0089 & 0.5015$\pm$0.0187 & 0.7553$\pm$0.0015 & 0.5283$\pm$0.2328 \\
 & MLM & 0.5441$\pm$0.0107 & 0.5466$\pm$0.0314 & 0.5348$\pm$0.0237 & 0.4972$\pm$0.0142 & 0.7509$\pm$0.0135 & 0.5695$\pm$0.1427 \\
 & TLM & 0.5358$\pm$0.0186 & 0.5501$\pm$0.0128 & 0.5474$\pm$0.0137 & 0.5069$\pm$0.0119 & 0.7586$\pm$0.0057 & 0.4599$\pm$0.0943 \\
\hline
\multirow{5}{*}{NER} & 2-LM & 0.8200$\pm$0.0042 & 0.8092$\pm$0.0053 & 0.8259$\pm$0.0035 & 0.8626$\pm$0.0022 & 0.7215$\pm$0.0122 & 0.7274$\pm$0.0093 \\
 & 2-MT & 0.8670$\pm$0.0017 & 0.8651$\pm$0.0022 & 0.8727$\pm$0.0018 & 0.8897$\pm$0.0042 & 0.7934$\pm$0.0039 & 0.8685$\pm$0.0046 \\
 & CLM & 0.7950$\pm$0.0064 & 0.8053$\pm$0.0028 & 0.8099$\pm$0.0044 & 0.8129$\pm$0.0021 & 0.6622$\pm$0.0182 & 0.5994$\pm$0.1880 \\
 & MLM & 0.8635$\pm$0.0123 & 0.8580$\pm$0.0142 & 0.8706$\pm$0.0055 & 0.8739$\pm$0.0199 & 0.7629$\pm$0.0172 & 0.4113$\pm$0.2254 \\
 & TLM & 0.7908$\pm$0.0028 & 0.8024$\pm$0.0081 & 0.8067$\pm$0.0047 & 0.8120$\pm$0.0032 & 0.6758$\pm$0.0312 & 0.3094$\pm$0.0000 \\
\hline
\multirow{5}{*}{POS} & 2-LM & 0.8925$\pm$0.0039 & 0.7365$\pm$0.0025 & 0.8496$\pm$0.0034 & 0.8088$\pm$0.0059 & 0.8984$\pm$0.0055 & 0.7769$\pm$0.0102 \\
 & 2-MT & 0.9314$\pm$0.0024 & 0.7826$\pm$0.0235 & 0.8866$\pm$0.0074 & 0.8842$\pm$0.0059 & 0.9285$\pm$0.0029 & 0.8660$\pm$0.0088 \\
 & CLM & 0.8752$\pm$0.0042 & 0.7854$\pm$0.0024 & 0.8573$\pm$0.0041 & 0.7906$\pm$0.0195 & 0.8264$\pm$0.0104 & 0.5932$\pm$0.0194 \\
 & MLM & 0.9177$\pm$0.0068 & 0.8079$\pm$0.0259 & 0.8851$\pm$0.0019 & 0.8313$\pm$0.0079 & 0.9226$\pm$0.0048 & 0.8602$\pm$0.0132 \\
 & TLM & 0.8782$\pm$0.0045 & 0.7830$\pm$0.0067 & 0.7421$\pm$0.2503 & 0.7876$\pm$0.0271 & 0.8247$\pm$0.0088 & 0.6201$\pm$0.0071 \\
\hline
\multirow{5}{*}{NLI} & 2-LM & 0.5771$\pm$0.0067 & 0.5760$\pm$0.0088 & 0.5658$\pm$0.0085 & 0.4766$\pm$0.0058 & 0.5629$\pm$0.0052 & 0.5350$\pm$0.0070 \\
 & 2-MT & 0.6183$\pm$0.0054 & 0.6151$\pm$0.0082 & 0.5991$\pm$0.0073 & 0.5302$\pm$0.0086 & 0.5887$\pm$0.0041 & 0.5678$\pm$0.0032 \\
 & CLM & 0.4240$\pm$0.2315 & 0.5589$\pm$0.0355 & 0.5493$\pm$0.0404 & 0.4729$\pm$0.1123 & 0.5507$\pm$0.0265 & 0.4554$\pm$0.1199 \\
 & MLM & 0.5927$\pm$0.0189 & 0.5719$\pm$0.0487 & 0.5282$\pm$0.0964 & 0.4618$\pm$0.0453 & 0.5775$\pm$0.0069 & 0.5247$\pm$0.0221 \\
 & TLM & 0.4428$\pm$0.1751 & 0.4728$\pm$0.1731 & 0.5345$\pm$0.1076 & 0.4558$\pm$0.0722 & 0.5061$\pm$0.0771 & 0.3816$\pm$0.1562 \\
\hline
\end{tabular}
\end{table}

\end{document}
=====END FILE=====