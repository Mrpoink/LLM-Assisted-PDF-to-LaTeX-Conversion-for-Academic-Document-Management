=====FILE: main.tex=====
\documentclass[11pt,a4paper]{article}
\usepackage[utf8]{inputenc}
\usepackage[T1]{fontenc}
\usepackage{geometry}
\geometry{a4paper, margin=1in}
\usepackage{amsmath,amssymb}
\usepackage{graphicx}
\usepackage{booktabs}
\usepackage{array}
\usepackage{multirow}
\usepackage{caption}
\usepackage{hyperref}
\hypersetup{
    colorlinks=true,
    linkcolor=blue,
    filecolor=magenta,
    urlcolor=cyan,
}
\urlstyle{same}

\title{LLM-based Code-Switched Text Generation for Grammatical Error Correction}
\author{Tom Potter\thanks{University of Manchester, \texttt{thomas.potter@postgrad.manchester.ac.uk}} \and Zheng Yuan\thanks{King's College London, \texttt{zheng.yuan@kcl.ac.uk}}}
\date{}

\begin{document}

\maketitle

\begin{abstract}
With the rise of globalisation, code- switching (CSW) has become a ubiquitous part of multilingual conversation, posing new challenges for natural language processing (NLP), especially in Grammatical Error Correction (GEC). This work explores the complexities of applying GEC systems to CSW texts. Our objectives include evaluating the performance of state- of- the- art GEC systems on an authentic CSW dataset from English as a Second Language (ESL) learners, exploring synthetic data generation as a solution to data scarcity, and developing a model capable of correcting grammatical errors in monolingual and CSW texts. We generated synthetic CSW GEC data, resulting in one of the first substantial datasets for this task, and showed that a model trained on this data is capable of significant improvements over existing systems. This work targets ESL learners, aiming to provide educational technologies that aid in the development of their English grammatical correctness without constraining their natural multilingualism.
\end{abstract}

\section{Introduction}
Code- switching (CSW), the practice of fluidly alternating between two or more languages in conversation, has become commonplace in recent years. This linguistic phenomenon, emerging as a natural consequence of multilingualism, is now widely accepted in social and professional settings (Yow et al., 2018). Many works have highlighted the utility and cultural importance of CSW in general conversation (Beatty- Martinez et al., 2020; Falbo and LaCroix, 2021). Further research indicates that these advantages extend to language learning, with CSW offering many pedagogical benefits. These include increasing students' access to content and improving their confidence. Nguyen et al. (2022) discuss the mechanisms for this, where students

Example 1:According to the test, [lacks in me $\rightarrow$ my shortcomings] are 下 and 主人样. Example 2: When we [call $\rightarrow$ say] 5 5 5 5 5 5 5 5 5 5 6 5 5 5 5 5 5 5 5 5 7 5 5 5 5 5 5 5 5 5 5 8 5 5 5 5 5 5 5 5 5 9 5 5 5 5 5 5 5 5 5 5 10 5 5 5 5 5 5 5 5 5 11 5 5 5 5 5 5 5 5 5 12 5 5 5 5 5 5 5 5 5 13 5 5 5 5 5 5 5 5 5 14 5 5 5 5 5 5 5 5 5 15 5 5 5 5 5 5 5 5 5 16 5 5 5 5 5 5 5 5 5 17 5 5 5 5 5 5 5 5 5 18 5 5 5 5 5 5 5 5 5 19 5 5 5 5 5 5 5 5 5 20 5 5 5 5 5 5 5 5 5 21 5 5 5 5 5 5 5 5 5 22 5 5 5 5 5 5 5 5 5 23 5 5 5 5 5 5 5 5 5 24 5 5 5 5 5 5 5 5 5 25 5 5 5 5 5 5 5 5 5 26 5 5 5 5 5 5 5 5 5 27 5 5 5 5 5 5 5 5 5 28 5 5 5 5 5 5 5 5 5 29 5 5 5 5 5 5 5 5 5 30 5 5 5 5 5 5 5 5 5 31 5 5 5 5 5 5 5 5 5 32 5 5 5 5 5 5 5 5 5 33 5 5 5 5 5 5 5 5 5 34 5 5 5 5 5 5 5 5 5 35 5 5 5 5 5 5 5 5 5 36 5 5 5 5 5 5 5 5 5 37 5 5 5 5 5 5 5 5 5 38 5 5 5 5 5 5 5 5 5 39 5 5 5 5 5 5 5 5 5 40 5 5 5 5 5 5 5 5 5 41 5 5 5 5 5 5 5 5 5 42 5 5 5 5 5 5 5 5 5 43 5 5 5 5 5 5 5 5 5 44 5 5 5 5 5 5 5 5 5 45 5 5 5 5 5 5 5 5 5 46 5 5 5 5 5 5 5 5 5 47 5 5 5 5 5 5 5 5 5 48 5 5 5 5 5 5 5 5 5 49 5 5 5 5 5 5 5 5 5 50 5 5 5 5 5 5 5 5 5 51 5 5 5 5 5 5 5 5 5 52 5 5 5 5 5 5 5 5 5 53 5 5 5 5 5 5 5 5 5 54 5 5 5 5 5 5 5 5 5 55 5 5 5 5 5 5 5 5 5 56 5 5 5 5 5 5 5 5 5 57 5 5 5 5 5 5 5 5 5 58 5 5 5 5 5 5 5 5 5 59 5 5 5 5 5 5 5 5 5 60 5 5 5 5 5 5 5 5 5 61 5 5 5 5 5 5 5 5 5 62 5 5 5 5 5 5 5 5 5 63 5 5 5 5 5 5 5 5 5 64 5 5 5 5 5 5 5 5 5 65 5 5 5 5 5 5 5 5 5 66 5 5 5 5 5 5 5 5 5 67 5 5 5 5 5 5 5 5 5 68 5 5 5 5 5 5 5 5 5 69 5 5 5 5 5 5 5 5 5 70 5 5 5 5 5 5 5 5 5 71 5 5 5 5 5 5 5 5 5 72 5 5 5 5 5 5 5 5 5 73 5 5 5 5 5 5 5 5 5 74 5 5 5 5 5 5 5 5 5 75 5 5 5 5 5 5 5 5 5 76 5 5 5 5 5 5 5 5 5 77 5 5 5 5 5 5 5 5 5 78 5 5 5 5 5 5 5 5 5 79 5 5 5 5 5 5 5 5 5 80 5 5 5 5 5 5 5 5 5 81 5 5 5 5 5 5 5 5 5 82 5 5 5 5 5 5 5 5 5 83 5 5 5 5 5 5 5 5 5 84 5 5 5 5 5 5 5 5 5 85 5 5 5 5 5 5 5 5 5 86 5 5 5 5 5 5 5 5 5 87 5 5 5 5 5 5 5 5 5 88 5 5 5 5 5 5 5 5 5 89 5 5 5 5 5 5 5 5 5 90 5 5 5 5 5 5 5 5 5 91 5 5 5 5 5 5 5 5 5 92 5 5 5 5 5 5 5 5 5 93 5 5 5 5 5 5 5 5 5 94 5 5 5 5 5 5 5 5 5 95 5 5 5 5 5 5 5 5 5 96 5 5 5 5 5 5 5 5 5 97 5 5 5 5 5 5 5 5 5 98 5 5 5 5 5 5 5 5 5 99 5 5 5 5 5 5 5 5 5 100 5 5 5 5 5 5 5 5 5 101 5 5 5 5 5 5 5 5 5 102 5 5 5 5 5 5 5 5 5 103 5 5 5 5 5 5 5 5 5 104 5 5 5 5 5 5 5 5 5 105 5 5 5 5 5 5 5 5 5 106 5 5 5 5 5 5 5 5 5 107 5 5 5 5 5 5 5 5 5 108 5 5 5 5 5 5 5 5 5 109 5 5 5 5 5 5 5 5 5 110 5 5 5 5 5 5 5 5 5 111 5 5 5 5 5 5 5 5 5 112 5 5 5 5 5 5 5 5 5 113 5 5 5 5 5 5 5 5 5 114 5 5 5 5 5 5 5 5 5 115 5 5 5 5 5 5 5 5 5 116 5 5 5 5 5 5 5 5 5 117 5 5 5 5 5 5 5 5 5 118 5 5 5 5 5 5 5 5 5 119 5 5 5 5 5 5 5 5 5 120 5 5 5 5 5 5 5 5 5 121 5 5 5 5 5 5 5 5 5 122 5 5 5 5 5 5 5 5 5 123 5 5 5 5 5 5 5 5 5 124 5 5 5 5 5 5 5 5 5 125 5 5 5 5 5 5 5 5 5 126 5 5 5 5 5 5 5 5 5 127 5 5 5 5 5 5 5 5 5 128 5 5 5 5 5 5 5 5

\subsection{2.2.1 Step 1: CSW Text Generation}
Three different synthetic data generation techniques have been explored to generate CSW data.

variable syntax, semantics and pragmatics, add additional complexity to this task. Monolingual seq2seq GEC models, e.g. T5 (Rothe et al., 2021), struggle with CSW text as they fail to represent the non- English inputs, resulting in their inability to output the CSW text. On the other hand, multilingual seq2seq models and edit- based GEC models like GECToR (Omelianchuk et al., 2020) can handle CSW text but struggle with the ambiguity present at language switching points. This ambiguity challenges the models' ability to accurately correct the text.

This paper aims to bridge this gap. Firstly, to address the data scarcity issue, we propose a method for generating high- quality synthetic CSW GEC data, using which we produce, to our knowledge, one of the first substantial datasets labelled for this task. Secondly, we train a token classificationstyle GEC system, tailored to correct errors in texts produced by ESL learners. This demographic is significant for our study as they not only present consistent CSW patterns but also stand to benefit greatly from a GEC system capable of handling CSW text.

\section{Data}
\subsection{Genuine CSW GEC Dataset}
One of the only datasets labelled for GEC which does not remove CSW text, is the Lang8 dataset (Mizumoto et al., 2013), sourced from the Lang- 8 language learning platform. This dataset, when filtered to contain entries where CSW is present, offers a foundation of authentic data, comprises 5,875 pairs of ungrammatical and corrected sentences across 6 CSW language pairs: English- Japanese (81.9\%), EnglishKorean (13.0\%), English- Traditional Chinese (3.4\%), English- Russian (1.2\%), English- Thai (0.5\%) and English- Arabic (0.1\%).

The crowd- sourced nature of Lang- 8 required manual validation to ensure accuracy. We tasked an annotator with the responsibility of verifying the original corrections in the dataset, as well as combing for missed errors, incorrect annotations and over- annotations.

\subsection{Synthetic CSW GEC Data Generation}
Given the small size of the available CSW GEC dataset, we introduced a 2- step approach to synthetic CSW GEC data generation. First, we gener

ated grammatically correct CSW sentences. This is followed by the introduction of errors.

\subsubsection{Step 1: CSW Text Generation}
Three different synthetic data generation techniques have been explored to generate CSW data.

Translation- based CSW Text Generation required a monolingual corpus, a machine translation (MT) algorithm, and a sentence parser. To generate a CSW utterance, we used the Stanford Parser v4.5.4 (Manning et al., 2014) to build a syntactic parse tree. We then randomly selected and translated a subtree using the ArgosTranslate MT package4 (Finlay, 2023; Klein et al., 2017). This method generates plausible CSW text. However, performance is dependent on the strength of the parsing and translation algorithms; and the style of language within the corpus. To approximate the style of our authentic CSW text, we used corrected monolingual sentences from the Lang- 8 corpus.

Parallel Corpus- based CSW Text Generation avoids the need for a translation algorithm. Instead, we used the same Stanford Parser, this time with Spanish, French and German configurations; and the AWESOME word- alignment model5 (Dou and Neubig, 2021), to identify parts of parallel corpora labelled for MT with similar syntactic structure. For this method, we use the Europarl corpus (Koehn et al., 2012) due to the grammatical quality of its English component. Under the Equivalence Constraint Theory (Rizvi et al., 2021), these areas are where CSW is likely to take place. We, therefore, randomly chose overlapping subtrees as candidates for injection of non- English text. Although this method does not require MT, it is reliant on performant word- alignment and parsing systems; these are rare for many languages.

LLM Prompting- based CSW Text Generation The other methods of generating CSW text rely on injecting a second language into existing monolingual corpora. Hence, they are not able to recreate one of the main switching styles shown by ESL learners - CSW as a genuine pragmatic strategy. A common reason for this style of switching is when quoting another language. It is difficult to recreate this style using a monolingual foundation as sentences like The Japanese word for "dog" is "犬" seldom appear in authentic monolingual corpora.

\begin{table}[h]
\centering
\caption{Quantitative Description of the Genuine and Generated CSw Datasets Using Various CSw Metrics.}
\label{tab:metrics}
\begin{tabular}{lcccc}
\toprule
Metric & Genuine CSW & LLM CSW & Translation CSW & Corpus CSW \\
\midrule
CMI & 15.52 & 16.14 & 27.81 & 11.42 \\
M-Index & 0.007 & 0.004 & 0.015 & 0.006 \\
I-Index & 0.21 & 0.21 & 0.30 & 0.20 \\
Burstiness & -0.07 & -0.04 & 0.03 & -0.11 \\
CF1 & 6.38 & 5.82 & 17.13 & 2.55 \\
CF2 & 19.77 & 19.03 & 31.11 & 16.04 \\
CF3 & 18.34 & 17.61 & 30.05 & 14.20 \\
\bottomrule
\end{tabular}
\end{table}

To generate diverse CSw texts without relying on existing corpora or inaccurate alignment algorithms, we leveraged the strong general knowledge of Large Language Models (LLMs). We demonstrated that OpenAI's GPT- 3.5 (Brown et al., 2020) can create high- quality CSw sentences when shown examples of authentic utterances. Along with genuine CSw texts, we supplied a one- shot example of how to use the switching styles of an existing CSw text to generate a new sentence.6

Comparison of Synthetic CSw Text We used several CSw metrics to quantify the qualities of CSw texts: Code Mixing Index (CMI) (Gambäck and Das, 2016), Multilingual Index (M- Index) (Barnett et al., 2000), Probability of Switching (I- Index) (Guzman et al., 2017), Burstiness (Goh and Barabasi, 2008), and Complexity Factor (CF1- 3) (Ghosh et al., 2017). Table \ref{tab:metrics} shows the value of each metric for our genuine CSw dataset, as well as for these 3 synthetic CSw datasets. We can see that the LLM prompting- based dataset was superior in its similarity to the authentic CSw data. Using this method, we generated a corpus of 73,293 utterances covering over 20 English language pairs, including the 6 language pairs in the original dataset.7

\subsubsection{Step 2: Synthetic Error Generation}
text[[115, 682, 486, 857], [511, 236, 690, 251]]
Several works have shown the effectiveness of rule- based error injection for GEC data generation. Many use the PIE- synthetic dataset (Awasthi et al., 2019), a perturbed version of the 1BW corpus (Chelba et al., 2013). For each sentence, the authors introduce between 0 and 4 errors of random type. We extended this work by introducing a new subset of error types that are not only more common in ESL learners, but also are areas where the SOTA performance collapses when faced with CSw text: noun, pronoun, word order, determiner, and punctuation errors.8

To increase the diversity of errors, we adopted a second style of error injection, Backtranslation (Stahlberg and Kumar, 2021). By swapping the source and target sentences of a monolingual dataset, we trained a GECToR- based system to induce errors in our synthetic CSw sentences.

Using these methods, we created two datasets: Syn- CSw PIE and Syn- CSw Rev- GECToR. After removing pairs containing no injected errors, we are left with 70,180 and 18,159 sentences each.

\section{CSw GEC Systems}
For our GEC system targeting CSw texts, we chose a GECToR model (Omelianchuk et al., 2020), with a RoBERTa- base foundation, due to its proven efficacy with limited training data and stronger performance on CSw texts compared to seq2seq models. We added a new CSw class to the error detection head, adding the ability to detect CSw tokens.

Following Tarnavskyi et al. (2022), we used a 3- stage training schedule. In the first, we used the same distilled 1BW corpus, and added all our synthetic CSw GEC data. For the second, we used several GEC datasets: NUCLE (Dahlmeier et al., 2023), FCE (Yannakoudakis et al., 2011), W\&I Locusness (Bryant et al., 2019), Lang- 8 and our 2 synthetic CSw datasets. As our genuine CSw dataset is a subset of the Lang- 8 corpus, we checked and removed any duplicates. Following previous works, we finished training using the W\&I Locusness dataset due to its superior quality. In this final stage, we added a sampled subset of our synthetic CSw sentences and \(90\%\) of our authentic CSw data, ensuring exposure to synthetic and genuine CSw text. At each stage, we reserved \(5\%\) of the data for validation. Finally, we tuned inference parameters using a grid- search to optimise the \(F_{0.5}\) on the final validation set. By beginning with pre

\begin{table}[h]
\centering
\caption{ERRANT- based Precision, Recall and \(F_{0.5}\) Scores of Baselines and Our Model Throughout Training}
\label{tab:errant}
\begin{tabular}{llccc}
\toprule
Model & Dataset & Precision & Recall & $F_{0.5}$ \\
\midrule
Baseline GECToR & BEA-2019 & [MISSING] & [MISSING] & [MISSING] \\
Baseline GECToR & CSW (10\%) & [MISSING] & [MISSING] & [MISSING] \\
Baseline T5 & BEA-2019 & [MISSING] & [MISSING] & [MISSING] \\
Baseline T5 & CSW (10\%) & [MISSING] & [MISSING] & [MISSING] \\
Our Model Stage 1 & BEA-2019 & [MISSING] & [MISSING] & [MISSING] \\
Our Model Stage 1 & CSW (10\%) & [MISSING] & [MISSING] & [MISSING] \\
Our Model Stage 2 & BEA-2019 & [MISSING] & [MISSING] & [MISSING] \\
Our Model Stage 2 & CSW (10\%) & [MISSING] & [MISSING] & [MISSING] \\
Our Model Stage 3 & BEA-2019 & [MISSING] & [MISSING] & [MISSING] \\
Our Model Stage 3 & CSW (10\%) & [MISSING] & [MISSING] & [MISSING] \\
\bottomrule
\end{tabular}
\end{table}

training on large amounts of lower- quality data in the early stages, this multi- stage learning process allows the model to first build a robust GEC foundation before refining it with high quality data in the latter stages. This approach allows the model to learn incrementally, reducing the risk of the model being overwhelmed by the complexity of the task from the outset.

\section{Results and Analysis}
\subsection{Baseline Comparisons}
We compared our model against two wellestablished systems: a RoBERTa- base GECToR model (Omelianchuk et al., 2020), with near SOTA performance on the BEA- 2019 test set (Bryant et al., 2019), and a seq2seq T5 model (Rothe et al., 2021). To assess these models, we evaluated their performance on the BEA- 2019 test set and the remaining \(10\%\) of our authentic CSW data. The ERRANT (Bryant et al., 2017) GEC evaluation results, as outlined in Table \ref{tab:errant}, demonstrate a clear degradation in performance when these two systems are applied to CSW texts. The ERRANT toolkit detects and classifies edits between source and target sentence pairs into predefined error categories. It enables the comparison of a proposed set of edits with a reference set, providing a way of calculating metrics, such as precision and recall, across these categories.

\subsection{Detailed Model Performance}
text[[115, 775, 486, 917], [510, 259, 881, 403]]
The progression of our model throughout training provided insights into its evolving capabilities and effectiveness of our synthetic data. We monitored several metrics, including the ERRANT precision, recall and \(F_{0.5}\) score, for the BEA- 2019 test set and the remaining unused \(10\%\) of our genuine CSW dataset. These metrics, as displayed in Table \ref{tab:errant}, indicate a steady improvement in the ability to handle CSW texts. Notably, the performance on the CSW dataset shows a significant leap in the final stages, where the contribution of our synthetic dataset is largest. This improvement in CSW text handling did slightly compromise the model's performance on monolingual GEC tasks, as seen on the BEA- 2019 test set. This suggests a trade- off inherent in specialising the model for CSW contexts. However, our model remains competitive amongst SOTA monolingual GEC systems of its size.

Three illustrative examples of our model's corrections, taken from the CSW test set, can be seen in Figure \ref{fig:examples}. The first example demonstrates a case where the model has correctly identified all of the changes required, including the incorrect capitalisation of a word, a missing word, and some missing punctuation. The second example shows a "near miss"; here, the model has correctly identified the majority of the changes required but dropped the "I" whilst rearranging the start of the sentence. Finally, the third example presents a scenario where the model has fallen slightly short, failing to recognise the need for "were" instead of "was" in this hypothetical context.

\begin{figure}[h]
\centering
\fbox{\parbox{\linewidth}{
Gold Correction 1: We have many [New $\rightarrow$new] words for [$\emptyset$$\rightarrow$the] unemployed[$\emptyset$:] "이태백"[$\emptyset$,] "백수"[$\emptyset$,] "백조"\\
Proposed Correction 1: We have many [New $\rightarrow$new] words for [$\emptyset$$\rightarrow$the] unemployed[$\emptyset$:] "이태백"[$\emptyset$,] "백수"[$\emptyset$,] "백조"\\
\vspace{5mm}
Gold Correction 2: [I and my girlfriend $\rightarrow$My girlfriend and I] looked [$\emptyset$$\rightarrow$at a] picture called "無原罪の聖母" (Immaculate Conception).\\
Proposed Correction 2: [I and my girlfriend $\rightarrow$My girlfriend and] looked [$\emptyset$$\rightarrow$at a] picture called "無原罪の聖母" (Immaculate Conception).\\
\vspace{5mm}
Gold Correction 3: If he [was a $\rightarrow$were] Japanese, I suppose I [replied $\rightarrow$would reply] like this: "ああ、駅ならこの道を真っすぐ行けばすぐですよ、800mくらい先です".\\
Proposed Correction 3: If he was [a $\rightarrow$$\emptyset$] Japanese, I suppose I [replied $\rightarrow$would reply] like this: "ああ、駅ならこの道を真っすぐ行けばすぐですよ、800mくらい先です".
}}
\caption{Three examples of model’s proposed corrections from the CSW test set.}
\label{fig:examples}
\end{figure}

\subsection{Inference Tweaking and Error Thresholds}
The inference tweaking phase was crucial in tuning the balance between precision and recall. The changes made here, particularly lowering the minimum error thresholds before the model makes an edit, indicated a clear attempt to force the model to make more corrections.10 While this slightly lowered precision on monolingual errors, it significantly enhanced the performance on CSW text.

To determine that the improved performance of our proposed model was not entirely due to the different inference configuration, we conducted a similar grid search for the existing GECToR model. However, instead of using the Stage 3 validation dataset, as we did with our model, we used the CSW test set directly. The highest F0.5 achieved by the baseline model was 56.46, providing evidence that our proposed model beats all inference configurations of the previous GECToR system when applied to CSW texts.

\subsection{Synthetic Data Impact}
The synthetic CSW text and error injection methods were central to this project. The resemblance of our synthetic text to real ESL learner data, as shown by the similarity metrics in Table \ref{tab:metrics}, is a testament to the effectiveness of our chosen generation method. The improvements in F0.5 scores provide further evidence of this.

Our extended PIE-synthetic dataset aimed to introduce four error types common in ESL students: noun, pronoun, punctuation and word errors. When compared to the monolingual GECToR, our model is stronger in all of these areas.11 This provides strong evidence that the targeted approach to error injection was successful in boosting the model’s ability in these areas.

\section{Conclusion}
The primary aim of this paper was to build a GEC system capable of effectively correcting English errors in CSW text, whilst maintaining competitive performance on monolingual data. To address the scarcity of CSW data, we explored methods of generating synthetic CSW text. We used several CSW metrics to establish that the LLM prompting- based approach was the most capable of generating text resembling the content in our genuine dataset. From there, we used two error injection methods to create the first substantial datasets labelled for CSW GEC. This significantly expanded the training data available. Importantly, it also opened up opportunities for future research in CSW GEC and CSW NLP more generally. We demonstrated the efficacy of our synthetic data generation techniques by training the first GEC model aimed at correcting errors in CSW texts. Our model showed a clear improvement in performance on CSW data, surpassing the SOTA in this area.

\section{Limitations}
This research, while comprehensive, encounters several limitations that highlight areas for potential improvement. One primary limitation lies in the overrepresentation of Japanese in the genuine CSW dataset. This raises questions about the model’s applicability to a broader range of language pairs. This is an unfortunate consequence of using a dataset sourced from Lang-8, a Japanese language learning network. Although our method demonstrated that it could generate texts from a wider range of language pairs, it is possible that all CSW data used shows a bias towards Japanese styles of CSW. Such a bias in our system could inadvertently lead to reduced accessibility and effectiveness for ESL learners who CSW with languages other than Japanese. If this system were to be used as an aid in ESL education, steps should be taken to ensure that it does not contribute to existing inequalities present in language learning platforms. English learning tools should be accessible regardless of the student’s native language and future work should focus on developing more inclusive datasets to help mitigate these risks.

Another possible limitation relates to the style of model chosen. The sequence tagging method was selected due to its lower data requirements, but this decision may have constrained the capabilities of the model. Many of the errors typical of ESL students require complex restructuring of 16961

 the sentence - a notably difficult task for edit- based GEC systems. Although the data needs are more substantial, it is likely that NMT GEC systems may fare better as they are not constrained by a limited vocabulary of edits.

To assess the likeness of our generated CSw text, we introduced several common CSw metrics. Although useful, these metrics are not very sophisticated, and often struggle to accurately capture the nuances of CSw patterns across different subpopulations. These language patterns can have a substantial impact on the optimal approaches to problems across CSw NLP, and hence, the field would benefit from further research in this area. Ideally, we would have conducted a human study to evaluate the quality of our synthetic data. However, given the constraints of the project, it was not possible, and we acknowledge this as a limitation of our work.

Finally, we reported results for a RoBERTa- base GECToR system. Although we also tested other base models, including BERT, DeBERTa and ELECTRA, we did not look at larger models or ensemble systems. Future extensions could explore this area, building upon the observation that larger models or simple voting ensembles can yield better results than the smaller base models (Tarnavskyi et al., 2022).

In summary, whilst the current work makes significant contributions to the field of GEC for CSw text, these limitations indicate crucial areas for further research and development.

\section*{References}
\begin{thebibliography}{00}
\bibitem{Awasthi2019} Abhijeet Awasthi, Sunita Sarawagi, Rasna Goyal, Sabyasachi Ghosh, and Vihari Piratla. 2019. Parallel iterative edit models for local sequence transduction. In \emph{Proceedings of the 2019 Conference on Empirical Methods in Natural Language Processing and the 9th International Joint Conference on Natural Language Processing (EMNLP- IJCNLP)}, pages 4259- 4269, Hong Kong, China. Association for Computational Linguistics.

\bibitem{Barnett2000} Ruthanna Barnett, Eva Codo, Eva Eppler, Montse Forcadell, Penelope Gardner- Chlros, Roeland van Hout, Melissa Moyer, Maria Carme Torras, Maria Teresa Turell, Mark Sebba, Marianne Starren, and Sietse Wensing. 2000. The lides coding manual: A document for preparing and analyzing language interaction data version 1.1- july, 1999. \emph{International Journal of Bilingualism}, 4(2):131- 132.

\bibitem{BeattyMartinez2020} Anne L Beatty- Martinez, Christian A Navarro- Torres, and Paola E Dussias. 2020. Codeswitching: A bilingual toolkit for opportunistic speech planning. \emph{Frontiers in Psychology}, 11:1699.

\bibitem{Brown2020} Tom B. Brown, Benjamin Mann, Nick Ryder, Melanie Subbiah, Jared Kaplan, Prafulla Dhariwal, Arvind Neelakantan, Pranav Shyam, Girish Sastry, Amanda Askell, Sandhini Agarwal, Ariel Herbert- Voss, Gretchen Krueger, Tom Henighan, Rewon Child, Aditya Ramesh, Daniel M. Ziegler, Jeffrey Wu, Clemens Winter, Christopher Hesse, Mark Chen, Eric Sigler, Mateusz Litwin, Scott Gray, Benjamin Chess, Jack Clark, Christopher Berner, Sam McCandlish, Alec Radford, Ilya Sutskever, and Dario Amodei. 2020. Language models are few- shot learners. \emph{CoRR}, abs/2005.14165.

\bibitem{Bryant2019} Christopher Bryant, Mariano Felice, Oistein E. Andersen, and Ted Briscoe. 2019. The BEA- 2019 shared task on grammatical error correction. In \emph{Proceedings of the Fourteenth Workshop on Innovative Use of NLP for Building Educational Applications}, pages 52- 75, Florence, Italy. Association for Computational Linguistics.

\bibitem{Bryant2017} Christopher Bryant, Mariano Felice, and Ted Briscoe. 2017. Automatic annotation and evaluation of error types for grammatical error correction. In \emph{Proceedings of the 55th Annual Meeting of the Association for Computational Linguistics (Volume 1: Long Papers)}, pages 793- 805, Vancouver, Canada. Association for Computational Linguistics.

\bibitem{Chan2024} Kelvin Wey Han Chan, Christopher Bryant, Li Nguyen, Andrew Caines, and Zheng Yuan. 2024. Grammatical error correction for code- switched sentences by learners of English. In \emph{Proceedings of the 2024 Joint International Conference on Computational Linguistics, Language Resources and Evaluation (LRECCOLING 2024)}, pages 7926- 7938, Torino, Italia. ELRA and ICCL.

\bibitem{Chelba2013} Ciprian Chelba, Tomas Mikolov, Mike Schuster, Qi Ge, Thorsten Brants, and Phillipp Koehn. 2013. One billion word benchmark for measuring progress in statistical language modeling. \emph{CoRR}, abs/1312.3005.

\bibitem{Dahlmeier2023} Daniel Dahlmeier, Hwee Tou Ng, Siew Mei Wu, et al. 2023. Nus corpus of learner english (nucle). National University of Singapore, NLP Group. Available: https://www.comp.nus.edu.sg/nlp/corpora.html.

\bibitem{Dou2021} Zi- Yi Dou and Graham Neubig. 2021. Word alignment by fine- tuning embeddings on parallel corpora. In \emph{Proceedings of the 16th Conference of the European Chapter of the Association for Computational Linguistics: Main Volume}, pages 2112- 2128, Online. Association for Computational Linguistics.

\bibitem{Falbo2021} Arianna Falbo and Travis LaCroix. 2021. Est- ce que vous compute? code- switching, cultural identity, and AI. \emph{CoRR}, abs/2112.08256.

\bibitem{Finlay2023} P.J. Finlay. 2023. Argos translate. Open- source offline translation library written in Python.

\bibitem{Gamback2016} Bjorn Gambäck and Amitava Das. 2016. Comparing the level of code- switching in corpora. In \emph{Proceedings of the Tenth International Conference on Language Resources and Evaluation (LREC'16)}, pages 1850- 1855, Portoroz, Slovenia. European Language Resources Association (ELRA).

\bibitem{Ghosh2017} Souvick Ghosh, Satanu Ghosh, and Dipankar Das. 2017. Complexity metric for code- mixed social media text. \emph{Computation $\gamma$ Sistemas}, 21(4):693- 701.

\bibitem{Goh2008} K.- I. Goh and A.- L. Barabasi. 2008. Burstiness and memory in complex systems. \emph{Europhysics Letters}, 81(4):48002.

\bibitem{Guzman2017} Gualberto Guzman, Joseph Ricard, Jacqueline Serigos, Barbara E. Bullock, and Almeida Jacqueline Toribio. 2017. Metrics for Modeling Code- Switching Across Corpora. In \emph{Proc. Interspeech 2017}, pages 67- 71.

\bibitem{Klein2017} Guillaume Klein, Yoon Kim, Yuntian Deng, Jean Senellart, and Alexander M. Rush. 2017. OpenNMT: Open- source toolkit for neural machine translation. In \emph{Proceedings of the 55th Annual Meeting of the Association for Computational Linguistics (ACL 2017), System Demonstrations}, pages 67- 72, Vancouver, Canada. Association for Computational Linguistics.

\bibitem{Koehn2012} Philipp Koehn et al. 2012. European parliament proceedings parallel corpus.

\bibitem{Manning2014} Christopher Manning, Mihai Surdeanu, John Bauer, Jenny Finkel, Steven Bethard, and David McClosky. 2014. The Stanford CoreNLP natural language processing toolkit. In \emph{Proceedings of 52nd Annual Meeting of the Association for Computational Linguistics: System Demonstrations}, pages 55- 60, Baltimore, Maryland. Association for Computational Linguistics.

\bibitem{Mizumoto2013} Tomoya Mizumoto, Toshikazu Tajiri, Takuya Fujino, Seiji Kasahara, Mamoru Komachi, Masaaki Nagata, and Yuji Matsumoto. 2013. NAIST Lang- 8 Learner Corpora. Language Learner Corpora compiled from Lang- 8 SNS. Available for research and educational purposes.

\bibitem{Nguyen2022} Li Nguyen, Zheng Yuan, and Graham Seed. 2022. Building educational technologies for code- switching: Current practices, difficulties and future directions. \emph{Languages}, 7:220.

\bibitem{Omelianchuk2020} Kostiantyn Omelianchuk, Vitaliy Atrashevskyi, Artem Chernodub, and Oleksandr Skurzhanskyi. 2020. GECToR - grammatical error correction: Tag, not rewrite. In \emph{Proceedings of the Fifteenth Workshop on Innovative Use of NLP for Building Educational Applications}, pages 163- 170, Seattle, WA, USA $\rightarrow$ Online. Association for Computational Linguistics.

\bibitem{Rizvi2021} Mohd Sanad Zaki Rizvi, Anirudh Srinivasan, Tanuja Ganu, Monojit Choudhury, and Sunayana Sitaram. 2021. GCM: A toolkit for generating synthetic code- mixed text. In \emph{Proceedings of the 16th Conference of the European Chapter of the Association for Computational Linguistics: System Demonstrations}, pages [MISSING].

\bibitem{Rothe2021} Sascha Rothe, Jonathan Mallinson, Eric Malmi, Sebastian Krause, and Aliaksei Severyn. 2021. A simple recipe for multilingual grammatical error correction. In \emph{Proceedings of the 59th Annual Meeting of the Association for Computational Linguistics and the 11th International Joint Conference on Natural Language Processing (Volume 2: Short Papers)}, pages 702- 707, Online. Association for Computational Linguistics.

\bibitem{Stahlberg2021} Felix Stahlberg and Shankar Kumar. 2021. Synthetic data generation for grammatical error correction with tagged corruption models. In \emph{Proceedings of the 16th Workshop on Innovative Use of NLP for Building Educational Applications}, pages 37- 47, Online. Association for Computational Linguistics.

\bibitem{Tarnavskyi2022} Maksym Tarnavskyi, Artem Chernodub, and Kostiantyn Omelianchuk. 2022. Ensembling and knowledge distilling of large sequence taggers for grammatical error correction. In \emph{Accepted for publication at 60th Annual Meeting of the Association for Computational Linguistics (ACL 2022)}, Dublin, Ireland.

\bibitem{Yannakoudakis2011} Helen Yannakoudakis, Ted Briscoe, and Ben Medlock. 2011. A new dataset and method for automatically grading ESOL texts. In \emph{Proceedings of the 49th Annual Meeting of the Association for Computational Linguistics: Human Language Technologies}, pages 180- 189, Portland, Oregon, USA. Association for Computational Linguistics.

\bibitem{Yow2018} W. Quin Yow, Jessica S. H. Tan, and Suzanne Flynn. 2018. Code- switching as a marker of linguistic competence in bilingual children. \emph{Bilingualism: Language and Cognition}, 21(5):1075- 1090.
\end{thebibliography}

\appendix
\section{Example LLM Prompt}
Figure 3 shows an example LLM prompt used to generate synthetic CSW sentences from genuine examples. As we are using a private subset of the Lang- 8 dataset, we are not permitted to share any of the CSW texts.

\begin{figure}[h]
\centering
\fbox{\parbox{\linewidth}{
Settings: [no prose]\\
For each of the following code- switched sentences, generate a new sentence that uses the same two languages and a similar style of code- switching. The topic should be different. Ensure you use the correct grammar in the English portion of the sentence. Make sure that each sentence contains 2 languages. Only return the sentences and their number. You must follow all of the instructions.\\
For example, given the source sentence and label:\\
1. This food is called "一"An acceptable answer would be:1. This animal is called a "\\
Do not include any other information in the generated sentences. The 10 real examples are as follows:\\
1. [CSW SENTENCE]\\
2. [CSW SENTENCE]\\
3. [CSW SENTENCE]\\
4. [CSW SENTENCE]\\
5. [CSW SENTENCE]\\
6. [CSW SENTENCE]\\
7. [CSW SENTENCE]\\
8. [CSW SENTENCE]\\
9. [CSW SENTENCE]\\
10. [CSW SENTENCE]}}
\caption{Example LLM prompt.}
\label{fig:prompt}
\end{figure}

\section{Training Data Schedule}
In this section, we explicitly detail the data used at each stage of the training process.

Stage 1 For the initial pre- training stage, we used the distilled dataset proposed by the SOTA (Tarnavskyi et al., 2022). This dataset was constructed by extracting corrections from the monolingual 1BW corpus (Chelba et al., 2013) using the highest performing GECToR ensemble. Through this dataset, we shuffled our PIE- synthetic CSW dataset. We deemed this dataset to be of lower quality than its Rev- GECToR counterpart. Consequently, it was used earlier in the training process. This provided roughly 1,200,000 examples for the initial training phase of which we split between train and validation sets according to a ratio of 19:1. Our synthetic CSW sentences comprised approximately \(5.65\%\) of this dataset. We aimed to keep this percentage small in this phase of the training process to allow the model to first learn to correct errors in monolingual texts. In later stages, we boosted the contribution of the CSW data.

Stage 2 For the second stage, we shuffled several GEC datasets. These are NUCLE (Dahlmeier et al., 2023), FCE (Yannakoudakis et al., 2011), W\&I Locrness (Bryant et al., 2019), Lang- 8 (Mizumoto et al., 2013) and our 2 newly created CSW datasets. As our genuine CSW dataset is a subset of the private Lang- 8 corpus, we checked and removed any duplicates. Table \ref{tab:stage2} shows the overall contributions of each corpus towards the stage 2 dataset. Similar to the previous stage, the data was split into train and validation sets.

\begin{table}[h]
\centering
\caption{Sentence Count and Contribution of Stage 2 Datasets}
\label{tab:stage2}
\begin{tabular}{lr}
\toprule
Dataset & Sentences \\
\midrule
Lang-8 & 985,683 (80.54\%) \\
W\&I Locness & 68,608 (5.61\%) \\
NUCLE & 54,258 (4.43\%) \\
FCE & 26,929 (2.20\%) \\
Syn-CSW PIE & 70,181 (5.73\%) \\
Syn-CSW Rev-GECToR & 18,160 (1.48\%) \\
\hline
Total & 1,223,819 \\
\bottomrule
\end{tabular}
\end{table}

Stage 3 For the final stage, we combined the high quality W\&I Locrness dataset with a sampled subset of the genuine CSW data and a sampled subset of the synthetic CSW texts. Again, the stage 3 dataset is split into train and validation sets. The remaining unused subset of the genuine CSW dataset was retained for testing purposes. Table \ref{tab:stage3} details the contributions to this stage from each dataset.

\begin{table}[h]
\centering
\caption{Sentence Count and Contribution of Stage 3 Datasets}
\label{tab:stage3}
\begin{tabular}{lr}
\toprule
Dataset & Sentences \\
\midrule
W\&I Locness & 68,608 (67.23\%) \\
Syn-CSW Rev-GECToR & 18,160 (17.80\%) \\
Syn-CSW PIE & 10,000 (9.80\%) \\
CSW Genuine & 5,279 (5.17\%) \\
\hline
Total & 102,047 \\
\bottomrule
\end{tabular}
\end{table}

\section{Inference Hyperparameters}
After training our model, we used the validation dataset from stage 3 to tune 2 inference parameters. These are:

additional confidence This value is added to the probability of the SKEEP token. If this value is high, recall is likely to decrease and precision increase. The grid search found the best value of this to be 0.

min\_error\_probability For a change to be made to a sentence, the probability of at least one token in the sentence being an error must be higher than the min\_error\_probability. If this value is high, then precision is likely to be higher and recall lower. The grid search found the best value of this to be 0.4.

\section{Error Type Analysis of Proposed Model}
By exploring the ERRANT error classifications of our proposed model when applied to the CSW test dataset, we can further explore the effectiveness of our synthetic data in addressing the problematic areas identified in Appendix B. A breakdown of the precision, recall and \(\mathrm{F}_{0.5}\) score for each of the previously identified categories is shown in Table \ref{tab:error_analysis}.

\begin{table}[h]
\centering
\caption{Precision (P), Recall (R) and \(F_{0.5}\) Score of Our Proposed Model for Targeted Error Types in the Genuine CSW Test Dataset}
\label{tab:error_analysis}
\begin{tabular}{lccc}
\toprule
Category & P & R & F0.5 \\
\midrule
NOUN & 0.2857 & 0.0833 & 0.1923 \\
PRON & 0.7647 & 0.4643 & 0.6771 \\
PUNCT & 0.7143 & 0.1139 & 0.3460 \\
WO & 0.7778 & 0.2000 & 0.4930 \\
\bottomrule
\end{tabular}
\end{table}

\end{document}
=====END FILE=====


=====FILE: refs.bib=====
@inproceedings{Awasthi2019,
  title={Parallel iterative edit models for local sequence transduction},
  author={Awasthi, Abhijeet and Sarawagi, Sunita and Goyal, Rasna and Ghosh, Sabyasachi and Piratla, Vihari},
  booktitle={Proceedings of the 2019 Conference on Empirical Methods in Natural Language Processing and the 9th International Joint Conference on Natural Language Processing (EMNLP-IJCNLP)},
  pages={4259--4269},
  year={2019},
  address={Hong Kong, China},
  publisher={Association for Computational Linguistics}
}

@article{Barnett2000,
  title={The lides coding manual: A document for preparing and analyzing language interaction data version 1.1-july, 1999},
  author={Barnett, Ruthanna and Codo, Eva and Eppler, Eva and Forcadell, Montse and Gardner-Chlros, Penelope and van Hout, Roeland and Moyer, Melissa and Torras, Maria Carme and Turell, Maria Teresa and Sebba, Mark and Starren, Marianne and Wensing, Sietse},
  journal={International Journal of Bilingualism},
  volume={4},
  number={2},
  pages={131--132},
  year={2000}
}

@article{BeattyMartinez2020,
  title={Codeswitching: A bilingual toolkit for opportunistic speech planning},
  author={Beatty-Martinez, Anne L and Navarro-Torres, Christian A and Dussias, Paola E},
  journal={Frontiers in Psychology},
  volume={11},
  pages={1699},
  year={2020}
}

@article{Brown2020,
  title={Language models are few-shot learners},
  author={Brown, Tom B and Mann, Benjamin and Ryder, Nick and Subbiah, Melanie and Kaplan, Jared and Dhariwal, Prafulla and Neelakantan, Arvind and Shyam, Pranav and Sastry, Girish and Askell, Amanda and Agarwal, Sandhini and Herbert-Voss, Ariel and Krueger, Gretchen and Henighan, Tom and Child, Rewon and Ramesh, Aditya and Ziegler, Daniel M and Wu, Jeffrey and Winter, Clemens and Hesse, Christopher and Chen, Mark and Sigler, Eric and Litwin, Mateusz and Gray, Scott and Chess, Benjamin and Clark, Jack and Berner, Christopher and McCandlish, Sam and Radford, Alec and Sutskever, Ilya and Amodei, Dario},
  journal={CoRR},
  volume={abs/2005.14165},
  year={2020}
}

@inproceedings{Bryant2019,
  title={The BEA-2019 shared task on grammatical error correction},
  author={Bryant, Christopher and Felice, Mariano and Andersen, Oistein E. and Briscoe, Ted},
  booktitle={Proceedings of the Fourteenth Workshop on Innovative Use of NLP for Building Educational Applications},
  pages={52--75},
  year={2019},
  address={Florence, Italy},
  publisher={Association for Computational Linguistics}
}

@inproceedings{Bryant2017,
  title={Automatic annotation and evaluation of error types for grammatical error correction},
  author={Bryant, Christopher and Felice, Mariano and Briscoe, Ted},
  booktitle={Proceedings of the 55th Annual Meeting of the Association for Computational Linguistics (Volume 1: Long Papers)},
  pages={793--805},
  year={2017},
  address={Vancouver, Canada},
  publisher={Association for Computational Linguistics}
}

@inproceedings{Chan2024,
  title={Grammatical error correction for code-switched sentences by learners of English},
  author={Chan, Kelvin Wey Han and Bryant, Christopher and Nguyen, Li and Caines, Andrew and Yuan, Zheng},
  booktitle={Proceedings of the 2024 Joint International Conference on Computational Linguistics, Language Resources and Evaluation (LRECCOLING 2024)},
  pages={7926--7938},
  year={2024},
  address={Torino, Italia},
  publisher={ELRA and ICCL}
}

@article{Chelba2013,
  title={One billion word benchmark for measuring progress in statistical language modeling},
  author={Chelba, Ciprian and Mikolov, Tomas and Schuster, Mike and Ge, Qi and Brants, Thorsten and Koehn, Phillipp},
  journal={CoRR},
  volume={abs/1312.3005},
  year={2013}
}

@misc{Dahlmeier2023,
  title={Nus corpus of learner english (nucle)},
  author={Dahlmeier, Daniel and Ng, Hwee Tou and Wu, Siew Mei and others},
  year={2023},
  howpublished={National University of Singapore, NLP Group. Available: https://www.comp.nus.edu.sg/nlp/corpora.html}
}

@inproceedings{Dou2021,
  title={Word alignment by fine-tuning embeddings on parallel corpora},
  author={Dou, Zi-Yi and Neubig, Graham},
  booktitle={Proceedings of the 16th Conference of the European Chapter of the Association for Computational Linguistics: Main Volume},
  pages={2112--2128},
  year={2021},
  address={Online},
  publisher={Association for Computational Linguistics}
}

@article{Falbo2021,
  title={Est-ce que vous compute? code-switching, cultural identity, and AI},
  author={Falbo, Arianna and LaCroix, Travis},
  journal={CoRR},
  volume={abs/2112.08256},
  year={2021}
}

@misc{Finlay2023,
  title={Argos translate},
  author={Finlay, P.J.},
  year={2023},
  howpublished={Open-source offline translation library written in Python}
}

@inproceedings{Gamback2016,
  title={Comparing the level of code-switching in corpora},
  author={Gambäck, Björn and Das, Amitava},
  booktitle={Proceedings of the Tenth International Conference on Language Resources and Evaluation (LREC'16)},
  pages={1850--1855},
  year={2016},
  address={Portoroz, Slovenia},
  publisher={European Language Resources Association (ELRA)}
}

@article{Ghosh2017,
  title={Complexity metric for code-mixed social media text},
  author={Ghosh, Souvick and Ghosh, Satanu and Das, Dipankar},
  journal={Computation $\gamma$ Sistemas},
  volume={21},
  number={4},
  pages={693--701},
  year={2017}
}

@article{Goh2008,
  title={Burstiness and memory in complex systems},
  author={Goh, K.-I. and Barabasi, A.-L.},
  journal={Europhysics Letters},
  volume={81},
  number={4},
  pages={48002},
  year={2008}
}

@inproceedings{Guzman2017,
  title={Metrics for Modeling Code-Switching Across Corpora},
  author={Guzman, Gualberto and Ricard, Joseph and Serigos, Jacqueline and Bullock, Barbara E. and Toribio, Almeida Jacqueline},
  booktitle={Proc. Interspeech 2017},
  pages={67--71},
  year={2017}
}

@inproceedings{Klein2017,
  title={OpenNMT: Open-source toolkit for neural machine translation},
  author={Klein, Guillaume and Kim, Yoon and Deng, Yuntian and Senellart, Jean and Rush, Alexander M.},
  booktitle={Proceedings of the 55th Annual Meeting of the Association for Computational Linguistics (ACL 2017), System Demonstrations},
  pages={67--72},
  year={2017},
  address={Vancouver, Canada},
  publisher={Association for Computational Linguistics}
}

@misc{Koehn2012,
  title={European parliament proceedings parallel corpus},
  author={Koehn, Philipp and others},
  year={2012}
}

@inproceedings{Manning2014,
  title={The Stanford CoreNLP natural language processing toolkit},
  author={Manning, Christopher and Surdeanu, Mihai and Bauer, John and Finkel, Jenny and Bethard, Steven and McClosky, David},
  booktitle={Proceedings of 52nd Annual Meeting of the Association for Computational Linguistics: System Demonstrations},
  pages={55--60},
  year={2014},
  address={Baltimore, Maryland},
  publisher={Association for Computational Linguistics}
}

@misc{Mizumoto2013,
  title={NAIST Lang-8 Learner Corpora},
  author={Mizumoto, Tomoya and Tajiri, Toshikazu and Fujino, Takuya and Kasahara, Seiji and Komachi, Mamoru and Nagata, Masaaki and Matsumoto, Yuji},
  year={2013},
  howpublished={Language Learner Corpora compiled from Lang-8 SNS. Available for research and educational purposes}
}

@article{Nguyen2022,
  title={Building educational technologies for code-switching: Current practices, difficulties and future directions},
  author={Nguyen, Li and Yuan, Zheng and Seed, Graham},
  journal={Languages},
  volume={7},
  pages={220},
  year={2022}
}

@inproceedings{Omelianchuk2020,
  title={GECToR - grammatical error correction: Tag, not rewrite},
  author={Omelianchuk, Kostiantyn and Atrashevskyi, Vitaliy and Chernodub, Artem and Skurzhanskyi, Oleksandr},
  booktitle={Proceedings of the Fifteenth Workshop on Innovative Use of NLP for Building Educational Applications},
  pages={163--170},
  year={2020},
  address={Seattle, WA, USA $\rightarrow$ Online},
  publisher={Association for Computational Linguistics}
}

@inproceedings{Rizvi2021,
  title={GCM: A toolkit for generating synthetic code-mixed text},
  author={Rizvi, Mohd Sanad Zaki and Srinivasan, Anirudh and Ganu, Tanuja and Choudhury, Monojit and Sitaram, Sunayana},
  booktitle={Proceedings of the 16th Conference of the European Chapter of the Association for Computational Linguistics: System Demonstrations},
  pages={[MISSING]},
  year={2021}
}

@inproceedings{Rothe2021,
  title={A simple recipe for multilingual grammatical error correction},
  author={Rothe, Sascha and Mallinson, Jonathan and Malmi, Eric and Krause, Sebastian and Severyn, Aliaksei},
  booktitle={Proceedings of the 59th Annual Meeting of the Association for Computational Linguistics and the 11th International Joint Conference on Natural Language Processing (Volume 2: Short Papers)},
  pages={702--707},
  year={2021},
  address={Online},
  publisher={Association for Computational Linguistics}
}

@inproceedings{Stahlberg2021,
  title={Synthetic data generation for grammatical error correction with tagged corruption models},
  author={Stahlberg, Felix and Kumar, Shankar},
  booktitle={Proceedings of the 16th Workshop on Innovative Use of NLP for Building Educational Applications},
  pages={37--47},
  year={2021},
  address={Online},
  publisher={Association for Computational Linguistics}
}

@inproceedings{Tarnavskyi2022,
  title={Ensembling and knowledge distilling of large sequence taggers for grammatical error correction},
  author={Tarnavskyi, Maksym and Chernodub, Artem and Omelianchuk, Kostiantyn},
  booktitle={Accepted for publication at 60th Annual Meeting of the Association for Computational Linguistics (ACL 2022)},
  year={2022},
  address={Dublin, Ireland}
}

@inproceedings{Yannakoudakis2011,
  title={A new dataset and method for automatically grading ESOL texts},
  author={Yannakoudakis, Helen and Briscoe, Ted and Medlock, Ben},
  booktitle={Proceedings of the 49th Annual Meeting of the Association for Computational Linguistics: Human Language Technologies},
  pages={180--189},
  year={2011},
  address={Portland, Oregon, USA},
  publisher={Association for Computational Linguistics}
}

@article{Yow2018,
  title={Code-switching as a marker of linguistic competence in bilingual children},
  author={Yow, W. Quin and Tan, Jessica S. H. and Flynn, Suzanne},
  journal={Bilingualism: Language and Cognition},
  volume={21},
  number={5},
  pages={1075--1090},
  year={2018}
}
=====END FILE=====


=====FILE: figures/README.txt=====
[ILLEGIBLE] This directory would contain figure images if they were provided in the source PDF. The LaTeX code uses \fbox environments with the caption text to denote missing figures.
=====END FILE=====