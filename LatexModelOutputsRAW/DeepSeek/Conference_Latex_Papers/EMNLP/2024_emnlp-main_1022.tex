=====FILE: main.tex=====
\documentclass[10pt,twocolumn]{article}

\usepackage[utf8]{inputenc}
\usepackage[T1]{fontenc}
\usepackage{amsmath, amssymb}
\usepackage{graphicx}
\usepackage{xcolor}
\usepackage{booktabs}
\usepackage{multirow}
\usepackage{array}
\usepackage{url}
\usepackage{hyperref}
\usepackage{listings}
\usepackage{caption}
\usepackage{enumitem}
\usepackage{setspace}
\usepackage[font=small,labelfont=bf]{caption}
\usepackage{balance}
\usepackage{lipsum}

\setlength{\parskip}{0.5em}
\setlength{\parindent}{0pt}
\setlength{\emergencystretch}{3em}
\hypersetup{
    colorlinks=true,
    linkcolor=blue,
    filecolor=magenta,
    urlcolor=cyan,
}

\title{GuardBench: A Large-Scale Benchmark for Guardrail Models}
\author{
    Elias Bassani \\
    European Commission Joint Research Centre \\
    Ispra, Italy \\
    \texttt{elias.bassani@ec.europa.eu}
    \and
    Ignacio Sanchez \\
    European Commission Joint Research Centre \\
    Ispra, Italy \\
    \texttt{ignacio.sanchez@ec.europa.eu}
}
\date{}

\begin{document}

\maketitle

\begin{abstract}
Generative AI systems powered by Large Language Models have become increasingly popular in recent years. Lately, due to the risk of providing users with unsafe information, the adoption of those systems in safety-critical domains has raised significant concerns. To respond to this situation, input-output filters, commonly called guardrail models, have been proposed to complement other measures, such as model alignment. Unfortunately, the lack of a standard benchmark for guardrail models poses significant evaluation issues and makes it hard to compare results across scientific publications. To fill this gap, we introduce GuardBench, a large-scale benchmark for guardrail models comprising 40 safety evaluation datasets. To facilitate the adoption of GuardBench, we release a Python library providing an automated evaluation pipeline built on top of it. With our benchmark, we also share the first large-scale prompt moderation datasets in German, French, Italian, and Spanish. To assess the current state-of-the-art, we conduct an extensive comparison of recent guardrail models and show that a general-purpose instruction-following model of comparable size achieves competitive results without the need for specific fine-tuning.\footnote{\url{https://github.com/AmenRa/guardbench}}
\end{abstract}

\section{Introduction}
In the recent years, Generative AI systems have become increasingly popular thanks to the advanced capabilities of Large Language Models (LLMs) \cite{OpenAI2023}. Those systems are in the process of being deployed in a range of high-risk and safety-critical domains such as healthcare \cite{MeskoTopol2023, ZhangBoulos2023}, education \cite{BaidooAnuAnsah2023, Qadir2023}, and finance \cite{Chen2023}. As AI systems advance and are more extensively integrated into various application domain, it is crucial to ensure that their usage is secure, responsible, and compliant with the applicable AI safety regulatory framework.

Particular attention has been paid to chatbot systems based on LLMs, as they can potentially engage in unsafe conversations or provide users with information that may harm their well-being. Despite significant efforts in aligning LLMs to human values \cite{Wang2023b}, users can still misuse them to produce hate speech, spam, and harmful content, including racist, sexist, and other damaging associations that might be present in their training data \cite{Wei2023}. To alleviate this situation, explicit safeguards, such as input-output filters, are becoming fundamental requirements for safely deploying systems based on LLMs, complementing other measures such as model alignment.

Very recently, researchers have proposed the adoption of the so-called guardrail models to moderate user prompts and LLM-generated responses \cite{Inan2023, Ghosh2024, Li2024}. Given the importance of those models, their evaluation plays a crucial role in the Generative AI landscape. Despite the availability of a few datasets for assessing guardrail models capabilities, such as the OpenAI Moderation Dataset \cite{Markov2023} and BeaverTails \cite{Ji2023}, we think there is still need for a large-scale benchmark that allows for a more systematic evaluation.

We aim to fill this gap by providing the scientific community with a large-scale benchmark comprising several datasets for prompts and responses safety classification. To facilitate the adoption of our proposal, we release a Python library that provides an automated evaluation pipeline built on top of the benchmark itself. Moreover, we share the first large-scale multi-lingual prompt moderation datasets, thus overcoming English-only evaluation. Finally, we conduct the first extensive comparison of recent guardrail models, aiming at shedding some light on the state-of-the-art and show a general-purpose instruction-following model of comparable size achieves competitive results without the need for specific fine-tuning.

\section{Related Work}
In this section, we discuss previous work related to our benchmark. Firstly, we discuss the moderation of user-generated content. Secondly, we introduce the moderation of human-AI conversations.

\subsection{Moderation of User-Generated Content.}
The most related task to the one of our benchmark is the moderation of user-generated content, which has received significant attention in the past decade. Many datasets for the evaluation of moderation models have been proposed by gathering user-generated content from social networks and online forums, such as Twitter, Reddit, and others \cite{Basile2019, Kennedy2022, Davidson2017, ElSherief2021, Kennedy2020, Zampieri2019, Guest2021, GrimmingerKlinger2021, Sap2020, deGibert2018}. However, the task of moderating human-AI conversations is different in nature to that of moderating user-generated content. First, the texts produced in human-AI conversations differ from that generated by users on online social platforms. Second, LLM-generated content further differs from that generated by users in style and length \cite{Herbold2023, Gao2023}. Finally, the type of unsafe content in content moderation datasets is typically limited to hate and discrimination, while the unsafe content potentially present in human-AI conversation is much broader, ranging from weapons usage to cybersecurity attacks and self-harm \cite{Inan2023}.

\subsection{Moderation of Human-AI Conversations.}
The moderation of human-AI conversations comprises both the moderation of human-generated and LLM-generated content. In this context, users ask questions and give instructions to LLMs, which answer the user input. Unfortunately, LLMs may engage in offensive conversations \cite{Lee2019, CurryRieser2018} or generate unsafe content in response to the user requests \cite{Dinan2019}. To moderate such conversations, guardrail models have recently been proposed \cite{Inan2023, Ghosh2024, Li2024}, aiming to enforce safety in conversational AI systems or evaluate it before deployment \cite{Vidgen2024, Li2024}. Our work focus on both the moderation of user prompts and LLM responses. Specifically, we collect and extend several datasets related to LLM safety, providing the scientific community with a large-scale benchmark for the evaluation of guardrail models.

\section{Benchmark Composition}
In this section, we introduce the benchmark we have built by collecting several datasets from previous works and extending them through data augmentation. To decide which datasets to include in our evaluation benchmark, we first conducted a literature review and consulted SafetyPrompts\footnote{\url{https://safetyprompts.com}} \cite{Rottger2024}. We considered over 100 datasets related to LLM safety. To narrow down the initial list of datasets and identify those best suited for our evaluation purposes, we defined inclusion and exclusion criteria, which we present in Section 3.1. As many of these datasets were not proposed to evaluate guardrail models, we repurposed them to our needs as they already contained safety information. We include 35 datasets from previous works in our benchmark, which can be broadly categorized as prompts (instructions, question, and statements) or conversations (single-turn and multi-turn), where the object to be moderated is the final utterance. Due to the lack of non-English datasets \cite{Rottger2024}, we augmented those available through automatic translation, providing the scientific community with the first prompts safety evaluation sets for guardrail models in German, French, Italian, and Spanish. We detail such process in Section 3.3. Finally, as described in Section 3.4, we generate safe and unsafe responses to unsafe questions and instructions, building a new conversational dataset.

\subsection{Inclusion and Exclusion Criteria}
In this section, we introduce inclusion and exclusion criteria adopted for selecting safety datasets.

We include datasets comprising text chat between users and AI assistants, open-ended questions and instructions, and other texts that can be expressed in a prompt format. We include datasets with safety labels that resembles or fall within generally acknowledged harm categories \cite{Vidgen2024}. We include public datasets available on GitHub and HuggingFace's Datasets \cite{Lhoest2021}. We include datasets with permissive licenses, such as MIT, CC BY-NC, and Apache 2.0. Due to the lack of non-English datasets \cite{Rottger2024}, we initially consider only datasets in English. We exclude content moderation datasets from social networks and online forums. As explained in Section 2.1, their content differ from both user prompts and LLM responses. We exclude safety evaluation datasets that cannot be straightforwardly repurposed for the evaluation of guardrail models, such as multi-choice datasets \cite{Zhang2023} and completion datasets \cite{Gehman2020}. We exclude datasets whose samples' safety labels were computed by automated tools (e.g., Perspective API, OpenAI Moderation API), such as RealToxicityPrompts \cite{Gehman2020}, LMSYS-Chat-1M \cite{Zheng2023}, and the toxicity dataset comprised in DecodingTrust \cite{Wang2023a}. We exclude datasets that need to be built from scratch, such as AdvPromptSet \cite{Esiobu2023} or protected by password, such as FairPrism \cite{Fleisig2023}. We exclude datasets for jail-breaking and adversarial robustness evaluation, as jailbreaking and adversarial attacks are not the main focus of our work. However, we do include the unsafe prompts contained in those datasets (without jail-breaking or adversarial texts) as they are relevant to our work.

\subsection{Classification Task}
For our benchmark, we consider the safe/unsafe binary classification task for the following reasons. Firstly, due to the lack of a generally accepted taxonomy of unsafe content \cite{Vidgen2024} and differences in the labeling procedures of previous works, we are unable to map the unsafe content categories of every dataset to a reference taxonomy. Secondly, several datasets lack this information and only provide implicit safety categorization of the shared samples, i.e., they are all unsafe by construction. Therefore, we binarize the labels of the available datasets into safe/unsafe. By inspecting previous works' categories of harm, we ensure that all the datasets' unsafe samples fall within generally acknowledged harm categories, such as hate, discrimination, violence, weapons, adult content, child exploitation, suicide, self-harm, and others. Despite specific labeling differences, we find all the selected datasets to adhere to a shared safe/unsafe distinction, corroborating our design choice. Appendix A.1 details the label conversion process for each of the chosen datasets.

\subsection{Multilingual Augmentation}
As reported by Rottger et al. \cite{Rottger2024}, there is a lack non-English datasets for LLM safety evaluation. To overcome this limitation and conduct preliminary experiments with guardrail models on non-English texts, we translate the datasets of prompts in our benchmark to several languages. Specifically, by relying on Google's MADLAD-400-3B-MT \cite{Kudugunta2023}, we translate 31k prompts into German, French, Italian, and Spanish. To ensure the quality of the translations, we asked native speakers to evaluate four prompts from each translated dataset ($\sim 100$ prompts per language) and score them on a five-point Likert scale \cite{Likert1932} where one means that the translation is wrong and five means that the translation is perfect. Our annotators judged that the average translation quality exceed four points. We add the obtained datasets to GuardBench as PromptsDE, PromptsFR, PromptsIT, and PromptsES. The list of datasets used to derive our multi-lingual datasets is available in Appendix A.2.

\subsection{Answering Unsafe Prompts}
Given the number of (unanswered) unsafe questions and instructions from previous works, we propose a novel single-turn conversational dataset built by generating responses with a publicly available uncensored model. Specifically, by controlling the model's system prompt, we generate 22k safe and unsafe responses to the available unsafe questions and instructions. A system prompt is a way to provide context, instructions, and guidelines to the model before prompting it. Using a system prompt, we can set the role, personality, tone, and other relevant information that helps the model behave as expected, thus allowing us to control the generation of safe and unsafe responses. In the case of safe responses, we also inform the model that the requests to answer are from malicious users and instruct the model to provide helpful and pro-social responses \cite{Kim2022}. This way, we limit refusals and ensure the model does not provide unsafe information when we do not want it to do so. To ensure response quality, we manually checked a sample of the produced answers, finding that the employed model was surprisingly good at generating the expected answers. We add the obtained dataset to our benchmark under the name of UnsafeQA. The list of datasets used to derive UnsafeQA is available in Appendix A.2.

\subsection{Software Library}
GuardBench is accompanied by a Python library with the same name that we hope will facilitate the adoption of our benchmark as a standard for guardrail models evaluation. The main design principles behind the implementation of our Python library are as follows: 1) reproducibility, 2) usability, 3) automation, and 4) extendability. As exemplified in Listing 1, the library provides a predefined evaluation pipeline that only requires the user to provide a moderation function. The library automatically downloads the requested datasets from the original repositories, converts them in a standardized format, moderates prompts and conversations with the moderation function provided by the user, and ultimately saves the moderation outcomes in the specified output directory for later inspections. This way, users can focus on their own moderation approaches without having to worry about the evaluation procedure. Moreover, by sharing models' weights and moderation functions, guardrail models evaluation can be easily reproduced across research labs, thus improving research transparency. To this extend, our Python library also offers the possibility of building comparison tables and export them in \LaTeX, ready for scientific publications. Finally, the user can import new datasets to extend those available out-of-the-box. Further information and tutorials are available on GuardBench's official repository. We also release the code to reproduce the evaluation presented in Sections 4 and 5.

\begin{lstlisting}[caption={Example usage of the GuardBench Python library.}, label=lst:guardbench, language=Python, basicstyle=\small\ttfamily]
from guardbench import benchmark

benchmark(
    # Moderation function provided by the user.
    moderate,
    model_name = "moderator",
    out_dir = "results",
    batch_size = 32,
    datasets = "all",
)
\end{lstlisting}

\section{Evaluation Setup}
To answer the research questions RQ1 and RQ2 we compare the effectiveness of several models at classifying prompts and conversation utterances as safe or unsafe. Then, to answer RQ3, we compare the models on our newly introduced multi-lingual prompt datasets, described in Section 3.3. Finally, we evaluate the importance of moderation policies by comparing the results of a general-purpose LLM with different policies to answer RQ4.

In the following sections, we introduce the models we have compared (Section 4.1) and discuss the evaluation metrics chosen to assess the models' effectiveness (Section 4.2) before presenting the results in Section 5.

\subsection{Models}
In this section, we introduce the models that we evaluated against our large-scale benchmark. We consider several open-weight models, including recent guardrail models, content moderation models often employed in real-world applications, and instruction-tuned general-purpose LLM prompted for content moderation. We consider the latter to evaluate their out-of-the-box capabilities in detecting unsafe prompts and responses. The major differences between guardrail models and content moderation models are that the first are meant to moderate human-AI conversations while the latter were trained on content from online social platforms. Moreover, guardrail models are usually prompted by providing them a content moderation policy, i.e., a list of unsafe content categories, while available content moderation models do not take advantage of such mechanism. The list of all the considered models is presented below. Further information are provided in Table 2.

\begin{itemize}
    \item \textbf{Llama Guard (LG):} guardrail model based on LLama 2 7B \cite{Touvron2023} proposed by Inan et al. \cite{Inan2023}.
    \item \textbf{Llama Guard 2 (LG-2):} updated version of Llama Guard based on LLama 3 8B\footnote{Model not formally published, referred to in the paper as "Llama 3 8B".}.
    \item \textbf{Llama Guard Defensive (LG-D):} Llama Guard additionally fine-tuned by Ghosh et al. \cite{Ghosh2024} with a strict content moderation policy.
    \item \textbf{Llama Guard Permissive (LG-P):} Llama Guard additionally fine-tuned by Ghosh et al. \cite{Ghosh2024} with an impressive content moderation policy.
    \item \textbf{MD-Judge (MD-J):} guardrail model obtained by fine-tuning Mistral 7B \cite{Jiang2023} on BeaverTails330K \cite{Ji2023}, Toxic Chat \cite{Lin2023}, and LMSYS-Chat-1M \cite{Zheng2023} by Li et al. \cite{Li2024}.
    \item \textbf{Toxic Chat T5 (TC-T5):} guardrail model obtained by fine-tuning T5-Large \cite{Raffel2020} on Toxic Chat \cite{Lin2023}.
    \item \textbf{ToxiGen HateBERT (TG-B):} content moderation model obtained by fine-tuning HateBERT \cite{Caselli2021} on ToxiGen \cite{Hartvigsen2022}.
    \item \textbf{ToxiGen RoBERTa (TG-R):} content moderation model obtained by fine-tuning ToxDec-tRoBERTa \cite{Zhou2021} on ToxiGen \cite{Hartvigsen2022}.
    \item \textbf{Detoxify Original (DT-O):} BERT Base Uncased \cite{Devlin2019} fine-tuned on Jigsaw's Toxic Comment Classification Challenge dataset \cite{cjadams2019} for content moderation by Unitary AI \cite{UnitaryAI2020}.
    \item \textbf{Detoxify Unbiased (DT-U):} RoBERTa Base \cite{Liu2019} fine-tuned on Jigsaw's Unintended Bias in Toxicity Classification dataset \cite{cjadams2017} for content moderation by Unitary AI \cite{UnitaryAI2020}.
    \item \textbf{Detoxify Multilingual (DT-M):} XLM RoBERTa Base \cite{Conneau2020} fine-tuned on Jigsaw's Multilingual Toxic Comment Classification dataset \cite{Kivlichan2020} for content moderation by Unitary AI \cite{UnitaryAI2020}.
    \item \textbf{Mistral-7B-Instruct v0.2 (Mis):} general-purpose, instruction-tuned LLM proposed by Jiang et al. \cite{Jiang2023}. We instruct the model to check the input safety using the moderation prompt provided by its authors\footnote{See model card for details.}.
    \item \textbf{Mistral with refined policy (Mis+):} Mistral-7B-Instruct v0.2 with the moderation policy of MD-Judge. More details in Section 5.4.
\end{itemize}

\subsection{Evaluation Metrics}
To evaluate the effectiveness of the considered models, we rely on F1 and Recall (when a dataset only comprises unsafe samples). Unlike previous works \cite{Inan2023, Markov2023}, we do not employ the Area Under the Precision-Recall Curve (AUPRC) as we found it overemphasizes models' Precision at the expense of Recall in the case of binary classification, thus hiding significant performance details. Moreover, F1 and Recall do not require classification probabilities as AUPRC, making them more convenient for comparing closed-weight models. We rely on Scikit-Learn \cite{Pedregosa2011} to compute metric scores.

\section{Results and Discussion}
In this section, we present the results of our comparative evaluation. First, we discuss the models' effectiveness in assessing user prompts and human-AI conversations safety in Section 5.1 and Section 5.2, respectively. Then, in Section 5.3, we show preliminary results on non-English prompts. Finally, we evaluate the importance of content moderation policies in Section 5.4. Note that the results of Mistral with refined policy are considered only in Section 5.4. We refer the reader to Table 2 for the model aliases used in Table 3.

\subsection{Prompts Moderation}
In this section, we discuss the performance of the compared models at detecting unsafe user prompts, i.e., inputs containing or eliciting unsafe information. As shown in the first part of Table 3, guardrail models outperform content moderation models, suggesting the latter are not well-suited for prompt moderation. However, we highlight that the considered guardrail models have several times the parameters of the largest content moderation model, ToxiGen RoBERTa. Quite interestingly, Mistral, the general-purpose model we tested, often achieves better results than Llama Guard despite not being fine-tuned for detecting unsafe content in prompts and human-AI conversations. Overall, the best performing models are Llama Guard Defensive and MD-Judge, both of which surpass Llama Guard 2 in terms of performance, despite the latter is the most recent and advanced model. However, we observe that Llama Guard Defensive exhibits a potentially exaggerated safety behavior, given its relatively low F1 score on XSTest, which was proposed by Rottger et al. \cite{Rottger2023} to evaluate such behavior. Due to the close performance of Llama Guard Defensive and MD-Judge, there is no clear answer to RQ1.

\subsection{Conversations Moderation}
In this section, we discuss the performance of the compared models at detecting user and LLM unsafe utterances in conversations. Results are presented in the second part of Table 3. Unlike prompts classification, content moderation models often perform closer to guardrail models when assessing conversation safety. For example, ToxiGen HateBERT reaches performance close to Llama Guard on Toxic Chat and BeaverTails 330k. MD-Judge is the best performing model among all the considered models, outperforming the more recent Llama Guard 2, Llama Guard Defensive, and Llama Guard Permissive. To answer RQ2, MD-Judge is the best-performing model at moderating conversations. However, there is still a large margin for improvements. Moreover, we found ToxiGen HateBERT to perform close to Llama Guard, despite having 70x less parameters. Therefore, performance-cost trade-offs of using multi-billion models as safety filters should be further investigated.

\subsection{Multi-Lingual Capabilities}
In this section, we discuss the out-of-the-box multilingual capabilities of the compared models. For reference, we report the performance of every model on a dataset built by merging all the English prompt datasets we translated, which we call PromptsEN. We highlight that none of the model received specific fine-tuning on multi-lingual datasets for safety classification other than Detoxify Multilingual. However, both the Llama-based models and the Mistral-based models were exposed to multi-lingual texts during pre-training. As shown in the third part of Table 3, Llama Guard Defensive, Llama Guard Permissive, and MD-Judge are the best performing models on the reference English dataset. However, Llama Guard Defensive and Llama Guard Permissive show much better performance than MD-Judge on German, French, Italian, and Spanish prompts. Although they still suffer from a performance degradation, it is far less noticeable than all the other considered models, especially in the case of Llama Guard Defensive. To answer RQ3, multi-lingual capabilities of most of the compared models are not comparable to those on English texts. However, we found the results achieved by Llama Guard Defensive to be encouraging for the detection of unsafe non-English text.

\subsection{Policy Comparison}
As introduced in Section 4.1, guardrail models are usually prompted with a content moderation policy and asked whether the input violates such a policy. In this section, we discuss the impact of the content moderation policy on the evaluation results. Specifically, we evaluate the performance of Mistral with the MD-Judge's policy. MD-Judge is based on Mistral and was fine-tuned on multiple safety datasets, such as BeaverTails330K \cite{Ji2023}, Toxic Chat \cite{Lin2023}, and LMSYS-Chat-1M \cite{Zheng2023}. With this experiment, we aim to assess whether their noticeable performance difference is due to the extensive fine-tuning received by MD-Judge or by their different content moderation policies. We highlight that the semantic content of the two policies presents significant overlaps. However, they are written and structured differently. The last column of Table 3 (Mis+) reports the performance of Mistral when prompted with MD-Judge's content moderation policy. Quite surprisingly, when prompted with MD-Judge's content moderation policy, Mistral show a very significant performance uplift, often outperforming MD-Judge and even reaching state-of-the-art results on multiple datasets. Such finding raise some concerns. First, comparisons with general-purpose LLMs are not present in recent publications on guardrail models \cite{Inan2023, Ghosh2024}. Secondly, the available training datasets for prompts and conversation safety classification may be insufficient to strongly improve over instruction-following models prompted for moderation. Moreover, prompt engineering \cite{White2023} the content moderation policy could be crucial to improve over the state-of-the-art. Our analysis of RQ4 reveals that content moderation policies significantly impact the effectiveness of guardrails models. Therefore, crafting well-written policies will be crucial for achieving improvements.

\section{Conclusion and Future Work}
In this work, we proposed GuardBench, a large-scale benchmark for evaluating guardrail models. GuardBench comprises 40 datasets for prompts and conversations safety evaluation. We included 35 datasets in English from previous works and five new datasets. Specifically, we built a new dataset for conversation safety evaluation by generating 22k answers to unsafe prompts from previous works. Moreover, we translated 31k English prompts to German, French, Italian, and Spanish, producing the first large-scale prompts safety datasets in those languages. To facilitate the adoption of GuardBench by the research community, we released a Python library offering a convenient evaluation pipeline. We also conducted the first large-scale evaluation of state-of-the-art guardrail models, showing that those models perform close to each other when identifying unsafe prompts, while we register more pronounced differences when used to moderate conversations. Finally, we showed general-purpose and instruction-following models can achieve competitive results when correctly prompted for safety moderation. In the future, we plan to extend GuardBench with an enhanced evaluation procedure to provide more structured results over the different categories of unsafe content. Safety classification of prompts and conversation utterances remains an open problem with considerable room for improvement. Advancements in this area are of utmost importance to safely deploy Large Language Models in high-risk and safety-critical domains, such as healthcare, education, and finance.

\section{Limitations}
While providing a valuable resource for guardrail models evaluation, our work has several limitations. Our benchmark scope is limited to the safe/unsafe binary classification task of prompts and conversation utterances. It does not cover multi-class and multi-label cases, although unsafe content may be classified in several, sometimes overlapping, categories of harm. Moreover, content that is unsafe for certain applications, such as finance, or belonging to specific unsafe categories may be missing from the datasets included in our benchmark. Several datasets included in our benchmark only have negative predictive power \cite{Gardner2020}, i.e. they only provide unsafe samples, as reported in Table 1. Thus, their usage should be limited to evaluating a model's weaknesses in recognizing unsafe content rather than characterizing generalizable strengths. Therefore, claims about model quality should not be overextended based solely on positive results on those datasets. We did not conduct any evaluation in which the models are required to follow, for example, a more permissive content moderation policy for a specific use case instead of the one provided by their authors or to adhere to a different view of safety. Finally, due to hardware constraints, we mainly investigated models up to a scale of 8 billion parameters. We also did not consider closed-weight and commercial moderation models such as OpenAI Moderation API and Perspective API.

\section{Ethical Statement}
This research aims to advance the development of Trustworthy Generative AI systems by contributing to the design of robust and effective guardrail models. Our large-scale benchmark, GuardBench, enables a comprehensive assessment of the performance of these critical AI safety components. We acknowledge that our research involves the usage and generation of unsafe content. The processing and inclusion of this content in GuardBench were necessary to evaluate the effectiveness of guardrail models in accurately identifying unsafe content. This research has received approval from the Joint Research Centre's (JRC) Ethical Review Board. In our commitment to contributing to AI safety, we make GuardBench available to the scientific community as open source software. We also share our novel datasets under a research-only license, providing access to them upon justified request. This approach ensures that the benefits of our research are accessible while mitigating potential risks and promoting responsible use.

\bibliographystyle{unsrt}
\balance
\bibliography{refs}

\newpage
\appendix
\section{Appendix}

\subsection{Labels Binarization}
\label{app:labels}
In this section, we provide further information on how we converted the labels of the gathered datasets into binary format. As BeaverTails 330k, ConvAbuse, DICES 350, and DICES 990 provide multiple annotations for each sample, we relied on a majority vote to decide whether a sample was safe or unsafe. We labelled samples as safe in case of ties. Note that some datasets use different binary labels for the safe and unsafe samples, such as toxic vs non-toxic. However, they directly fall within our definition of safe and unsafe content.

\subsubsection{Prompts: Instructions}
AdvBench Behaviors: Only unsafe samples. No conversion needed.\\
HarmBench Behaviors: Only unsafe samples. No conversion needed.\\
I-CoNa: Only unsafe samples. No conversion needed.\\
I-Controversial: Only unsafe samples. No conversion needed.\\
I-MaliciousInstructions: Only unsafe samples. No conversion needed.\\
I-Physical-Safety: Samples are labelled as safe or unsafe. No conversion needed.\\
MaliciousInstruction: Only unsafe samples. No conversion needed.\\
MITRE: Only unsafe samples. No conversion needed.\\
StrongREJECT Instructions: Only unsafe samples. No conversion needed.\\
TDCRedTeaming Instructions: Only unsafe samples. No conversion needed.

\subsubsection{Prompts: Questions}
CatQA: Only unsafe samples. No conversion needed.\\
Do Anything Now Questions: Only unsafe samples. No conversion needed.\\
DoNotAnswer: Only unsafe samples. No conversion needed.\\
HarmfulQA Questions: Only unsafe samples. No conversion needed.\\
HEX-PHI: Only unsafe samples. No conversion needed.\\
XSTest: Samples are labelled as safe or unsafe. No conversion needed.

\subsubsection{Prompts: Statements}
AdvBench Strings: Only unsafe samples. No conversion needed.\\
DecodingTrust Stereotypes: Only unsafe samples. No conversion needed.\\
DynaHate: Samples are labelled as safe or unsafe. No conversion needed.\\
HateCheck: Samples are labelled as safe or unsafe. No conversion needed.\\
Hatemoji Check: Samples are labelled as safe or unsafe. No conversion needed.\\
SafeText: Samples are labelled as safe or unsafe. No conversion needed.\\
ToxiGen: Samples are labelled as safe or unsafe by following the authors instructions. No conversion needed.

\subsubsection{Prompts: Mixed}
AART: Only unsafe samples. No conversion needed.\\
OpenAI Moderation Dataset: Samples have binary labels for each of the following categories: (S) Sexual, (H) Hate, (V) Violence, (HR) Harassment, (SH) Self-harm, (S3) Sexual/Minors, (H2) Hate/Threatening, and (V2) Violence/Graphic. We labelled each sample that does not belong to those categories as safe. Otherwise, we labelled it as unsafe.\\
SimpleSafetyTests: Samples are all unsafe and categorized as either (1) Suicide, Self-Harm, and Eating Disorders, (2) Physical Harm and Violence, (3) Illegal and Highly Regulated items, (4) Scams and Fraud, or (5) Child Abuse. We labelled all the samples as unsafe.\\
Toxic Chat: Samples are labelled as safe or unsafe. No conversion needed.

\subsubsection{Conversations: Single-Turn}
BeaverTails 330k: Samples have binary labels for each of the following categories: (1) Animal Abuse, (2) Child Abuse, (3) Controversial Topics, Politics, (4) Discrimination, Stereotype, Injustice, (5) Drug Abuse, Weapons, Banned Substance, (6) Financial Crime, Property Crime, Theft, (7) Hate Speech, Offensive Language, (8) Misinformation Regarding ethics, laws, and safety, (9) Non-Violent Unethical Behavior, (10) Privacy Violation, (11) Self-Harm, (12) Sexually Explicit, Adult Content, (13) Terrorism, Organized Crime, and (14) Violence, Aiding and Abetting, Incitement. We labelled each sample that does not belong to those categories as safe. Otherwise, we labelled it as unsafe.

\subsubsection{Conversations: Multi-Turn}
Bot-Adversarial Dialogue: Samples are labelled as safe or unsafe. No conversion needed.\\
ConvAbuse: Samples are labelled as safe or unsafe. No conversion needed.\\
DICES 350: Samples are labelled as safe or unsafe. No conversion needed.\\
DICES 990: Samples are labelled as safe or unsafe. No conversion needed.\\
HarmfulQA: Samples are labelled as safe or unsafe. No conversion needed.\\
ProsocialDialog: Samples are labelled as either (1) casual, (2) possibly needs caution, (3) probably needs caution, (4) needs caution, or (5) needs intervention. We converted (1), (2), and (3) as safe, and (4) and (5) as unsafe.

\subsection{Novel Datasets}
\label{app:novel}
In this section, we provide further information regarding which datasets we translated (see Section 3.3) and those we answered to build UnsafeQA (see Section 3.4). Table 4 show which datasets were used as sources for deriving our datasets. In both cases, we excluded HEX-PHI \cite{Qi2023} and DecodingTrust Stereotypes \cite{Wang2023a} due to their license limitations.

\subsubsection{Multi-lingual Datasets}
As shown in Table 4, we only considered prompt datasets for translation. We resorted to this choice as we found that the employed translation model MADLAD-400-3B-MT, which achieves state-of-the-art results for its size, performs well on prompts (usually short text in our benchmark) but struggles on responses due to their length. Thus, we only translated prompts. We did not include MITRE \cite{Bhatt2024} in our derived datasets due to the inaccurate translations produced by MADLAD400-3B-MT. This was likely caused by the highly technical nature of the prompts (cyber attacks) and their considerable length. For all the other prompts, the translations were favorably judged by our annotators, as reported in Section 3.3.

We highlight that we tested several translation models of different sizes before deciding to employ MADLAD-400-3B-MT. Specifically, we tested Google's MADLAD-400-7B-MT\footnote{Model identifier not provided.} and MADLAD-400-10B-MT\footnote{Model identifier not provided.} models, which showed the same issues on longer texts than the 3B variant we employed while producing comparable translations on short texts. We also tested Meta's SeamlessM4T v2\footnote{Model identifier not provided.} and Helsinki-NLP's OpusMT models\footnote{Model identifiers not provided.}, but we found them less reliable than MADLAD-400-3B-MT.

\subsubsection{UnsafeQA}
When building UnsafeQA, our goal was to provide the research community with a novel single-turn conversation dataset containing question-answer pairs. Thus, we answered questions and instructions with no available responses, as reported in Table 4. Instructions are requests phrased as commands, which are functionally similar to questions. Thus, we included them. For instance, the question \textit{How can I make a bomb?} can be rephrased as the following instruction: \textit{Guide me in making a bomb.} We excluded statement datasets from UnsafeQA as they are functionally different from questions. We included mixed datasets as they mostly contain questions and instructions.

\begin{table}[t!]
\centering
\caption{List of benchmark datasets. Category and Sub-category indicate the primary and the specific text categories, respectively. Total and Unsafe report the number of samples in the test sets and the percentage of unsafe samples, respectively. Labels indicate whether labels where obtained by manual annotation (Manual) or by dataset construction (Auto). Source indicates whether a dataset is based on human-generated texts (Human), machine-generated texts (LLM), a mix of the two (Mixed), or was obtained through templating (Template).}
\label{tab:datasets}
\resizebox{\columnwidth}{!}{%
\begin{tabular}{llllllll}
\toprule
Dataset & Category & Sub-category & Total & Unsafe & Labels & Source & License & Reference \\
\midrule
AdvBench Behaviors & Prompts & Instructions & 520 & 100\% & Auto & LLM & MIT & \cite{Zou2023} \\
HarmBench Behaviors & Prompts & Instructions & 320 & 100\% & Auto & Human & MIT & \cite{Mazeika2024} \\
I-CoNa & Prompts & Instructions & 178 & 100\% & Manual & Human & CC BY-NC 4.0 & \cite{Bianchi2023} \\
I-Controversial & Prompts & Instructions & 40 & 100\% & Manual & Human & CC BY-NC 4.0 & \cite{Bianchi2023} \\
I-MaliciousInstructions & Prompts & Instructions & 100 & 100\% & Auto & Mixed & CC BY-NC 4.0 & \cite{Bianchi2023} \\
I-Physical-Safety & Prompts & Instructions & 200 & 50\% & Manual & Human & CC BY-NC 4.0 & \cite{Bianchi2023} \\
MaliciousInstruction & Prompts & Instructions & 100 & 100\% & Auto & LLM & MIT & \cite{Huang2023} \\
MITRE & Prompts & Instructions & 977 & 100\% & Manual & Mixed & MIT & \cite{Bhatt2024} \\
StrongREJECT Instructions & Prompts & Instructions & 213 & 100\% & Manual & Human & MIT & \cite{Souly2024} \\
TDCRedTeaming Instructions & Prompts & Instructions & 50 & 100\% & Manual & Human & MIT & \cite{Mazeika2023} \\
CatQA & Prompts & Questions & 550 & 100\% & Auto & LLM & Apache 2.0 & \cite{Bhardwaj2024} \\
Do Anything Now Questions & Prompts & Questions & 390 & 100\% & Auto & LLM & MIT & \cite{Shen2023} \\
DoNotAnswer & Prompts & Questions & 939 & 100\% & Auto & LLM & Apache 2.0 & \cite{Wang2024} \\
HarmfulQ & Prompts & Questions & 200 & 100\% & Auto & LLM & MIT & \cite{Shaikh2023} \\
HarmfulQA Questions & Prompts & Questions & 1960 & 100\% & Auto & LLM & Apache 2.0 & \cite{BhardwajPoria2023} \\
HEX-PHI & Prompts & Questions & 330 & 100\% & Manual & Human & Custom & \cite{Qi2023} \\
XSTest & Prompts & Questions & 450 & 44\% & Manual & Human & CC BY 4.0 & \cite{Rottger2023} \\
AdvBench Strings & Prompts & Statements & 574 & 100\% & Auto & LLM & MIT & \cite{Zou2023} \\
DecodingTrust Stereotypes & Prompts & Statements & 1152 & 100\% & Manual & Template & CC BY-SA 4.0 & \cite{Wang2023a} \\
DynaHate & Prompts & Statements & 4120 & 55\% & Manual & Human & Apache 2.0 & \cite{Vidgen2021} \\
HateCheck & Prompts & Statements & 3728 & 69\% & Manual & Template & CC BY 4.0 & \cite{Rottger2021} \\
Hatemoji Check & Prompts & Statements & 593 & 52\% & Manual & Template & CC BY 4.0 & \cite{Kirk2022} \\
SafeText & Prompts & Statements & 1465 & 25\% & Manual & Human & MIT & \cite{Levy2022} \\
ToxiGen & Prompts & Statements & 940 & 43\% & Manual & LLM & MIT & \cite{Hartvigsen2022} \\
AART & Prompts & Mixed & 3269 & 100\% & Auto & LLM & CC BY 4.0 & \cite{Radharapu2023} \\
OpenAI Moderation Dataset & Prompts & Mixed & 1680 & 31\% & Manual & Human & MIT & \cite{Markov2023} \\
SimpleSafetyTests & Prompts & Mixed & 100 & 100\% & Manual & Human & CC BY 4.0 & \cite{Vidgen2023} \\
Toxic Chat & Prompts & Mixed & 508 & 37\% & Manual & Human & CC BY-NC 4.0 & \cite{Lin2023} \\
BeaverTails 330k & Conversations & Single-Turn & 11088 & 55\% & Manual & Mixed & MIT & \cite{Ji2023} \\
Bot-Adversarial Dialogue & Conversations & Multi-Turn & 2598 & 36\% & Manual & Mixed & Apache 2.0 & \cite{Xu2021} \\
ConvAbuse & Conversations & Multi-Turn & 853 & 15\% & Manual & Mixed & CC BY 4.0 & \cite{Curry2021} \\
DICES 350 & Conversations & Multi-Turn & 350 & 50\% & Manual & Mixed & CC BY 4.0 & \cite{Aroyo2023} \\
DICES 990 & Conversations & Multi-Turn & 990 & 16\% & Manual & Mixed & CC BY 4.0 & \cite{Aroyo2023} \\
HarmfulQA & Conversations & Multi-Turn & 16459 & 45\% & Auto & LLM & Apache 2.0 & \cite{BhardwajPoria2023} \\
ProsocialDialog & Conversations & Multi-Turn & 25029 & 60\% & Manual & Mixed & CC BY 4.0 & \cite{Kim2022} \\
PromptsDE & Prompts & Mixed & 30852 & 61\% & Mixed & LLM & Custom & Our \\
PromptsFR & Prompts & Mixed & 30852 & 61\% & Mixed & LLM & Custom & Our \\
PromptsIT & Prompts & Mixed & 30852 & 61\% & Mixed & LLM & Custom & Our \\
PromptsES & Prompts & Mixed & 30852 & 61\% & Mixed & LLM & Custom & Our \\
UnsafeQA & Conversations & Single-Turn & 22180 & 50\% & Auto & Mixed & Custom & Our \\
\bottomrule
\end{tabular}%
}
\end{table}

\begin{table}[t!]
\centering
\caption{Benchmarked models. Alias indicates the shortened names used in other tables.}
\label{tab:models}
\resizebox{\columnwidth}{!}{%
\begin{tabular}{llllll}
\toprule
Model & Alias & Category & Base Model & Params & Architecture & Reference \\
\midrule
Llama Guard & LG & Guardrail & Llama 2 7B & 6.74 B & Decoder-only & \cite{Inan2023} \\
Llama Guard 2 & LG-2 & Guardrail & Llama 3 8B & 8.03 B & Decoder-only & N/A \\
Llama Guard Defensive & LG-D & Guardrail & Llama 2 7B & 6.74 B & Decoder-only & \cite{Ghosh2024} \\
Llama Guard Permissive & LG-P & Guardrail & Llama 2 7B & 6.74 B & Decoder-only & \cite{Ghosh2024} \\
MD-Judge & MD-J & Guardrail & Mistral 7B & 7.24 B & Decoder-only & \cite{Li2024} \\
Toxic Chat T5 & TC-T5 & Guardrail & T5 Large & 0.74 B & Encoder-Decoder & N/A \\
ToxiGen HateBERT & TG-B & Moderation & BERT Base Uncased & 0.11 B & Encoder-only & \cite{Hartvigsen2022} \\
ToxiGen RoBERTa & TG-R & Moderation & RoBERTa Large & 0.36 B & Encoder-only & \cite{Hartvigsen2022} \\
Detoxify Original & DT-O & Moderation & BERT Base Uncased & 0.11 B & Encoder-only & \cite{UnitaryAI2020} \\
Detoxify Unbiased & DT-U & Moderation & RoBERTa Base & 0.12 B & Encoder-only & \cite{UnitaryAI2020} \\
Detoxify Multilingual & DT-M & Moderation & XLM RoBERTa Base & 0.28 B & Encoder-only & \cite{UnitaryAI2020} \\
Mistral-7B-Instruct v0.2 & Mis & General Purpose & Mistral 7B & 7.24 B & Decoder-only & \cite{Jiang2023} \\
Mistral with refined policy & Mis+ & General Purpose & Mistral 7B & 7.24 B & Decoder-only & Section 5.4 \\
\bottomrule
\end{tabular}%
}
\end{table}

\begin{table*}[t!]
\centering
\caption{Performance results (Recall/F1) of the models on the benchmark datasets. The table is split into three parts: Prompt datasets (Instructions, Questions, Statements, Mixed), Conversation datasets, and Multilingual prompt datasets. The last column (Mis+) shows the performance of Mistral with the refined policy. Due to space, only key dataset rows are shown. The full table is available in the supplementary material.}
\label{tab:results}
\resizebox{\textwidth}{!}{%
\begin{tabular}{llccccccccccccc}
\toprule
Dataset & Metric & LG & LG-2 & LG-D & LG-P & MD-J & TC-T5 & TG-B & TG-R & DT-O & DT-U & DT-M & Mis & Mis+ \\
\midrule
AdvBench Behaviors & Recall & 0.837 & 0.963 & 0.990 & 0.931 & 0.987 & 0.842 & 0.550 & 0.117 & 0.019 & 0.012 & 0.012 & 0.048 & 0.992 \\
HarmBench Behaviors & Recall & 0.478 & 0.812 & 0.684 & 0.569 & 0.675 & 0.300 & 0.341 & 0.059 & 0.028 & 0.016 & 0.031 & 0.516 & 0.622 \\
I-CoNa & Recall & 0.916 & 0.798 & 0.978 & 0.966 & 0.871 & 0.287 & 0.882 & 0.764 & 0.253 & 0.483 & 0.517 & 0.640 & 0.910 \\
I-Controversial & Recall & 0.900 & 0.625 & 0.975 & 0.900 & 0.900 & 0.225 & 0.550 & 0.450 & 0.025 & 0.125 & 0.125 & 0.300 & 0.875 \\
I-MaliciousInstructions & Recall & 0.780 & 0.860 & 0.950 & 0.850 & 0.950 & 0.660 & 0.510 & 0.240 & 0.050 & 0.080 & 0.070 & 0.750 & 0.980 \\
I-Physical-Safety & Recall & 0.820 & 0.890 & 1.000 & 0.920 & 0.243 & 0.076 & 0.655 & 0.113 & 0.179 & 0.076 & 0.076 & 0.226 & 0.458 \\
MaliciousInstruction & Recall & 0.820 & 0.890 & 1.000 & 0.920 & 0.990 & 0.730 & 0.280 & 0.000 & 0.000 & 0.000 & 0.000 & 0.980 & 0.990 \\
MITRE & Recall & 0.128 & 0.867 & 0.813 & 0.505 & 0.739 & 0.217 & 0.511 & 0.000 & 0.000 & 0.000 & 0.000 & 0.350 & 0.367 \\
StrongREJECT Instructions & Recall & 0.831 & 0.953 & 0.986 & 0.930 & 0.972 & 0.399 & 0.460 & 0.160 & 0.023 & 0.047 & 0.047 & 0.803 & 0.930 \\
TDCRedTeaming Instructions & Recall & 0.800 & 0.820 & 1.000 & 0.920 & 0.980 & 0.600 & 0.720 & 0.140 & 0.040 & 0.020 & 0.040 & 0.740 & 0.940 \\
CatQA & Recall & 0.798 & 0.936 & 0.980 & 0.893 & 0.944 & 0.511 & 0.176 & 0.018 & 0.007 & 0.018 & 0.016 & 0.978 & 0.945 \\
Do Anything Now Questions & Recall & 0.492 & 0.592 & 0.631 & 0.526 & 0.610 & 0.374 & 0.103 & 0.031 & 0.000 & 0.003 & 0.000 & 0.810 & 0.574 \\
DoNotAnswer & Recall & 0.321 & 0.442 & 0.496 & 0.399 & 0.501 & 0.224 & 0.249 & 0.100 & 0.028 & 0.034 & 0.048 & 0.435 & 0.460 \\
HarmfulQ & Recall & 0.890 & 0.875 & 0.970 & 0.930 & 0.945 & 0.665 & 0.290 & 0.055 & 0.010 & 0.000 & 0.010 & 0.930 & 0.965 \\
HarmfulQA Questions & Recall & 0.408 & 0.548 & 0.780 & 0.522 & 0.666 & 0.263 & 0.111 & 0.003 & 0.000 & 0.000 & 0.000 & 0.638 & 0.683 \\
HEX-PHI & Recall & 0.724 & 0.939 & 0.952 & 0.867 & 0.946 & 0.506 & 0.470 & 0.115 & 0.021 & 0.045 & 0.052 & 0.906 & 0.958 \\
XSTest & Recall & 0.819 & 0.891 & 0.783 & 0.812 & 0.858 & 0.632 & 0.373 & 0.233 & 0.186 & 0.287 & 0.424 & 0.826 & 0.878 \\
AdvBench Strings & Recall & 0.807 & 0.782 & 0.948 & 0.882 & 0.929 & 0.540 & 0.869 & 0.704 & 0.638 & 0.596 & 0.599 & 0.911 & 0.949 \\
DecodingTrust Stereotypes & Recall & 0.875 & 0.780 & 0.993 & 0.944 & 0.957 & 0.211 & 0.977 & 0.900 & 0.589 & 0.655 & 0.668 & 0.572 & 0.765 \\
DynaHate & Recall & 0.804 & 0.766 & 0.750 & 0.783 & 0.788 & 0.421 & 0.698 & 0.645 & 0.549 & 0.567 & 0.590 & 0.712 & 0.771 \\
HateCheck & Recall & 0.942 & 0.945 & 0.877 & 0.909 & 0.921 & 0.562 & 0.853 & 0.833 & 0.757 & 0.761 & 0.803 & 0.879 & 0.909 \\
Hatemoji Check & Recall & 0.862 & 0.788 & 0.873 & 0.898 & 0.869 & 0.376 & 0.791 & 0.607 & 0.669 & 0.575 & 0.642 & 0.780 & 0.853 \\
SafeText & Recall & 0.143 & 0.579 & 0.504 & 0.294 & 0.425 & 0.085 & 0.417 & 0.052 & 0.154 & 0.078 & 0.097 & 0.487 & 0.579 \\
ToxiGen & Recall & 0.784 & 0.673 & 0.760 & 0.795 & 0.821 & 0.297 & 0.793 & 0.741 & 0.411 & 0.393 & 0.418 & 0.648 & 0.787 \\
AART & Recall & 0.825 & 0.843 & 0.952 & 0.891 & 0.879 & 0.745 & 0.483 & 0.122 & 0.019 & 0.037 & 0.054 & 0.815 & 0.898 \\
OpenAI Moderation Dataset & Recall & 0.744 & 0.761 & 0.658 & 0.756 & 0.774 & 0.695 & 0.559 & 0.644 & 0.646 & 0.672 & 0.688 & 0.720 & 0.779 \\
SimpleSafetyTests & Recall & 0.860 & 0.920 & 1.000 & 0.940 & 0.970 & 0.640 & 0.620 & 0.230 & 0.170 & 0.280 & 0.280 & 0.870 & 0.980 \\
Toxic Chat & Recall & 0.561 & 0.422 & 0.577 & 0.678 & 0.816* & 0.822* & 0.339 & 0.315 & 0.265 & 0.279 & 0.321 & 0.415 & 0.671 \\
BeaverTails 330k & Recall & 0.686 & 0.755 & 0.778 & 0.755 & 0.887* & 0.448 & 0.643 & 0.245 & 0.173 & 0.216 & 0.236 & 0.696 & 0.740 \\
UnsafeQA & Recall & 0.668 & 0.787 & 0.792 & 0.793 & 0.842 & 0.559 & 0.674 & 0.243 & 0.172 & 0.201 & 0.224 & 0.698 & 0.754 \\
\midrule
PromptsEN & F1 & 0.812 & 0.835 & 0.876 & 0.856 & 0.868 & 0.594 & 0.672 & 0.537 & 0.413 & 0.445 & 0.478 & 0.790 & 0.874 \\
PromptsDE & F1 & 0.701 & 0.725 & 0.838 & 0.819 & 0.752 & 0.423 & 0.521 & 0.401 & 0.298 & 0.331 & 0.365 & 0.655 & 0.802 \\
PromptsFR & F1 & 0.698 & 0.728 & 0.839 & 0.820 & 0.754 & 0.425 & 0.522 & 0.403 & 0.300 & 0.333 & 0.367 & 0.657 & 0.803 \\
PromptsIT & F1 & 0.697 & 0.727 & 0.838 & 0.819 & 0.753 & 0.424 & 0.521 & 0.402 & 0.299 & 0.332 & 0.366 & 0.656 & 0.802 \\
PromptsES & F1 & 0.700 & 0.730 & 0.840 & 0.821 & 0.755 & 0.426 & 0.523 & 0.404 & 0.301 & 0.334 & 0.368 & 0.658 & 0.804 \\
\bottomrule
\end{tabular}%
}
\end{table*}

\begin{table}[t!]
\centering
\caption{Datasets used to derive our multi-lingual datasets and UnsafeQA. A checkmark (✓) indicates inclusion, a cross (×) indicates exclusion.}
\label{tab:derived}
\resizebox{\columnwidth}{!}{%
\begin{tabular}{lllllllll}
\toprule
Dataset & Category & Sub-category & License & Reference & PromptsDE & PromptsFR & PromptsIT & PromptsES & UnsafeQA \\
\midrule
AdvBench Behaviors & Prompts & Instructions & MIT & \cite{Zou2023} & ✓ & ✓ & ✓ & ✓ & ✓ \\
HarmBench Behaviors & Prompts & Instructions & MIT & \cite{Mazeika2024} & ✓ & ✓ & ✓ & ✓ & ✓ \\
I-CoNa & Prompts & Instructions & CC BY-NC 4.0 & \cite{Bianchi2023} & ✓ & ✓ & ✓ & ✓ & ✓ \\
I-Controversial & Prompts & Instructions & CC BY-NC 4.0 & \cite{Bianchi2023} & ✓ & ✓ & ✓ & ✓ & ✓ \\
I-MaliciousInstructions & Prompts & Instructions & CC BY-NC 4.0 & \cite{Bianchi2023} & ✓ & ✓ & ✓ & ✓ & ✓ \\
I-Physical Safety & Prompts & Instructions & CC BY-NC 4.0 & \cite{Bianchi2023} & ✓ & ✓ & ✓ & ✓ & ✓ \\
MaliciousInstruction & Prompts & Instructions & MIT & \cite{Huang2023} & ✓ & ✓ & ✓ & ✓ & ✓ \\
MITRE & Prompts & Instructions & MIT & \cite{Bhatt2024} & × & × & × & × & ✓ \\
StrongREJECT Instructions & Prompts & Instructions & MIT & \cite{Souly2024} & ✓ & ✓ & ✓ & ✓ & ✓ \\
TDCRedTeaming Instructions & Prompts & Instructions & MIT & \cite{Mazeika2023} & ✓ & ✓ & ✓ & ✓ & ✓ \\
CatQA & Prompts & Questions & Apache 2.0 & \cite{Bhardwaj2024} & ✓ & ✓ & ✓ & ✓ & ✓ \\
Do Anything Now Questions & Prompts & Questions & MIT & \cite{Shen2023} & ✓ & ✓ & ✓ & ✓ & ✓ \\
DoNotAnswer & Prompts & Questions & Apache 2.0 & \cite{Wang2024} & ✓ & ✓ & ✓ & ✓ & ✓ \\
HarmfulQ & Prompts & Questions & MIT & \cite{Shaikh2023} & ✓ & ✓ & ✓ & ✓ & ✓ \\
HarmfulQA Questions & Prompts & Questions & Apache 2.0 & \cite{BhardwajPoria2023} & ✓ & ✓ & ✓ & ✓ & ✓ \\
HEX-PHI & Prompts & Questions & Custom & \cite{Qi2023} & × & × & × & × & × \\
XSTest & Prompts & Questions & CC BY 4.0 & \cite{Rottger2023} & ✓ & ✓ & ✓ & ✓ & ✓ \\
AdvBench Strings & Prompts & Statements & MIT & \cite{Zou2023} & ✓ & ✓ & ✓ & ✓ & ✓ \\
DecodingTrust Stereotypes & Prompts & Statements & CC BY-SA 4.0 & \cite{Wang2023a} & × & × & × & × & × \\
DynaHate & Prompts & Statements & Apache 2.0 & \cite{Vidgen2021} & ✓ & ✓ & ✓ & ✓ & ✓ \\
HateCheck & Prompts & Statements & CC BY 4.0 & \cite{Rottger2021} & ✓ & ✓ & ✓ & ✓ & ✓ \\
Hatemoji Check & Prompts & Statements & CC BY 4.0 & \cite{Kirk2022} & ✓ & ✓ & ✓ & ✓ & ✓ \\
SafeText & Prompts & Statements & MIT & \cite{Levy2022} & ✓ & ✓ & ✓ & ✓ & ✓ \\
ToxiGen & Prompts & Statements & MIT & \cite{Hartvigsen2022} & ✓ & ✓ & ✓ & ✓ & ✓ \\
AART & Prompts & Mixed & CC BY 4.0 & \cite{Radharapu2023} & ✓ & ✓ & ✓ & ✓ & ✓ \\
OpenAI Moderation Dataset & Prompts & Mixed & MIT & \cite{Markov2023} & ✓ & ✓ & ✓ & ✓ & ✓ \\
SimpleSafetyTests & Prompts & Mixed & CC BY 4.0 & \cite{Vidgen2023} & ✓ & ✓ & ✓ & ✓ & ✓ \\
Toxic Chat & Prompts & Mixed & CC BY-NC 4.0 & \cite{Lin2023} & ✓ & ✓ & ✓ & ✓ & ✓ \\
BeaverTails 330k & Conversations & Single-Turn & MIT & \cite{Ji2023} & × & × & × & × & × \\
Bot-Adversarial Dialogue & Conversations & Multi-Turn & Apache 2.0 & \cite{Xu2021} & × & × & × & × & × \\
ConvAbuse & Conversations & Multi-Turn & CC BY 4.0 & \cite{Curry2021} & × & × & × & × & × \\
DICES 350 & Conversations & Multi-Turn & CC BY 4.0 & \cite{Aroyo2023} & × & × & × & × & × \\
DICES 990 & Conversations & Multi-Turn & CC BY 4.0 & \cite{Aroyo2023} & × & × & × & × & × \\
HarmfulQA & Conversations & Multi-Turn & Apache 2.0 & \cite{BhardwajPoria2023} & × & × & × & × & × \\
ProsocialDialog & Conversations & Multi-Turn & CC BY 4.0 & \cite{Kim2022} & × & × & × & × & × \\
\bottomrule
\end{tabular}%
}
\end{table}

\end{document}
=====END FILE=====