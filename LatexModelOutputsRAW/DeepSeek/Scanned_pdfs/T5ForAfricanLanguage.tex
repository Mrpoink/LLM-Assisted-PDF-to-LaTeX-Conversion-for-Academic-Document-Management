=====FILE: main.tex=====
\documentclass[10pt,twocolumn]{article}

\usepackage{graphicx}
\usepackage{amsmath}
\usepackage{amssymb}
\usepackage{booktabs}
\usepackage{array}
\usepackage{xcolor}
\usepackage{url}
\usepackage{hyperref}
\usepackage{geometry}
\geometry{letterpaper, margin=0.75in}

\title{Enhancing Pretraining Data Quality for African Languages: \\ The WURA Corpus and AfriTeVa V2 Model}
\author{
1 University of Waterloo \\
2 University of Washington \\
3 Masakhane \\
4 University College London \\
\texttt{aooladip@uwaterloo.ca}
}
\date{}

\begin{document}
\maketitle

\begin{abstract}
In this study, we highlight the importance of enhancing the quality of pretraining data in multilingual language models. Existing web crawls have demonstrated quality issues, particularly in the context of low-resource languages. Consequently, we introduce a new multilingual pretraining corpus for 16 African languages, designed by carefully auditing existing pretraining corpora to understand and rectify prevalent quality issues. To compile this dataset, we undertake a rigorous examination of current data sources for thirteen languages within one of the most extensive multilingual web crawls, mC4, and extract cleaner data through meticulous auditing and improved web crawling strategies. Subsequently, we pretrain a new T5-based model on this dataset and evaluate its performance on multiple downstream tasks. Our model demonstrates better downstream effectiveness over existing pretrained models across four NLP tasks, underscoring the critical role data quality plays in pretraining language models in low-resource scenarios. Specifically, on cross-lingual QA evaluation, our new model is more than twice as effective as multilingual T5. All code, data and model are publicly available at \url{https://github.com/castorini/AfriTeVa-keji}.
\end{abstract}

\section{Introduction}
As language models have scaled up in size and multilingual capability in recent years, commensurate effort has followed to curate pretraining data (Raffel et al., 2020) to support this growth and improve the alignment of language models.
Earlier multilingual models such as mBERT (Devlin et al., 2019) and XLM-R (Conneau et al., 2019) were trained on monolingual data from Wikipedia and/or other large-scale web crawls which included only a few African languages. The introduction of mC4 (Xue et al., 2021), a document-level dataset spanning 101 languages helped alleviate this coverage gap. However, previous work (Kreutzer et al., 2022) has shown that mC4 and other existing large-scale pretraining corpora have numerous quality issues, particularly for the low-resource African languages they contain.

Against this backdrop, indigenous efforts to build language resources for African languages have converged to two approaches: (1) Small high-quality data (e.g., 1GB) pretraining where most data are from the clean or verified sources like news domain (Ogueji et al., 2021). (2) Large aggregation of all available data (e.g., 15 - 42 GB) from noisy or unverified sources like CC-100 (Conneau et al., 2020), and mC4, combined with high-quality sources like news corpora (Adelani et al., 2022; Alabi et al., 2022; Adebara et al., 2022).

This tradeoff between quantity and quality is forced by the unavailability of large, quality pretraining data for African languages. Motivated by this need, we introduce a new multilingual pretraining corpus in 20 African languages. We draw from Kreutzer et al. (2022)'s audit of existing pretraining corpora to understand prevailing quality issues. For mC4, they cite a high ratio both of sentences in incorrect languages (15.98\% average) and nonlinguistic content (11.40\% average). We trace these issues to the quality of data sources used in mC4 for the languages in our study and design heuristics to effectively extract clean monolingual text.

More notably, we demonstrate how large-scale web crawls and document-level datasets, such as mC4, can be enhanced through meticulous auditing of their document sources i.e., base URLs (e.g., www.voahausa.com). Interestingly, for numerous credible sources, mC4 encompasses fewer documents than what is actually available. We conduct our own web crawl of these sources, collecting more documents than what is present in mC4 for the respective languages. We consolidate the result of our efforts (cleaning and crawling) with data from other sources, notably Wikipedia, and include four high-resource languages - Arabic, English, French \& Portuguese.

To evaluate the quality of our new corpus, we pretrain a new T5-based LM on the collected dataset and benchmark its performance on multiple downstream tasks. Our model demonstrates improved effectiveness over existing pretrained LMs further highlighting the importance of carefully curated datasets for pretraining language models in low-resource scenarios. Our model was significantly better than the baseline mT5 models across four different downstream tasks. Specifically, on cross-lingual QA evaluation, our new model achieves more than double the performance of multilingual T5.

\section{WURA Dataset}
We present WURA, a multilingual dataset comprising 16 African languages and 4 high-resource languages popularly spoken on the African continent - Arabic, English, French, and Portuguese.

The curation of WURA was carried out in a three-part process: - (i) Auditing and cleaning mC4 (ii) Crawling indigenous websites and (iii) Combination with existing language resources.

\subsection{Auditing and Cleaning mC4}
\subsubsection{Language Contamination}
Kreutzer et al. (2022) reports mC4's high ratio of non-linguistic content and sentences in incorrect languages, with African languages being of particular concern. The authors report significant loss (up to \(50\%\)) in recall of correct in-language sentences as they increased precision of their automatic language classification.

Our manual audit of mC4 corroborates the documented issues. We highlight three important findings: (1) The distribution of mC4 document sources has a long tail. Many individual news publications yield thousands of documents in the mC4. (2) Documents from news publications are more likely to be of higher quality i.e., both in-language and grammatical compared to documents from other web sources. (3) Some documents are from websites which translate content using online translation tools. Such documents are often a mix of in-language and noisy or non-linguistic text, and may best be filtered at sentence-level. Noting all of these issues and findings, we filter at three levels:

\textbf{Corpus-level.} We first rank unique websites in descending order of the number of documents they contribute to the mC4 corpus for each language. Then, we select the top \(20\%\) of websites for each language and collect documents sourced from websites in this list. This preserves high potential sources for further document level filtering.

\textbf{Document-level.} At document level, we filter out documents that do not contain at least 5 stopwords in them (Caswell et al., 2020) using stopwords from Stopword Lists for African Languages dataset.\footnote{[MISSING]}

\textbf{Passage-level.} After document-level filtering, we chunk the dataset into passages of roughly 512 tokens. Finally, we filter out passages that contain fewer than 4 unique words or contain repetition for more than \(20\%\) of its word length; have more than \(40\%\) of its characters are numeric or contain markers of possibly offensive content such as included in the Toxicity-200 dataset (NLLB Team et al., 2022) for the relevant language.

While Kreutzer et al. (2022)'s audit of mC4 did not yield a significant amount of offensive content (\(0.06\%\) of sentences they audited) and our web crawls mainly focused on verified news publications, these filters ensure that non-linguistic and offensive contents are removed at the passage level.

\subsubsection{mC4 is a Great Source!}
Xue et al. (2021)'s inclusion of the URL each document is sourced from makes the mC4 corpus even more useful as a data source. Commonly, multiple articles are collected from the same base website, e.g., news publications. For many news publications that provide a sitemap, we find that there are fewer articles in mC4 than is actually available on the websites. Further, mC4 only covers up to August, 2020 so updating the crawls up to the current day yields more data.

We initiate focused crawls for such websites and this leads to significant increase (\(>100\%\) for Hausa and Somali) in the amount of articles available per language. For all languages we consider except Chichewa, Sesotho, Xhosa and Zulu, we collect 1.39M articles (see Table \ref{tab:dataset_stats}) from credible sources found in mC4.

\subsection{Combination with Existing Language Resources and Non-African Languages}
Following previous works (Alabi et al., 2022; Adebara et al., 2022), we include certain non-African languages in our pretraining data. Specifically, we include over 240,000 articles newly crawled from 10 African news websites reporting in English, French and Portuguese. We also include a sample of 1.5M Wikipedia articles for English and French, as well as Wikipedia articles written in Egyptian Arabic. For the African languages, we include all Wikipedia articles. Finally, we deduplicate using the document URLs. In doing this, we prioritize news articles in our focused crawls over their existing counterparts in mC4.

\textbf{Final Dataset Statistics} Table \ref{tab:dataset_stats} presents a statistical summary of our dataset. The combined dataset from crawling, combining with existing sources and deduplication amounts to \(\sim 30\mathrm{GB}\) of data across all languages and \(\sim 19\mathrm{GB}\) for African languages.

\section{Experimental Setup}
\subsection{Model}
Using t5x and seqio (Roberts et al., 2022), we pretrain a T5 (Shazeer, 2020; Raffel et al., 2020) model with a subword-tokenizer of vocabulary size 150,000. We pretrain for 524,288 steps on the span-corruption objective using the Adafactor optimizer. Each training batch consists of 512 examples, each with an input of 512 tokens and an output of 114 tokens. Our new model is known as AfriTeVa V2, a 428M parameter model.

\subsection{Downstream Tasks}
\subsubsection{Cross-lingual Question Answering}
We evaluated our models on the test set of AfriQA Ogundepo et al. (2023), a cross-lingual question answering dataset with questions in 10 African languages and gold passages in English or French. We evaluated in zero-shot generative cross-lingual QA settings using in-lang queries and the provided gold passages in English.

\subsubsection{Machine Translation}
We evaluated using MAFAND-MT (Adelani et al., 2022) - a machine translation benchmark in the news domain. MAFAND-MT contains few thousand parallel training sentences (2,500-30,000 sentences) for 16 African languages, ideal for evaluating the effective adaptation of pretrained LMs to new languages and domains.

\subsubsection{Summarization}
For summarization, we use XL-Sum (Hasan et al., 2021), an abstractive summarization dataset which covers 44 languages, including 9 African languages. The authors establish strong baselines on both low and high-resource languages in the dataset through multilingual finetuning of mT5.

\subsubsection{Text Classification}
We use the news topic classification dataset recently introduced by Adelani et al. (2023) for 16 African languages, MasakhNews. The authors establish multiple baselines on the dataset using both classical machine learning models and finetuning or prompting language models.

\subsection{Baseline Models}
We compare our new model, AfriTeVa V2, with the base variants of existing multilingual T5 models: mT5 (Xue et al., 2021), ByT5 (Xue et al., 2022) and FlanT5 (Chung et al., 2022), as well as African metric models: AfriTeVa (Ogundepo et al., 2022), AfriMT5 \& AfriByT5 (Adelani et al., 2022).

mT5 was pretrained on the mC4 corpus which is the starter point for this work while ByT5 is the byte-level adaptation of the mT5 model. FlanT5 is T5 instruction-finetuned for improved performance. AfriTeVa, AfriMT5 and AfriByT5 models provide a closer comparison given the nature and focus of our research. While AfriTeVa is a T5 model pretrained on a small corpus (\(\sim 1\mathrm{GB}\)), AfriMT5 \& AfriByT5 are adapted from mT5 and ByT5 models using continual pretraining. Apart from AfriTeVa, AfriTeVa V2 has \(\sim 26\%\) less parameters than the other baseline models.

\section{Result and Discussion}
\subsection{Downstream Performance}
In this section, we compare AfriTeVa V2 to baseline models on selected tasks. For each downstream task, we evaluate under the same conditions. We performed per-language finetuning for machine translation \& text classification, multilingual finetuning over 35K steps for summarization.

\subsubsection{Cross-lingual Question Answering:}
AfriTeVa V2 achieves very impressive results in the cross-lingual question-answering task, especially for languages in our pretraining data. We finetune on the train set of Squad 2.0 (Rajpurkar et al., 2016) dataset and evaluate the models performance on the test set AfriQA. We compare performance on generative gold passage answer prediction, with in-language queries and English passages. Table \ref{tab:qa_results} shows that AfriTeVa V2 achieves much better F1 scores and Exact Match accuracies (\(\sim 2\times\)) across 6 out of 7 languages compared to using mT5-Base as the backbone model.

\subsubsection{Machine Translation}
We observe higher BLEU scores when translating from African languages into English than in the reverse direction. According to Table \ref{tab:mt_results}, we achieve a better score on average, topping mT5 and AfriMT5 base models by \(\sim 1 - 3\) points. While both ByT5-style models show greater effectiveness over the mT5 models, AfriTeVa V2 consistently improves over both results for all languages except ibo and pcm, an English-based creole language.

\subsubsection{Summarization}
We perform multilingual training for 35,000 steps and sample each batch from a single language. Table \ref{tab:xl_sum_results} shows we match the performance of mT5 on orm \& pcm and gain improvements over baseline Rouge scores for the other languages we consider, with yor benefiting the most.

\subsubsection{Text Classification}
Our results for the news classification task are presented in Table \ref{tab:news_class_results}. We finetune AfriTeVa V2 on MasakhaNews for each language, framing it as a text-to-text task by predicting the class of each article in the decoding sequence and report results of 3 random seeds. On average, AfriTeVa V2 yields better F1 scores across all languages and has the best F1 score on 10 out of 16 languages.

\subsection{Discussion}
\subsubsection{Results for Nigerian Pidgin}
AfriTeVa V2 does not outperform baselines for text classification, machine translation and summarization on Nigerian Pidgin (pcm). We note that AfriTeVa V2 was not pretrained on Nigerian Pidgin. As Nigerian Pidgin is an English-based creole, models pretrained on large amounts of English text are expected to be performant for the language. However, AfriTeVa V2 was pretrained on far less English text than the baselines we compare to, save for AfriTeVa. Still, we obtains results for Nigerian Pidgin that are competitive with the best baselines across the evaluation tasks.

\subsubsection{Impact of Data Quality on LMs}
Previous works have shown the correlation between the quality of the data used in pretraining a model and the performance of the trained model (Rae et al., 2021; Kreutzer et al., 2022; Hernandez et al., 2022). AfriTeVa V2's improvement over baselines in downstream tasks suggests that this is true. We note that AfriTeVa V2 outperforms the larger AfriMT5 \& AfriByT5 (Alabi et al., 2022) which were trained on unfiltered mC4 corpus.

However, our pretraining dataset, WURA, contains \(\sim 1.5\times\) more data than mC4 contains across 16 African languages. Thus, more experiments are needed to separate the effects of scale from that of data quality.

\section{AfriTeVa V2 Large Model}
We also pre-train a large variant of AfriTeVa V2 using the same configuration of the T5-large model except for the vocabulary size which we set to be 150,000, similar to the configuration of AfriTeVa V2 (base) as detailed in subsection 3.1. We present the effectiveness of scaling to a large model size on summarization and news topic classification tasks in Appendix C.4

\section{Related Work}
Absence of a large monolingual corpus has always been the major challenge of leveraging the benefits of self-supervised pretraining for building representation and language models for African languages. The most available corpus are mostly from religious corpus like Bible (Resnik et al., 1999) or JW300 (Agic and Vulić, 2019), Wikipedia and Common Crawl archive. The latter often has significant quality issues (Kreutzer et al., 2022).

Earlier works on building word representation models for African languages showed the importance of developing FastText embeddings with small high-quality data (Alabi et al., 2020) over pretrained FastText embeddings developed from noisier common crawl data. Obtaining such high-quality data is tedious since it involved curating several verified sources manually. Thus, previous works have prioritized filtering of the common crawl data to produce better quality dataset for pretraining (Conneau et al., 2020; Ortiz Suárez et al., 2019; Xue et al., 2021; Bapna et al., 2022). However, quality issues still persist in those filtered corpora. An alternative to this is basically aggregating high quality data for African languages mostly from verified sources (Ogueji et al., 2021; Leong et al., 2022; Palen-Michel et al., 2022). However, this often results in smaller sized corpus.

The current models with impressive performance on African languages simply aggregate both low-quality data and high-quality data for pretraining (Alabi et al., 2022; Adebara et al., 2022). The quality of these models implies that there must be significant portions of the data that are of good quality. To this end, we systematically and rigorously filtered these low-quality data from mC4 corpus for African languages, similar to the OSCAR dataset approach. To the best of our knowledge, no previous work has done this. OSCAR dataset only has few documents for African languages e.g., 37.2MB for Afrikaans dataset while our filtered corpus has more than 4.5 GB.

\section{Conclusion}
In this work, we look to address the lack of large, quality pretraining dataset for African languages. While previous works have highlighted quality issues in existing pretraining dataset such as mC4, we demonstrate how these datasets can be enhanced by auditing their document sources and incorporating rigorous data filtering methods. To highlight the effectiveness of our approach and the relevance of this new dataset, we train a new T5 model, AfriTeVa V2, on our dataset. Our experiments show significant improvements across existing NLP benchmarks for African languages underscoring the impact of qualitative pretraining data in training language models.

\section{Limitations}
The representativeness of our dataset poses a potential limitation. Despite our efforts to collect data from multiple African news websites, it is possible that our dataset does not fully capture the breadth and diversity of African news articles. The reliance on specific websites and the utilization of the mC4 dataset, along with existing corpora, may introduce inherent bias that our work does not address.

Furthermore, our implementation of several-level filtering techniques, including the removal of non-linguistic content in the target language, does not guarantee the complete removal of all text in different languages or other toxic contents that may be present in the existing corpus.

Lastly, we acknowledge the need for future work to include more African languages. Our dataset only covers 16 languages, limiting the generalizability of our findings across the wide range of languages spoken in Africa.

\section{Acknowledgements}
This research was supported in part by the Natural Sciences and Engineering Research Council (NSERC) of Canada and an AI for Social Good grant from the Waterloo AI Institute. Computational resources were provided by Compute Ontario and Compute Canada. We also thank the Google TRC program for providing us free cloud TPU access.

\bibliographystyle{unsrt}
\begin{thebibliography}{100}

\bibitem{abadji2022}
Julien Abadji, Pedro Ortiz Suarez, Laurent Romary, and Benoit Sagot.
\newblock Towards a Cleaner Document-Oriented Multilingual Crawled Corpus.
\newblock \emph{ArXiv}, abs/2201.06642, 2022.

\bibitem{adebara2022}
Ife Adebara, AbdelRahim Elmadany, Muhammad Abdul-Mageed, and Alcides Alcoba Inciarte.
\newblock SERENGETI: Massively Multilingual Language Models for Africa.
\newblock \emph{ArXiv}, abs/2212.10785, 2022.

\bibitem{adelani2022}
David Ifeoluwa Adelani, Jesujoba Oluwadara Alabi, Angela Fan, Julia Kreutzer, Xiaoyu Shen, Machel Reid, Dana Ruiter, Dietrich Klakow, Peter Nabende, Ernie Chang, Tajuddeen Rabiu Gwadabe, Freshia Sackey, Bonaventure F. P. Dossou, Chris C. Emezue, Colin Leong, Michael Beukman, Shamsuddeen Hassan Muhammad, Guyo Dub Jarso, Oreen Yousuf, Andre Niyongabo Rubungo, Gilles Hacheme, Eric Peter Wairagala, Muhammad Umair Nasir, Benjamin Ayoade Ajibade, Tunde Oluwaseyi Ajayi, Yvonne Wambui Gitau, Jade Z. Abbott, Mohamed Ahmed, Millicent A. Ochieng, Anuoluwapo Aremu, Perez Ogayo, Jonathan Mukiibi, Fatoumata Ouoba Kabore, Godson Kalipe, Derguene Mbaye, Allahsera Auguste Tapo, Victoire Memdjokam Koagne, Edwin Munkoh-Buabeng, Valencia Wagner, Idris Abdulmumin, Ayodele Awokoya, Happy Buzaaba, Blessing K. Sibanda, Andiswa Bukula, and Sam Manthalu.
\newblock A Few Thousand Translations Go a Long Way! Leveraging Pre-trained Models for African News Translation.
\newblock In \emph{North American Chapter of the Association for Computational Linguistics}, 2022.

\bibitem{adelani2023}
David Ifeoluwa Adelani, Marek Masiak, Israel Abebe Azime, Jesujoba Oluwadara Alabi, Atnafu Lambebo Tonja, Christine Mwase, Odunayo Ogundepo, Bonaventure F. P. Dossou, Akintunde Oladipo, Doreen Nixdorf, Chris C. Emezue, Sana Sabah Al-Azzawi, Blessing K. Sibanda, Davis David, Lolwethu Ndolela, Jonathan Mukiibi, Tunde Oluwaseyi Ajayi, Tatiana Moteu Ngoli, Brian Odhiambo, Abraham Toluwase Owodunni, Nnaemeka C. Obiefuna, Shamsuddeen Hassan Muhammad, Saheed Salahudeen Abdullahi, Mesay Gemeda Yigezu, Tajuddeen Rabiu Gwadabe, Idris Abdulmumin, Mahlet Taye Bame, Oluwabusayo Olufunke Awoyomi, Iyanuoluwa Shode, Tolulope Anu Adelani, Habiba Abdulganiy Kailani, Abdul-Hakeem Omotayo, Adetola Adeeko, Afolabi Abeeh, Anuoluwapo Aremu, Olanrewaju Samuel, Clemencia Siro, Wangari Kimotho, Onyekachi Raphael Ogbu, Chinedu E. Mbonu, Chiamaka Ijeoma Chukwuneke, Samuel Fanijo, Jessica Ojo, Oyinkanola F. Awosan, Tadesse Kebede Guge, Sakayo Toadoum Sari, Pamela Nyatsine, Freedmore Sidume, Oreen Yousuf, Mardiyyah Oduwole, Ussen Kimanuka, Kanda Patrick Tshinu, Thina Diko, Siyanda Nxakama, Abdulmejid Tuni Johar, Sinodos Gebre, Muhidin A. Mohamed, Shafie Abdi Mohamed, Fuad Mire Hassan, Mages Ahmed Mehamed, Evrard Ngabire, and Pontus Stenetorp.
\newblock MasakhaNEWS: News Topic Classification for African languages.
\newblock \emph{ArXiv}, abs/2304.09972, 2023.

\bibitem{agic2019}
Zeljko Agic and Ivan Vulić.
\newblock JW300: A Wide-Coverage Parallel Corpus for Low-Resource Languages.
\newblock In \emph{Proceedings of the 57th Annual Meeting of the Association for Computational Linguistics}, pages 3204-3210, Florence, Italy. Association for Computational Linguistics, 2019.

\bibitem{ahia2023}
Orevaoghene Ahia, Sachin Kumar, Hila Gonen, Jungo Kasai, David R. Mortensen, Noah A. Smith, and Yulia Tsvetkov.
\newblock Do All Languages Cost the Same? Tokenization in the Era of Commercial Language Models.
\newblock \emph{ArXiv}, abs/2305.13707, 2023.

\bibitem{alabi2020}
Jesujoba Alabi, Kwabena Amponsah-Kaakyire, David Adelani, and Cristina Espana-Bonet.
\newblock Massive vs. Curated Embeddings for Low-Resourced Languages: The Case of Yoruba and Twi.
\newblock In \emph{Proceedings of the Twelfth Language Resources and Evaluation Conference}, pages 2754-2762, Marseille, France. European Language Resources Association, 2020.

\bibitem{alabi2022}
Jesujoba Oluwadara Alabi, David Ifeoluwa Adelani, Marius Mosbach, and Dietrich Klakow.
\newblock Adapting Pre-trained Language Models to African Languages via Multilingual Adaptive Fine-Tuning.
\newblock In \emph{International Conference on Computational Linguistics}, 2022.

\bibitem{bapna2022}
Ankur Bapna, Isaac Caswell, Julia Kreutzer, Orhan Firat, Daan van Esch, Aditya Siddhant, Mengmeng Niu, Pallavi N. Baljekar, Xavier Garcia, Wolfgang Macherey, Theresa Breiner, Vera Axelrod, Jason Riesa, Yuan Cao, Mia Xu Chen, Klaus Macherey, Maxim Krikun, Pidong Wang, Alexander Gutkin, Apurva Shah, Yanping Huang, Z. Chen, Yonghui Wu, and Macduff Hughes.
\newblock Building Machine Translation Systems for the Next Thousand Languages.
\newblock \emph{ArXiv}, abs/2205.03983, 2022.

\bibitem{caswell2020}
Isaac Caswell, Theresa Breiner, Daan van Esch, and Ankur Bapna.
\newblock Language ID in the Wild: Unexpected Challenges on the Path to a Thousand-Language Web Text Corpus.
\newblock \emph{ArXiv}, abs/2010.14571, 2020.

\bibitem{chung2022}
Hyung Won Chung, Le Hou, S. Longpre, Barret Zoph, Yi Tay, William Fedus, Eric Li, Xuezhi Wang, Mostafa Dehghani, Siddhartha Brahma, Albert Webson, Shixiang Shane Gu, Zhuyun Dai, Mirac Suzgun, Xinyun Chen, Aakanksha Chowdhery, Dasha Valter, Sharan Narang, Gaurav Mishra, Adams Wei Yu, Vincent Zhao, Yanping Huang, Andrew M. Dai, Hongkun Yu, Slav Petrov, Ed Huai hsin Chi, Jeff Dean, Jacob Devlin, Adam Roberts, Denny Zhou, Quoc V. Le, and Jason Wei.
\newblock Scaling Instruction-Finetuned Language Models.
\newblock \emph{ArXiv}, abs/2210.11416, 2022.

\bibitem{conneau2019}
Alexis Conneau, Kartikay Khandelwal, Naman Goyal, Vishrav Chaudhary, Guillaume Wenzek, Francisco Guzman, Edouard Grave, Myle Ott, Luke Zettlemoyer, and Veselin Stoyanov.
\newblock Unsupervised Cross-lingual Representation Learning at Scale.
\newblock In \emph{Annual Meeting of the Association for Computational Linguistics}, 2019.

\bibitem{conneau2020}
Alexis Conneau, Kartikay Khandelwal, Naman Goyal, Vishrav Chaudhary, Guillaume Wenzek, Francisco Guzman, Edouard Grave, Myle Ott, Luke Zettlemoyer, and Veselin Stoyanov.
\newblock Unsupervised Cross-lingual Representation Learning at Scale.
\newblock In \emph{Proceedings of the 58th Annual Meeting of the Association for Computational Linguistics}, pages 8440-8451, Online. Association for Computational Linguistics, 2020.

\bibitem{devlin2019}
Jacob Devlin, Ming-Wei Chang, Kenton Lee, and Kristina Toutanova.
\newblock BERT: Pre-training of Deep Bidirectional Transformers for Language Understanding.
\newblock In \emph{North American Chapter of the Association for Computational Linguistics}, 2019.

\bibitem{dione2023}
Cheikh M. Bamba Dione, David Ifeoluwa Adelani, Peter Nabende, Jesujoba Oluwadara Alabi, Thapelo Sindane, Happy Buzaaba, Shamsuddeen Hassan Muhammad, Chris C. Emezue, Perez Ogayo, Anuoluwapo Aremu, Catherine Gitau, Derguene Mbaye, Jonathan Mukiibi, Blessing K. Sibanda, Bonaventure F. P. Dossou, Andiswa Bukula, Rooweither Mabaya, Allahsera Auguste Tapo, Edwin Munkoh-Buabeng, Victoire Memdjokam Koagne, Fatoumata Ouoba Kabore, Amelia Taylor, Godson Kalipe, Tebogo Macucwa, Vukosi Marivate, Tajuddeen Rabiu Gwadabe, Mboning Tchiaze Elvis, Ikechukwu E. Onyenne, Gration Gualbert Atindogbe, Tolulope Anu Adelani, Idris Akinade, Olanrewaju Samuel, Marie-Rosette Nahimana, Théogene Musabeyezu, Emile Niyomutabazi, Ester Chimhenga, Kudzai Gotosa, Patrick Miza, Apelete Agbolo, Seydou T. Traore, Chinedu Uchechukwu, Aliyu Yusuf, Muhammad Abubakar Abdullahi, and Dietrich Klakow.
\newblock MasakhapOS: Part-of-Speech Tagging for Typologically Diverse African Languages.
\newblock In \emph{Annual Meeting of the Association for Computational Linguistics}, 2023.

\bibitem{lample2019}
Guillaume Lample and Alexis Conneau.
\newblock Crosslingual Language Model Pretraining.
\newblock In \emph{Neural Information Processing Systems}, 2019.

\bibitem{hasan2021}
Tahmid Hasan, Abhik Bhattacharjee, Md. Saiful Islam, Kazi Samin, Yuan-Fang Li, Yong-Bin Kang, M. Sohel Rahman, and Rifat Shahriyar.
\newblock XL-Sum: Large-Scale Multilingual Abstractive Summarization for 44 Languages.
\newblock In \emph{Findings}, 2021.

\bibitem{hernandez2022}
Danny Hernandez, Tom Brown, Tom Conerly, Nova DasSarma, Dawn Drain, Sheer El-Showk, Nelson Elhage, Zac Hatfield-Dodds, Tom Henighan, Tristan Hume, et al.
\newblock Scaling Laws and Interpretability of Learning from Repeated Data.
\newblock \emph{ArXiv}, abs/2205.10487, 2022.

\bibitem{kreutzer2022}
Julia Kreutzer, Isaac Caswell, Lisa Wang, Ahsan Wahab, Daan van Esch, Nasanbayar Ulzii-Orshikh, Allahsera Tapo, Nishant Subramani, Artem Sokolov, Clayton Sikasote, Monang Setyawan, Supehakmungkol Sarin, Sokhar Samb, Benoît Sagot, Clara Rivera, Annette Rios, Isabel Papadimitriou, Salomey Osei, Pedro Ortiz Suarez, Iroro Orife, Kelechi Ogueji, Andre Niyongabo Rubungo, Toan Q. Nguyen, Mathias Müller, André Müller, Shamsuddeen Hassan Muhammad, Nanda Muhammad, Ayanda Mnyakeni, Jamshidbek Mirzakhalov, Tapiwanashe Matangira, Colin Leong, Nze Lawson, Sneha Kudugunta, Yacine Jernite, Mathias Jenny, Orhan Firat, Bonaventure F. P. Dossou, Sakhile Dlamini, Nisansa de Silva, Sakine Çabuk Balli, Stella Biderman, Alessia Battisti, Ahmed Baruwa, Ankur Bapna, Pallavi Baljekar, Israel Abebe Azime, Ayodele Awokoya, Duygu Ataman, Orevaoghene Ahia, Oghenefego Ahia, Sweta Agrawal, and Mofetoluwa Adeyemi.
\newblock Quality at a Glance: An Audit of Web-Crawled Multilingual Datasets.
\newblock \emph{Transactions of the Association for Computational Linguistics}, 10:50-72, 2022.

\bibitem{kudo2018}
Taku Kudo and John Richardson.
\newblock SentencePiece: A Simple and Language Independent Subword Tokenizer and Detokenizer for Neural Text Processing.
\newblock In \emph{Conference on Empirical Methods in Natural Language Processing}, 2018.

\bibitem{leong2022}
Colin Leong, Joshua Nemecek, Jacob Mansdorfer, Anna Filighera, Abraham Owodunni, and Daniel Whitenack.
\newblock Bloom Library: Multimodal Datasets in 300+ Languages for a Variety of Downstream Tasks.
\newblock In \emph{Proceedings of the 2022 Conference on Empirical Methods in Natural Language Processing}, pages 8608-8621, Abu Dhabi, United Arab Emirates. Association for Computational Linguistics, 2022.

\bibitem{nllb2022}
NLLB Team, Marta R. Costa-jussà, James Cross, Onur Çelebi, Maha Elbayad, Kenneth Heafield, Kevin Heffernan, Elahe Kalbassi, Janice Lam, Daniel Licht, Jean Maillard, Anna Sun, Skyler Wang, Guillaume Wenzek, Al Youngblood, Bapi Akula, Loic Barrault, Gabriel Mejia-Gonzalez, Prangthip Hansanti, John Hoffman, Semarley Jarrett, Kaushik Ram Sadagopan, Dirk Rowe, Shannon Spruit, Chau Tran, Pierre Andrews, Necip Fazil Ayan, Shruti Bhosale, Sergey Edunov, Angela Fan, Cynthia Gao, Vedanuj Goswami, Francisco Guzmán, Philipp Koehn, Alexandre Mourachko, Christophe Ropers, Safiyyah Saleem, Holger Schwenk, and Jeff Wang.
\newblock No language left behind: Scaling human-centered machine translation.
\newblock \emph{ArXiv}, abs/2207.04672, 2022.

\bibitem{ogueji2021}
Kelechi Ogueji, Yuxin Zhu, and Jimmy J. Lin.
\newblock Small Data? No Problem! Exploring the Viability of Pretrained Multilingual Language Models for Low-resourced Languages.
\newblock \emph{Proceedings of the 1st Workshop on Multilingual Representation Learning}, 2021.

\bibitem{ogundepo2023}
Odunayo Ogundepo, Tajuddeen R. Gwadabe, Clara E. Rivera, Jonathan H. Clark, Sebastian Ruder, David Ifeoluwa Adelani, Bonaventure F. P. Dossou, Abdou Aziz DIOP, Claytone Sikasote, Gilles Hacheme, Happy Buzaaba, Ignatius Ezeani, Rooweither Mabuyia, Salomey Osei, Chris Emezue, Albert Njoroge Kahira, Shamsuddeen H. Muhammad, Akintunde Oladipo, Abraham Toluwase Owodunni, Atnafu Lambebo Tonja, Iyanuoluwa Shode, Akari Asai, Tunde Oluwaseyi Ajayi, Clementia Siro, Steven Arthur, Mofetoluwa Adeyemi, Orevaoghene Ahia, Aremu Anuoluwapo, Oyinkansola Awosan, Chiamaka Chukwuneke, Bernard Opoku, Awokoya Ayodele, Verrah Otende, Christine Mwase, Boyd Sinkala, Andre Niyongabo Rubungo, Daniel A. Ajisafe, Emeka Felix Onwuegbuzia, Habib Mbow, Emile Niyomutabazi, Eunice Mukonde, Falalu Ibrahim Lawan, Ibrahim Said Ahmad, Jesujoba O. Alabi, Martin Namukombo, Mbonu Chinedu, Mofya Phiri, Neo Putini, Ndumiso Mngoma, Priscilla A. Amuok, Ruqayya Nasir Iro, and Sonia Adhiambo.
\newblock AfriQA: Cross-lingual Open-Retrieval Question Answering for African Languages, 2023.

\bibitem{ogundepo2022}
Odunayo Jude Ogundepo, Akintunde Oladipo, Mofetoluwa Adeyemi, Kelechi Ogueji, and Jimmy Lin.
\newblock AfriTeVA: Extending Small Data Pretraining Approaches to Sequence-to-Sequence Models.
\newblock In \emph{Proceedings of the Third Workshop on Deep Learning for Low-Resource Natural Language Processing}, pages 126-135, 2022.

\bibitem{ortiz2019}
Pedro Javier Ortiz Suárez, Benoît Sagot, and Laurent Romary.
\newblock Asynchronous Pipelines For Processing Huge Corpora on Medium to Low-Resource Infrastructures.
\newblock \emph{Proceedings of the Workshop on Challenges in the Management of Large Corpora (CMLC-7) 2019. Cardiff, 22nd July 2019}, pages 9 - 16, Mannheim. Leibniz-Institut für Deutsche Sprache, 2019.

\bibitem{palen2022}
Chester Palen-Michel, June Kim, and Constantine Lignos.
\newblock Multilingual Open Text Release 1: Public Domain News in 44 Languages.
\newblock In \emph{Proceedings of the Thirteenth Language Resources and Evaluation Conference}, pages 2080-2089, Marseille, France. European Language Resources Association, 2022.

\bibitem{petrov2023}
Aleksandar Petrov, Emanuele La Malfa, Philip H. S. Torr, and Adel Bibi.
\newblock Language Model Tokenizers Introduce Unfairness Between Languages.
\newblock \emph{ArXiv}, abs/2305.15425, 2023.

\bibitem{rae2021}
Jack W Rae, S Borgeaud, T Cai, K Millican, J Hoffmann, HF Song, J Aslanides, S Henderson, R Ring, S Young, et al.
\newblock Scaling Language Models: Methods, Analysis \& Insights from Training Gopher (2021).
\newblock \emph{ArXiv}, abs/2112.11446, 2021.

\bibitem{raffel2020}
Colin Raffel, Noam Shazeer, Adam Roberts, Katherine Lee, Sharan Narang, Michael Matena, Yanqi Zhou, Wei Li, and Peter J Liu.
\newblock Exploring the Limits of Transfer Learning With a Unified Text-to-Text Transformer.
\newblock \emph{The Journal of Machine Learning Research}, 21(1):5485-5551, 2020.

\bibitem{rajpurkar2016}
Pranav Rajpurkar, Jian Zhang, Konstantin Lopyrev, and Percy Liang.
\newblock SQuAD: 100,000+ Questions for Machine Comprehension of Text.
\newblock In \emph{Conference on Empirical Methods in Natural Language Processing}, 2016.

\bibitem{resnik1999}
Philip Resnik, Mari Broman Olsen, and Mona T. Diab.
\newblock The Bible as a Parallel Corpus: Annotating the 'Book of 2000 Tongues'.
\newblock \emph{Computers and the Humanities}, 33:129-153, 1999.

\bibitem{roberts2022}
Adam Roberts, Hyung Won Chung, Anselm Levskaya, Gaurav Mishra, James Bradbury, Daniel Andor, Sharan Narang, Brian Lester, Colin Gaffney, Afroz Mohiuddin, Curtis Hawthorne, Aitor Lewkowycz, Alexandru Salcianu, Marc van Zee, Jacob Austin, Sebastian Goodman, Livio Baldini Soares, Haitang Hu, Sasha Tsvyashchenko, Aakanksha Chowdhery, Jasmijn Bastings, Jannis Bulian, Xavier García, Jianmo Ni, Andrew Chen, Kathleen Kenealy, J. Clark, Stephan Lee, Daniel H Garrette, James Lee-Thorp, Colin Raffel, Noam M. Shazeer, Marvin Ritter, Maarten Bosma, Alexandre Passos, Jeremy B. Maitin-Shepard, Noah Fiedel, Mark Omernick, Brennan Saeta, Ryan Sepassi, Alexander Spiridonov, Joshua Newlan, and Andrea Gesmundo.
\newblock Scaling Up Models and Data with t5x and seqio.
\newblock \emph{ArXiv}, abs/2203.17189, 2022.

\bibitem{shazeer2020}
Noam M. Shazeer.
\newblock GLU Variants Improve Transformer.
\newblock \emph{ArXiv}, abs/2002.05202, 2020.

\bibitem{xue2022}
Linting Xue, Aditya Barua, Noah Constant, Rami Al-Rfou, Sharan Narang, Mihir Kale, Adam Roberts, and Colin Raffel.
\newblock ByT5: Towards a Token-Free Future with Pre-trained Byte-to-Byte Models.
\newblock \emph{Transactions of the Association for Computational Linguistics}, 10:291-306, 2022.

\bibitem{xue2021}
Linting Xue, Noah Constant, Adam Roberts, Mihir Kale, Rami Al-Rfou, Aditya Siddhant, Aditya Barua, and Colin Raffel.
\newblock mT5: A Massively Multilingual Pre-trained Text-to-Text Transformer.
\newblock In \emph{Proceedings of the 2021 Conference of the North American Chapter of the Association for Computational Linguistics: Human Language Technologies}, pages 483-498, Online. Association for Computational Linguistics, 2021.

\bibitem{acs2019}
Judit Acs.
\newblock Exploring BERT's Vocabulary, 2019.

\end{thebibliography}

\appendix
\section{Data}
\subsection{mC4 Audit}
We aim to tease out heuristics that are guaranteed to help us quickly and reliably extract high-quality monolingual text across the African languages in mC4. First, we reduce the source URL of each document to its hostname and keep a list of unique hostnames that exist for each language. For each language, we first sample a hostname then sample 20 documents sourced from the sampled hostname. This sampling strategy not only allows to audit more documents and sources faster, it allows us trace existing quality issues to the source URLs that produced the documents. We follow non-expert auditing strategies proposed by Kreutzer et al. (2022). Additionally, we also visit the hostname URL to ascertain its purpose for speakers of the language and translate paragraphs in the document using Google Translate.

\subsection{Web Crawling}
We open-source Otelemuye, an extensible framework for large scale web-crawls. In our work, we crawl at a safe pace that does not degrade the website's performance and respect the rules websites publish in their robots.txt. Where possible, we include the category under which each article was published. This information may be useful for identification of the domains in our dataset. We also release a list of the top document URLs for each language and invite native speakers to audit these sources to help us improve the quality of WURA.

\section{Tokenization}
In multilingual settings, the design of tokenizers has great impact on the downstream utility and cost of inference of language models across languages (Petrov et al., 2023; Ahia et al., 2023). We characterize the performance of our tokenizers using fertility (Acs., 2019), defined as the number of subwords created per word (or per dataset) by the tokenizer. We compute fertility on the languages covered by MasakhanePOS (Dione et al., 2023).

We train multiple unigram language models on our dataset using Sentencepiece (Kudo and Richardson, 2018) with vocabulary sizes ranging from 100,000 to 250,000. As shown in Table \ref{tab:dataset_stats} above, our dataset sizes varies over orders of magnitude between languages. To alleviate unfair treatment of the lowest-resourced of the languages we consider, we follow Guillaume Lample and Alexis Conneau (2019) to learn the unigram language models on sentences sampled according to a multinomial distribution with probabilities \(q_{i = 1\ldots N}\) calculated as follows:

\(q_{i} = \frac{p_{i}^{\alpha}}{\sum_{j = 1}^{N}p_{j}^{\alpha}}\) where \(p_{i} = \frac{n_{i}}{\sum_{k = 1}^{N}n_{k}}\) and \(\alpha = 0.3\) \(N\) denotes the number of languages and \(n_{i}\), the number of sentences in language \(i\). We denote this as sampling configuration \(①\). We also investigate a sampling configuration \(②\) in which we further upsample languages which still do not have adequate representation after sampling sentences with the calculated probabilities. Simply, after calculating probabilities using \(①\), we upsample by a factor of 10 for ibo, kin, nya, sna, sot, tir, xho, and a factor of 5 for amh, arz, mlg, som. We make this choice of upsampling factor taking into consideration the maximum amount of data we can train with given our CPU resources. The fertility of tokenizers trained on the sentences obtained by both sampling configurations are presented in Table \ref{tab:tokenizer_fertility}.

Across both configurations \(①\) \& \(②\), we obtain the best tradeoff between fertility distributions across the languages and vocabulary size at 150,000. Tokenizers obtained from \(②\) perform better across board, improving fertility markedly for ibo, kin, nya, sna, xho, yor and zul without affecting fertility for hau and swa negatively.

\section{AfriTeVa V2 Large}
We also pretrain a large variant of AfriTeVa V2 and present its effectiveness on summarization (Table \ref{tab:xl_sum_large}) and classification (Table \ref{tab:news_large}). For summarization, we finetune both models for 10 epochs and make inference using beam search with width of 4. We gain improvements over the base model across both tasks, particularly for summarization where ibo benefits the most.

\newpage
\onecolumn
\section*{Tables}
\begin{table}[htbp]
\centering
\caption{MasakhaNews classification results: Evaluation is done using the weighted F1 score and the scores presented are averaged across 3 seeds. AfriTeVa V2 surpasses mT5-base by up to 10 points. The average scores excluding languages not in the mC4 corpus are also provided in AVGSL.}
\label{tab:news_class_results}
\small
\begin{tabular}{l*{17}{r}}
\toprule
Model & Size & amh & eng & fra & hau & ibo & lin & lug & orm & pcm & run & sna & som & swa & tir & xho & yor & AVG & AVGSL \\
\midrule
AfriTeVa-base & 229M & 87.0 & 80.3 & 71.9 & 85.8 & 79.9 & 82.8 & 60.2 & 82.9 & 95.2 & 80.0 & 84.4 & 58.0 & 80.7 & 55.2 & 69.4 & 86.4 & 77.5 & 78.4 \\
mT5-base & 580M & 78.2 & 89.8 & 59.0 & 82.7 & 76.8 & 80.8 & 75.0 & 79.2 & 96.1 & 85.7 & 90.4 & 75.0 & 76.1 & 65.1 & 71.8 & 86.2 & 79.2 & 78.6 \\
FlanT5-base & 580M & 54.5 & 92.4 & 88.9 & 84.5 & 86.6 & 90.6 & 84.1 & 85.8 & 97.8 & 87.3 & 90.6 & 76.0 & 79.0 & 41.5 & 90.8 & 88.9 & 82.5 & 83.2 \\
AfriMT5-base & 580M & 90.2 & 90.3 & 87.4 & 87.9 & 88.0 & 88.6 & 84.8 & 83.9 & 96.6 & 91.0 & 91.5 & 77.8 & 84.4 & 80.8 & 91.6 & 88.8 & 87.7 & 87.8 \\
AfriTeVa V2 & 428M & 92.8 & 90.6 & 88.0 & 89.4 & 86.1 & 86.0 & 91.1 & 90.8 & 96.8 & 92.3 & 93.3 & 75.7 & 87.0 & 86.4 & 93.6 & 92.3 & 89.5 & 88.9 \\
\bottomrule
\end{tabular}
\end{table}

\begin{table}[htbp]
\centering
\caption{MAFAND-MT results: Evaluation is done using the BLEU score and we obtain significantly better performance on average across all languages in both the en-xx and xx-en directions, except for ibo and pcm.}
\label{tab:mt_results}
\small
\begin{tabular}{l*{7}{r} *{7}{r}}
\toprule
\multirow{2}{*}{Model} & \multicolumn{7}{c}{en-xx} & \multicolumn{7}{c}{xx-en} \\
\cmidrule(lr){2-8} \cmidrule(lr){9-15}
 & hau & ibo & pcm & swa & yor & zul & AVG & hau & ibo & pcm & swa & yor & zul & AVG \\
\midrule
mT5-base & 2.8 & 18.0 & 34.1 & 25.1 & 4.8 & 11.7 & 16.1 & 5.8 & 18.9 & 42.2 & 29.5 & 12.3 & 22.4 & 21.9 \\
AfriMT5-base & 5.1 & 19.6 & 35.0 & 26.7 & 6.2 & 13.2 & 17.5 & 10.4 & 19.5 & 44.6 & 30.6 & 13.8 & 24.0 & 23.8 \\
ByT5-base & 8.3 & 21.8 & 30.1 & 24.4 & 7.5 & 14.0 & 17.7 & 12.9 & 21.0 & 39.4 & 27.1 & 11.5 & 22.8 & 22.5 \\
AfriByT5-base & 9.3 & 22.7 & 30.0 & 24.7 & 7.6 & 15.3 & 18.3 & 13.5 & 20.7 & 39.5 & 27.0 & 11.9 & 24.0 & 22.8 \\
AfriTeVa V2 & 13.4 & 20.7 & 31.1 & 28.0 & 12.1 & 15.6 & 20.3 & 16.2 & 16.7 & 40.5 & 31.0 & 17.6 & 28.4 & 25.1 \\
\bottomrule
\end{tabular}
\end{table}

\begin{table}[htbp]
\centering
\caption{XL-SUM results: Performance based on Rouge-1, Rouge-2 and Rouge-L. AfriTeVa V2 is generally more effective than mT5.}
\label{tab:xl_sum_results}
\small
\begin{tabular}{l r r r r r r r r}
\toprule
Model & hau & ibo & orm & pcm & som & swa & yor & AVG \\
\midrule
mT5 & 39.4/17.7/31.7 & 31.6/10.2/24.5 & 18.7/6.2/16.2 & 38.0/15.1/29.9 & 31.6/11.6/24.2 & 37.7/17.9/30.9 & 31.7/11.7/25.1 & 32.7/12.9/26.1 \\
AfriTeVa V2 & 41.0/18.8/32.8 & 33.4/12.7/25.6 & 18.5/6.1/16.0 & 37.7/14.6/29.1 & 33.3/12.8/26.1 & 38.1/17.8/30.9 & 38.9/16.7/29.9 & 34.4/14.2/27.2 \\
\bottomrule
\end{tabular}
\end{table}

\begin{table}[htbp]
\centering
\caption{Cross-lingual Question Answering Results: F1 and Exact Match (EM) Accuracy scores on the test set of AfriQA (Onguendo et al., 2023). For both metrics, AfriTeVa V2 outperforms mT5 except for twi.}
\label{tab:qa_results}
\small
\begin{tabular}{l r r r r r r r r}
\toprule
\multirow{2}{*}{Metric} & \multirow{2}{*}{Model} & \multicolumn{7}{c}{Language} \\
\cmidrule(lr){3-9}
 & & bem & hau & ibo & kin & twi & yor & zul & AVG \\
\midrule
\multirow{4}{*}{F1} & mT5 & 2.9 & 25.8 & 41.7 & 25.5 & 5.3 & 11.9 & 24.7 & 17.6 \\
 & AfriTeVa-Base & 3.5 & 4.6 & 5.5 & 4.8 & 5.4 & 6.1 & 4.4 & 4.9 \\
 & AfriMT5-Base & 6.4 & 39.7 & 40.7 & 30.3 & 5.3 & 21.8 & 31.9 & 25.2 \\
 & AfriTeVa V2 & 5.7 & 45.4 & 57.1 & 45.4 & 2.1 & 37.6 & 45.9 & 34.2 \\
\midrule
\multirow{4}{*}{EM} & mT5 & 1.1 & 22.3 & 34.7 & 20.2 & 3.5 & 7.8 & 20.9 & 13.9 \\
 & AfriTeVa-Base & 2.0 & 2.7 & 4.2 & 3.2 & 3.1 & 3.9 & 3.1 & 3.2 \\
 & AfriMT5-Base & 4.2 & 33.0 & 33.0 & 23.1 & 2.9 & 15.7 & 25.5 & 19.6 \\
 & AfriTeVa V2 & 5.2 & 36.7 & 47.7 & 33.7 & 1.4 & 29.5 & 37.8 & 27.4 \\
\bottomrule
\end{tabular}
\end{table}

\begin{table}[htbp]
\centering
\caption{Tokenizer Fertilities: We measure the fertilities of our tokenizers with varying vocabulary sizes using the MasakhanePOS dataset. The 150k tokenizer gives the best trade-off in size and fertility scores across all languages, especially in the second sampling configuration.}
\label{tab:tokenizer_fertility}
\small
\begin{tabular}{l l r r r r r r r r r}
\toprule
Sampling & Vocab Size & hau & ibo & kin & nya & sna & swa & xho & yor & zul \\
\midrule
Config ① & 100,000 & 1.29 & 1.62 & 1.80 & 1.90 & 1.76 & 1.24 & 2.37 & 2.05 & 2.22 \\
 & 150,000 & 1.25 & 1.53 & 1.67 & 1.74 & 1.64 & 1.21 & 2.20 & 1.97 & 2.06 \\
 & 200,000 & 1.23 & 1.49 & 1.57 & 1.67 & 1.56 & 1.19 & 2.10 & 1.92 & 1.96 \\
 & 250,000 & 1.22 & 1.47 & 1.54 & 1.63 & 1.53 & 1.19 & 2.03 & 1.90 & 1.91 \\
\midrule
Config ② & 100,000 & 1.25 & 1.43 & 1.52 & 1.65 & 1.54 & 1.29 & 2.07 & 1.67 & 1.90 \\
 & 150,000 & 1.21 & 1.39 & 1.43 & 1.51 & 1.45 & 1.25 & 1.94 & 1.59 & 1.77 \\
 & 200,000 & 1.20 & 1.37 & 1.38 & 1.45 & 1.38 & 1.23 & 1.86 & 1.55 & 1.69 \\
\bottomrule
\end{tabular}
\end{table}

\begin{table}[htbp]
\centering
\caption{WURA Dataset Statistics: We provide the count of crawled articles, Wikipedia articles, original mC4 articles, and final size before passage-level filtering for each language. In total, we have \(\sim 4.7M\) articles, more than 1.5 times what mC4 contains across 16 African languages.}
\label{tab:dataset_stats}
\footnotesize
\begin{tabular}{l r r r r r r}
\toprule
Language & \# Crawled Articles & \# Wikipedia Articles & \# mC4 Articles & \# Combined Articles & \# De-duped Articles & Size (GB) Articles \\
\midrule
\textbf{African Languages in mC4} \\
Afrikaans (afr) & 139,977 & 107,860 & 978,740 & 1,226,577 & 1,158,680 & 4.8 \\
Amharic (amh) & 22,831 & 15,713 & 112,843 & 151,387 & 150,958 & 1.2 \\
Chichewa (nya) & — & 1,135 & 42,917 & 44,052 & 44,052 & 0.4 \\
Hausa (hau) & 247,507 & 25,957 & 147,028 & 420,492 & 399,866 & 0.9 \\
Igbo (ibo) & 6,196 & 16,158 & 34,802 & 57,156 & 57,095 & 0.2 \\
Malagasy (mlg) & 35,839 & 95,612 & 110,841 & 242,292 & 240,233 & 0.5 \\
Sesotho (sot) & — & 1,076 & 41,547 & 42,623 & 42,623 & 0.2 \\
Shona (sna) & 10,637 & 10,847 & 48,337 & 69,821 & 67,762 & 0.5 \\
Somali (som) & 585,928 & 11,241 & 513,028 & 1,110,197 & 1,084,982 & 2.3 \\
Swahili (swa) & 265,733 & 77,017 & 831,162 & 1,173,912 & 1,151,393 & 3.5 \\
Xhosa (xho) & — & 1,554 & 24,992 & 26,546 & 26,546 & 0.1 \\
Yoruba (yor) & 28,463 & 32,915 & 20,463 & 81,841 & 81,632 & 0.1 \\
Zulu (zul) & — & 11,331 & 61,387 & 72,718 & 72,718 & 0.7 \\
\midrule
\textbf{African Languages not in mC4} \\
Afaan Oromoo (orm) & 18,675 & 1,535 & — & 22,410 & 22,410 & 0.06 \\
Kinyarwanda (kin) & 17,218 & 7,423 & — & 32,437 & 32,437 & 0.10 \\
Tigrinya (tir) & 8,728 & 427 & — & 9,155 & 9,155 & 0.03 \\
\midrule
\textbf{Total (African)} & 1,393,097 & 422,536 & 2,968,087 & 4,793,623 & 4,652,549 & 18.9 \\
\midrule
\textbf{Other Languages} \\
Arabic (arz) & — & 1,617,402 & — & 1,617,402 & 1,617,402 & 0.72 \\
English (eng) & 31,727 & 1,500,000 & — & 1,531,727 & 1,531,727 & 4.0 \\
French (fra) & 103,529 & 1,500,000 & — & 1,603,529 & 1,603,529 & 3.6 \\
Portuguese (por) & 107,670 & 1,102,551 & — & 1,210,221 & 1,210,221 & 2.3 \\
\midrule
\textbf{Total (All)} & 1,636,023 & 6,142,489 & 2,968,087 & 10,756,502 & 10,615,428 & 29.5 \\
\bottomrule
\end{tabular}
\end{table}

\begin{table}[htbp]
\centering
\caption{XL-SUM results: Performance based on Rouge-1, Rouge-2 and Rouge-L. AfriTeVa V2 Large outperforms AfriTeVa V2 Base across all languages considered.}
\label{tab:xl_sum_large}
\small
\begin{tabular}{l r r r r r r r r}
\toprule
Model & hau & ibo & orm & pcm & som & swa & yor & AVG \\
\midrule
AfriTeVa V2 (Base) & 37.3/16.3/29.6 & 22.6/8.1/17.7 & 16.1/5.7/14.1 & 37.0/14.5/29.1 & 29.3/10.1/23.2 & 34.2/15.5/27.9 & 36.2/15.1/26.9 & 30.9/12.6/24.6 \\
AfriTeVa V2 (Large) & 38.1/16.2/29.5 & 34.9/12.8/25.9 & 16.8/5.2/14.4 & 38.8/14.9/30.0 & 29.8/10.0/23.1 & 38.5/18.1/31.4 & 38.2/16.0/27.6 & 34.2/13.9/26.7 \\
\bottomrule
\end{tabular}
\end{table}

\begin{table}[htbp]
\centering
\caption{MasakhaNews Classification Results: Evaluation is done using the weighted F1 score and the scores presented are averaged across 3 seeds. AfriTeVa V2 Large marginally improves overs Base results.}
\label{tab:news_large}
\small
\begin{tabular}{l*{17}{r}}
\toprule
Model & amh & eng & fra & hau & ibo & lin & lug & orm & pcm & run & sna & som & swa & tir & xho & yor & AVG & AVGSL \\
\midrule
AfriTeVa V2 (Base) & 92.8 & 90.6 & 88.0 & 89.4 & 86.1 & 86.0 & 91.1 & 90.8 & 96.8 & 92.3 & 93.3 & 75.7 & 87.0 & 86.4 & 93.6 & 92.3 & 89.5 & 88.9 \\
AfriTeVa V2 (Large) & 92.4 & 91.1 & 88.2 & 89.8 & 88.4 & 90.2 & 92.1 & 88.2 & 96.9 & 92.6 & 93.2 & 77.9 & 86.0 & 86.0 & 94.6 & 91.8 & 90.0 & 89.3 \\
\bottomrule
\end{tabular}
\end{table}

\end{document}
=====END FILE=====