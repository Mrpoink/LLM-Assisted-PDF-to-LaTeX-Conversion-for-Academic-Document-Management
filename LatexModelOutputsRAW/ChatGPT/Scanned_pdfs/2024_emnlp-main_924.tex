=====FILE: main.tex=====
\documentclass[11pt]{article}

\usepackage[margin=1in]{geometry}
\usepackage{times}
\usepackage{latexsym}
\usepackage{amsmath,amssymb}
\usepackage{graphicx}
\usepackage{booktabs}
\usepackage{array}
\usepackage{url}
\usepackage[hidelinks]{hyperref}
\usepackage{caption}
\usepackage{multicol}

\title{Is It Really Long Context if All You Need Is Retrieval?\
Towards Genuinely Difficult Long Context NLP}

\author{
Omer Goldman\thanks{Equal contribution} \and
Alon Jacovi\footnotemark[1] \and
Aviv Slobodkin\footnotemark[1] \and
Aviya Maimon\footnotemark[1] \and
Ido Dagan \and
Reut Tsarfaty\
Bar-Ilan University\
\texttt{[omer.goldman@gmail.com](mailto:omer.goldman@gmail.com)}
}

\date{}

\begin{document}
\maketitle

\begin{abstract}
\input{sections/abstract}
\end{abstract}

\begin{multicols}{2}

\input{sections/introduction}
\input{sections/task_design}
\input{sections/taxonomy}
\input{sections/under_explored}
\input{sections/discussion}
\input{sections/acknowledgments}
\input{sections/conclusions}
\input{sections/limitations}

\end{multicols}

\input{sections/references}

\appendix
\input{sections/appendix_a}

\end{document}
=====END FILE=====

=====FILE: sections/abstract.tex=====
Improvements in language models' capabilities
have pushed their applications towards longer
contexts, making long-context evaluation and
development an active research area. However,
many disparate use cases are grouped together
under the umbrella term of ``long-context'', de%
flned simply by the total length of the model's
input, including -- for example -- Needle-in-a%
Haystack tasks, book summarization, and in%
formation aggregation. Given their varied diffi%
culty, in this position paper we argue that con%
flating different tasks by their context length
is unproductive. As a community, we require
a more precise vocabulary to understand what
makes long-context tasks similar or different.
We propose to unpack the taxonomy of long%
context based on the properties that make them
more dfficult with longer contexts. We propose
two orthogonal axes of difficulty: (I) Disper%
slon.' How hard is it to find the neeessary infor%
mation in the context? (II) Scope: How much
necessary information is there to find? We sur%
vey the literature on long context, provide justi%
fication for this taxonomy as an informative de%
scriptor, and situate the literature with respect
to it. We conclude that the most difficult and in%
teresting settings, whose necessary information
is very long and highly dispersed within the
input, is severely under-explored. By using a
descriptive vocabulary and discussing the rele%
vant properties of dfficulty in long context, we
can implement more informed research in this
area. We call for a careful design of lasks and
benchmarks with distinctly long context, tak%
ing into account the characteristics that make it
qualitatively different from shorter context.
=====END FILE=====

=====FILE: sections/introduction.tex=====
\section{Introduction}

The ability to deal with ever-longer contexts has
been one of the most notable trends among the
emerging capabilities of large language models
(LLMs). Starting with a few hundred tokens as the
maximal input length of the first attention-based
LLMs (Devlin et al., 2019; Raffel et al., 2020),
contemporary models are -- technically -- able to
process up to 128k and even 1M tokens (Gemini
Team Google, 7024; OpenAI, 2024).

\begin{figure}[t]
\centering
\fbox{\parbox{0.95\columnwidth}{\centering IMAGE NOT PROVIDED}}
\caption{A taxonomy of long context tasks based on
the distribution of the needed information in the text.
Tasks with larger scope and higher dispersion are more
difficult (indicated by shade) and more indicative of the
long context capabilities of large language models.}
\label{fig:taxonomy_overview}
\end{figure}

The demand to evaluate LLMs in this setting has
led to a line of research on designing long-context
tasks and benchmarks, in order to systematically
understand models' capabilities and drive their de%
velopment. Howevel the field has generally a sole
recurring descriptor to define such measurements
by -- simply, the length of the context. For exam%
ple, long-context benchmarks group tasks mostly
by length in words (e.g., Shaham et al., 2022; Ba
et al., 2023; Zhang et al., 2024b). Thi s le ads to qu al-%
itatively different measurements being conflated
together, with conclusions about long-context ca%
pabilities being extended from one class of tasks
to others. The community is, of course, aware that,
for example, tasks which require a small part of
the input are different from tasks that require a
large part of it. But we ask the more general ques%
tion: What are the properties that differentiate tasks
when conditioned on their context length? What
can we accomplish with such a distinction?

In this position paper, we claim that the current
landscape of works on long-context evaluation will
greatly beneflt from a more flne-grained charac%
terization of long-context task design. We argue
that judging LLMs by their ability to process long
sequences, while disregarding the task they pro%
cess them for, overlooks the characteristics that
make longer inputs more difficult, and interesting
to research, to begin with (\S2).

For example, Needle in a Haystack tasks (NIAH;
Ivgi et al., 2023; Mohtashami and Jaggi, 2023) in%
volve queries whose main challenge is finding the
relevant information in a long context, without re%
quiring much further processing. Synthetic NIAH
datasets are, of course, easier than their natural
equivalents (Ivgi et al., 2023), but the `natural vs.
artificial'' classification is not informative in our
setting, since it applies equally for tasks regardless
of context length. What, then, is an informative
property? What makes long-context tasks different
from each other? For example, multiple-needle
variants of NIAH (Hsieh et al., 2024), or those rhat
position the `needles'' closer or farther apart (Levy
et al., 2024). Evidently, ``the number of tokens in
the input'' is not a sufficient descriptor.

To resolve this roadblock, we present a taxon%
omy of long-context tasks for the different factors
that make them harder when controlling for context
length (\S3). This taxonomy is derived by surveying
the long-context literature and surfacing the most
salient points of distinction between various tasks.
We focus on (I) how difficult it is to find and extracr
the required information from the input (its disper%
sion in the input), and (II) the absolute quantity of
required information to solve the task (its scope).
See Figure~\ref{fig:taxonomy_overview} for a summary.

To understand this categorization and its utility,
we review the literature on long-context evaluation
and position the works with respect to those factors,
We find that the most challenging setting, in which
a large quantity of required information is present
in a dispersed manner that is difficult to extract, is
signifl cantly under-explored (\S4).

Finally, acknowledging the inherent and legiti%
mate reasons behind the focus on context length as
the sole descriptor of difficulty, we discuss possible
paths forward for designing more reliable measure%
ments of long-context capabilities when utilizing a
more nuanced vocabulary (\S5).
=====END FILE=====

=====FILE: sections/task_design.tex=====
\section{Task Design in Long Context}

Evaluating the performance of NLP models over
very long contexts is a fast-changing area of re%
search (Bishop et al., 2024; Wu et al., 2024). Mea%
surements are regularly updated to account for new
capabilities which improve with extrapolation ar%
chitectures (Vaswani et al., 2017; St et al., 2024)
and training data (He et al., 2023; Chen et al., 2023).
Evaluators were tasked with designing measure%
ments of long-context capabilities cheaply, effi%
ciently, and quickly, while matching realistic use
cases as much as possible. The most common
way of differentiating long-context tasks, besides
the context's length, is whether they are naturally%
constructed or synthetically-constructed (Tay et al.,
2020; Ba et al., 2023; Hsieh et al., 2024).

\paragraph{Natural construction.}
A simple yet effective
way of ``moving the goalpost'' for context length
is by modeling long-context tasks based on short%
context tasks. This was done, for example, with QA
(Kodiskf et al., 2018, cf.\ Dunn et al., 2017), sum%
marization (Huang et al., 2027a, cf.\ Narayan et al.,
2018), and NLI (Koreeda and Manning, 2021a, cf.
Williams et al., 2018). Specialized domains like
legal (Bruno and Roth, 2022; Nguyen et al., 2024)
and literature (Wang et al., 2022; Kryscinski et al.,
2022) often involve longer texts, turning typically
short-context tasks such as QA and NLI into long%
context scenarios. Another more native methodol%
ogy is to create new tasks which inherently require
a long context, such as multi-document summariza%
tion (Fabbri et al., 2019; Angelidis et al., Z0ZI),
survey generation (Gao et al., 2024), and structured
data aggregation (Caciularu et al., 2OZ4). Both
methodologies share the constraint that, due to their
natural construction (i.e., using natural text), once
created, they are difficult to modify for longer con%
texts as models' long-context capabilities improve.

\paragraph{Synthetic construction.}
A more flexible ap%
proach, sacrificing natural construction for length
control, is to use distractors to synthetically in%
crease the context length. This method allows for
cheap and efficient (in terms of task construction
cost) evaluation of models' full context length ca%
pabilities, with difficulty adjusted by controlling
the distractors. Tasks like Needle-in-a-Haystack
(NIAH; Ivgi et al., 2023; Kamradt, 2023) and
PassKey retrieval (Mohtashami and Jaggi, 2023)
were created to evaluate a model's ability to pin%
point specific information amid lengthy distrac%
tors. Flexible and effective against existing models,
they became standard benchmarks for evaluating
new long-context models (GLM Team, 2024; Jiatg
et al., 2024). Followup studies have complicated
these tasks by increasing the number of critical de%
tails to locate (Arora et al., 2023; Liu et al., 2024a)
and changing their position within the input (Liu
et al., 2024b; Levy et al., 2024).

\paragraph{Limitations of the status quo.}
NIAH-like tasks
aim to assess information retrieval capabilities,
yet many `naturally constructed'' QA and reading%
comprehension tasks with trivial questions about
a long context accomplish the same goal. At the
same time, `multiple needles'' NIAH can increase
difficulty not by increasing the quantity of nee%
dles or length of input, but by adding distractors
between needles (Levy et al., 2024). What can sys%
tematically explain the different variables at play,
in order to inform better task design in the future?

Clearly, there are underlying qualitative drffer%
ences that discriminate between these various tasks
besides their natural and synthetic constructions,
and besides their actual context length. Therefore,
we require a more informative vocabulary to dis%
cuss the goals of each task design, what it accom%
plishes, and what it does not, in terms of measuring
long-context capabilities.
=====END FILE=====

=====FILE: sections/taxonomy.tex=====
\section{What Makes Long Context More than Retrieval?}

We require a taxonomy to capture task difficulty
variations beyond mere ``number of tokens''. We fo%
cus on the information that is canonically required
to solve the task as the conditioning variable. Our
classification can be summarized via the following
two questions, when asked about a given task:

\begin{enumerate}
\item[(I)] How dfficult is it to find and extract the required information?
\item[(II)] How much information is needed to be found?
\end{enumerate}

Assuming that some highlighting of the relevant
information is needed to solve the task (see Fig%
ure~\ref{fig:taxonomy_overview}), the latter question asks how much text is
highlighted, while the former addresses the chal%
lenge of locating the relevant text for highlighting.

For instance, consider the task of summarizing
a book, in comparison to a NIAH task of identi%
fying a numerical detail in a long financial report
(e.g., ``how much did the company earn in 2015?'').
Although both tasks involve long texts, the informa%
tion required and its accessibility vary significantly.
The NIAH task focuses on localized, identifiable in%
formation, while summarization requires extracting
key details dispersed throughout the text, tangled
together with irrelevant content. Therefore, we
can say that the book summarization task is more
difficult on both axes (I) and (II).

Below we give more formal descriptions of the
two axes characterized by the questions above.

\paragraph{(I) Dispersion.}
Although the question above intu%
itively defines ``difficulty of information finding'',
we offer a more concrete description. Between two
similar tasks, we consider the information harder
to find in one task compared to another if: (1) it
is more obscured (e.g., linguistically, semantically,
contextually, etc); (2) it is more sparse, such that
it is interspersed with non-required information;
(3) its indicators are less redundant, such that there
are fewer places in the document where the same
information is available.

\paragraph{(II) Scope.}
The property of scope is simpler,
and refers to the minimal quantity of information
needed to solve the task. Importantly, we are not
concerned with precise metric for ``quantity of in%
formation'' at this stage -- it can refer to quantity
of tokens, sentences, relations, cells in a table, etc.
Any metric that reliably captures difflculty for an
established solver is sufflcient for our purposes.

\paragraph{Illustrative example.}
To illustrate, consider the
Wikipedia entry for New York City and a simple
question: `What is the estimated population of the
city?'' Since the answer needs a small snippet of
information, we say that the task has small scope.
And since it is easily accessible, we say that it
has low dispersion. Consider, instead, the ques%
tion `how many syllables are in this document?'' --
since this question requires the entire document to
answer, we say that it has large scope, but if we
consider counting syllables as straightforward, then
we say its dispersiorz is still low. Finally with the
question `Was the city's mayor elected before or
after the city was affected by Hurricane Sandy?'' --
since it requires snippets from at least two different
areas of the text, we can say that when compared to
the question about the city's population, the disper%
sion is higher, but not as high as for the question
`What makes the city a prominent place on the
world stage?'' which poses a challenge on both
axes.
=====END FILE=====

=====FILE: sections/under_explored.tex=====
\section{Challenging Long Context Is Under-Explored}

\begin{figure}[t]
\centering
\fbox{\parbox{0.95\columnwidth}{\centering IMAGE NOT PROVIDED}}
\caption{This flgure illustrates our subjective judgment
on the distribution of long-context benchmarks for each
task, categorized by their scope and dispersion charac%
teristics, with the four quadrants being marked by the
dashed lines. Difficulty is expressed by shade, where
red is more difficult and green in easier. Notably, some
tasks, like Question-answering (QA), appear in multi%
ple quadrants, as different benchmarks demand varylng
levels of scope and dispersion (e.g., a single fact versus
multiple facts spread across a document). For a detailed
breakdown of benchmarks and their task associations.
refer to Appendix A.}
\label{fig:benchmarks_distribution}
\end{figure}

Revisiting the works surveyed in \S2, they clearly
differ with respect to both scope and dispersion.

\paragraph{With respect to dispersion.}
The information
needed for tasks ranges from easily accessible to
highty dispersed and difficult to detect. On low
dispersion we have NIAH (Kamradt, 2A23; Mo%
htashami and Jaggi, 2A23) and a myriad of fac%
tual single-hop QA datasets (Tseng et al., 2OI6;
KoEiskf et al., 2Ol7; Kwiatkowski et al., 2019;
Dasigi et al., 2021, inter alia) in which the answer
is relatively accessible. Adding more snippets of
information separated by distractors, either in the
form of several needles (Arora et al., 2023; Hsieh
et al., 2024) or of hops in a multi-hop question
(Trivedi et al., 2A22; Zhao et al., 2022), complicates
the information detection due to the need to find
at least two snippets (Levy et al., 2024), thereby
increasing dispersion. Dispersion can also be in%
creased by making the detection of the information
less straightforward (e.g., Pang et al., 2022) or re%
quiring aggregation (Shaham et al., 2023). Lastly,
summarization tasks are of a very high dispersion
(Huang et al., 202La; Wang et al., 2022), as they
require the non-trivial detection of salient facts that
are interwoven with the irrelevant text.

\paragraph{With respect to scope.}
Thsks overwhelmingly tar%
get relatively small scope. In addition to the afore%
mentioned NIAH tasks and their variants, many
QA datasets apply as well (Li et al., 2023; ZJtao
et al., 2023; Reddy et al., 2024, inter alia). A some%
what higher scope is achieved by datasets for query%
based summarization (Zhong et al., 2021; Wang
et al., 2022), and QA datasets with more obfuscated
answers that require reading the text surrounding
the answer for its verification (An et al., 2023; He
et al., 2023). Although much higher on the scope
ladder, book summarization is still limited in its
scope as datasets include texts that are only of up
to zok tokens (Huang et al., 2021a; Chen et al.,
2022a; Shaham et al., 2023). Currently, tasks with
the highest scope, requiring information across the
entire input length, are artificial and of low disper%
sion, like common words extraction (Hsieh et al.,
2024).

\paragraph{Conclusion.}
Figure~\ref{fig:benchmarks_distribution} summarizes the above clas%
sification of tasks and datasets. Note that without
a concrete definition of dispersion and scope, the
plot is only an illustration that involves a good
deal of subjective judgements. However, we con%
clude that (1) the majority of tasks designed to
challenge LLMs in a long-context setting target
either scope or dispersion, such that (2) tasks that
push current models' capabilities on both axes are
under-represented in the current landscape.
=====END FILE=====

=====FILE: sections/discussion.tex=====
\section{Discussion: Towards Genuinely Difficult Long-Context Task Design}

\paragraph{Challenges.}
Designing meaningful long-context
tasks amidst rapid model progress is profoundly
challenging, making the deficiency in tasks repre%
senting difficulty on both the dispersion and scope
axes unsurprising. One source of this challenge is
the lack of diverse, coherent long texts, as models'
context windows can now be comparable to the
length of the New Testament\footnote{rwww.readinglength.com/book/isbn-0190909005}
and the Odyssey.\footnote{2www.readinglength.com/book/isbn-0140268863}
The methodologies discussed in \S2 for creating
long context tasks -- lengthening short context tasks
and synthetically creating length-adjustable tasks
-- are preferred for their straightforward definition
and the incremental adjustments they require for
existing data. They rely on the common understand%
ing of machine comprehension as formulated with
short context in mind (Dunietz et al., 2020), and
therefore they are intuitive and easy to comprehend
for NLP researchers without domain expertise (e.g.,
in law or biomedical fields that have long contexts).

\paragraph{Future work.}
The goals laid forward in this
work are clear: For more durable and robust mea%
surement of long-context capabilities, we must de%
sign tasks that explicitly target both the dispersion
and scope capabilities of models. How can this be
achieved? As mentioned, one possible avenue is to
rely more on /asfr.r that require domain expertise,
such as legal documents (Bruno and Roth, 2022),
flnancial reports (Reddy et al., 2024), biomedical
publications (Stylianou et al., 202l), and so on, In
specialized domains, it is common that dispersion
will be naturally higher (Zhao et al., 2022). Tasks
that involve implicit aggregations over structured
data, slch as table manipulation (Caciularu et al.,
2024), are possible avenues for increasing both
scope and dispersion synthetically by leveraging
the data structure. In this work, we argue that an
explicit vocabulary for such properties of difficulty
is what can enable more informed techniques to
design difficult tasks in the future.

information, and quantity of information, are both
vague terms that can only be grounded in the con%
text of a specific family of tasks and use-cases. We
intend for this work to serve as a call to action and a
tool for a shared vocabulary in the interest of more
informed long-context task design in the future,
rather than to anchor the taxonomy to a speciflc
and fragile point in time.

\paragraph{Retrieval is still interesting.}
Although we ar%
gue that small scope and low dispersion tasks are
the least indicative of the model's ability to long%
context capabilities, tasks that are well-served by
implicit retrieval or by traditional retrieval-based
pipelines are certainly relevant and useful in a va%
riety of common use-cases (Stylianou et al., 2021;
Bruno and Roth, 2022; Gao et al., 2023).

\paragraph{Other uses for a long-context window.}
This pa%
per deals only with long inputs that serve as inputs
to a task. The long context of course can have other
purposes as well, like containing many in-context
examples (Bertsch et al., 2424) or containing other
modalities and structures (Jiang et al., 2023).
=====END FILE=====

=====FILE: sections/acknowledgments.tex=====
\section*{Acknowledgments}
The authors would like to thank Gabriel Stanovsky
for the fruitful discussions.
=====END FILE=====

=====FILE: sections/conclusions.tex=====
\section{Conclusions}

We present a taxonomy of factors that make long%
context tasks more challenging compared to short
ones. This is in contrast with the existing litera%
ture that refers only to the length of the input as
the hallmark of long context, and as a result ends
up conflating tasks of different character when as%
sessing the ability of models to understand longer
text. We reviewed works on evaluation in a long%
context setting and found that the most challenging
setting, in which the information needed is of large
scope and is highly dispersed within the input, is
under-explored. Finally, we suggested some leads
for future work to tackle this imbalance towards a
more informative long context evaluation.
=====END FILE=====

=====FILE: sections/limitations.tex=====
\section{Limitations}

Formality. In the context of this work, we have
deliberately adhered to a taxonomy based on an
lntultlve oescnptlon' ln tne mterest ot utlllty to a
wide diversity of research and flexibility for future
work. Difficulty in searching for and exftacting
[ILLEGIBLE]
=====END FILE=====

=====FILE: sections/references.tex=====
\begin{thebibliography}{99}

\bibitem{} Shmuel Amar, Liat Schiff, Ori Ernst, Asi Shefer, Ori
Shapira, and Ido Dagan. 2023. OpenAsp: A bench%
mark for multi-document open aspect-based summa%
rization. lt Proceedings of the 2023 Conference on
Empirical Methods in Natural Language Processing,
pages 1967--1991, Singapore. Association for Com%
putational Linguistics.

\bibitem{} Chenxin An, Shansan Gong, Ming Zhor5 Xingjian
Zhao, Mukai Li, Jun Zhang, Lingpeng Kong, and
Xipeng Qiu. 2023. L-eval: Instituting standard%
ized evaluation for long context language models.
Preprint, arXiv:2307. 1 1088.

\bibitem{} Stefanos Angelidis, Reinald Kim Amplayo, Yoshihiko
Suhara, Xiaolan Wang, and Mirella Lapata. 202l.
Extractive opinion summarization in quantized ffans%
former spaces. Transactions of the Association for
Computational Linguistics, 9:277--293.

\bibitem{} Simran Arolar. S.abri Eyuboglu, Aman Timalsina, Isys
Johnson, Michael Poli, James Zou, Atri Rudra, and
christopier Re. 2023. Zoology: Measuring and im%
wide diversity of research and flexibility for future proving recall in efficient lariguage modeli. arXly
work. Difficulty in searching for and exftacting preprint arXiv:2312.04927.

\bibitem{} Dennis Aumiller and Michael Gera. 2022. Klexikon:
A German dataset for joint summarization and sim%
plification. In Proceedings of the Thirteenth l-nn%
guage Resources and Eyaluation Conference, pages
2693--2701, Marseille, France. European Language
Resources Association.

\bibitem{} Yushi Bai, Xin Lv, Jiajie Zhang, Hongchang Lyu,
Jiankai Tang, Zhidian Huang, Zhengxiao Du, Xiao
Liu, Aohan Zeng, Lei Hou, Yuxiao Dong, Jie Tang,
and Juanzi Li. 2023. Longbench: A bilingual, mul%
titask benchmark for long context understanding.
P rep rint, arXiv: 2308. 14508,

\bibitem{} Amanda Bertsch, Maor lvgi, Uri AIon, Jonathan Berant,
Matthew R. Gormley, and Graham Neubig. 2024.
In-context learning with long-context models: An
in-depth exploration. Preprint, arXiv:2405.00200.

\bibitem{} Jennifer A Bishop, Qianqian Xie, and Sophia Anani%
adou. 2024. Longdocfactscore: Evaluating the fac%
tuality of long document abstractive summarisation.
P re p r int, arXiv :2309 .124 5 5 .

\bibitem{} Odellia Boni, Guy Feigenblat, Guy Lev, Michal
Shmueli-Scheuer, Benjamin Sznajder, and David
Konopnicki. 2021. Howsumm: A mulfi-document
summarization dataset derived from wikihow articles.
P rep rint, arXiv; 2 I 1 0.03 I 79.

\bibitem{} William Bruno and Dan Roth. 2022. Lawngnli: A
long-premise benchmark for in-domain generaliza%
tion from short to long contexts and for implication%
based retrieval. P reprint, arXiv :22L2.03222,

\bibitem{} Avi Caciularu, Alon Jacovi, Eyal Ben-David, Sasha
Goldshtein, Tal Schuster, Jonathan Herzig, Gal Eli%
dan, and Amir Globerson. 2024. Tact: Advancing
complex aggregative reasoning with information ex%
traction tools. Preprint, arXiv :2406.036 I 8.

\bibitem{} Mingda Chen, Zewei Chu, Sam Wiseman, and Kevin
Gimpel. 2022a. SummScreen: A dataset for abstrac%
tive screenplay summarization. In Proceedings of the
60th Annual Meeting of the Association for Compu%
tational Linguistics (Volume 1: Long Papers), pages
8602--8615, Dublin, Ireland. Association for Compu%
tational Linguistics.

\bibitem{} Mingda Chen, Zewei Chu, Sam Wiseman, and Kevin
Gimpel. 2022b. Summscreen: A dataset for
absfractive screenplay summarization. Preprint,
arXiv:2104.07091.

\bibitem{} Yukang Chen, Shengju Qian, Haotian Tang, Xin Lai,
Zhijian Liu, Song Han, and Jiaya Jia. 2023. Longlora:
Efficient fine-tuning of long-context large language
models. ArXiv, abs/2309.12307 .

\bibitem{} Arman Cohan, Franck Demoncourt, Doo Soon Kim,
Trung Bui, Seokhwan Kim, Walter Chang, and Nazli
Goharian. 2018. A discourse-aware attention model
for abstractive summarization of long documents.
P rep rint, arXiv: I 804.05685.

\bibitem{} Anze Xe Ying Sheng Lianmin Zheng Joseph E. Gonza%
lez Ion Stoica Xuezhe Ma Dacheng Li*, Rulin Shao*
and Hao Zhang. 2023. How long can open-source
llms truly promise on context length?

\bibitem{} Pradeep Dasigi, Kyle Lo, IzBeltagy, Arman Cohan,
Noah A. Smith, and Matt Gardner. 202L. A dataset
of information-seeking questions and answers an%
chored in research papers, In Proceedings of the
2021 Conference of the North American Chapter of
the As s ociation for C omputational Linguistic s : Hu%
man lnnguage Technologies, pages 4599--4610, On%
line. Association for Computational Linguistics.

\bibitem{} Jacob Devlin, Ming-Wei Chang, Kenton Lee, and
Kristina Toutanova. 2019. BERT: Pre-training of
deep bidirectional transformers for language under%
standing. In Proceedings of the 2019 Conference of
the North American Chapter of the Association for
Computational Linguistics : Human Language Tech%
nologies, Volume 1 (Long and Short Papers), pages
4171--4186, Minneapolis, Minnesota, Association for
Computational Linguistics.

\bibitem{} Zican Dong, Tianyi Thng, Junyi Li, Wayne XinZhao,
and Ji-Rong Wen. 2024. BAMBOO: A comprehen%
sive benchmark for evaluating long text modeling ca%
pacities of large language models. In Proceedings of
the 2024 loint International Conference on Compu%
tational Linguistics, Language Resources and Eyal%
uation (LREC-COLING 2024), pages 2086--2099,
Torino, Italia. ELRA and ICCL.

\bibitem{} Jesse Dunietz, Greg Burnham, Akash Bharadwaj, Owen
Rambow, Jennifer Chu-Canoll, and Dave Femrcci.
2020. To test machine comprehension, start by defln%
ing comprehension. In Proceedings of the 58th An%
nual Meeting of the Association for Computational
Linguistics, pages 7839--7859, Online. Association
for Computational Linguistics,

\bibitem{} Matthew Dunn, Levent Sagun, Mike Higgins, V Ugur
Guney, Volkan Cirik, and Kyunghyun Cho. 2017.
Searchqa: A new q&a dataset augmented with
context from a search engine, arXiv preprint
arXiv:1704.05179.

\bibitem{} Alexander R. Fabbri, kene Li, Tianwei She, Suyi Li, and
Dragomir R. Radev. 2019. Multi-news: a large-scale
multi-document summarization dataset and abstrac%
tive hierarchical model. P rep rint, arXiv :1906.01749.

\bibitem{} Song Feng, Siva Sankalp Patel, Hui Wan, and Sachindra
Joshi. 2021. MultiDoc2Dial: Modeling dialogues
grounded in multiple documents. In Proceedings of
the 2021 Conference on Empirical Methods in Natu%
ral Language Processing, pages 6162--6176, Online
and Punta Cana, Dominican Republic. Association
for Computational Linguistics,

\bibitem{} Fan Gao, Hang Jiang, Rui Yang, Qingcheng Zeng,
Jinghui Lu, Moritz Blum, Dairui Liu, Tianwei She,
Yuang Jiang, and Irene LL. 2024. Large language
models on wikipedia-style survey generation: an eval%
uation in nlp concepts. P reprint, arXiv: 2308. I 04 I 0.

\bibitem{} Luyu Gao, Zhuyun Dai, Panupong Pasupat, Anthony
Chen, Arun Tejasvi Chaganty, Yicheng Fan, Vin%
cent Y. Zhao, Ni Lao, Hongrae Lee, Da-Cheng Juan,
and Kelvin Gtu. 2023. Rarr: Researching and re%
vising what language models say, using language
models. P reprint, arXiv :2210.087 26.

\bibitem{} Gemini Team Google. 2024. Gemini 1.5: Unlocking
multimodal understanding across millions of tokens
of context. Preprint, arXiv:2403.05530.

\bibitem{} GLM Team. 2024. GLM-4-9b-chat technical report.

\bibitem{} Daya Guo, Canwen Xu, Nan Duan, Jian Yin, and Ju%
lian McAuley. 2023. Longcoder: A long-range
pre-trained language model for code completion.
P rep rint, uKv :2306.1 4893.

\bibitem{} Junqing He, Kunhao Pan, Xiaoqun Dong, Zhuoyang
Song, Yibo Liu, Yuxin Liang, Hao Wang, Qianguo
Sun, Songxin Zhang, Zejian Xie, and Jiaxing Zhang.
2023. Never lost in the middle: Improving large
language models via attention strengthening question
answering. P reprint, arXv:23 I 1.091 98.

\bibitem{} Dan Hendrycks, Collin Burns, Anya Chen, and
Spencer Ball. 2021. Cuad; An expert-annotated
nlp dataset for legal contract review. Preprint,
arXiv:2103.06268.

\bibitem{} Xanh Ho, Anh-Khoa Duong Nguyen, Saku Sugawara,
and Akiko Aizawa. 2020. Constructing a multi%
hop QA dataset for comprehensive evaluation of
reasoning steps. In Proceedings of the 28th Inter%
nat i o nal C onfe re nc e on C o mp utati o nal Lin g uis tic s,
pages 6609--6625, Barcelona, Spain (Online). Inter%
national Committee on Computational Linguistics.

\bibitem{} Cheng-Ping Hsieh, Simeng Sun, Samuel Kriman, Shan%
tanu Acharya, Dima Rekesh, Fei Jia, Yang Zhang,
and Boris Ginsburg. 2024. Ruler: What's the real
context size of your long-context language models?
P rep rint, arXiv :240 4.066 5 4.

\bibitem{} Yebowen Hu, Timothy Ganter, Hanieh Deilamsalehy,
Franck Dernoncourt, Hassan Foroosh, and Fei Liu.
2023. MeetingBank: A benchmark dataset for meet%
ing summarization. In Proceedings of the 61st An%
nual Meeting of the Association for Computational
Linguistics (Volume 1: Long Papers), pages 16409--
16423, Toronto, Canada. Association for Computa%
tional Linguistics.

\bibitem{} Luyang Huang, Shuyang Cao, Nikolaus Parulian, Heng
Ji, and Lu Wang. 2021a. Efficient attentions for long
document summarization. ln Proceedings of the 2021
Conference of the North American Chapter of the
As s ociation for Computational Linguistic s : Human
Lon guag e Te chno lo g ie s, pages 1419--1436, Online.
Association for Computational Linguistics.

\bibitem{} Luyang Huiurg, Shuyang Cao, Nikolaus Parulian,
Heng Ji, and Lu Wang. 2021b. Efficient atten%
tions for long document summarizatiot. Preprint,
arXiv:2104.02112.

\bibitem{} Maor Ivgi, Uri Shaham, and Jonathan Berant. 2023. Et%
ficient long-text understanding with short-text mod%
els. Transactions of the Association for Computa%
tional Linguistic s, I I :284--299.

\bibitem{} Albert Q. Jiang, Alexandre Sablayrolles, Antoine
Roux, Arthur Mensch, Blanche Savary, Chris
Bamford, Devendra Singh Chaplot, Diego de las
Casas, Emma Bou Hanna, Florian Bressand, Gi%
anna Lengyel, Guillaume Bour, Guillaume Lam%
ple, L6lio Renard Lavaud, Lucile Saulnier, Marie%
Anne Lachaux, Pierre Stock, Sandeep Subramanian,
Sophia Yang, Szymon Antoniak, Teven Le Scao,
Th6ophile Gervet, Thibaut Lavril, Thomas Wang,
Timoth6e Lacroix, and William El Sayed. 2024. Mix%
tral of experts. Preprtnt, arXiv:2401.04088.

\bibitem{} Jinhao Jiang, Kun Zhou, Zican Dong, Keming Ye, Xin
Zhao, and Ji-Rong Wen. 2023. StructGPT: A general
framework for large language model to reason over
structured data. In Proceedings of the 2023 Con%
ference on Empirical Methods in Natural lnnguage
P roces sing, pages 9237--9251, Singapore. Associa%
tion for Computational Linguistics.

\bibitem{} Gregory Kamradt. 2023. Needle in a haystack - pressure
testing LLMs. GitHub.

\bibitem{} Yuta Koreeda and Christopher Manning. 2021a. Con%
tractNl-l: A dataset for document-level natural lan%
guage inference for contracts. In Findings of the
Association for Computational Linguistics : EMNLP
2021, pages 1907--1919, Punta Cana, Dominican Re%
public. Association for Computational Linguistics.

\bibitem{} Yuta Koreeda and Christopher D. Manning. 2021b.
Contractnli: A dataset for document-level natu%
ral language inference for contracts. Preprint,
arXiv:2110.01799.

\bibitem{} Anastassia Kornilova and Vladimir Eidelman. 2019.
BillSum: A corpus for automatic summarization of
US legislation. In Proceedings of the 2nd Workshop
on New Frontiers in Summarization, pages 48--56,
Hong Kong, China. Association for Computational
Linguistics.

\bibitem{} Tomd5 KoEiskf, Jonathan Schwarz, Phil Blunsom,
Chris Dyer, Karl Moritz Hermann, Gdbor Melis,
and Edward Grefenstette. 2017. The narra%
tiveqa reading comprehension challenge. Preprint,
arXiv:1712.07040,

\bibitem{} Tom65 Kodiskf, Jonathan Schwarz, Phil Blunsom, Chris
Dyer, Karl Moritz Hermann, G6bor Melis, and Ed%
ward Grefenstette. 2018. The NarrativeQA Reading
Comprehension Challenge. Tiansactions of the Asso%
c tati o n fo r C omp ut at i o n al Lin g ui s t ic s, 6:317--328.

\bibitem{} Wojciech Kryscinski, Nazneen Rajani, Divyansh Agar%
wal, Caiming Xiong, and Dragomir Radev. 2022.
BOOKSUM: A collection of datasets for long-form
narrative summarization. In Findings of the Associ%
ation for C omputattonal Linguistic s : EMNLP 2022,
pages 6536--6558, Abu Dhabi, United Arab Emirates.
Association for Computational Linguistics.

\bibitem{} Wojciech KrySciriski, Nazneen Rajani, Divyansh Agar%
wal, Caiming Xiong, and Dragomir Radev. 2022.
Booksum: A collection of datasets for long-form nar%
rative summarization. P rep rint, arXiv : 2 I 05.08209.

\bibitem{} Sayali Kulkami, Sheide Chammas, Wan Zhu, Fei Sha,
and Eugene Ie. 2020. Aquamuse: Automatically
generating datasets for query-based multi-document
summarization. Preprint, arXiv :2010J2694.

\bibitem{} Yuri Kuratov, Aydar Bulatov, Petr Anokhin, Dmitry
Sorokin, Artyom Sorokin, and Mikhail Burtsev.
2024. In search of needles in a 1lm haystack: Re%
curent memory finds what llms miss. Preprint,
arXiv:'402.10790.

\bibitem{} Tom Kwiatkowski, Jennimaria Palomaki, Olivia Red%
fleld, Michael Collins, Ankur Parikh, Chris Alberti,
Danielle Epstein, Illia Polosukhin, Jacob Devlin, Ken%
ton Lee, Kristina Toutanova, Llion Jones, Matthew
Kelcey, Ming'Wei Chang, Andrew M. Dai, Jakob
Uszkoreit, Quoc Le, and Slav Petrov. 2019. Natu%
ral questions: A benchmark for question answering
research. Trans actions of the As s ociation for C ompu%
tational Linguistic s, 7:452--466.

\bibitem{} Mosh Levy, Alon Jacoby, and Yoav Goldberg. 2024.
Same task, more tokens: the impact of input length on
the reasoning performance of large language models.
P re p r int, arXiv :2402. I 48 48.

\bibitem{} Jiaqi Li, Mengmeng Wang, Zilong Zheng, and Muhan
Zhang. 2023. Loogle: Can long-context lan%
guage models understand long contexts? Preprint,
arXiv:2311,04939.

\bibitem{} Hao Liu, Wilson Yan, Matei Zahaia, and Pieter
Abbeel. 2024a. Worldmodel on million-length video
and language with ringattention, arXiv preprint
arXiv:2402.08268.

\bibitem{} Nelson F Liu, Kevin Lin, John Hewitt, Ashwin Paran%
jape, Michele Bevilacqua, Fabio Petroni, and Percy
Liang. 2024b. Lost in the middle: How language
models use long contexts. Transactions of the Asso%
c i at ion fo r C omputati onal Lin gui s t ic s, 12:157--173.

\bibitem{} Shuaiqi Liu, Jiannong Cao, Ruosong Yang, and
Zhiyuan Wen. 2023a. Long text and multi-table
summarization: Dataset and method. Preprint,
arXiv:2302.03815.

\bibitem{} Tianyang Liu, Canwen Xu, and Julian McAuley.
2023b. Repobench: Benchmarking repository%
level code auto-completion systems. Preprint,
arXiv:2306.0309 I .

\bibitem{} Chaitanya Malaviya, Subin Lee, Sihao Chen, Elizabeth
Sieber, Mark Yatskar, and Dan Roth. 2024. Ex%
pertQA: Expert-curated questions and attributed an%
swers. In Proceedings of the 2024 Conference of
the North American Chapter of the Association for
Computational Ltnguistics : Human Language kch%
nologies (Volume 1: Long Papers), pages 3025*3045,
Mexico City, Mexico. Association for Computational
Linguistics.

\bibitem{} Amirkeivan Mohtashami and Martin Jaggi. 2023. Land%
mark attention: Random-access infinite context
length for transformers. In Workshop on Efficient
Systems for Foundation Modek@ ICML2023.

\bibitem{} Shashi Narayan, Shay B. Cohen, and Mirella Lapaa.
2018. Don't give me the details, just the summary!
topic-aware convolutional neural networks for ex%
treme summarization. In Proceedings of the 2018
Conference on Empirical Methods in Natural lnn%
guage Processlng, pages L797--1807, Brussels, Bel%
gium. Association for Computational Linguistics.

\bibitem{} Chau Nguyen, Phuong Nguyen, Thanh Tran, Dat
Nguyen, An Trieu, Tin Pham, Anh Dang, and Le%
Minh Nguyen. 2024. Captain at coliee 2023: Ef%
flcient methods for legal information retrieval and
entailment tasks. P reprint, wXiv :2401,03551.

\bibitem{} OpenAI. 2024. GPT-4 technical report. Preprint,
arXiv:2303.08774.

\bibitem{} Arka Pal, Deep Karkhanis, Manley Roberts, Samuel
Dooley, Arvind Sundararajan, and Siddartha Naidu.
2023. Giraffe: Adventures in expanding context
lengths in llms. Preprint, arXiv:2308,10882.

\bibitem{} Richard Yuanzhe Pang, Alicia Parrish, Nitish Joshi,
Nikita Nangia, Jason Phang, Angelica Chen, Vishakh
Padmakumar, Johnny Ma, Jana Thompson, He He,
and Samuel Bowman. 2022. QuALITY Question
answering with long input texts, yes! In Proceedings
of the 2022 Conference of the North American Chap%
ter of the Association for Computational Linguistics:
Human Language Technologies, pages 5336--5358,
Seattle, United States. Association for Computational
Linguistics.

\bibitem{} Archiki Prasad, Trung Bui, Seunghyun Yoon, Hanieh
Deilamsalehy, Franck Demoncourt, and Mohit
Bansal. 2023. MeetingQA: Extractive question%
answering on meeting transcripts. In Proceedings
of the 61st Annual Meeting of the Association for
Computational Linguistics (Volume 1: Long Papers),
pages 15000--15025, Toronto, Canada. Association
for Computational Lingui stics.

\bibitem{} Hongjing Qian, Yutao Zhl, Zhicheng Dou, Haoqi Gu,
Xinyu Zhang, Zhetg Liu, Ruofei Lai, Zhao Cao,
Jian-Yun Nie, and Ji-Rong Wen. 2023. Webbrain:
Learning to generate factually correct articles for
queries by grounding on large web corpus. Prepint,
arXiv:2304.04358.

\bibitem{} Jack W. Rae, Anna Potapenko, Siddhant M. Jayakumar,
and Timothy P. Lillicrap. 2019. Compressive trans%
formers for long-range sequence modelling. Preprint,
arXiv:1911.05507.

\bibitem{} Colin Raffel, Noam Shazeer, Adam Roberts, Katherine
Lee, Sharan Narang, Michael Matena, Yanqi Zhou,
Wei Li, and Peter J Liu. 2020. Exploring the lim%
its of transfer leaming with a unifled text-to-text
ffansformer, Journal of machine learning research,
2t(140):1--67.

\bibitem{} Varshini Reddy, Rik Koncel-Kedziorski, Viet Dac Lai,
Michael Krumdick, Charles Lovering, and Chris Tan%
ner. 2024. Docfinqa: A long-context flnancial rea%
soning dataset. P reprint, arXv: 240 1.069 I 5.

\bibitem{} William Saunders, Catherine Yeh, Jeff Wu, Steven Bills,
Long Ouyang, Jonatlan Ward, and lar Leike. 2022.
Self-critiquing models for assisting human evaluators.
P re p r int, ar'Xiv :2206.05 802.

\bibitem{} Uri Shaham, Maor lvgi, Avia Efrat, Jonathan Berant,
and Omer [xvy. 2023. ZeToSCROLLS: A zero-shot
benchmark for long text understanding. lt Find%
ings of the Association for Computational Linguis%
tics: EMNLP 2023, pages 7977--7989, Singapore.
Association for Computational Linguistics.

\bibitem{} Uri Shaham, Elad Segal, Maor Ivgi, Avia Efrat, Ori
Yoran, Adi Haviv, Ankit Gupta, Wenhan Xiong,
Mor Geva, Jonathan Berant, and Omer Levy. 2022.
SCROLLS: Standardized CompaRison over long lan%
guage sequences. In Proceedings of the 2022 Con%
ference on Empirical Methods in Natural Language
Processing, pages 12007--12021, Abu Dhabi, United
Arab Emirates. Association for Computational Lin%
guistics.

\bibitem{} Eva Sharma, Chen Li, and Lu Wang. 2019. BIG%
PAIENT: A large-scale dataset for abstractive and
coherent summarization. It Proceedings of the 57th
Annual Meeting of the Association for Computational
Lin gui s tic s, pages 2204--22 I 3, Florence, Italy. Asso%
ciation for Computational Linguistics.

\bibitem{} Nikolaos Stylianou, Panagiotis Kosmoliaptsis, and Ioan%
nis Vlahavas. 2021. Improved biomedical entity
recognition via longer context modeling. In Artifi%
cial Intelligence Applications and Innovations: lTth
IFIP WG 12.5 Intemational Conference, AIN 2021,
Hersonissos, Crete, Greece, June 25--27, 2021, Pro%
ceedings I 7, pages 45--56. Springer.

\bibitem{} Jianlin Su, Murtadha Ahmed, Yu Lu, Shengfeng Pan,
Wen Bo, and Yunfeng Lin. 2024. Roformer: En%
hanced transformer with rotary position embedding.
Neuro computing, 568:127 063.

\bibitem{} Sotaro Takeshita, Tommaso Green, Ines Reinig, Kai
Eckert, and Simone Ponzetto. 2024. ACLSUm: A
new dataset for aspect-based summarization of scien%
tific publications. In Proceedings of the 2024 Con%
ference of the North American Chapter of the Asso%
ciation for C omputational Lin guistic s : Human Lan%
gua g e Te c hno lo g ie s (Volume I : Lon g Pap e rs ), pages
666U6675, Mexico City, Mexico. Association for
Computational Linguistics.

\bibitem{} Yi Tay, Mostafa Dehghani, Samira Abnar, Yikang Shen,
Dara Bahri, Philip Pham, Jinfeng Rao, Liu Yang,
Sebastian Ruder, and Donald Metzler. 2020. Long
range arena: A benchmark for efficient transformers.
P rep rint, arXv:20 1 1.04006.

\bibitem{} Harsh Trivedi, Niranjan Balasubramanian, Thshar Khot,
and Ashish Sabharwal, 2022. Musique; Multi%
hop questions via single-hop question composition.
P reprint, arXiv:2 108.00573.

\bibitem{} Bo-Hsiang Tseng, Sheng-Syun Shen, Hung-Yi Lee, and
Lin-Shan Lee. 20 1 6. Towards machine comprehen%
sion of spoken content: Initial toefl listening compre%
hension test by machine.

\bibitem{} Ashish Vaswani, Noam M. Shazeer, Niki Parmar, Jakob
Uszkoreit, Llion Jones, Aidan N. Gomez, Lukasz
Kaiser, and Illia Polosukhin. 2017. Attention is all
you need. In Neural Information Processing Systems.

\bibitem{} Alex Wang, Richard Yuanzhe Pang, Angelica Chen, Ja%
son Phang, and Samuel R. Bowman. 2022. SQUAL%
ITY: Building a long-document summarization
dataset the hard way, In Proceedings of the 2022 Con%
ference on Empirical Methods in Nanral Language
Processing, pages 1139--1156, Abu Dhabi, United
Arab Emirates. Association for Computational Lin%
guistics.

\bibitem{} Adina Williams, Nikita Nangia, and Samuel Bowman.
2018. A broad-coverage challenge corpus for sen%
tence understanding through inference. In Proceed%
ings of the 2018 Conference of the North American
Chapter of the Association for Computational Lin%
guistics: Human Language Technologies, Volume
1 (Long Papers), pages 1112--1122, New Orleans,
Louisiana. Association for Computational Linguis%
tics.

\bibitem{} Yunshu Wu, Hayate Iso, Pouya Pezeshkpour, Nikita
Bhutani, and Estevam Hruschka. 2024. Less is
more for long document summary evaluation by llms.
P re p rint, arXiv :23 09 .07 3 82.

\bibitem{} Zhilin Yang, Peng Qi, Saizheng Zhary, Yoshua Ben%
gio, William W Cohen, Ruslan Salakhutdinov, and
Christopher D. Manning. 2018. Hotpotqa: A dataset
for diverse, explainable multi-hop question answer%
ing. P rep rint, arXiv: I 809.09600.

\bibitem{} Jiebin Zhang, Eugene J. Yu, Qinyu Chen, Chen%
hao Xiong, Dawei Zhu, Han Qian, Mingbo Song,
Xiaoguang Li, Qun Liu, and Sujian Li. 2024a.
Retrieval-based full-length wikipedia generation for
emergent events. P reprint, arXiv :2402.18264.

\bibitem{} Xinrong Zhang, Yingfa Chen, Shengding Hu, Zi%
hang Xu, Junhao Chen, Moo Khai Hao, Xu Han,
Zhen Leng Thai, Shuo Wang, Zhiyuan Liu, and
Maosong Sun. 2024b. oobench: Extending long
context evaluation beyond 100k tokens. Preprint,
atXiv:'402.13718.

\bibitem{} Yilun Zhao, Yunxiang Li, Chenying Li, and Rui Zhang.
2022. MultiHiertt: Numerical reasoning over multi
hierarchical tabular and textual data. In Proceedings
of the 60th Annual Meeting of the Association for
Computational Linguistics (Volume 1: Long Papers),
pages 6588--6600, Dublin, Ireland. Association for
Computational Linguistics.

\bibitem{} Yilun Zhao, Yitao Long, Hongjun Liu, Linyong Nan,
Lyuhao Chen, Ryo Kamoi, Yixin Liu, Xiangru Tang,
Rui Zhang, and Arman Cohan. 2023. Docmath-eval:
Evaluating numerical reasoning capabilities of llms
in understanding long documents with tabular data.
ArXiv, abs/231 1.09805.

\bibitem{} Ming Zhong, Da Yin, Tao Yu, Ahmad Zaidi, Mutethia
Mutuma, Rahul Jha, Ahmed Hassan Awadallah, Asli
Celikyilmaz, Yang Liu, Xipeng Qiu, and Dragomir
Radev. 2021. QMSum: A new benchmark for query%
based multi-domain meeting summarization. In Pro%
ceedings of the 2021 Conference of the North Amer%
ican Chapter of the Association for Computational
Lin g ui s tic s : H uman lan gua g e Te chnol o g i e s, pages
5905--5921, Online. Association for Computational
Linguistics.

\bibitem{} Yijie Zhou, Kejian Shi, Wencai Zhang, Yixin Liu, Yilun
Zhao, and Arman Cohan. 2023. Odsum: New bench%
marks for open domain multi-document summariza%
tion. P reprint, arXiv; 2309.08960.

\end{thebibliography}
=====END FILE=====

=====FILE: sections/appendix_a.tex=====
\section{A Benchmark Scope-Dispersion Classification}

In Table~\ref{tab:scope_dispersion} we delineate the different long-context
benchmarks, as well as classify them accord%
ing to how challenging they are scope-wise and
dispersion-wise.

\begin{table*}[t]
\centering
\caption{classification of long-context benchmarks in terms of scope and Dispersion.}
\label{tab:scope_dispersion}
\begin{tabular}{p{0.48\textwidth} p{0.48\textwidth}}
\toprule
\textbf{LOW SCOPE} & \textbf{HIGH SCOPE} \
\midrule
\textbf{[ILLEGIBLE] Dispersion [ILLEGIBLE]} &
\textbf{[ILLEGIBLE] Dispersion [ILLEGIBLE]} [4pt]

\textbf{Retrieval / QA / Related (as listed)}\
Qilper (Dtuigi et al., 2021)\
NamtiveQA (Kodiski et al., 2018)\
Shon-dependency QA (Li €t at., 2023)\
MultiFieldQA (Bai et at..2023)\
LitM (QA) (Liu et al.. 2024b)\
L-eval (MC QA) (An et al., 2023)\
NQ (Kwiatkowski et al.. 2019)\
RULER (single-hop QA) (Hsieh et at.. 2024)\
MeetingQA (Prasad et al.. 2023)\
BABIlong (tmks 1,4-6,9-10) (Kurarov er aL, 202,1)\
Giraffe (2 tasks) (Pal et al.. 2023)\
\emph{Retrieval}\
LitM (Key-value Retrieval) (Liu et aL. 2024b)\
MultiDoc2Dial (cSP) (Feng er al.. 2021)\
TopicRet (Dacheng Li* ard Zhang, 2023)\
Wiki-GenBen (Zhang er al.. 2024a)\
RULER (S-NIAH & MK-MAH) (Hsieh et al.. 2024)\
LongchaFlines (Pal er al.. 2023)\
\emph{NLI}\
LawogNLI (Bruno md Roth, 2022)\
ContractNll (Koreeda and Mmning, 2021b)\
Htllucination Delectinn I Dong et al.. 2024,\
FLenQA (3 t6ks) (Levy et al., 2024)\
\emph{Fill-mask}\
Cloze (Li et al., 2023)\
\emph{NLG}\
MultiDoc2Dial (Feng et a1..2021)\
QuALITY (Pang et aL. 2022)\
Long-dependency QA (Li et al., 2023)\
DuReader (Bai er al.. 2023)\
SFcition QA (An er dl..2023)\
ExpenQA (Malaviya et al., 2024)\
DocFinQA (Reddy er al., 2024)\
BABlLong (tilks 2-3,12) (Kurarov er al., 2024)\
Bamboo (QA) (Dong er al., 2024)[6pt]

\textbf{Multi-hop QA}\
MuSiQue (Trivedi et al.. 2022)\
HotpotQA (Yang er al., 2018)\
Multi-hop Tracing (Hsieh et al.. 2024)\
RULER (multi-hop QA) (Hsieh et al.. 2024)\
ZWilMultihopQA (Ho et ai.. 2020)\
\emph{NLI}\
FLenQA (3 rand. placement tsk) (kvy et il.,2024)\
Legal Textual Entailment (Nguyen et al.. 2024)\
\emph{Code Understanding}\
LCC (Cuo et al.. 2023)\
RepoBench-P (Liu et al.. 2023b)\
Codeu (An et al., 2023)\
PrivateEval (Dong er al., 2024)\
\emph{Classification}\
LRA (tmks 2. 4-6) (Tay et al., 2020)\
\emph{Retrieval}\
COLIEE (tasks 1,3,4) (Nguyen et al., 2024)\
RULER (MV-NIAH & MQ-NIAH) (Hsieh et aI.,2024)\
\emph{Next Token Prediction}\
PG-19 (Rae et al., 2019)\
Bamboo (LM) (Dong et al., 2024)\
\emph{Reasoning}\
DocMath-Eval (Zhao et al.. 2023)\
BABllong (tsks 14-20) (Kuratov et al., 2024)\
\emph{Aggregation}\
RULER (2 Aggr tuks) (Hsieh et al., 2024)\
BABILong (ilks 7-8) (Kuratov er a].. 2024)\
\emph{NLU}\
Academic Feedback Generation (An et al., 2m3)\
CUAD (Hendrycks et al., 2021)\
Factuality Evaluation\
LongSciverify (Bishop et al., 2024)\
Coreference Resolution\
11,13) (Kuratov et al.. 2024)\

&
\textbf{[ILLEGIBLE] Dispersion [ILLEGIBLE]}\
QMSum (Zhong et al., 2021)\
SQUALITY (Wmg et al.. 2022)\
Related Work Summarization (An et al., 2023)\
SPACE (Angelidt et al., 2021)\
WebBrain-c (Qim et al., 2023)\
AquaMuse (Kulkami et al.. 2020)\
FINDSUm-Liquidity (Liu et al., 2023a)\
Aggregation\
ZeToSCROLLS (SpaceDigest & Booksumson) (Shaham et at.. 2023)\
Pdsagecount (Bai et al., 2023)\
FINDSUm-ROO (Liu et al., 2023a)\
Aspecr based Summarization\
ACLSUm (Takeshita et al., 2024)\
OpenAsp (Amar et al.. 2023)\
Text Sorting\
Bamboo (Showsson & ReporSurnSon) (Dong er al.. 2024)\
Retrieval\
PilsageRetrieval (Bai er al.. 2023)\
LFQA\
LongFQA (An et a1..2023)\
NLI\
L€gal Cde Entailment (Nguyen er al., 2024)\
CovReport (HuaDg et al., 2021b)\
SummScreenFD (Chen er al.. 2022b)\
Loogle (Summarizatioo) (Li et a1., 2023)\
VCSTI,I (Bai et al., 2023)\
Self-cririquing (Saunden et al.. 2022)\
Abstract Generation (An et al.. 2023)\
Multi-News (Fabbri et al.. 2019)\
BiSPatent (Shma er al.. 2019)\
Scientific Sumrnrization (Cohan er al., 2018)\
Billsum (Komilova ild Eidelman, 2019)\
Howsumm (Boni et a1.,2021)\
ODSum (Zhou et al.. 2023)\
Klexikon (Summtrization) (Aumiller and Gertz, 2022)\
Booksum (Kryiciiski et al., 2022)\
MeetingBank (Hu er al.. 2023)\
Text Simplification\
Klexikon (Simplification) (Aumiller md Genz, 2022)\
Reasoning\
LonB Listops (Tay er al., 2020)\
Retrieval\
LRA (ask 3) (Tay et al., 2020)\
\
\
\
\
\
\
\
\
\
\
\
\
\
\
\
\
\
\
\
\
\
\
\
\
\
\
\
\
\
\
\
\
\
\
\
\
\
\
\
\
\
\
\
\
\
\
\
\
\
\
\
\
\
\
\
\
\

\
\
\
\
\
\
\
\
\
\
\
\
\
\
\
\
\
\
\
\
\
\
\
\
\
\
\
\
\
\
\
\
\
\
\
\
\
\
\
\
\
\
\
\
\
\
\
\
\
\
\
\
\
\
\
\
\

\
\
\
\
\
\
\
\
\
\
\
\
\
\
\
\
\
\
\
\
\
\
\
\
\
\
\
\
\
\
\
\
\
\
\
\
\
\
\
\
\
\
\
\
\
\
\
\
\
\
\
\
\
\
\
\
\
\
\
\
\
\
\
\
\
\
\
\
\
\
\
\
\
\
\
\
\
\
\
\
\
\
\
\
\
\
\
\
\
\
\
\
\
\
\
\
\
\
\
\
\
\
\
\
\
\
\
\
\
\
\
\
\
\
\
\
\
\
\
\
\
\
\
\
\
\
\
\
\
\
\
\
\
\
\
\
\
\

\end{tabular}
\end{table*}
=====END FILE=====

=====FILE: figures/README.txt=====
Source PDF: 
=====END FILE=====
