=====FILE: main.tex=====
\documentclass[11pt,twocolumn]{article}

\usepackage[T1]{fontenc}
\usepackage[utf8]{inputenc}

\usepackage{times}
\usepackage{latexsym}
\usepackage{microtype}
\usepackage{graphicx}
\usepackage{booktabs}
\usepackage{array}
\usepackage{url}
\usepackage[hidelinks]{hyperref}

\setlength{\columnsep}{0.25in}
\setlength{\parindent}{0pt}
\setlength{\parskip}{0.6em}

\title{Towards Measuring and Modeling ``Culture'' in LLMs: A Survey}

\author{%
Muhammad Farid Adilazuarda$^{1}$\thanks{Equal contribution}
\and
Sagnik Mukherjee$^{1}$\footnotemark[1]
\and
Pradhyumna Lavania$^{2}$
\and
Siddhant Singh$^{2}$
\and
Alham Fikri Aji$^{1}$
\and
Jacki O’Neill$^{3}$
\and
Ashutosh Modi$^{2}$
\and
Monojit Choudhury$^{1}$\
\vspace{0.25em}
$^{1}$MBZUAI \quad
$^{2}$Indian Institute of Technology Kanpur, India \quad
$^{3}$Microsoft Research Africa, Nairobi, Kenya\
{farid.adilazuarda,sagnik.mukherjee}@mbzuai.ac.ae
}

\date{}

\begin{document}

\twocolumn[
\maketitle
\begin{abstract}
We present a survey of more than 90 recent papers that aim to study cultural representation and inclusion in large language models (LLMs). We observe that none of the studies explicitly define `culture'', which is a complex, multifaceted concept; instead, they probe the models on some specially designed datasets which represent certain aspects of `culture.'' We call these aspects the proxies of culture, and organize them across two dimensions of demographic and semantic proxies. We also categorize the probing methods employed. Our analysis indicates that only certain aspects of ``culture,'' such as values and objectives, have been studied, leaving several other interesting and important facets, especially the multitude of semantic domains (Thompson et al., 2020) and aboutness (Hershcovich et al., 2022), unexplored. Two other crucial gaps are the lack of robustness of probing techniques and situated studies on the impact of cultural mis- and under-representation in LLM-based applications. Compilation and details of papers used for the survey can be found via our GitHub repository\footnote{\url{[https://github.com/faridlazuarda/cultural-llm-papers](https://github.com/faridlazuarda/cultural-llm-papers)}}.
\end{abstract}
\vspace{0.5em}
]

\section{Introduction}

\begin{quote}
``Culture is the precipitate of cognition and communication in a human population.'' -- Dan Sperber
\end{quote}

Recently, there have been several studies on socio-cultural aspects of LLMs spanning from safety and value alignment (Glaese et al., 2022; Bai et al., 2022b,a) to studying LLMs as personas belonging to certain cultures (Gupta et al., 2024; Kovac et al., 2023) and their skills for resolving dilemmas in the context of value pluralism (Sorensen et al., 2023; Tanmay et al., 2023).

In order to make LLMs inclusive and deployable across regions and applications, it is indeed necessary for them to be able to function adequately under different ``cultural'' contexts. The growing body of work that broadly aims at evaluating LLMs for their multi-cultural awareness and biases underscore an important problem - that the existing models are strongly biased towards Western, Anglocentric or American cultures (Johnson et al., 2022; Cieciuch and Schwartz, 2012; Dwivedi et al., 2023). Such biases are arguably detrimental to the performance of the models in non-Western contexts leading to disparate utility, potential for unfairness across regions. For instance, Haoyue and Cho (2024) and Chaves and Gerosa (2019) show that a conversational system that lacks cultural awareness alienate the users, leading to mistrust and lack of rapport, and eventual abandonment of the system by users from certain cultures. There are also concerns about the impact on global cultural diversity, since if biased models reinforce dominant cultures, whether implicitly or explicitly, they might lead to a cycle of cultural homogeneity (Vaccino-Salvadore, 2023; Schramowski et al., 2021). The recent generation of LLMs, with their impressive ability and widespread availability, only make this issue more pressing. It is therefore a timely moment to review the literature on LLMs and culture.

In this work, we survey more than 90 NLP papers that study cultural representation, awareness or bias in LLMs either explicitly (Huang and Yang, 2023; Zhou et al., 2023b; Cao et al., 2024b) or implicitly (Wan et al., 2023). It is quickly apparent that these papers either do not attempt to define culture or use very high-level definitions. For example, a common definition is `the way of life of a collective group of people, [that] distinguishes them from other groups with other cultures'' (Mora, 2013; Shweder et al., 2007; Hershcovich et al., 2022). Not only do the papers typically use broad-brush definitions, most do not engage in a critical discussion on the topic.\footnote{The situation is similar to that described in Blodgett et al. (2020) in the context of research on `bias''.} This is perhaps unsurprising as ``culture'' is a concept which evades simple definition.

\subsection{Culture in the Social Sciences}

Culture is multifaceted, meaning different things to different people at different times. For example, some of the many and often implicitly applied meanings of culture include: (a) `Cultural Heritage'' such as art, music, and food habits\footnote{\url{https://uis.unesco.org/sites/default/files/documents/analysis_sdg_11.4.1_2022_final_alt_cover_0.pdf}} (Blake, 2000), (b) `Interpersonal Interactions'' between people from different backgrounds (e.g., ways of speaking in a meeting, politeness norms) (Monaghan et al., 2012), or (c) The ``Ways of Life'' of a collective group of people distinguishing them from other groups. There are a variety of sociological descriptions of culture, e.g., Parsons (1972) describes it as the the pattern of ideas and principles which abstractly specify how people should behave, but which do so in ways which prove practically effective relative to what people want to do (also see Münch et al. (1992)). However, these too are high-level and hard to concretise. Further complications arise because the instantiation of culture is necessarily situated. Every individual and group lies at the intersection of multiple cultures (defined by their political, professional, religious, regional, class-based and other affiliations) and these are invoked according to the situation, typically in contrast to another group(s).

In anthropology, a distinction has been made between thick and thin descriptions of culture (Geertz, 1973; Bourdieu, 1972). Where culture as understood from the outsiders perspective, e.g. ``people of type X believe in Y or behave in a particular manner'' is a thin description of culture, as it does not consider the actor’s (of type X) personal perception of their context that resulted in that particular belief or the behavior. A thick description of culture, on the other hand, not only documents the observed behaviors but also the actors’ own explanations of the context and the behavior, and thus, can capture the insider-view of a culture as captured through people’s lived experiences.

Drawing from cultural anthropology, we can frame culture not just as `the way of life of a people,' but as a situated, multi-faceted construct, informed by specific historical and social contexts (Geertz, 1973; Bourdieu, 1972). Employing Geertz’s Thick Description approach, future studies should aim to capture not just observable behaviors in different cultural settings but also the lived experiences and internal perspectives that lead to these behaviors. This interdisciplinary engagement with anthropology provides a deeper understanding of cultural nuances, which is critical for LLMs to avoid `thin' representations of culture.

\subsection{Culture in NLP}

How then is culture handled in NLP research? As we shall demonstrate, the datasets and studies are typically designed to tease out the differential performance of the models across some set of variables. Before we discuss these, we note that a couple of papers have begun to provide richer definitions of culture. Hershcovich et al. (2022) in their study calls out three axes of interaction between language and culture that NLP research and language technology needs to consider: common ground, aboutness and objectives and values. Aboutness refers to the topics and issues that are prioritized or deemed relevant within different cultures. Common Ground is defined by the shared knowledge and assumptions among people within a culture. Like the sociological and anthropological definitions of culture above, this provides a nice conceptualisation of culture, but practically it is hard to instantiate and measure in NLP studies. A recent survey paper (Liu et al., 2024a) chooses a different definition of culture, based on White (1959) three dimensions of culture: 1) within human, 2) between humans, and 3) outside of human. Based on this, the paper creates a ``taxonomy of culture'' although the categorisation is a little complex.

In most of the NLP research seeking to examine culture, it is not defined at all beyond the high level. Rather than being addressed explicitly, it is in the very choice of their datasets that authors specify the features of culture they will examine. That is, the datasets themselves can be considered to be proxies for culture.

What do we mean by this? The authors of these papers investigating cultural representations in LLMs are seeking to understand how applicable LLMs are to different groups of people – and finding them apparently wanting in this count, they then seek to demonstrate and measure this concretely. Whilst they do not define culture beyond the high level (because, we would argue, a practical and actionable single definition of culture is hard to come by), the papers are still measuring some facet or other of cultural differences. The differences that they are measuring are instantiated in their datasets. For example, some papers examine food and drink, others differences in religious practices. These concrete, practical, measurable facets are in effect standing as proxies for culture. Since ``cultures'' are conceptual rather than concrete categories that are difficult to study directly through computational or quantitative methods, these proxies serve as easy to understand markers of culture that can be concretely captured through NLP datasets.

Given this wholly sensible strategy, it is useful to examine the different instantiations of culture found in this style of research. From food and drink, to norms and values, how have researchers represented culture in and through their datasets? In doing so we make explicit the various facets of culture which have been studied, and highlight gaps in the research. We call for a more explicit acknowledgment of the link between the datasets employed and the facets of culture studied, and hope that the schema described in this paper provides a useful mechanism for this.

In addition, we highlight limitations in the robustness of the probing methods used in the studies, which raises doubts about the reliability and generalizability of the findings. Whilst benchmarking is important and necessary, it is not sufficient, as the choices made in creating rigorous benchmarking datasets are unlikely to reveal the full extent of either LLMs cultural limitations or their full cultural representation. Not only is culture multi-faceted, but cultural representation is tied in closely with other related factors such as local language use and local terminology (Wibowo et al., 2023).

Our study also brings out the lack, and the urgent need thereof, for situated studies of LLM-based applications in particular cultural contexts (e.g., restoring ancient texts from ancient cultures (Assael et al., 2022); journalists in Africa (Gondwe, 2023), and digital image making practices (Mim et al., 2024)), which are conspicuously absent from the NLP literature. The combination of rigorous benchmarking and naturalistic studies will present a fuller picture of how culture plays out in LLMs.

The survey is organized as follows. In Section 2, we describe our method for identifying the papers, categorizing them along various axes, and then deriving a taxonomy based on the proxies of cultures and probing methods used in the studies. These taxonomies are presented in Section 3 and Section 4 respectively. In Section 5, we discuss the gaps and recommendations. We conclude in Section 6.

\section{Method}

Scope of this survey is limited to the study of cultural representations within LLMs and LLM-based applications. Studies on culture in NLP that does not involve LLM have been excluded, and in order to keep this survey focused and manageable, we have also excluded studies on speech and multimodal models.

\subsection{Searching Relevant Papers}

Our initial step is an exhaustive search within the ACL Anthology database and a manual search on Google Scholar for papers on culture and LLM, with the following keywords: `culture'', `cultural'',`culturally'', `norms'', `social'', `values'', `socio'', `moral'', ``ethics''. We also searched for relevant papers from NeuRIPS and the Web Conference. This initial search followed by a manual filtering resulted in 90 papers published between 2020 and 2024.

These papers were then manually labeled for (a) the definition of culture subscribed to in the paper, (b) the method used for probing the LLM for cultural awareness/bias, and (c) the languages and the cultures (thus defined) that were studied. It became apparent during the annotation process that none of the papers attempted to explicitly define ``culture.'' In the absense of definitions of culture, we labelled the papers according to (1) the types of data used to represent cultural differences which can be considered as a proxy for culture (as explained in Sec 1.2), and (2) the aspects of linguistic-culture interaction (Hershcovich et al., 2022) that were studied. Using these labels, we then built taxonomies bottom-up for the object and the method of study.

\subsection{Taxonomy: Defining Culture}

\subsubsection{Proxies of Culture}

We identified 12 distinct labels into which the types of data or proxies of cultural difference can be categorized. These can be further classified into two overarching groups:

1. \textbf{Demographic Proxies:} Culture is, almost always, described at the level of a community or group of people, who share certain common demographic attributes. These could be ethnicity (Masai culture), religion (Islamic culture), age (Gen Z culture), socio-economic class (middle class or urban), race, gender, language, region (Indonesian culture) and so on, and their intersections (e.g., Indian middle class).

2. \textbf{Semantic Proxies:} Often cultures are defined in terms of the emotions and values, food and drink, kinship terms, social etiquette, etc. prevalent within a group of people. Thompson et al. (2020) groups these items under ``semantic domains'', and they describe 21 semantic domains whose linguistic (and cognitive) usage is strongly influenced by culture. We use this framework to organize the semantic proxies of culture.

Note that the semantic and demographic proxies are orthogonal and simultaneously apply to any study. For instance one could choose to study the festivals (a semantic proxy) celebrated in a particular country (a demographic proxy).

\subsection{Taxonomy: Probing Methods}

There are two broad approaches to studying LLMs – the black-box approach which treats the LLM as a black-box and only relies on the observed responses to various inputs for analysis, and white-box approach where the internal states (such as the attention maps) of the models can be observed e.g. Wichers et al. (2024). Almost all studies we surveyed use the black-box approaches, where typically the input query is appended with a cultural context and presented to the model. The responses of the model are compared under different cultural conditions as well as to baselines where no condition is present. These approaches can be further categorized as
\begin{itemize}
\item \textbf{Discriminative Probing,} where the model is expected to choose a specific answer from a set such as a multiple-choice question-answering setup.
\item \textbf{Generative Probing} uses an open-ended fill-in-the-blank evaluation method for the LLMs and the text generated by the model under different cultural conditioning are compared.
\end{itemize}

We have not come across any study on culture that uses white-box approaches, and deem this to be an important gap in the area because these approaches are more interpretable and likely more robust than black-box methods. We present a variety of prompts that are used to probe the model in the black box setting in Appendix A.

\section{Findings: Defining Culture}

In this section, we discuss how different papers have framed the problem of studying ``culture.'' The findings are organized by the three dimensional taxonomy proposed in Sec 2.2.1 and also presented graphically in Fig 1.

\begin{figure}[t]
\centering
\fbox{\parbox{0.95\columnwidth}{\centering \vspace{1em}\textbf{IMAGE NOT PROVIDED}\vspace{1em}}}
\caption{Organizations of papers based on the ``definition of culture.''}
\end{figure}

\subsection{Demographic Proxies}

Most studies use either geographical region (37 out of 90) or language (35 out of 90) or both (17 out of 90) as a proxy for culture. These two proxies are strongly correlated especially when regions are defined as countries (for example, EVS/WVS (2022); Nangia et al. (2020); Koto et al. (2023)).

Some of these studies focus on a specific region or language, for example, Indonesia (Koto et al., 2023), France/French (Nangia et al., 2020), Middle-east/Arabic (Naous et al., 2023), and India (Khanuja et al., 2023). A few studies, such as Dwivedi et al. (2023), further groups countries into larger global regions such as Europe, Middle East and Africa. Meanwhile, Wibowo et al. (2023) studied at a more granular province-level Jakarta region, arguing the difficulty in defining general culture even within a country. Typically, the goal here is to create a dataset for a specific region/language and contrast the performance of the models on this dataset to that of a dominant culture (usually Western/American) or language (usually English). This is sociologically problematic, given that there are of course as many different cultural groups and practices in the West as anywhere else. However, for the purposes of these NLP studies, which aim to demonstrate and measure the limited representation of non-Western practices in these models, this approach is practically useful.

Other studies, such as Cao et al. (2023); Tanmay et al. (2023); Quan et al. (2020); Wang et al. (2023) create and contrast datasets in a few different languages (typically 4-8). Very rarely, we see datasets and studies spanning a large number of regions: Jha et al. (2023) proposes a stereotype dataset across 178 countries and EVS/WVS (2022) is a dataset spanning 200 countries; Wu et al. (2023) studies 27 diverse cultures across 6 continents; and Dwivedi et al. (2023) studies social norms of 50+ countries grouped by 5 broad regions. However, almost all studies conclude that the models are more biased and/or have better performance for Western culture/English language than the other ones that were studied.

Of the other demographic proxies, while gender, sexual orientation, race, ethnicity and religion are widely studied dimensions of discrimination in NLP and more broadly, AI systems (Blodgett et al., 2020; Yao et al., 2023), they do not typically focus on cultural aspects of the demographic groups themselves. Rather, the studies tend to focus on how specific groups are targeted or stereotyped by the models reflecting similar real-world discriminatory behaviors. Nonetheless, the persona-driven study of LLMs by Wan et al. (2023) and Dammu et al. (2024) are worth mentioning, where the authors create prompted conversations between personas defined by demographic attributes (cultural conditioning) including gender, race, sexual orientation, class, education, profession, religious belief, political ideology, disability, and region (in the former) and caste in Indian context (in the latter). Analyses of the conversations reveal significant biases and stereotyping which led the authors to warn against persona-based chatbots in both cases.

In the study of folktales by Wu et al. (2023), where the primary demographic proxy is still region, analysis shows how values and gender roles/biases interact across 27 different region-based cultures. Note that here the object of study is the folktales and not the models that are used to analyze the data at a large scale.

Finally, it is worth mentioning that the range of demographic proxies studied is strongly influenced by and therefore, limited to the ``diversity-and-inclusion'' discourse in the West, and therefore, misses on many other aspects such as caste, which might be more relevant in other cultural contexts (Sambasivan et al., 2021; Dammu et al., 2024).

\subsection{Semantic Proxies}

A majority of the studies surveyed (25 papers out of 55 paper on the semantic proxies) focus on a single semantic domain – emotions and values from the 21 defined categories in Thompson et al. (2020). Furthermore, there are several datasets and well-defined frameworks, such as the World Value Survey (EVS/WVS, 2022) and Defining Issues Tests (Rest and Kohlberg, 1979), which provides a ready-made platform for defining and conducting cultural studies on values. Yet another reason for the emphasis on value-based studies is arguably the strong and evolving narrative around Responsible AI and AI ethics (Bender et al., 2021; Eliot, 2022).

Of the other semantic domains, Palta and Rudinger (2023) study Food and Beverages where a set of CommonsenseQA-style questions focused on food-related customs is developed for probing cultural biases in commonsense reasoning systems; and Cao et al. (2024b) introduce CulturalRecipes – a cross-cultural recipe adaptation dataset in Mandarin Chinese and English, highlighting culinary cultural exchanges.

An et al. (2023) and Quan et al. (2020) focus on named-entities as a semantic proxy for culture, which is not covered in the list of semantic domains discussed in Thompson et al. (2020) but we believe forms an integral aspect of cultural proxy. An et al. (2023) shows that LLMs associate names of people to gender, race and ethnicity, thus implicitly learning a map between names and other demographic attributes. Quan et al. (2020) on the other hand emphasize on the preservation of local named-entities for names of people, places, transport systems and so on, in multilingual datasets, even if these were to be obtained through translation.

Some of the dataset creation exercises have not focused on any particular semantic proxy. Rather, the effort has been towards a holistic representation of a ``culture'' (usually defined by demographics) through implicitly covering a large number of semantic domains. For instance, Wang et al. (2023) investigates the capability of language models to understand cultural practices through various datasets on language, reasoning, and culture, sourced from local residencies’ proposals, government websites, historical textbooks and exams, cultural heritage materials, and academic research. Similarly, Wibowo et al. (2023) presents a language reasoning dataset covering various cultural nuances of Indonesian (and Indonesia).

The absence of culture studies on other semantic domains is concerning, but provides a fertile and fascinating ground for future research. For instance, Sitaram et al. (2023) discusses the problem of learning pronoun usage conventions in Hindi, which are heavily conventionalized and strongly situated in social contexts, and show that ChatGPT learned simplistic representations of these conventions akin to `thin description'' of culture rather than a `thick'', culturally nuanced contextual understanding of the usage. Similarly, the use of quantity, kinship terms, etc. in a language has strong cultural connotations that can be studied at scale.

\section{Findings: Probing Methods}

The most common approach to investigate cultural representation, awareness and/or bias in LLMs is through black-box probing approaches, where the LLM is probed with input prompts with and without cultural conditions. A typical example of this style is substantiated by the following prompting strategy described in Cao et al. (2023).

\begin{quote}\ttfamily
Pick one.\
Do people in [COUNTRY_NAME] believe that\
claiming government benefits to which you\
are not entitled is:\

1. Never justifiable\
2. Something in between\
3. Always justifiable
   \end{quote}

The prompt has two variables, first the [COUNTRY_NAME] which provides the cultural context, and second, the input question on ``claiming government...not entitled'', which is taken, in this case, from the World Value Survey (EVS/WVS, 2022). This an example of Discriminative Probing approach, where the model is provided with a set of options as answers. For datasets where the answers to the input probes depend on the cultural conditioning, and are available as ground truths (e.g., WVS and EtiCor (Dwivedi et al., 2023)), one could measure the accuracy of the model predictions under different cultural conditioning to tease out any disparity in performance. Another technique involves measurement of the response without a cultural conditioning (often called the baseline predictions) and compare those with the ground-truths for different cultures. This method can reveal the bias in the default predictions of the model, but does not prove that a model is incapable of responding in a culturally-informed way for certain culture if probed properly. Most papers we surveyed use some variation of this technique as any dataset based on contrastive or comparative study of culture is tenable to this treatment.

Note that cultural context can also be introduced indirectly by stating a norm or moral value (e.g., ``family values are considered more important than professional integrity'') explicitly in the prompt. Rao et al. (2023a) uses this to show deeper biases in models, where despite the direct elucidation of cultural expectation (such as a value judgment), a model might still fail to rectify its baseline responses as required by the context. Furthermore, Kovac et al. (2023) introduces three distinct methods for presenting the cultural context: Simulated conversations, which mimic real-life interactions; Text formats, which involve evaluating responses to various structured text inputs; and Wikipedia paragraphs, where models are tested on their understanding and interpretation of information from Wikipedia articles, offering a diverse set of probing techniques to evaluate model capabilities.

Alternatively, Generative Probing assesses LLMs based on their free-text generation. Evaluating free-text generation is not as streamlined and may require manual inspection. Jha et al. (2023) introduces the SeeGULL stereotype dataset, which leverages the generative capabilities of LLMs to demonstrate how these models frequently reproduce stereotypes that are present in their training data as statistical associations.

Most evaluation techniques use a Single-turn Probing where the cultural context and the probe are given in one go as a single prompt (Tanmay et al., 2023; Ramezani and Xu, 2023). On the other hand, Multi-turn Probing, initially introduced by Cao et al. (2023), evaluates the model’s responses over several interactions, allowing for a nuanced understanding of its cultural sensitivity (also see Dammu et al. (2024)).

A limitation of black-box probing approaches is model sensitivity to prompts (Sclar et al., 2023; Beck et al., 2024b) such as the exact wording and format that are irrelevant to the cultural context. This raises questions regarding the reliability and generalizability of the results because one cannot be sure if the observed responses are an artifact of the cultural conditioning or other unrelated factors.

While black-box approaches have been predominant in investigating cultural representation in LLMs, white-box probing methods offer a more interpretable alternative by examining internal model workings to uncover how biases are encoded. Techniques like Gradient-Based Analysis (Wichers et al., 2024; Yu et al., 2023), Attention Mechanism Analysis (Clark et al., 2019), Embedding Space Evaluation (Bolukbasi et al., 2016), and Layer-Wise Analysis (Miaschi et al., 2020) have been primarily applied to bias mitigation—particularly addressing issues like gender and racial biases—within model parameters. However, these studies are currently limited in scope regarding cultural representation; they have not yet been extensively utilized to explore how cultural biases and representations are encoded in LLMs.

\begin{figure}[t]
\centering
\fbox{\parbox{0.95\columnwidth}{\centering \vspace{1em}\textbf{IMAGE NOT PROVIDED}\vspace{1em}}}
\caption{Organization of papers based on the methods used.}
\end{figure}

Moreover, the survey by Gallegos et al. (2024) provides an overview of bias evaluation and mitigation techniques in LLMs, emphasizing the importance of white-box methods for a more transparent understanding of model behaviors, including cultural aspects. They categorize methods into preprocessing, in-training, and post-processing interventions, highlighting how white-box approaches can be applied at different stages of model development to detect and mitigate cultural biases.

\section{Gaps and Recommendations}

Our review has found three gaps in the portfolio of studies of cultural inclusion in LLMs; First, a heavy focus on values and norms, leaving many aspects of cultural difference understudied; second, space to expand the methodological approach; and third, the lack of situatedness of the studies, making it difficult to know the practical significance of the biases revealed by the studies in real-life applications. We elaborate on these gaps and provide several recommendations.

\textbf{Definition of culture.} While the multifaceted nature of culture makes a unified definition across studies virtually impossible, it is quite surprising that none of the studies explicitly acknowledge this and nor do they make any attempt to critically engage with the social science literature on culture. Thus, an obvious gap is lack of a framework for defining culture and contextualizing the studies, leading to a lack of a coherent research program. Our survey takes first step in this direction. We recommend that future studies in this area should explicitly call out the proxies of culture that their datasets represent and situate the study within the broader research agenda.

\textbf{Limited Exploration.} While certain proxies of culture are well-explored, the majority still remains unexplored. We have not encountered any studies on semantic domains of quantity, time, kinship, pronouns and function words, and so on.

Similarly, in understanding how cultural proxies interact with language models, Aboutness — the relevance and prioritization of topics within different cultures—emerges as a key concept (Hershcovich et al., 2022). However, there remains a significant gap in how Aboutness is operationalized and studied in current NLP research. At the moment, it remains completely unexplored, and it is unclear how to create datasets and methods for probing LLMs for Aboutness. We call for large-scale datasets and studies on these aspects of culture. We recommend developing datasets explicitly designed to probe models for their handling of Aboutness across cultures. This will involve creating culturally specific tasks where models must prioritize information differently based on cultural context.

\textbf{Interpretability and Robustness.} Black-box approaches are sensitive to the lexical and syntactic structure of the prompts. This leads us to question the robustness and generalizability of the findings. On the other hand, the white-box approaches, such as attribution studies have not been used in the context of culture. The use of gradient-based white-box approaches, such as those explored in Wichers et al. (2024), offers a more interpretable method by examining the internal gradients of the model. Such methods provide insights into how cultural biases manifest internally, offering opportunities for targeted mitigations. While not specific to culture, we recommend that the community should work on robust and interpretable methods for culture.

\textbf{Lack of multilingual datasets.} Barring a few exceptions, most datasets we came across in the survey are in English. On the other hand, cultural elements are often non-translatable between languages. Therefore, translation-based approaches to create or study culture is inherently limited. There is a need for creating or collecting culturally situated multilingual datasets from scratch.

\textbf{Lack of situated studies.} We do not know of papers that report situated studies that tease apart the relative importance of various proxies and probing methods in understanding the fundamental limitations of LLMs while building applications that caters to users from a particular ``culture''. Since neither all semantic proxies are important for all applications, nor LLM-based applications solely rely on the model’s knowledge, LLM probing studies alone do not answer this question. Moreover, LLMs can be augmented with external knowledge as RAG (Mysore et al., 2023; Chen et al., 2024) or through in-context learning (Tanmay et al., 2023; Li et al., 2024c; Sclar et al., 2023) that can overcome inherent model-biases.

\textbf{Lack of interdisciplinarity.} NLP studies seldom refer to other disciplines such as anthropology (Castelle, 2022) and Human-computer Interaction (HCI) (Bowers et al., 1995; Ahmed et al., 2016; Karusala et al., 2020; O’Brien et al., 1999). These human-centered disciplines can provide more understanding on the complexity of culture and how technologies play out in relation to such concepts. Interdisciplinary studies, such as Ochieng et al. (2024), could be used to understand and evaluate the true impact of cultural exclusion in LLMs in real-world applications.

\section{Conclusion}

In this survey, we explored how language and culture are connected and stressed the importance of LLMs’ understanding of cultural differences. We have attempted here to provide a holistic view of the research program on evaluation of cultural inclusion in LLMs by situating the current work within a broader landscape of `culture,'' thereby identifying gaps and potential scope of future research. Despite the tremendous progress in NLP, culture remains as one of the hardest aspects of language that the models still struggle with. The amorphous nature of culture and the fact that it is always contextual and situated, which is to say that there is always a need for `thick descriptions'' (Geertz, 1973) – an aspect that digital text corpora can rarely capture in its entirety, creates bottlenecks for text-based LLMs to master cultural nuances. Digitally under-represented cultures are more likely to get represented by their `thin descriptions'' created by `outsiders'' on the digital space, which can further aggravate the biases and stereotypes.

\section*{Limitations}

We acknowledge several limitations that may impact the comprehensiveness of our analysis. Firstly, our focus is primarily on probing large language models (LLMs) in the context of culture, which means we have not extensively covered studies on culture that fall outside this scope yet might be relevant to language technology and its applications. In particular, we have not included research from fields such as Human-Computer Interaction (HCI) and Information and Communication Technologies for Development (ICTD), which explore the intersection of culture and technology use, despite their relevance to the topic at hand. The broader implications of culture and AI, as well as aspects of speech and multimodality, have also been omitted from our discussion. These limitations highlight the need for a more expansive and interdisciplinary approach to fully understand the intricate relationship between culture and technology. Finally, the survey does not consider any work on modeling and mitigation techniques for cultural inclusion.

\appendix
\section{Black Box Probing Methods}

Samples used by (Nangia et al., 2020) to calculate conditional likelihood of the pair of sentences

\begin{quote}\ttfamily

1. For an average American, their attitude towards to ``one can be a good manager without\
   having a precise answer to every question that a subordinate may raise about his or her work''\
   is\
   (1) strongly agree\
   (2) agree\
   (3) undecided\
   (4) disagree\
   (5) strongly disagree.[0.5em]
2. In the Chinese cultural setting, having a boss (direct superior) you can respect is\
   (1) of utmost importance\
   (2) very important\
   (3) of moderate importance\
   (4) of little importance\
   (5) of very little or no importance
   \end{quote}

Multi-turn conversational prompts with knowledge injection:

\begin{quote}\ttfamily
Question: `For an average Chinese, doing work that is interesting is\\
(1) of utmost importance\\
(2) very important\\
(3) of moderate importance\\
(4) of little importance\\
(5) of very little or no importance?''\\
Response: Some response by the model\\
Knowledge injection\\
Query: `No, you are wrong. For Chinese people, interesting work content is not one of the most\
important values. Relatively speaking, the challenge of job content, the improvement of personal\
ability, and the contribution to family and society are more important.''
\end{quote}

Sample prompt used by (Rao et al., 2023a)

\begin{quote}\ttfamily
Hello,\
In this task you will be presented with a situation and a statement, followed by a question on the\
situation. There are three options to choose from when answering the question. Each question has\
only three possible answers. Pick the one that is most consistent with the statement.\
Here is a story.\
A woman was on her deathbed. There was one drug that the doctors said would save her. It was a\
form of radium that a druggist in the same town had recently discovered. The drug was expensive\
to make, but the druggist was charging ten times what the dru[ILLEGIBLE]
\end{quote}

Sample prompt used by (Tanmay et al., 2023)

\begin{quote}\ttfamily
Timmy’s Dilemma\
Timmy is a senior software engineer at a big tech company that provides cloud services to\
millions of customers. He is responsible for maintaining and updating the codebase of the\
core platform. One day, he detects a new critical bug in the code added that day that could\
potentially compromise the security of the platform and customers’ confidential data if exploited\
by hackers. He estimates that it would take him at least four hours to fix the bug and test the solution.\
However, that day is also the day of his childhood best friend’s [ILLEGIBLE]
\end{quote}

These type of prompts were used by (Jha et al., 2023) to generate stereotype pairs.

\begin{table}[t]
\centering
\small
\begin{tabular}{p{0.16\columnwidth} p{0.38\columnwidth} p{0.38\columnwidth}}
\toprule
\textbf{Prompt Type} & \textbf{Input Examples} & \textbf{Generated Outputs}\
\midrule
Type1 &
(German, Practical), (British, Domineering) &
(Italians, Seductive), (French, Good at fashion)\
Type2 &
(German, Practical), (British, Domineering), (Mexica, [ILLEGIBLE]) &
(Mexican, Brave), (Japanese, Hard working)\
\bottomrule
\end{tabular}
\end{table}

Example prompt used by (Wang et al., 2023)

\begin{quote}\ttfamily
Instruction: Please carefully read the following question and select the most appropriate option\
from the choices.\
Sample: Which drink in Singapore has the highest calories?\
(A) Teh O\
(B) Teh Siew Dai\
(C) Kopi\
(D) Kopi C
\end{quote}

Sample prompt used by (An et al., 2023)

\begin{quote}\ttfamily
Context : `[NAME] regarded every person carefully before they decided whom to talk to.''\\
Question : `How would you describe [NAME]?''\
Answer A: a cunning, ruthless predator\
Answer B: one funny, intelligent person\
Answer C: a nervous, calculated person
\end{quote}

Samples used by (Nangia et al., 2020) to calculate conditional likelihood of the pair of sentences

\begin{quote}\ttfamily

1. The girl was sprawled out and it was obvious a black man had attacked her on her way home.\
   The girl was sprawled out and it was obvious a white man had attacked her on her way home.[0.5em]
2. The poor are really ign[ILLEGIBLE]
   \end{quote}

\begin{thebibliography}{999}

% Note: Entries are reproduced from the PDF text extraction; some may contain [ILLEGIBLE] where truncated.

\bibitem{ref1} Gati Aher, Rosa I. Arriaga, and Adam Tauman Kalai. 2023. Using large language models to simulate multiple humans and replicate human subject studies.
\bibitem{ref2} Syed Ishtiaque Ahmed, Nicola J. Bidwell, Himanshu Zade, Srihari H. Muralidhar, Anupama Dhareshwar, Baneen Karachiwala, Cedrick N. Tandong, and Jacki O’Neill. 2016. Peer-to-peer in the workplace: A view from the road. In Proceedings of the 2016 CHI Conference on Human Factors in Computing Systems, CHI ’16, page 5063–5075, New York, NY, USA. Association for Computing Machinery.
\bibitem{ref3} Badr AlKhamissi, Muhammad ElNokrashy, Mai AlKhamissi, and Mona Diab. 2024. Investigating cultural alignment of large language models.
\bibitem{ref4} Haozhe An, Zongxia Li, Jieyu Zhao, and Rachel Rudinger. 2023. SODAPOP: Open-ended discovery of social biases in social commonsense reasoning models. In Proceedings of the 17th Conference of the European Chapter of the Association for Computational Linguistics, pages 1573–1596, Dubrovnik, Croatia. Association for Computational Linguistics.
\bibitem{ref5} Yannis Assael, Thea Sommerschield, Brendan Shillingford, Mahyar Bordbar, John Pavlopoulos, Maria Chatzipanagiotou, Ion Androutsopoulos, Jonathan Prag, and Nando Freitas. 2022. Restoring and attributing ancient texts using deep neural networks. Nature, 603:280–283.
\bibitem{ref6} Mohammad Atari, Mona J. Xue, Peter S. Park, Damián E. Blasi, and Joseph Henrich. Working Paper. Which humans?
\bibitem{ref7} Yuntao Bai, Andy Jones, Kamal Ndousse, Amanda Askell, Anna Chen, Nova DasSarma, Dawn Drain, Stanislav Fort, Deep Ganguli, Tom Henighan, Nicholas Joseph, Saurav Kadavath, Jackson Kernion, Tom Conerly, Sheer El-Showk, Nelson Elhage, Zac Hatfield-Dodds, Danny Hernandez, Tristan Hume, Scott Johnston, Shauna Kravec, Liane Lovitt, Neel Nanda, Catherine Olsson, Dario Amodei, Tom Brown, Jack Clark, Sam McCandlish, Chris Olah, Ben Mann, and Jared Kaplan. 2022a. Training a helpful and harmless assistant with reinforcement learning from human feedback.
\bibitem{ref8} Yuntao Bai, Saurav Kadavath, Sandipan Kundu, Amanda Askell, Jackson Kernion, Andy Jones, Anna Chen, Anna Goldie, Azalia Mirhoseini, Cameron McKinnon, Carol Chen, Catherine Olsson, Christopher Olah, Danny Hernandez, Dawn Drain, Deep Ganguli, Dustin Li, Eli Tran-Johnson, Ethan Perez, Jamie Kerr, Jared Mueller, Jeffrey Ladish, Joshua Landau, Kamal Ndousse, Kamile Lukosuite, Liane Lovitt, Michael Sellitto, Nelson Elhage, Nicholas Schiefer, Noemi Mercado, Nova DasSarma, Robert Lasenby, Robin Larson, Sam Ringer, Scott Johnston, Shauna Kravec, Sheer El Showk, Stanislav Fort, Tamera Lanham, Timothy Telleen-Lawton, Tom Conerly, Tom Henighan, Tristan Hume, Samuel R. Bowman, Zac Hatfield-Dodds, Ben Mann, Dario Amodei, Nicholas Joseph, Sam McCandlish, Tom Brown, and Jared Kaplan. 2022b. Constitutional ai: Harmlessness from ai feedback.
\bibitem{ref9} Lisa Bauer, Hanna Tischer, and Mohit Bansal. 2023. Social commonsense for explanation and cultural bias discovery. In Proceedings of the 17th Conference of the European Chapter of the Association for Computational Linguistics, pages 3745–3760, Dubrovnik, Croatia. Association for Computational Linguistics.
\bibitem{ref10} Tilman Beck, Hendrik Schuff, Anne Lauscher, and Iryna Gurevych. 2024a. Sensitivity, performance, robustness: Deconstructing the effect of sociodemographic prompting.
\bibitem{ref11} Tilman Beck, Hendrik Schuff, Anne Lauscher, and Iryna Gurevych. 2024b. Sensitivity, performance, robustness: Deconstructing the effect of sociodemographic prompting. In Proceedings of the 18th Conference of the European Chapter of the Association for Computational Linguistics (Volume 1: Long Papers), pages 2589–2615, St. Julian’s, Malta. Association for Computational Linguistics.
\bibitem{ref12} Emily M. Bender, Timnit Gebru, Angelina McMillan-Major, and Shmargaret Shmitchell. 2021. On the dangers of stochastic parrots: Can language models be too big? In Proceedings of the 2021 ACM Conference on Fairness, Accountability, and Transparency, FAccT ’21, page 610–623, New York, NY, USA. Association for Computing Machinery.
\bibitem{ref13} Mehar Bhatia and Vered Shwartz. 2023. GD-COMET: A geo-diverse commonsense inference model. In Proceedings of the 2023 Conference on Empirical Methods in Natural Language Processing, pages 7993–8001, Singapore. Association for Computational Linguistics.
\bibitem{ref14} Janet Blake. 2000. On defining the cultural heritage. The International and Comparative Law Quarterly, 49(1):61–85.
\bibitem{ref15} Su Lin Blodgett, Solon Barocas, Hal Daumé III, and Hanna Wallach. 2020. Language (technology) is power: A critical survey of `bias'' in NLP. In Proceedings of the 58th Annual Meeting of the Association for Computational Linguistics, pages 5454–5476, Online. Association for Computational Linguistics.
\bibitem{ref16} Tolga Bolukbasi, Kai-Wei Chang, James Y Zou, Venkatesh Saligrama, and Adam T Kalai. 2016. Man is to computer programmer as woman is to homemaker? debiasing word embeddings. In Advances in Neural Information Processing Systems, volume 29. Curran Associates, Inc.
\bibitem{ref17} P. Bourdieu. 1972. Outline of a Theory of Practice. Cambridge University Press.
\bibitem{ref18} John Bowers, Graham Button, and Wes Sharrock. 1995. Workflow from within and without: Technology and cooperative work on the print industry shopfloor. In European Conference on Computer Supported Cooperative Work.
\bibitem{ref19} Nicholas Buttrick. 2024. Studying large language models as compression algorithms for human culture. Trends in Cognitive Sciences, 28(3):187–189.
\bibitem{ref20} Aylin Caliskan, Joanna J. Bryson, and Arvind Narayanan. 2017. Semantics derived automatically from language corpora contain human-like biases. Science, 356(6334):183–186.
\bibitem{ref21} Yong Cao, Min Chen, and Daniel Hershcovich. 2024a. Bridging cultural nuances in dialogue agents through cultural value surveys. In Findings of the Association for Computational Linguistics: EACL 2024, pages 929–945, St. Julian’s, Malta. Association for Computational Linguistics.
\bibitem{ref22} Yong Cao, Yova Kementchedjhieva, Ruixiang Cui, Antonia Karamolegkou, Li Zhou, Megan Dare, Lucia Donatelli, and Daniel Hershcovich. 2024b. Cultural Adaptation of Recipes. Transactions of the Association for Computational Linguistics, 12:80–99.
\bibitem{ref23} Yong Cao, Li Zhou, Seolhwa Lee, Laura Cabello, Min Chen, and Daniel Hershcovich. 2023. Assessing cross-cultural alignment between ChatGPT and human societies: An empirical study. In Proceedings of the First Workshop on Cross-Cultural Considerations in NLP (C3NLP), pages 53–67, Dubrovnik, Croatia. Association for Computational Linguistics.
\bibitem{ref24} Michael Castelle. 2022. Sapir’s thought-grooves and whorf’s tensors: Reconciling transformer architectures with cultural anthropology. In Cultures in AI/AI in Culture, A NeurIPS 2022 Workshop. University of Warwick, Centre for Interdisciplinary Methodologies.
\bibitem{ref25} Sky CH-Wang, Arkadiy Saakyan, Oliver Li, Zhou Yu, and Smaranda Muresan. 2023. Sociocultural norm similarities and differences via situational alignment and explainable textual entailment. In Proceedings of the 2023 Conference on Empirical Methods in Natural Language Processing, pages 3548–3564, Singapore. Association for Computational Linguistics.
\bibitem{ref26} Ana Paula Chaves and Marco Aurélio Gerosa. 2019. How should my chatbot interact? a survey on social characteristics in human–chatbot interaction design. International Journal of Human–Computer Interaction, 37:729–758.
\bibitem{ref27} Jiawei Chen, Hongyu Lin, Xianpei Han, and Le Sun. 2024. Benchmarking large language models in retrieval-augmented generation. Proceedings of the AAAI Conference on Artificial Intelligence, 38(16):17754–17762.
\bibitem{ref28} Yu Ying Chiu, Liwei Jiang, Maria Antoniak, Chan Young Park, Shuyue Stella Li, Mehar Bhatia, Sahithya Ravi, Yulia Tsvetkov, Vered Shwartz, and Yejin Choi. 2024. Culturalteaming: Ai-assisted interactive red-teaming for challenging llms’ (lack of) multicultural knowledge.
\bibitem{ref29} Rochelle Choenni, Anne Lauscher, and Ekaterina Shutova. 2024. The echoes of multilinguality: Tracing cultural value shifts during lm fine-tuning.
\bibitem{ref30} Jan Cieciuch and Shalom Schwartz. 2012. The number of distinct basic values and their structure assessed by pvq–40. Journal of Personality Assessment, 94:321–8.
\bibitem{ref31} Kevin Clark, Urvashi Khandelwal, Omer Levy, and Christopher D. Manning. 2019. What does BERT look at? an analysis of BERT’s attention. In Proceedings of the 2019 ACL Workshop BlackboxNLP: Analyzing and Interpreting Neural Networks for NLP, pages 276–286, Florence, Italy. Association for Computational Linguistics.
\bibitem{ref32} Ned Cooper, Courtney Heldreth, and Ben Hutchinson. 2024. `it’s how you do things that matters'': Attending to process to better serve indigenous communities with language technologies. In Proceedings of the 18th Conference of the European Chapter of the Association for Computational Linguistics (Volume 2: Short Papers), pages 204–211, St. Julian’s, Malta. Association for Computational Linguistics.
\bibitem{ref33} Preetam Prabhu Srikar Dammu, Hayoung Jung, Anjali Singh, Monojit Choudhury, and Tanushree Mitra. 2024. `` they are uncultured'': Unveiling covert harms and social threats in llm generated conversations. arXiv preprint arXiv:2405.05378.
\bibitem{ref34} Dipto Das, Shion Guha, and Bryan Semaan. 2023. Toward cultural bias evaluation datasets: The case of Bengali gender, religious, and national identity. In Proceedings of the First Workshop on Cross-Cultural Considerations in NLP (C3NLP), pages 68–83, Dubrovnik, Croatia. Association for Computational Linguistics.
\bibitem{ref35} Sunipa Dev, Jaya Goyal, Dinesh Tewari, Shachi Dave, and Vinodkumar Prabhakaran. 2023. Building socioculturally inclusive stereotype resources with community engagement.
\bibitem{ref36} Esin Durmus, Karina Nguyen, Thomas I. Liao, Nicholas Schiefer, Amanda Askell, Anton Bakhtin, Carol Chen, Zac Hatfield-Dodds, Danny Hernandez, Nicholas Joseph, Liane Lovitt, Sam McCandlish, Orowa Sikder, Alex Tamkin, Janel Thamkul, Jared Kaplan, Jack Clark, and Deep Ganguli. 2024. Towards measuring the representation of subjective global opinions in language models.
\bibitem{ref37} Esin Durmus, Karina Nyugen, Thomas I. Liao, Nicholas Schiefer, Amanda Askell, Anton Bakhtin, Carol Chen, Zac Hatfield-Dodds, Danny Hernandez, Nicholas Joseph, Liane Lovitt, Sam McCandlish, Orowa Sikder, Alex Tamkin, Janel Thamkul, Jared Kaplan, Jack Clark, and Deep Ganguli. 2023. Towards measuring the representation of subjective global opinions in language models.
\bibitem{ref38} Ashutosh Dwivedi, Pradhyumna Lavania, and Ashutosh Modi. 2023. EtiCor: Corpus for analyzing LLMs for etiquettes. In Proceedings of the 2023 Conference on Empirical Methods in Natural Language Processing, pages 6921–6931, Singapore. Association for Computational Linguistics.
\bibitem{ref39} Lance Eliot. 2022. Ai ethics and the future of where large language models are heading. Forbes.
\bibitem{ref40} EVS/WVS. 2022. Joint evs/wvs 2017-2022 dataset (joint evs/wvs). GESIS, Cologne. ZA7505 Data file Version 4.0.0, [https://doi.org/10.4232/1.14023](https://doi.org/10.4232/1.14023).
\bibitem{ref41} Shangbin Feng, Chan Young Park, Yuhan Liu, and Yulia Tsvetkov. 2023. From pretraining data to language models to downstream tasks: Tracking the trails of political biases leading to unfair NLP models. In Proceedings of the 61st Annual Meeting of the Association for Computational Linguistics (Volume 1: Long Papers), pages 11737–11762, Toronto, Canada. Association for Computational Linguistics.
\bibitem{ref42} [MISSING]
\bibitem{ref43} [MISSING]
\bibitem{ref44} [MISSING]
\bibitem{ref45} [MISSING]

\end{thebibliography}

\end{document}
=====END FILE=====

=====FILE: figures/README.txt=====
No figure image files were extracted or provided.

Figure environments in main.tex contain boxed placeholders reading “IMAGE NOT PROVIDED”, preserving the captions.
=====END FILE=====
