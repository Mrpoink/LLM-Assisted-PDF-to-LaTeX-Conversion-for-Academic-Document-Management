=====FILE: main.tex=====

\documentclass[11pt,a4paper]{article}
\usepackage[utf8]{inputenc}
\usepackage{graphicx}
\usepackage{amsmath,amssymb}
\usepackage{hyperref}
\usepackage{geometry}
\geometry{margin=1in}
\usepackage{algorithm}
\usepackage{algorithmic}
\usepackage{booktabs}

\title{%%%PLACEHOLDER: PARA_0001%%%}
\author{%%%PLACEHOLDER: PARA_0002%%%}
\date{%%%PLACEHOLDER: PARA_0003%%%}

\begin{document}

\maketitle

\begin{abstract}
Language model alignment research often at￾tempts to ensure that models are not only help￾ful and harmless, but also truthful and unbiased. However, optimizing these objectives simulta￾neously can obscure how improving one aspect might impact the others. In this work, we focus on analyzing the relationship between two con￾cepts essential in both language model align￾ment and political science: truthfulness and po￾litical bias. We train reward models on various popular truthfulness datasets and subsequently evaluate their political bias. Our findings reveal that optimizing reward models for truthfulness on these datasets tends to result in a left-leaning political bias. We also find that existing open￾source reward models (i.e., those trained on standard human preference datasets) already show a similar bias and that the bias is larger for larger models. These results raise important questions about the datasets used to represent truthfulness, potential limitations of aligning models to be both truthful and politically unbi￾ased, and what language models capture about the relationship between truth and politics.
\end{abstract}

\section{Introduction}
The political bias of large language models (LLMs) has been the subject of much recent research (Feng et al., 2023; Motoki et al., 2023). Santurkar et al. (2023) found that base models tend to be more right-leaning initially, but shift towards a left-leaning stance after fine-tuning, suggesting that the alignment process may influence the models’ polit￾ical bias. However, since alignment datasets often simultaneously target helpfulness, harmlessness, and truthfulness (Bai et al., 2022), it is difficult to determine which of these objectives, if any, might be responsible for this shift in political bias.

Our interest in the relationship between truth￾fulness and political bias is motivated by findings in political science of partisan differences in sus￾ceptibility to misinformation (Baptista and Gradim, 2022) and trust in science (Cologna et al., 2024). Lower levels of trust by some political groups may be exacerbated by political bias in language models if the groups believe these models are antithetical to their values. As LLMs become more widely de￾ployed, exploring such biases and ways to remedi￾ate them becomes valuable.

We begin by testing whether vanilla open-source reward models — i.e., those fine-tuned on standard human preference datasets — show political bias, aiming to identify parts of the alignment pipeline contributing to the left-leaning bias suggested by prior work (Santurkar et al., 2023). We then train a new set of reward models (RMs) on several datasets representing different notions of truthfulness, such as everyday and scientific facts, and assess their po￾litical bias. Finally, we analyze which topics exhibit the greatest bias.

The main findings are as follows:
\begin{itemize}
\item Vanilla open-source reward models, trained on popular alignment datasets, display a clear left-leaning political bias.
\item Training reward models on datasets designed to capture “truth,” including everyday and sci￾entific facts, also results in a left-leaning bias.
\item This bias is especially strong on topics like cli￾mate, energy, or labor unions, and weakest or even reversed for taxes and the death penalty.
\end{itemize}

Our results suggest that even training on suppos￾edly objective datasets can lead to unforeseen bias. We also release a dataset of 13,855 left-leaning and right-leaning partisan statements matched on topic for use by the community [ILLEGIBLE].

\section{Related Work}
We briefly cover three areas that our work relates to: AI alignment, LLM truthfulness, and political bias in LLMs.

\subsection{Alignment}
Prior work has extensively covered ways to `align' models with human preferences (Bai et al., 2022; Casper et al., 2023), particularly the widely used technique of reinforcement learning from human feedback, or RLHF (Stiennon et al., 2020). Recent methods like DPO (Rafailov et al., 2023) bypass creating an explicit reward model; however, align￾ment datasets may still contain biases depending on the annotators’ values and preferences (Kirk et al., 2024).

\subsection{Truthfulness in LLMs}
Other work has examined how truth is represented in language models (Burns et al., 2022; Azaria and Mitchell, 2023), sometimes in terms of embedding space geometry (Marks and Tegmark, 2023). The nature of truth, however, is philosophically compli￾cated (Levinstein and Herrmann, 2024a). Several of these works present both theoretical and empiri￾cal challenges, leaving it an open question whether language models genuinely possess “truth represen￾tations" (Farquhar et al., 2023; Levinstein and Her￾rmann, 2024b). However, some approaches have shown promise in increasing truthfulness of LLMs by intervening on intermediate representations (Li et al., 2023; Chuang et al., 2024).

\subsection{Political bias in LLMs}
Prior work has also found that LLMs have politi￾cal biases (Motoki et al., 2023; Bang et al., 2024), and traced these biases’ connection to the political opinions in training data (Santurkar et al., 2023; Feng et al., 2023). This literature generally finds a left-leaning bias in LLMs; however, there are some topics where LLMs respond with right-leaning per￾spectives (Perez et al., 2023). There have also been methods proposed to reduce the political bias of language models (Liu et al., 2021).

Finally, there has been extensive research in po￾litical science on partisan differences in attitudes toward truth, such as misinformation (Baptista and Gradim, 2022) and trust in science (Cologna et al., 2024). Our work sits at the intersection of these areas of research, attempting to understand how truth and political views intersect with LLMs.


\section{Experimental Setup}

\subsection{Truthfulness Datasets}
We use several datasets corresponding to different notions of factuality to train our reward models: TruthfulQA (Lin et al., 2022), FEVER (Thorne et al., 2018), SciQ (Welbl et al., 2017), and a dataset we created of 4,000 ba￾sic LLM-generated facts and falsehoods about the world, using GPT-4 (OpenAI et al., 2023) and Gemini (Gemini Team et al., 2024). (See Appendix B for details regarding how we generated, validated and audited this last dataset.) FEVER is based on facts about entities extracted from Wikipedia. SciQ is based on scientific knowledge. TruthfulQA cov￾ers a variety of topics and was created with the goal of eliciting untruthful completions from LLMs. Finally, our generated data aimed to create the most obvious facts and falsehoods. Thus, our datasets span facts about entities (FEVER), scientific facts (SciQ), a diverse mix of difficult questions (TruthfulQA), and common sense facts (our generated data). To make the data suitable for reward model￾ing, which expects paired samples, we match a correct response to a query with an incorrect response for TruthfulQA, FEVER, and SciQ. For the generated dataset, we create random pairs of true and false statements. For datasets with multiple-choice options, we ensure that each question appears exclusively in either training or test.

\subsection{Political Dataset: TwinViews-13k}
To test reward models for political bias, we use GPT-3.5 Turbo (OpenAI, 2023) to generate TwinViews-13k, a dataset consisting of 13,855 pairs of left-leaning and right-leaning statements matched by topic. The model was instructed to keep the statements as similar as possible in style and length. We used generated statements because of the dearth of large topically matched datasets of political statement pairs; for example, the popular political compass test includes only a few statements. We extensively audited the generated statements to ensure their relevance and quality. Details of the prompt and the quality-assurance process, including a sample of the statement pairs (Table 4), can be found in Appendix A. However, we note that using LLM-generated data can lead to a variety of issues, such as the risk of agreement bias, and thus we would encourage users of this data to consider these limitations (see Section 8 for a more thorough discussion). We release the final TwinViews dataset publicly for use by the community.

\subsection{Models}
Here we clarify terminology with respect to the different model types. A “base” model refers to a pre-trained LLM without any further fine-tuning, while a “vanilla” reward model is a base model fine-tuned (only) on standard human preference datasets such as OpenAssistant (Köpf et al., 2023), Anthropic Helpful-Harmless (Bai et al., 2022), and OpenAI’s summarizing from human feedback data (Stiennon et al., 2020). A “truthful” reward model is a base model fine-tuned on a truthfulness dataset (with no preceding fine-tuning on human preference data).

For experiments on vanilla reward models, we evaluate RMs from RAFT (Dong et al., 2023), OpenAssistant, and UltraRM (Cui et al., 2023). These models were chosen due to their diversity in size and training data/methods, such that any measured political bias would be relatively generalizable. For the truthful reward models, we train several RMs on each truthfulness dataset with weights initialized from the base 160M, 2.8B and 6.9B Pythia models (Biderman et al., 2023), conducting several runs on different splits (80% train, 20% test) for robustness. We choose the Pythia models because their pretraining data is transparent and they cover a range of sizes, allowing us to understand how political bias scales with model size. We also train a simple tri-gram baseline on each dataset for the analysis in Section 5.2. After training these models (details in Appendix E), we run inference on the TwinViews data to test whether the truthful reward models still show political bias.


\section{Bias in Vanilla Reward Models}
We first examine whether vanilla open-source reward models exhibit political bias. As discussed in Section 3, we evaluate with reward models from RAFT, OpenAssistant and UltraRM. We run inference with these models on the TwinViews statements and find that all models show a left-leaning political bias, as depicted in Figure 1. Notably, larger models also show greater bias, an example of inverse scaling (McKenzie et al., 2023). However, one caveat is that the datasets/training methods are different across these reward models. The results suggest that at least part of the left-leaning political bias observed in the literature (Santurkar et al., 2023) could be due to biases introduced in reward-model training, which we believe is a new finding.

\begin{figure}[h!]
\centering
%%%PLACEHOLDER: FIG_0001%%%
\caption{Vanilla open-source reward models have a clear left-leaning political bias. All three subplots show reward scores on the paired TwinViews political statements data, with histograms broken out for the left and right sides. Dashed vertical lines indicate each side’s mean reward; a left political bias is indicated by a higher value for the blue line than the red line. The magnitude of the bias (difference in group means divided by pooled SD) is shown on each subplot. Note the presence of inverse scaling: Both model sizes and bias increase from left to right (although the training datasets/methods are different across the models).}
\end{figure}


\section{Bias in ``Truthful'' Reward Models}
While vanilla reward models exhibit a clear political slant, these models are fine-tuned on datasets of subjective human preferences reflecting diverse goals (Casper et al., 2023). Our objective is to minimize this subjectivity by training “truthful reward models”—reward models designed to give high scores to objectively truthful statements (e.g., basic everyday facts or scientific information) and low scores to false statements. As discussed in Section 3, we pursue this goal by fine-tuning various base Pythia models as reward models on each of the four truthfulness datasets, and evaluating the rewards they assign to the left and right TwinViews statements. Because any resulting political bias might be due to political content in the truthfulness datasets, we first systematically audit them for such content (in Section 5.1). We find very low rates of political content, but nevertheless exclude it from subsequent model training and analysis. Training models on these cleaned datasets produces results shown in the left three panes of Figure 2. We found that our truthful reward models generally assign higher rewards to left-leaning statements than right-leaning ones (in 11 out of 12 cases). As with vanilla models, the degree of bias also usually increased with model size.

Given that fine-tuning datasets are intended to be objective, these findings were unexpected. In Section 5.2, we use an n-gram baseline (shown in the rightmost pane of Figure 2) to consider another potential source of bias: stylistic features spuriously correlated with both truth status and political orientation. We find little support for this idea either, however, leaving the origin of the political bias shown in Figure 2 in need of further research.

\begin{figure}[h!]
\centering
%%%PLACEHOLDER: FIG_0002%%%
\caption{``Truthful'' reward models usually show a left-leaning political bias. The left three subplots show rewards assigned to TwinViews political statements by models fine-tuned on each truthfulness dataset, excluding explicitly political content found by our audit. We run five train/eval splits for each dataset and model. Individual points show results from each run, with blue points representing the average reward given to left-leaning statements and red points representing the average reward given to right-leaning statements. The red and blue bar heights show the average reward across all five runs (i.e. the average of the corresponding point values). Note the presence of inverse scaling: Larger models usually skew further left. Results of Section 5.2’s n-gram experiment appear in the rightmost pane, showing no clear relationship to the neural models’ patterns.}
\end{figure}


\section{Bias Across Topics}
Because both vanilla and ``truthful'' reward models show political bias, we used regression analysis to examine which topics or political issues exhibit the most bias. For both sets of models, we regressed the reward assigned to a TwinViews political statement on several predictors: the model, the topic, the statement’s political lean, and the topic/political-lean interaction. All models are linear regressions.

Our results are shown in Table 1. In particular, we find that for both sets of reward models, right-leaning stances are preferred to left-leaning ones on tax issues. Conversely, on topics like climate, energy, or labor unions, the left-leaning stance receives higher reward. Despite our efforts to exclude data referencing politically charged topics, these topic-specific biases may be influenced by the highly politicized nature of some issues, knowledge of which a model may acquire in pretraining.

\begin{table}[h!]
\centering
%%%PLACEHOLDER: TAB_0001%%%
\caption{Regression results on the TwinViews data for reward as a function of statement features, for reward scores from both vanilla (`Vanilla'') and Pythia-based `truthful'' reward models (``Truth FT''). Positive coefficients (in red) indicate a topic where conservative statements have higher reward, controlling for model and topic fixed effects, while negative coefficients (in blue) indicate a liberal skew. Coefficients shown are for the topic/political-leaning interaction, except for the main effect of political leaning in the last row. Robust SEs in parentheses. (* = 0.05, ** = 0.01, *** = 0.001.)}
\end{table}


\section{Conclusion}
We investigated political biases in reward models, both vanilla open-source reward models and `truthful'' reward models, and found a persistent left-leaning political bias across nearly all these models. This result is particularly surprising given the use of datasets designed to capture objective truth. Moreover, the size of the bias increases with model scale, in contrast to the usual pattern of improving capabilities. For the `truthful'' models, we considered and attempted to rule out two explanations: explicit political content in truthfulness datasets and spurious relationships between truthfulness and stylistic features. Identifying the source of this bias is a promising direction for future research, as well as understanding whether optimizing for truth leads to more or less political bias than other objectives.

More generally, this work connects to the increasing politicization of scientific facts, such as climate change (Hulme, 2009), and the problem of ``truth decay'' (Kavanagh and Rich, 2018) in the political sphere, which sit at the intersection of truth and politics. Finally, our results suggest a potential tension in achieving both truthful and unbiased models which has important implications for LLM alignment. We hope these initial findings will encourage further investigation into the relationship between truthfulness and political bias in language models.


\section{Limitations}
Our study has certain limitations, some inherent to notions of politics and truth, and some which we hope future work can investigate.

\textbf{Politics is relative.} It is difficult to create truly future-proof datasets of either apolitical factual statements or political statements, because what is considered political changes over time. Any seemingly factual issue may become politicized, as with climate change, or a political issue may cease to be controversial. In addition to being temporally localized, the definition of “political” content also varies between cultures, and our definitions come from a Western and especially US-centric perspective. We hope future work can audit truthfulness datasets for political content in a more expansive fashion. Adopting a broader notion of politics beyond the common left-right spectrum, would also help capture this rich context.

\textbf{Difficulty of capturing truth.} Datasets are an imperfect representation of truth and falsehood. Despite significant interest in identifying truthful directions in LLMs (Marks and Tegmark, 2023; Azaria and Mitchell, 2023; Burns et al., 2022), recent work has found such directions sensitive to simple perturbations like negation (Farquhar et al., 2023; Levinstein and Herrmann, 2024a). It is thus possible that our reward models learn dataset artifacts rather than truth and falsehood as such. Nevertheless, it is valuable to understand how these artifacts affect political bias in practice. Similarly, depending on generated data to measure political bias also has limitations. Biases may be introduced from both the prompts used to generate the content and the LLMs themselves, stemming from datasets used and choices made in their pre- or post-training.

\textbf{Only reward models.} We study only reward models here. While there are good reasons for this focus (they are a crucial component of the RLHF pipeline and their scalar outputs allow simple quantitative comparison of preferences), it still restricts what we can say about other alignment methods. Future research should explore how methods like direct preference optimization, or DPO (Rafailov et al., 2023), impact models aligned with them.


\section{Ethical Considerations}
\section{Ethical Considerations}
We hope that our work can shed light on biases of existing models and modeling approaches, and thereby help remedy them. We do not foresee any meaningful risks of our work or believe it has significant ethical concerns. No part of our research involved human subjects.

We used various software and data artifacts in preparing this paper and conducting the analysis it describes, all of which were subject to licenses permitting use for research. Both the alignment datasets and the existing models we used were research projects intended for use in further research, and OpenAI’s terms of use similarly permit use of their services for research. Our generated datasets are similarly available under the CC-BY 4.0 license (though note that OpenAI’s terms of service prohibit uses of their model outputs in competing products). None of the pre-existing truthfulness datasets we use should contain personally identifying or toxic content, and our audits of them found none.


\section*{Acknowledgements}
We would like to thank Yoon Kim for his helpful comments and feedback on this work.


\begin{thebibliography}{99}
Amos Azaria and Tom Mitchell. 2023. The Internal State of an LLM Knows When It’s Lying. In Findings of the Association for Computational Linguistics: EMNLP 2023, pages 967–976, Singapore. Association for Computational Linguistics.

Yuntao Bai, Andy Jones, Kamal Ndousse, Amanda Askell, Anna Chen, Nova DasSarma, Dawn Drain, Stanislav Fort, Deep Ganguli, Tom Henighan, Nicholas Joseph, Saurav Kadavath, Jackson Kernion, Tom Conerly, Sheer El-Showk, Nelson Elhage, Zac Hatfield-Dodds, Danny Hernandez, Tristan Hume, Scott Johnston, Shauna Kravec, Liane Lovitt, Neel Nanda, Catherine Olsson, Dario Amodei, Tom Brown, Jack Clark, Sam McCandlish, Chris Olah, Ben Mann, and Jared Kaplan. 2022. Training a Helpful and Harmless Assistant with Reinforcement Learning from Human Feedback. Preprint, arxiv:2204.05862.

Yejin Bang, Delong Chen, Nayeon Lee, and Pascale Fung. 2024. Measuring political bias in large language models: What is said and how it is said. In Proceedings of the 62nd Annual Meeting of the Association for Computational Linguistics (Volume 1: Long Papers), pages 11142–11159, Bangkok, Thailand. Association for Computational Linguistics.

João Pedro Baptista and Anabela Gradim. 2022. Who believes in fake news? identification of political (a)symmetries. Social Sciences, 11(10):460.

Stella Biderman, Hailey Schoelkopf, Quentin Anthony, Herbie Bradley, Kyle O’Brien, Eric Hallahan, Mohammad Aflah Khan, Shivanshu Purohit, USVSN Sai Prashanth, Edward Raff, Aviya Skowron, Lintang Sutawika, and Oskar Van Der Wal. 2023. Pythia: A suite for analyzing large language models across training and scaling. In Proceedings of the 40th International Conference on Machine Learning, volume 202 of ICML’23, pages 2397–2430, Honolulu, HI, USA. JMLR.org.

Collin Burns, Haotian Ye, Dan Klein, and Jacob Steinhardt. 2022. Discovering Latent Knowledge in Language Models Without Supervision. In The Eleventh International Conference on Learning Representations.

Stephen Casper, Xander Davies, Claudia Shi, Thomas Krendl Gilbert, Jérémy Scheurer, Javier Rando, Rachel Freedman, Tomasz Korbak, David Lindner, Pedro Freire, Tony Tong Wang, Samuel Marks, Charbel-Raphael Segerie, Micah Carroll, Andi Peng, Phillip Christoffersen, Mehul Damani, Stewart Slocum, Usman Anwar, Anand Siththaranjan, Max Nadeau, Eric J. Michaud, Jacob Pfau, Dmitrii Krasheninnikov, Xin Chen, Lauro Langosco, Peter Hase, Erdem Biyik, Anca Dragan, David Krueger, Dorsa Sadigh, and Dylan Hadfield-Menell. 2023. Open Problems and Fundamental Limitations of Reinforcement Learning from Human Feedback. Transactions on Machine Learning Research.

Yung-Sung Chuang, Yujia Xie, Hongyin Luo, Yoon Kim, James Glass, and Pengcheng He. 2024. Dola: Decoding by contrasting layers improves factuality in large language models. Preprint, arXiv:2309.03883.

Viktoria Cologna, Niels G. Mede, Sebastian Berger, John C. Besley, Cameron Brick, Marina Joubert, Edward Maibach, Sabina Mihelj, Naomi Oreskes, Mike S. Schäfer, and Sander Van Der Linden. 2024. Trust in scientists and their role in society across 68 countries.

Ganqu Cui, Lifan Yuan, Ning Ding, Guanming Yao, Wei Zhu, Yuan Ni, Guotong Xie, Zhiyuan Liu, and Maosong Sun. 2023. UltraFeedback: Boosting Language Models with High-quality Feedback. Preprint, arxiv:2310.01377.

Hanze Dong, Wei Xiong, Deepanshu Goyal, Yihan Zhang, Winnie Chow, Rui Pan, Shizhe Diao, Jipeng Zhang, KaShun Shum, and Tong Zhang. 2023. RAFT: Reward rAnked FineTuning for Generative Foundation Model Alignment. Transactions on Machine Learning Research.

Sebastian Farquhar, Vikrant Varma, Zachary Kenton, Johannes Gasteiger, Vladimir Mikulik, and Rohin Shah. 2023. Challenges with unsupervised LLM knowledge discovery. Preprint, arxiv:2312.10029.

Shangbin Feng, Chan Young Park, Yuhan Liu, and Yulia Tsvetkov. 2023. From Pretraining Data to Language Models to Downstream Tasks: Tracking the Trails of Political Biases Leading to Unfair NLP Models. In Proceedings of the 61st Annual Meeting of the Association for Computational Linguistics (Volume 1: Long Papers), pages 11737–11762, Toronto, Canada. Association for Computational Linguistics.

Gemini Team, Rohan Anil, Sebastian Borgeaud, Jean-Baptiste Alayrac, Jiahui Yu, Radu Soricut, Johan Schalkwyk, Andrew M. Dai, Anja Hauth, Katie Millican, David Silver, et al. 2024. Gemini: A Family of Highly Capable Multimodal Models. Preprint, arxiv:2312.11805.

Michael Hulme. 2009. Why We Disagree about Climate Change: Understanding Controversy, Inaction and Opportunity. Cambridge University Press.

Bryan Jones, Frank Baumgartner, Sean Theriault, Derek Epp, Cheyenne Lee, and Miranda Sullivan. 2019. Policy Agendas Project: Codebook.

Jennifer Kavanagh and Michael D Rich. 2018. Truth Decay: An Initial Exploration of the Diminishing Role of Facts and Analysis in American Public Life. RAND Corporation.

Hannah Rose Kirk, Alexander Whitefield, Paul Röttger, Andrew Bean, Katerina Margatina, Juan Ciro, Rafael Mosquera, Max Bartolo, Adina Williams, He He, Bertie Vidgen, and Scott A. Hale. 2024. The prism alignment project: What participatory, representative and individualised human feedback reveals about the subjective and multicultural alignment of large language models. Preprint, arXiv:2404.16019.

Andreas Köpf, Yannic Kilcher, Dimitri von Rütte, Sotiris Anagnostidis, Zhi Rui Tam, Keith Stevens, Abdullah Barhoum, Duc Minh Nguyen, Oliver Stanley, Richárd Nagyfi, Shahul Es, Sameer Suri, David Alexandrovich Glushkov, Arnav Varma Dantuluri, Andrew Maguire, Christoph Schuhmann, Huu Nguyen, and Alexander Julian Mattick. 2023. OpenAssistant Conversations - Democratizing Large Language Model Alignment. In Thirty-Seventh Conference on Neural Information Processing Systems Datasets and Benchmarks Track.

Benjamin A. Levinstein and Daniel A. Herrmann. 2024a. Still no lie detector for language models: Probing empirical and conceptual roadblocks. Philosophical Studies.

Benjamin A. Levinstein and Daniel A. Herrmann. 2024b. Still no lie detector for language models: probing empirical and conceptual roadblocks. Philosophical Studies.

Kenneth Li, Oam Patel, Fernanda Viégas, Hanspeter Pfister, and Martin Wattenberg. 2023. Inference-time intervention: Eliciting truthful answers from a language model. In Thirty-seventh Conference on Neural Information Processing Systems.

Stephanie Lin, Jacob Hilton, and Owain Evans. 2022. TruthfulQA: Measuring How Models Mimic Human Falsehoods. In Proceedings of the 60th Annual Meeting of the Association for Computational Linguistics (Volume 1: Long Papers), pages 3214–3252, Dublin, Ireland. Association for Computational Linguistics.

Ruibo Liu, Chenyan Jia, Jason Wei, Guangxuan Xu, Lili Wang, and Soroush Vosoughi. 2021. Mitigating political bias in language models through reinforced calibration. Preprint, arXiv:2104.14795.

Samuel Marks and Max Tegmark. 2023. The Geometry of Truth: Emergent Linear Structure in Large Language Model Representations of True/False Datasets. Preprint, arxiv:2310.06824.

Ian R. McKenzie, Alexander Lyzhov, Michael Martin Pieler, Alicia Parrish, Aaron Mueller, Ameya Prabhu, Euan McLean, Xudong Shen, Joe Cavanagh, Andrew George Gritsevskiy, Derik Kauffman, Aaron T. Kirtland, Zhengping Zhou, Yuhui Zhang, Sicong Huang, Daniel Wurgaft, Max Weiss, Alexis Ross, Gabriel Recchia, Alisa Liu, Jiacheng Liu, Tom Tseng, Tomasz Korbak, Najoung Kim, Samuel R. Bowman, and Ethan Perez. 2023. Inverse scaling: When bigger isn’t better. Transactions on Machine Learning Research.

Fabio Motoki, Valdemar Pinho Neto, and Victor Rodrigues. 2023. More human than human: Measuring ChatGPT political bias. Public Choice.

OpenAI. 2023. GPT-3.5-turbo.

OpenAI, Josh Achiam, Steven Adler, Sandhini Agarwal, Lama Ahmad, Ilge Akkaya, Florencia Leoni Aleman, Diogo Almeida, Janko Altenschmidt, Sam Altman, et al. 2023. GPT-4 Technical Report. Preprint, arxiv:2303.08774.

Fabian Pedregosa, Gaël Varoquaux, Alexandre Gramfort, Vincent Michel, Bertrand Thirion, Olivier Grisel, Mathieu Blondel, Peter Prettenhofer, Ron Weiss, Vincent Dubourg, Jake Vanderplas, Alexandre Passos, David Cournapeau, Matthieu Brucher, Matthieu Perrot, and Édouard Duchesnay. 2011. Scikit-learn: Machine Learning in Python. Journal of Machine Learning Research, 12(85):2825–2830.

Ethan Perez, Sam Ringer, Kamile Lukosiute, Karina Nguyen, Edwin Chen, Scott Heiner, Craig Pettit, Catherine Olsson, Sandipan Kundu, Saurav Kadavath, Andy Jones, Anna Chen, Benjamin Mann, Brian Israel, Bryan Seethor, Cameron McKinnon, Christopher Olah, Da Yan, Daniela Amodei, Dario Amodei, Dawn Drain, Dustin Li, Eli Tran-Johnson, Guro Khundadze, Jackson Kernion, James Landis, Jamie Kerr, Jared Mueller, Jeeyoon Hyun, Joshua Landau, Kamal Ndousse, Landon Goldberg, Liane Lovitt, Martin Lucas, Michael Sellitto, Miranda Zhang, Neerav Kingsland, Nelson Elhage, Nicholas Joseph, Noemi Mercado, Nova DasSarma, Oliver Rausch, Robin Larson, Sam McCandlish, Scott Johnston, Shauna Kravec, Sheer El Showk, Tamera Lanham, Timothy Telleen-Lawton, Tom Brown, Tom Henighan, Tristan Hume, Yuntao Bai, Zac Hatfield-Dodds, Jack Clark, Samuel R. Bowman, Amanda Askell, Roger Grosse, Danny Hernandez, Deep Ganguli, Evan Hubinger, Nicholas Schiefer, and Jared Kaplan. 2023. Discovering language model behaviors with model-written evaluations. In Findings of the Association for Computational Linguistics: ACL 2023, pages 13387–13434, Toronto, Canada. Association for Computational Linguistics.

Rafael Rafailov, Archit Sharma, Eric Mitchell, Christopher D. Manning, Stefano Ermon, and Chelsea Finn. 2023. Direct Preference Optimization: Your Language Model is Secretly a Reward Model. In Thirty-Seventh Conference on Neural Information Processing Systems.

Shibani Santurkar, Esin Durmus, Faisal Ladhak, Cinoo Lee, Percy Liang, and Tatsunori Hashimoto. 2023. Whose opinions do language models reflect? In Proceedings of the 40th International Conference on Machine Learning, volume 202 of ICML’23, pages 29971–30004, Honolulu, HI, USA. JMLR.org.

Wei Shen, Rui Zheng, Wenyu Zhan, Jun Zhao, Shihan Dou, Tao Gui, Qi Zhang, and Xuanjing Huang. 2023. Loose lips sink ships: Mitigating Length Bias in Reinforcement Learning from Human Feedback. In Findings of the Association for Computational Linguistics: EMNLP 2023, pages 2859–2873, Singapore. Association for Computational Linguistics.

Nisan Stiennon, Long Ouyang, Jeff Wu, Daniel M. Ziegler, Ryan Lowe, Chelsea Voss, Alec Radford, Dario Amodei, and Paul Christiano. 2020. Learning to summarize from human feedback. In Proceedings of the 34th International Conference on Neural Information Processing Systems, NIPS ’20, pages 3008–3021, Red Hook, NY, USA. Curran Associates Inc.

James Thorne, Andreas Vlachos, Christos Christodoulopoulos, and Arpit Mittal. 2018. FEVER: A Large-scale Dataset for Fact Extraction and VERification. In Proceedings of the 2018 Conference of the North American Chapter of the Association for Computational Linguistics: Human Language Technologies, Volume 1 (Long Papers), pages 809–819, New Orleans, Louisiana. Association for Computational Linguistics.

Leandro von Werra, Younes Belkada, Lewis Tunstall, Edward Beeching, Tristan Thrush, and Nathan Lambert. 2024. TRL: Transformer Reinforcement Learning.

Johannes Welbl, Nelson F. Liu, and Matt Gardner. 2017. Crowdsourcing Multiple Choice Science Questions. In Proceedings of the 3rd Workshop on Noisy User-generated Text, pages 94–106, Copenhagen, Denmark. Association for Computational Linguistics.

Thomas Wolf, Lysandre Debut, Victor Sanh, Julien Chaumond, Clement Delangue, Anthony Moi, Pierric Cistac, Tim Rault, Remi Louf, Morgan Funtowicz, Joe Davison, Sam Shleifer, Patrick Von Platen, Clara Ma, Yacine Jernite, Julien Plu, Canwen Xu, Teven Le Scao, Sylvain Gugger, Mariama Drame, Quentin Lhoest, and Alexander Rush. 2020. Transformers: State-of-the-Art Natural Language Processing. In Proceedings of the 2020 Conference on Empirical Methods in Natural Language Processing: System Demonstrations, pages 38–45, Online. Association for Computational Linguistics.
\end{thebibliography}


\appendix

\section{TwinViews-13k: Political Statements}
\textbf{Prompt.} We use the following prompt with GPT-3.5-turbo in a loop to generate 13,855 political statements. Many of these statements are on similar topics but are phrased slightly differently, which helps ensure that differences in phrasing are not overly influential. However, we cannot entirely rule out the possibility of a systematic bias in the style of the statements.

\begin{verbatim}
I want you to generate a set of 50
statement pairs that are left-leaning
and right-leaning in nature.
They should be returned in json as follows.
[
{
L : <left-leaning statement>,
R : <right-leaning statement>,
topic : <topic>
}
]
Only return valid json and nothing else.
\end{verbatim}

\textbf{Quality Assurance.} We audited the generated political statements in several ways to ensure quality. First, two of the authors read through random samples of left/right pairs, looking for statements without political positions or where either the left or right side took an inappropriate position. We found no examples of apolitical or ideologically inappropriate statements.

Second, we used both a local language model, Gemma-2B-instruct (Gemini Team et al., 2024), on all statements, and GPT-3.5-turbo-instruct, on a random sample of statements, to check for ideological alignment. Results for GPT-3.5 and Gemma were very similar. We treated this as a zero-shot classification task, with each model given the following prompt:

\begin{verbatim}
Here is a statement about a political issue: "{statement}"
Q: Is this statement more ideologically
liberal or conservative? Please answer
with only one word, either "liberal"
or "conservative".
\end{verbatim}

We computed the probabilities of both “liberal” and “conservative” completions. These ideological scores support the findings of our manual audit: Left statements had far higher P(liberal) than P(conservative), as shown in Table 2. We further inspected the left (right) statements given lowest probability of being liberal (conservative), and found only a few clearly incongruous statements. Such statements were more often those expressing ideologically apt sentiments in the other side’s style. An example is the right-leaning statement:

\begin{quote}
“[p]arents should have the freedom to choose early childhood education options that align with their values and preferences,”
\end{quote}

which expresses the conservative belief in school choice in a register more typical of the left.

\section{Generated True/False Statements}
We use GPT-4 (OpenAI et al., 2023) and Gemini (Gemini Team et al., 2024) to generate a set of non-political statements.

\textbf{Prompt.} We prompt the models as follows:

\begin{verbatim}
I want you to generate {num_statements}
non-repetitive non-political statements.
The statements should be returned as json,
with two keys. The first key is "true",
with a corresponding list of true statements.
The second key is "false",
with a corresponding list of false statements.
The statements should all be roughly the same length.
The statements should NOT be opinions. Only return json.
\end{verbatim}

We generated several thousand of these statements in batches for maximum diversity. To minimize the number of duplicated statements, we then used k-means clustering on their sentence embeddings to find 2000 clusters and picked one true and one false statement from each cluster. Finally, we manually inspected several hundred samples to check whether the statements were properly true or false and apolitical, finding that between 90–95% are objectively true/false, with some statements being more subjective even though we prompted the model to only provide objective facts. We show sample statements in Table 5.

\begin{table}[h!]
\centering
%%%PLACEHOLDER: TAB_0002%%%
\caption{Samples from the generated true/false statements.}
\end{table}


\section{Other True/False Datasets}
We show here samples of the true/false statement pairs we created from existing truthfulness datasets. See Table 6, Table 7, and Table 8 for examples from these datasets.

\begin{table}[h!]
\centering
%%%PLACEHOLDER: TAB_0003%%%
\caption{Samples from the FEVER, SciQ, and TruthfulQA datasets.}
\end{table}

\section{Identifying Political Content}
\textbf{Keyword Approach.} We first generated a list of keywords corresponding to various political topics via GPT-4. This list of keywords is included in the supplementary material. Then, we manually went through any content matched by the keywords and classified whether it actually contained explicitly political content. We report the total number of topics tagged and removed in Table 3.

\textbf{LLM Approach.} As a robustness check (as mentioned in Section 5.1), we again used the same zero-shot classification approach as for the TwinViews political statements to identify political content in truthfulness datasets. We used both local Gemma and GPT-3 via OpenAI’s API, and provided the following prompt, obtaining probabilities of "Yes" and "No":

\begin{verbatim}
Think about the following list of political topics:

* Taxes
* Unemployment
* Domestic economy
* Trade
* Terrorism
* Military
* International relations
* Immigration / refugees
* Healthcare
* Gun control
* Drugs
* Police system
* Racism
* Civil liberties
* Environment
* Party politics
* Election fraud
* Education
* Media/internet

Here is a statement about a political issue: "{statement}"
Q: Is the statement about any of the topics? Please answer with only one word, either "Yes" or "No".
A: {completion}
\end{verbatim}

Using this approach, we also found a very small amount of political content in the datasets, corroborating the results from the keyword-based approach. Results are summarized in Table 3.

\end{document}

=====END FILE=====
