=====FILE: main.tex=====
% Source: 
\documentclass[11pt]{article}

\usepackage[margin=1in]{geometry}
\usepackage[T1]{fontenc}
\usepackage[utf8]{inputenc}
\usepackage{lmodern}
\usepackage{microtype}
\usepackage{url}
\usepackage{enumitem}

\title{Academics Can Contribute to Domain-Specialized Language Models}

\author{
Mark Dredze$^{1,2}$ \
\and
Genta Indra Winata$^{3}$\thanks{The project was completed during work at Bloomberg.} \
\and
Prabhanjan Kambadur$^{1}$ \
\and
Shijie Wu$^{4}$\footnotemark[1] \
\and
Ozan Irsoy$^{1}$ \
\and
Steven Lu$^{1}$ \
\and
Vadim Dabravolski$^{1}$ \
\and
David S Rosenberg$^{1}$ \
\and
Sebastian Gehrmann$^{1}$ \
\
$^{1}$Bloomberg \quad
$^{2}$Johns Hopkins University \quad
$^{3}$Capital One \quad
$^{4}$Anthropic \
\texttt{[mdredze@bloomberg.net](mailto:mdredze@bloomberg.net)}
}

\date{}

\begin{document}
\maketitle

\begin{abstract}
Commercially available models dominate academics leaderboards. While impressive, this has concentrated research on creating and adapting general-purpose models to improve NLP leaderboard standings for large language models. However, leaderboards collect many individual tasks and general-purpose models often underperform in specialized domains; domain-specific or adapted models yield superior results. This focus on large general-purpose models excludes many academics and draws attention away from areas where they can make important contributions. We advocate for a renewed focus on developing and evaluating domain- and task-specific models, and highlight the unique role of academics in this endeavor.
\end{abstract}

\section{Introduction}
Natural language processing (NLP) research has historically produced domain- and task-specific supervised models. The field has shifted course in the past few years, with a singular focus on general-purpose generative large language models (LLMs) that, rather than focusing on a single task or domain, do well across many tasks (Brown et al., 2020; Chowdhery et al., 2022; Workshop et al., 2022; Zhang et al., 2022; Touvron et al., 2023b). By training on massive amounts of data from many sources, these models can do well on extremely broad professional and linguistic examinations (Achiam et al., 2023; Anil et al., 2023), college-level knowledge questions (Hendrycks et al., 2021; Lai et al., 2023), and collections of reasoning tasks (Suzgun et al., 2023).

While the trend to develop a single, general-purpose generative model is a net positive change that has resulted in impressive results, it has also slowed down progress in other areas of NLP. First, we are less focused on problems that cannot be solved with a chat-like interface. Second, the best-performing LLMs are often commercial systems, which are sometimes opaque about training data, system architecture, and training details. Third, frequent model updates hinder reproducibility.

The resources required to train large general language models naturally constrain research to large organizations, and researchers (or academics) outside of these organizations have become dependent on closed commercial systems, or open systems with limited transparency regarding their training data. This is partly reflected in broader AI trends: Zhang et al. (2021) found that roughly 30% of papers at AI conferences (including *CL) have a Fortune 500 tech affiliation. Increased resources contribute to the success of transformer-based LLMs (Vaswani et al., 2017), with available hardware (Hooker, 2021) and benchmarks (Dehghani et al., 2021) both playing a deciding role in what models end up being developed. By optimizing the average score across hundreds of shallow tasks, we are smoothing out any signal that would be gained from deeply engaging with individual tasks. Developing domain-specific models can help identify model and training choices that yield improvements on tasks within those domains.

In this paper, we argue for renewed attention to domain-specific models with rigorous and domain-expert informed evaluations. Because many academics are excluded from LLM development due to resource constraints, attention has been drawn away from research areas where academics can make the greatest contributions: deep dives on specific challenging problems. Thus, we propose several research questions to reorient the research community towards developing domain-specific models and applications, where academics are uniquely suited to lead.

\section{LLMs: A Brief History}
While modern LMs date back to Jelinek (1976), we summarize very recent history to describe the current environment. In the wake of the popularization of neural word embeddings by word2vec (Mikolov et al., 2013), contextualized representations of language as features for supervised systems were realized by ELMo (Peters et al., 2018) followed by BERT (Devlin et al., 2019; Liu et al., 2019). BERT and subsequent models became the base models for supervised systems utilizing task-specific fine-tuning and continued pre-training for new domains (Gururangan et al., 2020), e.g., for clinical tasks ELMo (Schumacher and Dredze, 2019) and clinicalBERT (Huang et al., 2019).

Parallel work utilized transformers for autoregressive LLMs, resulting in GPT (Radford et al., 2018), GPT-2 (Radford et al., 2019), BART (Lewis et al., 2020a; Liu et al., 2020), CTRL (Keskar et al., 2019), T5 (Raffel et al., 2020; Xue et al., 2021), and XGLM (Lin et al., 2021). These models had some few-shot capabilities, but they could each be adapted (fine-tuned) for a specific task of interest. Some models were available to academics, though training a new model was beyond reach for many.

GPT-3 (Brown et al., 2020) greatly increased model size and changed our understanding of LLMs. Impressive in-context (few-shot) learning pushed the idea that a single large model could solve a wide range of tasks. While the cost of resources meant training was restricted to a few groups, work focused on training bigger models (Chowdhery et al., 2022; Anil et al., 2023; Zhang et al., 2022; Touvron et al., 2023a; Rae et al., 2021). While only a few could train large models, many studied how best to use them: prompt engineering (Liu et al., 2023), prompt tuning (Han et al., 2022; Wei et al., 2022), evaluation (Liang et al., 2022), among many other topics. Commercial LLM APIs, and eventually open source models (Zhang et al., 2022; Workshop et al., 2022; Touvron et al., 2023a,b; Groeneveld et al., 2024), facilitated this work. Ignat et al. (2024) noted the massive research shift to LLMs reflected in Google Scholar citations. Subsequent work in instruction tuning (Ouyang et al., 2022) and fine-tuning (Wei et al., 2022; Chung et al., 2022; Longpre et al., 2023) have further centralized research around general-purpose models. Many consider fine-tuning for specific applications to be obsolete: why would you tune a model for a specific task when you can tune a single model to do well on all tasks?\footnote{Distillation for task-specific models remains popular if smaller models are desired (Hsieh et al., 2023).}

Despite this view, multiple domain-specific LLMs have demonstrated that domain-specific data leads to models that outperform much larger models (Wu et al., 2023; Taylor et al., 2022). MedPaLM has shown that adapting even giant LLMs to a specific domain leads to vastly increased performance (Singhal et al., 2022, 2023).\footnote{We acknowledge that the biomedical domain is a rapidly developing area, and GPT-4 without fine-tuning was reported to surpass MedPaLM 2 (Nori et al., 2023).} Furthermore, the release of LLaMA (Touvron et al., 2023a) led quickly to Alpaca (Taori et al., 2023) and a wave of new fine-tuned versions of LLaMA for specific tasks. This trend strongly indicates that domain-specific models, especially for constrained sizes, are still highly relevant.

To be clear, our concern is not with closed models, which play an important role in the model ecosystem. Models range from full to limited to no access, with some closed models providing incredibly detailed information (Hoffmann et al., 2022; Rae et al., 2019; Wu et al., 2023) and others providing none (Achiam et al., 2023). Our lament over this focus on general models, either open or closed, is that it draws attention away from work on task- and domain-specific models and evaluations. Academics have become product testers, instead of focusing on tasks where they can play a unique role. Moreover, existing academic benchmarks increasingly serve a reduced purpose for commercial models; we are hill-climbing on benchmarks without a way to ensure existing LLMs have not been trained to excel on these benchmarks (Dodge et al., 2021). Furthermore, we rely on benchmarks in place of deep engagement with an application and its stakeholders.

\section{The Need for Domain-Specific LLMs}
In general, web data does not reflect the needs of all NLP systems. Historically, the community has developed systems for specialized domains such as finance, law, bio-medicine, and science. Accordingly, there have been efforts to build LLMs for these domains (Wu et al., 2023; Taylor et al., 2022; Singhal et al., 2022; Bolton et al., 2023; Luo et al., 2022; Lehman et al., 2023; García-Ferrero et al., 2024). We need a deep investment in how best to develop and evaluate these models in partnership with domain experts. How should we best integrate insights gained from the development of general-purpose models with these efforts? We propose several research directions.

\paragraph{How can general-purpose models inform domain-specific models?}
Building domain-specific models should benefit from insights and investments into general-purpose models. There are several strategies: training domain-specific models from scratch (Taylor et al., 2022; Bolton et al., 2023), mixing general and domain-specific data (Wu et al., 2023), and fine-tuning existing models (Singhal et al., 2022, 2023). Focusing on domain-specific needs, applications, and knowledge with guidance from topic experts will benefit us in acquiring a better model for specific NLP tasks. Which approach yields the best results for task performance and overall cost?

\paragraph{What is the role of in-context learning and fine-tuning?}
Both LIMA (Zhou et al., 2023) and MedPaLM (Singhal et al., 2022) use a small number of examples to tune a model. With expanding context size, we may soon rely entirely on in-context learning (Petroni et al., 2020). This blurs the lines between changing model parameters and conditioning during inference. Beyond inference speed tradeoffs between the two, there may be value in tuning on tens of thousands (or more) of examples. Which domain-specific examples are the most effective to include and in what manner?

\paragraph{How can LLMs be integrated with domain-specific knowledge?}
Specialized knowledge is key in many domains. RAG (Lewis et al., 2020b; Guu et al., 2020) and KILT-derived works (Petroni et al., 2021) focus on knowledge-intensive tasks by including retrieval steps. Work on attributed QA (Bohnet et al., 2022) takes a similar approach, as do search LLMs that require interaction with retrieved data (Nakano et al., 2021). Rich updated knowledge sources will always exist beyond the model, especially in environments like medicine, finance, and many academic disciplines.

\section{Evaluation of Domain-Specific Models}
The evaluation of NLP systems is at a crossroads, and the downstream usage of LLMs and evaluation approaches have diverged. Benchmarks assume that their results translate to insights into similar tasks and usefulness for commercial applications. But benchmarks have become increasingly narrow in scope, oftentimes assessing one metric on a single, often flawed, dataset (Mitchell et al., 2019; Kiela et al., 2021; Ethayarajh and Jurafsky, 2020). The primary evaluation approach for LLMs has been to evaluate on a broad set of these narrow benchmarks (Liang et al., 2022, HELM) (Srivastava et al., 2022, BIG-Bench). High average performance argues for a broad range of capabilities; however, one size may not fit all. Since specific uses of LLMs are typically much more narrow, we identify three major issues and associated research opportunities with this approach.

\paragraph{Depth-first Evaluation}
Current approaches focus on a single model doing everything well on average instead of being useful in a single domain. However, it is widely acknowledged that the standard benchmarks for most tasks are insufficient (e.g., for summarization, Fabbri et al., 2021; Goyal et al., 2022). Task-specific evaluations have thus adopted additional protocols that measure how well models transfer to different domains, how robust they are, and whether they stand up to concept drift (Mille et al., 2021; Dhole et al., 2021). These details disappear when benchmarking on 100+ tasks. Yet, a model’s usefulness is not solely defined by doing okay on everything but rather by how well it performs in specific and narrow tasks that provide value. This value is only realized if the model does not suffer from catastrophic failures.

Exemplar studies that perform deep dives on LLMs for specific tasks exist in healthcare (Zack et al., 2024; Eriksen et al., 2023; Ayers et al., 2023; Han et al., 2024; Chen et al., 2024; Strong et al., 2023), law (Blair-Stanek et al., 2023b,a; Magesh et al., 2024), and physics (Kim et al., 2024), among other areas. We encourage more work on evaluation practices for specific tasks that can handle various model setups and yield informative insights (Zhang et al., 2023; Liang et al., 2022).

\paragraph{Sound Metrics}
For convenience, most benchmark tasks are formulated as multiple choice question answering or classification. This is not how LLMs are often used. For much more common generation tasks, researchers have been ringing alarms about broken evaluations (Gehrmann et al., 2023). It is dubious whether we gain insights into non-task-specific generation through NLU benchmarks. If we are performing the depth-first evaluation of a generation task, a remaining hurdle – and why researchers fall back to NLU tasks – is the lack of robust metrics. While there is much recent work on better metrics (Celikyilmaz et al., 2020; Gehrmann et al., 2023), a troubling trend is the use of LLMs as evaluators (e.g., Sellam et al., 2020; Chiang et al., 2023). This approach poses many risks, including the implicit assumption that the evaluating model has access to the ground truth judgment. While there are some promising results, using an LLM out of the box should be avoided (e.g., Wang et al., 2023a,b). Moreover, it is unclear how to evaluate the evaluator when it is a non-deterministic API, or how to scale the development of learned metrics and quantify the strength of a metric.

\paragraph{Products are not Baselines}
If we really do want to evaluate 100+ tasks, there are many issues with the soundness of evaluation setups. At this scope, it is impossible to run careful ablation studies or to assess the effect of changes to methodology in a causal manner. Moreover, different LLMs respond differently to prompts. The BLOOM evaluation averaged over multiple prompts and found significant variance (Workshop et al., 2022). This variance leads to a lack of reproducibility: LLaMA (Touvron et al., 2023a) claimed high MMLU (Hendrycks et al., 2021) performance but didn’t release the prompts that led to them.\footnote{There was significant confusion surrounding model evaluation: \url{[https://huggingface.co/blog/open-llm-leaderboard-mmlu}}](https://huggingface.co/blog/open-llm-leaderboard-mmlu}}) Similarly, the evaluation scheme makes a difference (Liang et al., 2022, Fig. 33). High evaluation costs mean benchmarks pick a small number of setups (sometimes only one) for each task, which introduces further bias, making it hard to construct fair benchmarks on many tasks.

An additional issue with the current benchmarking approach is that the best-performing models are often commercial APIs. With limited transparency regarding data and training, we cannot fairly evaluate these models (e.g., data leakage). Furthermore, task-specific tuning may have been selected based on these specific benchmarks. Moreover, the underlying models change frequently, so it is unclear whether a result will hold for long.

These evaluation issues prompt significant open questions: 1) How do we develop consistent evaluation setups across models that give true measures of performance? 2) How do we develop evaluation setups and metrics more closely aligned with downstream usage? 3) How do we develop evaluation suites that support depth-first evaluation and not breadth-first benchmarking?

\section{The Role of Academics}
A focus on general-purpose LLMs has forced academics to work with large base models and perhaps, shifted the focus to solve problems of immediate industrial interest. Many academics feel excluded from current research trends (Ignat et al., 2024) and the academic and industry relationship is changing (Littman et al., 2022). Shifting attention back to domain-specific applications emphasizes areas where academics hold an advantage: partnerships with domain experts to invest in specific tasks, and consideration of broader societal needs.

Developing domain-specific models requires domain expertise and universities are diverse academic environments that house experts in many domains. Collaborations with these experts can identify data sources, tasks, and challenges important within each domain. Furthermore, these collaborations are the best avenues for better alignment of evaluations with use cases (Winata et al., 2024), and can support the development of proper metrics. These collaborations are necessary to explore wide open interdisciplinary topics, such as models for protein structure prediction (Tunyasuvunakool et al., 2021; Vig et al., 2021) and games as proxies for reasoning (Silver et al., 2016; Agostinelli et al., 2019; Schrittwieser et al., 2020). This includes developing domain-specific resources, which require domain experts to properly design and construct the datasets. Further, areas where industry underinvests are those where academics could focus attention. For example, low-resource languages are not served by a general-purpose multilingual LLM, nor will we reasonably have enough data to support current LLM training methods. Dialects and variations in languages are still wide open topics (Aji et al., 2022; Winata et al., 2023; Nicholas and Bhatia, 2023).

General-purpose LLMs are unlikely to solve problems in many important domains, with many open research problems that can only be solved by domain-specific approaches. Focusing on domain-specific knowledge will benefit us in acquiring a better model and developing application strategies more aligned with how humans learn domain-specific knowledge (Tricot and Sweller, 2014). For many interdisciplinary areas, subject matter experts are essential, and the problems must be defined clearly. The first pass from an LLM is often impressive, but it hides the trenches and areas where things are most interesting. We need a renewed focus on developing and evaluating domain-specific models and applications, an area where academics can play a leading role. Let us not be distracted by claims that a single model solves all tasks, and instead deeply explore and understand the needs and challenges of specific domains.

\section*{Limitations}
The literature that we explored in this opinion paper is limited to the area of LLMs. We study the history of LLMs from the literature on word embeddings, encoder-only, and generative transformers to the latest advancement of API-based LLMs.

\section*{Ethics Statement}
Our work does not include any experiments or use of data. No potential ethical issues in this work.

\section*{References}
\begin{thebibliography}{99}\setlength{\itemsep}{0pt}

\bibitem{Achiam2023}
Josh Achiam, Steven Adler, Sandhini Agarwal, Lama Ahmad, Ilge Akkaya, Florencia Leoni Aleman, Diogo Almeida, Janko Altenschmidt, Sam Altman, Shyamal Anadkat, et al.
\newblock 2023.
\newblock Gpt-4 technical report.
\newblock \emph{arXiv preprint arXiv:2303.08774}.

\bibitem{Agostinelli2019}
Forest Agostinelli, Stephen McAleer, Alexander Shmakov, and Pierre Baldi.
\newblock 2019.
\newblock Solving the Rubik’s Cube with deep reinforcement learning and search.
\newblock \emph{Nature Machine Intelligence}, 1(8):356--363.

\bibitem{Aji2022}
Alham Aji, Genta Indra Winata, Fajri Koto, Samuel Cahyawijaya, Ade Romadhony, Rahmad Mahendra, Kemal Kurniawan, David Moeljadi, Radityo Eko Prasojo, Timothy Baldwin, et al.
\newblock 2022.
\newblock One country, 700+ languages: NLP challenges for underrepresented languages and dialects in Indonesia.
\newblock In \emph{Proceedings of the 60th Annual Meeting of the Association for Computational Linguistics (Volume 1: Long Papers)}, pages 7226--7249.

\bibitem{Anil2023}
Rohan Anil, Andrew M Dai, Orhan Firat, Melvin Johnson, Dmitry Lepikhin, Alexandre Passos, Siamak Shakeri, Emanuel Taropa, Paige Bailey, Zhifeng Chen, et al.
\newblock 2023.
\newblock Palm 2 technical report.
\newblock \emph{arXiv preprint arXiv:2305.10403}.

\bibitem{Ayers2023}
John W Ayers, Adam Poliak, Mark Dredze, Eric C Leas, Zechariah Zhu, Jessica B Kelley, Dennis J Faix, Aaron M Goodman, Christopher A Longhurst, Michael Hogarth, et al.
\newblock 2023.
\newblock Comparing physician and artificial intelligence chatbot responses to patient questions posted to a public social media forum.
\newblock \emph{JAMA Internal Medicine}, 183(6):589--596.

\bibitem{BlairStanek2023a}
Andrew Blair-Stanek, Nils Holzenberger, and Benjamin Van Durme.
\newblock 2023a.
\newblock Can gpt-3 perform statutory reasoning?
\newblock In \emph{Proceedings of the Nineteenth International Conference on Artificial Intelligence and Law}, pages 22--31.

\bibitem{BlairStanek2023b}
Andrew Blair-Stanek, Nils Holzenberger, and Benjamin Van Durme.
\newblock 2023b.
\newblock Openai cribbed our tax example, but can gpt-4 really do tax?
\newblock \emph{arXiv preprint arXiv:2309.09992}.

\bibitem{Bohnet2022}
Bernd Bohnet, Vinh Q. Tran, Pat Verga, Roee Aharoni, Daniel Andor, Livio Baldini Soares, Jacob Eisenstein, Kuzman Ganchev, Jonathan Herzig, Kai Hui, Tom Kwiatkowski, Ji Ma, Jianmo Ni, Tal Schuster, William W. Cohen, Michael Collins, Dipanjan Das, Donald Metzler, Slav Petrov, and Kellie Webster.
\newblock 2022.
\newblock Attributed question answering: Evaluation and modeling for attributed large language models.
\newblock \emph{CoRR}, abs/2212.08037.

\bibitem{Bolton2023}
Elliot Bolton, David Hall, Michihiro Yasunaga, Tony Lee, Chris Manning, and Percy Liang.
\newblock 2023.
\newblock BioMedLM.
\newblock \url{[https://github.com/stanford-crfm/BioMedLM}](https://github.com/stanford-crfm/BioMedLM}).

\bibitem{Brown2020}
Tom Brown, Benjamin Mann, Nick Ryder, Melanie Subbiah, Jared D Kaplan, Prafulla Dhariwal, Arvind Neelakantan, Pranav Shyam, Girish Sastry, Amanda Askell, et al.
\newblock 2020.
\newblock Language models are few-shot learners.
\newblock \emph{Advances in Neural Information Processing Systems}, 33:1877--1901.

\bibitem{Celikyilmaz2020}
Asli Celikyilmaz, Elizabeth Clark, and Jianfeng Gao.
\newblock 2020.
\newblock Evaluation of text generation: A survey.
\newblock \emph{arXiv}, 2006.14799.

\bibitem{Chen2024}
Hanjie Chen, Zhouxiang Fang, Yash Singla, and Mark Dredze.
\newblock 2024.
\newblock Benchmarking large language models on answering and explaining challenging medical questions.
\newblock \emph{arXiv preprint arXiv:2402.18060}.

\bibitem{Chiang2023}
Wei-Lin Chiang, Zhuohan Li, Zi Lin, Ying Sheng, Zhanghao Wu, Hao Zhang, Lianmin Zheng, Siyuan Zhuang, Yonghao Zhuang, Joseph E Gonzalez, et al.
\newblock 2023.
\newblock Vicuna: An open-source chatbot impressing gpt-4 with 90%* chatgpt quality.
\newblock See \url{[https://vicuna.lmsys.org}](https://vicuna.lmsys.org}) (accessed 14 April 2023), 2(3):6.

\bibitem{Chowdhery2022}
Aakanksha Chowdhery, Sharan Narang, Jacob Devlin, Maarten Bosma, Gaurav Mishra, Adam Roberts, Paul Barham, Hyung Won Chung, Charles Sutton, Sebastian Gehrmann, Parker Schuh, Kensen Shi, Sasha Tsvyashchenko, Joshua Maynez, Abhishek Rao, Parker Barnes, Yi Tay, Noam Shazeer, Vinodkumar Prabhakaran, Emily Reif, Nan Du, Ben Hutchinson, Reiner Pope, James Bradbury, Jacob Austin, Michael Isard, Guy Gur-Ari, Pengcheng Yin, Toju Duke, Anselm Levskaya, Sanjay Ghemawat, Sunipa Dev, Henryk Michalewski, Xavier Garcia, Vedant Misra, Kevin Robinson, Liam Fedus, Denny Zhou, Daphne Ippolito, David Luan, Hyeontaek Lim, Barret Zoph, Alexander Spiridonov, Ryan Sepassi, David Dohan, Shivani Agrawal, Mark Omernick, Andrew M. Dai, Thanumalayan Sankaranarayana Pillai, Marie Pellat, Aitor Lewkowycz, Erica Moreira, Rewon Child, Oleksandr Polozov, Katherine Lee, Zongwei Zhou, Xuezhi Wang, Brennan Saeta, Mark Diaz, Orhan Firat, Michele Catasta, Jason Wei, Kathy Meier-Hellstern, Douglas Eck, Jeff Dean, Slav Petrov, and Noah Fiedel.
\newblock 2022.
\newblock Palm: Scaling language modeling with pathways.
\newblock \emph{CoRR}, abs/2204.02311.

\bibitem{Chung2022}
Hyung Won Chung, Le Hou, Shayne Longpre, Barret Zoph, Yi Tay, William Fedus, Eric Li, Xuezhi Wang, Mostafa Dehghani, Siddhartha Brahma, et al.
\newblock 2022.
\newblock Scaling instruction-finetuned language models.
\newblock \emph{arXiv}, 2210.11416.

\bibitem{Dehghani2021}
Mostafa Dehghani, Yi Tay, Alexey A. Gritsenko, Zhe Zhao, Neil Houlsby, Fernando Diaz, Donald Metzler, and Oriol Vinyals.
\newblock 2021.
\newblock The benchmark lottery.
\newblock \emph{CoRR}, abs/2107.07002.

\bibitem{Devlin2019}
Jacob Devlin, Ming-Wei Chang, Kenton Lee, and Kristina Toutanova.
\newblock 2019.
\newblock BERT: Pre-training of deep bidirectional transformers for language understanding.
\newblock In \emph{Proceedings of the 2019 Conference of the North American Chapter of the Association for Computational Linguistics: Human Language Technologies, Volume 1 (Long and Short Papers)}, pages 4171--4186, Minneapolis, Minnesota. Association for Computational Linguistics.

\bibitem{Dhole2021}
Kaustubh D. Dhole, Varun Gangal, Sebastian Gehrmann, Aadesh Gupta, Zhenhao Li, Saad Mahamood, Abinaya Mahendiran, Simon Mille, Ashish Srivastava, Samson Tan, Tongshuang Wu, Jascha Sohl-Dickstein, Jinho D. Choi, Eduard H. Hovy, Ondrej Dusek, Sebastian Ruder, Sajant Anand, Nagender Aneja, Rabin Banjade, Lisa Barthe, Hanna Behnke, Ian Berlot-Attwell, Connor Boyle, Caroline Brun, Marco Antonio Sobrevilla Cabezudo, Samuel Cahyawijaya, Emile Chapuis, Wanxiang Che, Mukund Choudhary, Christian Clauss, Pierre Colombo, Filip Cornell, Gautier Dagan, Mayukh Das, Tanay Dixit, Thomas Dopierre, Paul-Alexis Dray, Suchitra Dubey, Tatiana Ekeinhor, Marco Di Giovanni, Rishabh Gupta, Rishabh Gupta, Louanes Hamla, Sang Han, Fabrice Harel-Canada, Antoine Honore, Ishan Jindal, Przemyslaw K. Joniak, Denis Kleyko, Venelin Kovatchev, and et al.
\newblock 2021.
\newblock Nlaugmenter: A framework for task-sensitive natural language augmentation.
\newblock \emph{CoRR}, abs/2112.02721.

\bibitem{Dodge2021}
Jesse Dodge, Maarten Sap, Ana Marasovic, William Agnew, Gabriel Ilharco, Dirk Groeneveld, Margaret Mitchell, and Matt Gardner.
\newblock 2021.
\newblock Documenting large webtext corpora: A case study on the colossal clean crawled corpus.
\newblock In \emph{Proceedings of the 2021 Conference on Empirical Methods in Natural Language Processing}, pages 1286--1305.

\bibitem{Eriksen2023}
Alexander V Eriksen, S{"o}ren M{"o}ller, and Jesper Ryg.
\newblock 2023.
\newblock Use of gpt-4 to diagnose complex clinical cases.

\bibitem{Ethayarajh2020}
Kawin Ethayarajh and Dan Jurafsky.
\newblock 2020.
\newblock Utility is in the eye of the user: A critique of NLP leaderboards.
\newblock In \emph{Proceedings of the 2020 Conference on Empirical Methods in Natural Language Processing (EMNLP)}, pages 4846--4853, Online. Association for Computational Linguistics.

\bibitem{Fabbri2021}
Alexander R. Fabbri, Wojciech Kry{'s}ci{'n}ski, Bryan Mc-Cann, Caiming Xiong, Richard Socher, and Dragomir Radev.
\newblock 2021.
\newblock SummEval: Re-evaluating summarization evaluation.
\newblock \emph{Transactions of the Association for Computational Linguistics}, 9:391--409.

\bibitem{GarciaFerrero2024}
Iker Garc{'i}a-Ferrero, Rodrigo Agerri, Aitziber Atutxa Salazar, Elena Cabrio, Iker de la Iglesia, Alberto Lavelli, Bernardo Magnini, Benjamin Molinet, Johana Ramirez-Romero, German Rigau, et al.
\newblock 2024.
\newblock Medical mT5: an open-source multilingual text-to-text LLM for the medical domain.
\newblock \emph{arXiv preprint arXiv:2404.07613}.

\bibitem{Gehrmann2023}
Sebastian Gehrmann, Elizabeth Clark, and Thibault Sellam.
\newblock 2023.
\newblock Repairing the cracked foundation: A survey of obstacles in evaluation practices for generated text.
\newblock \emph{Journal of Artificial Intelligence Research}, 77:103--166.

\bibitem{Goyal2022}
Tanya Goyal, Junyi Jessy Li, and Greg Durrett.
\newblock 2022.
\newblock News summarization and evaluation in the era of GPT-3.
\newblock \emph{CoRR}, abs/2209.12356.

\bibitem{Groeneveld2024}
Dirk Groeneveld, Iz Beltagy, Pete Walsh, Akshita Bhagia, Rodney Kinney, Oyvind Tafjord, Ananya Harsh Jha, Hamish Ivison, Ian Magnusson, Yizhong Wang, et al.
\newblock 2024.
\newblock Olmo: Accelerating the science of language models.
\newblock \emph{arXiv preprint arXiv:2402.00838}.

\bibitem{Gururangan2020}
Suchin Gururangan, Ana Marasovic, Swabha Swayamdipta, Kyle Lo, Iz Beltagy, Doug Downey, and Noah A Smith.
\newblock 2020.
\newblock Don’t stop pretraining: Adapt language models to domains and tasks.
\newblock In \emph{Proceedings of the 58th Annual Meeting of the Association for Computational Linguistics}, pages 8342--8360.

\bibitem{Guu2020}
Kelvin Guu, Kenton Lee, Zora Tung, Panupong Pasupat, and Ming-Wei Chang.
\newblock 2020.
\newblock Retrieval augmented language model pre-training.
\newblock In \emph{Proceedings of the 37th International Conference on Machine Learning, ICML 2020, 13-18 July 2020, Virtual Event}, volume 119 of \emph{Proceedings of Machine Learning Research}, pages 3929--3938. PMLR.

\bibitem{Han2024}
Tessa Han, Aounon Kumar, Chirag Agarwal, and Himabindu Lakkaraju.
\newblock 2024.
\newblock Towards safe large language models for medicine.
\newblock In \emph{ICML 2024 Workshop on Models of Human Feedback for AI Alignment}.

\bibitem{Han2022}
Xu Han, Weilin Zhao, Ning Ding, Zhiyuan Liu, and Maosong Sun.
\newblock 2022.
\newblock PTR: Prompt tuning with rules for text classification.
\newblock \emph{AI Open}, 3:182--192.

\bibitem{Hendrycks2021}
Dan Hendrycks, Collin Burns, Steven Basart, Andy Zou, Mantas Mazeika, Dawn Song, and Jacob Steinhardt.
\newblock 2021.
\newblock Measuring massive multitask language understanding.
\newblock In \emph{International Conference on Learning Representations}.

\bibitem{Hoffmann2022}
Jordan Hoffmann, Sebastian Borgeaud, Arthur Mensch, Elena Buchatskaya, Trevor Cai, Eliza Rutherford, Diego de Las Casas, Lisa Anne Hendricks, Johannes Welbl, Aidan Clark, et al.
\newblock 2022.
\newblock An empirical analysis of compute-optimal large language model training.
\newblock \emph{Advances in Neural Information Processing Systems}, 35:30016--30030.

\bibitem{Hooker2021}
Sara Hooker.
\newblock 2021.
\newblock The hardware lottery.
\newblock \emph{Commun. ACM}, 64(12):58--65.

\bibitem{Hsieh2023}
Cheng-Yu Hsieh, Chun-Liang Li, Chih-kuan Yeh, Hootan Nakhost, Yasuhisa Fujii, Alex Ratner, Ranjay Krishna, Chen-Yu Lee, and Tomas Pfister.
\newblock 2023.
\newblock Distilling step-by-step! outperforming larger language models with less training data and smaller model sizes.
\newblock In \emph{Findings of the Association for Computational Linguistics: ACL 2023}, pages 8003--8017.

\bibitem{Huang2019}
Kexin Huang, Jaan Altosaar, and Rajesh Ranganath.
\newblock 2019.
\newblock Clinicalbert: Modeling clinical notes and predicting hospital readmission.
\newblock \emph{arXiv preprint arXiv:1904.05342}.

\bibitem{Ignat2024}
Oana Ignat, Zhijing Jin, Artem Abzaliev, Laura Biester, Santiago Castro, Naihao Deng, Xinyi Gao, Aylin Ece Gunal, Jacky He, Ashkan Kazemi, et al.
\newblock 2024.
\newblock Has it all been solved? open nlp research questions not solved by large language models.
\newblock In \emph{Proceedings of the 2024 Joint International Conference on Computational Linguistics, Language Resources and Evaluation (LREC-COLING 2024)}, pages 8050--8094.

\bibitem{Jelinek1976}
Frederick Jelinek.
\newblock 1976.
\newblock Continuous speech recognition by statistical methods.
\newblock \emph{Proceedings of the IEEE}, 64(4):532--556.

\bibitem{Keskar2019}
Nitish Shirish Keskar, Bryan McCann, Lav R Varshney, Caiming Xiong, and Richard Socher.
\newblock 2019.
\newblock Ctrl: A conditional transformer language model for controllable generation.
\newblock \emph{arXiv preprint arXiv:1909.05858}.

\bibitem{Kiela2021}
Douwe Kiela, Max Bartolo, Yixin Nie, Divyansh Kaushik, Atticus Geiger, Zhengxuan Wu, Bertie Vidgen, Grusha Prasad, Amanpreet Singh, Pratik Ringshia, Zhiyi Ma, Tristan Thrush, Sebastian Riedel, Zeerak Waseem, Pontus Stenetorp, Robin Jia, Mohit Bansal, Christopher Potts, and Adina Williams.
\newblock 2021.
\newblock Dynabench: Rethinking benchmarking in NLP.
\newblock In \emph{Proceedings of the 2021 Conference of the North American Chapter of the Association for Computational Linguistics: Human Language Technologies}, pages 4110--4124, Online. Association for Computational Linguistics.

\bibitem{Kim2024}
Eun-Ah Kim, Haining Pan, Nayantara Mudur, William Taranto, Subhashini Venugopalan, Yasaman Bahri, and Michael Brenner.
\newblock 2024.
\newblock Performing Hartree-Fock many-body physics calculations with large language models.
\newblock \emph{Bulletin of the American Physical Society}.

\bibitem{Lai2023}
Viet Lai, Chien Nguyen, Nghia Ngo, Thuat Nguyen, Franck Dernoncourt, Ryan Rossi, and Thien Nguyen.
\newblock 2023.
\newblock Okapi: Instruction-tuned large language models in multiple languages with reinforcement learning from human feedback.
\newblock In \emph{Proceedings of the 2023 Conference on Empirical Methods in Natural Language Processing: System Demonstrations}, pages 318--327.

\bibitem{Lehman2023}
Eric Lehman, Evan Hernandez, Diwakar Mahajan, Jonas Wulff, Micah J Smith, Zachary Ziegler, Daniel Nadler, Peter Szolovits, Alistair Johnson, and Emily Alsentzer.
\newblock 2023.
\newblock Do we still need clinical language models?
\newblock In \emph{Conference on health, inference, and learning}, pages 578--597. PMLR.

\bibitem{Lewis2020a}
Mike Lewis, Yinhan Liu, Naman Goyal, Marjan Ghazvininejad, Abdelrahman Mohamed, Omer Levy, Veselin Stoyanov, and Luke Zettlemoyer.
\newblock 2020a.
\newblock BART: Denoising sequence-to-sequence pre-training for natural language generation, translation, and comprehension.
\newblock In \emph{Proceedings of the 58th Annual Meeting of the Association for Computational Linguistics}, pages 7871--7880.

\bibitem{Lewis2020b}
Patrick S. H. Lewis, Ethan Perez, Aleksandra Piktus, Fabio Petroni, Vladimir Karpukhin, Naman Goyal, Heinrich K{"u}ttler, Mike Lewis, Wen-tau Yih, Tim Rockt{"a}schel, Sebastian Riedel, and Douwe Kiela.
\newblock 2020b.
\newblock Retrieval-augmented generation for knowledge-intensive NLP tasks.
\newblock In \emph{Advances in Neural Information Processing Systems}.

\bibitem{Liang2022}
Percy Liang, Rishi Bommasani, Tony Lee, Dimitris Tsipras, Dilara Soylu, Michihiro Yasunaga, Tony Lee, and et al.
\newblock 2022.
\newblock Holistic evaluation of language models.
\newblock \emph{CoRR}, abs/2211.09110.
\newblock [ILLEGIBLE]

\bibitem{Lin2021}
Xi Victoria Lin, Todor Mihaylov, Mikel Artetxe, Tianlu Wang, Shuohui Chen, Daniel Simig, Myle Ott, Naman Goyal, Shruti Bhosale, Jingfei Du, et al.
\newblock 2021.
\newblock Few-shot learning with multilingual language models.
\newblock \emph{arXiv preprint arXiv:2112.10668}.

\bibitem{Littman2022}
Michael L Littman, Ifeoma Ajunwa, Guy Berger, Craig Boutilier, Morgan Currie, Finale Doshi-Velez, Gillian Hadfield, Michael C Horowitz, Charles Isbell, Hiroaki Kitano, et al.
\newblock 2022.
\newblock Gathering strength, gathering storms: The one hundred year study on artificial intelligence (ai100) 2021 study panel report.
\newblock \emph{arXiv preprint arXiv:2210.15767}.

\bibitem{Liu2023}
Pengfei Liu, Weizhe Yuan, Jinlan Fu, Zhengbao Jiang, Hiroaki Hayashi, and Graham Neubig.
\newblock 2023.
\newblock Pretrain, prompt, and predict: A systematic survey of prompting methods in natural language processing.
\newblock \emph{ACM Computing Surveys}, 55(9):1--35.

\bibitem{Liu2020}
Yinhan Liu, Jiatao Gu, Naman Goyal, Xian Li, Sergey Edunov, Marjan Ghazvininejad, Mike Lewis, and Luke Zettlemoyer.
\newblock 2020.
\newblock Multilingual denoising pretraining for neural machine translation.
\newblock \emph{Transactions of the Association for Computational Linguistics}, 8:726--742.

\bibitem{Liu2019}
Yinhan Liu, Myle Ott, Naman Goyal, Jingfei Du, Mandar Joshi, Danqi Chen, Omer Levy, Mike Lewis, Luke Zettlemoyer, and Veselin Stoyanov.
\newblock 2019.
\newblock Roberta: A robustly optimized BERT pretraining approach.
\newblock \emph{arXiv}.

\bibitem{Longpre2023}
Shayne Longpre, Le Hou, Tu Vu, Albert Webson, Hyung Won Chung, Yi Tay, Denny Zhou, Quoc V Le, Barret Zoph, Jason Wei, et al.
\newblock 2023.
\newblock The FLAN collection: Designing data and methods for effective instruction tuning.
\newblock \emph{arXiv}, 2301.13688.

\bibitem{Luo2022}
Renqian Luo, Liai Sun, Yingce Xia, Tao Qin, Sheng Zhang, Hoifung Poon, and Tie-Yan Liu.
\newblock 2022.
\newblock Biogpt: generative pre-trained transformer for biomedical text generation and mining.
\newblock \emph{Briefings in bioinformatics}, 23(6):bbac409.

\bibitem{Magesh2024}
Varun Magesh, Faiz Surani, Matthew Dahl, Mirac Suzgun, Christopher D Manning, and Daniel E Ho.
\newblock 2024.
\newblock Hallucination-free? assessing the reliability of leading ai legal research tools.
\newblock \emph{arXiv preprint arXiv:2405.20362}.

\bibitem{Mikolov2013}
Tomas Mikolov, Kai Chen, Greg Corrado, and Jeffrey Dean.
\newblock 2013.
\newblock Efficient estimation of word representations in vector space.
\newblock \emph{arXiv preprint arXiv:1301.3781}.

\bibitem{Mille2021}
Simon Mille, Kaustubh D. Dhole, Saad Mahamood, Laura Perez-Beltrachini, Varun Gangal, Mihir Sanjay Kale, Emiel van Miltenburg, and Sebastian Gehrmann.
\newblock 2021.
\newblock Automatic construction of evaluation suites for natural language generation datasets.
\newblock In \emph{Advances in Neural Information Processing Systems}.

\bibitem{Mitchell2019}
Margaret Mitchell, Simone Wu, Andrew Zaldivar, Parker Barnes, Lucy Vasserman, Ben Hutchinson, Elena Spitzer, Inioluwa Deborah Raji, and Timnit Gebru.
\newblock 2019.
\newblock Model cards for model reporting.
\newblock In \emph{Proceedings of the Conference on Fairness, Accountability, and Transparency, FAT* 2019, Atlanta, GA, USA, January 29-31, 2019}, pages 220--229. ACM.

\bibitem{Nakano2021}
Reiichiro Nakano, Jacob Hilton, Suchir Balaji, Jeff Wu, Long Ouyang, Christina Kim, Christopher Hesse, Shantanu Jain, Vineet Kosaraju, William Saunders, et al.
\newblock 2021.
\newblock Webgpt: Browser-assisted question-answering with human feedback.
\newblock \emph{arXiv preprint arXiv:2112.09332}.

\bibitem{NicholasBhatia2023}
Gabriel Nicholas and Aliya Bhatia.
\newblock 2023.
\newblock Lost in translation: Large language models in non-english content analysis.
\newblock \emph{arXiv preprint arXiv:2306.07377}.

\bibitem{Nori2023}
Harsha Nori, Nicholas King, Scott Mayer McKinney, Dean Carignan, and Eric Horvitz.
\newblock 2023.
\newblock Capabilities of gpt-4 on medical challenge problems.
\newblock \emph{arXiv preprint arXiv:2303.13375}.

\bibitem{Ouyang2022}
Long Ouyang, Jeffrey Wu, Xu Jiang, Diogo Almeida, Carroll Wainwright, Pamela Mishkin, Chong Zhang, Sandhini Agarwal, Katarina Slama, Alex Ray, et al.
\newblock 2022.
\newblock Training language models to follow instructions with human feedback.
\newblock \emph{Advances in Neural Information Processing Systems}, 35:27730--27744.

\bibitem{Peters2018}
Matthew E. Peters, Mark Neumann, Mohit Iyyer, Matt Gardner, Christopher Clark, Kenton Lee, and Luke Zettlemoyer.
\newblock 2018.
\newblock Deep contextualized word representations.
\newblock \emph{arXiv}, 1802.05365.

\bibitem{Petroni2020}
Fabio Petroni, Patrick Lewis, Aleksandra Piktus, Tim Rockt{"a}schel, Yuxiang Wu, Alexander H. Miller, and Sebastian Riedel.
\newblock 2020.
\newblock How context affects language models’ factual predictions.
\newblock In \emph{Automated Knowledge Base Construction}.

\bibitem{Petroni2021}
Fabio Petroni, Aleksandra Piktus, Angela Fan, Patrick Lewis, Majid Yazdani, Nicola De Cao, James Thorne, Yacine Jernite, Vladimir Karpukhin, Jean Maillard, Vassilis Plachouras, Tim Rockt{"a}schel, and Sebastian Riedel.
\newblock 2021.
\newblock KILT: a benchmark for knowledge intensive language tasks.
\newblock In \emph{Proceedings of the 2021 Conference of the North American Chapter of the Association for Computational Linguistics: Human Language Technologies}, pages 2523--2544, Online.

\bibitem{Radford2018}
Alec Radford, Karthik Narasimhan, Tim Salimans, Ilya Sutskever, et al.
\newblock 2018.
\newblock Improving language understanding by generative pre-training.

\bibitem{Radford2019}
Alec Radford, Jeffrey Wu, Rewon Child, David Luan, Dario Amodei, Ilya Sutskever, et al.
\newblock 2019.
\newblock Language models are unsupervised multitask learners.
\newblock \emph{OpenAI blog}, 1(8):9.

\bibitem{Rae2021}
Jack W Rae, Sebastian Borgeaud, Trevor Cai, Katie Millican, Jordan Hoffmann, Francis Song, John Aslanides, Sarah Henderson, Roman Ring, Susan-nah Young, et al.
\newblock 2021.
\newblock Scaling language models: Methods, analysis & insights from training gopher.
\newblock \emph{arXiv preprint arXiv:2112.11446}.

\bibitem{Rae2019}
Jack W Rae, Anna Potapenko, Siddhant M Jayakumar, and Timothy P Lillicrap.
\newblock 2019.
\newblock Compressive transformers for long-range sequence modelling.
\newblock \emph{arXiv preprint arXiv:1911.05507}.

\bibitem{Raffel2020}
Colin Raffel, Noam Shazeer, Adam Roberts, Katherine Lee, Sharan Narang, Michael Matena, Yanqi Zhou, Wei Li, and Peter J Liu.
\newblock 2020.
\newblock Exploring the limits of transfer learning with a unified text-to-text transformer.
\newblock \emph{The Journal of Machine Learning Research}, 21(1):5485--5551.

\bibitem{Schrittwieser2020}
Julian Schrittwieser, Ioannis Antonoglou, Thomas Hubert, Karen Simonyan, Laurent Sifre, Simon Schmitt, Arthur Guez, Edward Lockhart, Demis Hassabis, Thore Graepel, et al.
\newblock 2020.
\newblock Mastering Atari, Go, Chess and Shogi by planning with a learned model.
\newblock \emph{Nature}, 588(7839):604--609.

\bibitem{SchumacherDredze2019}
Elliot Schumacher and Mark Dredze.
\newblock 2019.
\newblock Learning unsupervised contextual representations for medical synonym discovery.
\newblock \emph{JAMIA Open}, 2(4):538--546.

\bibitem{Sellam2020}
Thibault Sellam, Dipanjan Das, and Ankur Parikh.
\newblock 2020.
\newblock BLEURT: Learning robust metrics for text generation.
\newblock In \emph{Proceedings of the 58th Annual Meeting of the Association for Computational Linguistics}, pages 7881--7892.

\bibitem{Silver2016}
David Silver, Aja Huang, Chris J Maddison, Arthur Guez, Laurent Sifre, George Van Den Driessche, Julian Schrittwieser, Ioannis Antonoglou, Veda Panneershelvam, Marc Lanctot, et al.
\newblock 2016.
\newblock Mastering the game of Go with deep neural networks and tree search.
\newblock \emph{Nature}, 529(7587):484--489.

\bibitem{Singhal2022}
Karan Singhal, Shekoofeh Azizi, Tao Tu, S. Sara Mahdavi, Jason Wei, Hyung Won Chung, Nathan Scales, Ajay Kumar Tanwani, Heather Cole-Lewis, Stephen Pfohl, Perry Payne, Martin Seneviratne, Paul Gamble, Chris Kelly, Nathaneal Sch{"a}rli, Aakanksha Chowdhery, Philip Andrew Mansfield, Blaise Ag{"u}era y Arcas, Dale R. Webster, Gregory S. Corrado, Yossi Matias, Katherine Chou, Juraj Gottweis, Nenad Tomasev, Yun Liu, Alvin Rajkomar, Joelle K. Barral, Christopher Semturs, Alan Karthikesalingam, and Vivek Natarajan.
\newblock 2022.
\newblock Large language models encode clinical knowledge.
\newblock \emph{CoRR}, abs/2212.13138.

\bibitem{Singhal2023}
Karan Singhal, Tao Tu, Juraj Gottweis, Rory Sayres, Ellery Wulczyn, Le Hou, Kevin Clark, Stephen Pfohl, Heather Cole-Lewis, Darlene Neal, Mike Schaekermann, Amy Wang, Mohamed Amin, Sami Lachgar, Philip Andrew Mansfield, Sushant Prakash, Bradley Green, Ewa Dominowska, Blaise Ag{"u}era y Arcas, Nenad Tomasev, Yun Liu, Renee Wong, Christopher Semturs, S. Sara Mahdavi, Joelle K. Barral, Dale R. Webster, Gregory S. Corrado, Yossi Matias, Shekoofeh Azizi, Alan Karthikesalingam, and Vivek Natarajan.
\newblock 2023.
\newblock Towards expert-level medical question answering with large language models.
\newblock \emph{CoRR}, abs/2305.09617.

\bibitem{Srivastava2022}
Aarohi Srivastava, Abhinav Rastogi, Abhishek Rao, Abu Awal Md Shoeb, Abubakar Abid, Adam Fisch, Adam R. Brown, Adam Santoro, Aditya Gupta, Adri{`a} Garriga-Alonso, Agnieszka Kluska, Aitor Lewkowycz, Akshat Agarwal, Alethea Power, Alex Ray, Alex Warstadt, Alexander W. Kocurek, Ali Safaya, Ali Tazarv, Alice Xiang, Alicia Parrish, Allen Nie, Aman Hussain, Amanda Askell, Amanda Dsouza, Ameet Annasaheb Rahane, Anantharaman S. Iyer, Anders Andreassen, Andrea Santilli, Andreas Stuhlmuller, Andrew M. Dai, Andrew D. La, Andrew Kyle Lampinen, Andy Zou, Angela Jiang, Angelica Chen, Anh Vuong, Animesh Gupta, Anna Gottardi, Antonio Norelli, Anu Venkatesh, Arash Gholamidavoodi, Arfa Tabassum, Arul Menezes, Arun Kirubarajan, Asher Mullokandov, Ashish Sabharwal, Austin Herrick, Avia Efrat, Aykut Erdem, Ayla Karakacs, Bridget R. Roberts, Bao Sheng Loe, Barret Zoph, Bartlomiej Bojanowski, Batuhan Ozyurt, Behnam Hedayatnia, Behnam Neyshabur, Benjamin Inden, Benno Stein, Berk Ekmekci, Bill Yuchen Lin, Blake Stephen Howald, Cameron Diao, Cameron Dour, Catherine Stinson, Cedrick Argueta, C’esar Ferri Ram’irez, Chandan Singh, Charles Rathkopf, Chenlin Meng, Chitta Baral, Chiyu Wu, Chris Callison-Burch, Chris Waites, Christian Voigt, Christopher D. Manning, Christopher Potts, Cindy Tatiana Ramirez, Clara Rivera, Clemencia Siro, Colin Raffel, Courtney Ashcraft, Cristina Garbacea, Damien Sileo, Daniel H Garrette, Dan Hendrycks, Dan Kilman, Dan Roth, Daniel Freeman, Daniel Khashabi, Daniel Levy, Daniel Gonz’alez, Danny Hernandez, Danqi Chen, Daphne Ippolito, Dar Gilboa, David Dohan, D. Drakard, David Jurgens, Debajyoti Datta, Deep Ganguli, Denis Emelin, Denis Kleyko, Deniz Yuret, Derek Chen, Derek Tam, Dieuwke Hupkes, Diganta Misra, Dilyar Buzan, Dimitri Coelho Mollo, Diyi Yang, Dong-Ho Lee, Ekaterina Shutova, Ekin Dogus Cubuk, Elad Segal, Eleanor Hagerman, Elizabeth Barnes, Elizabeth P. Donoway, Ellie Pavlick, Emanuele Rodol{`a}, Emma FC Lam, Eric Chu, Eric Tang, Erkut Erdem, Ernie Chang, Ethan A. Chi, Ethan Dyer, Ethan J. Jerzak, Ethan Kim, Eunice Engefu Manyasi, Evgenii Zheltonozhskii, Fan Xia, Fatemeh Siar, Fernando Mart’inez-Plumed, Francesca Happ’e, François Chollet, Frieda Rong, Gaurav Mishra, Genta Indra Winata, Gerard de Melo, Germ{'a}n Kruszewski, Giambattista Parascandolo, Giorgio Mariani, Gloria Wang, Gonzalo Jaimovitch-L’opez, Gregor Betz, Guy Gur-Ari, Hana Galija-sevic, Han Sol Kim, Hannah Rashkin, Hanna Hajishirzi, Harsh Mehta, Hayden Bogar, Henry Shevlin, Hinrich Sch{"u}tze, Hiromu Yakura, Hongming Zhang, Hubert Wong, Ian Aik-Soon Ng, Isaac Noble, Jaap Jumelet, Jack Geissinger, John Kernion, Jacob Hilton, Jaehoon Lee, Jaime Fern{'a}ndez Fisac, J. Brooker Simon, James Koppel, James Zheng, James Zou, Jan Koco’n, Jana Thompson, Jared Kaplan, Jarema Radom, Jascha Narain Sohl-Dickstein, Jason Phang, Jason Wei, Jason Yosinski, Jekaterina Novikova, Jelle Bosscher, Jenni Marsh, Jeremy Kim, Jeroen Taal, Jesse Engel, Jesujoba Oluwadara Alabi, Jiacheng Xu, Jiaming Song, Jillian Tang, Jane W Waweru, John Burden, John Miller, John U. Balis, Jonathan Berant, Jorg Frohberg, Jos Rozen, José Hernández-Orallo, Joseph Boudeman, Joseph Jones, Joshua B. Tenenbaum, Joshua S. Rule, Joyce Chua, Kamil Kanclerz, Karen Livescu, Karl Krauth, Karthik Gopalakrishnan, Katerina Ignatyeva, Katja Markert, Kaustubh D. Dhole, Kevin Gimpel, Kevin Ochieng’ Omondi, Kory Wallace Mathewson, Kristen Chia-fullo, Ksenia Shkaruta, Kumar Shridhar, Kyle McDonell, Kyle Richardson, Laria Reynolds, Leo Gao, Li Zhang, Liam Dugan, Lianhui Qin, Lidia Contreras-Ochando, Louis-Philippe Morency, Luca Moschella, Luca Lam, Lucy Noble, Ludwig Schmidt, Luheng He, Luis Oliveros Col’on, Luke Metz, Lutfi Kerem Senel, Maarten Bosma, Maarten Sap, Maartje ter Hoeve, Madotto Andrea, Maheen Saleem Farooqi, Manaal Faruqui, Mantas Mazeika, Marco Baturan, Marco Marelli, Marco Maru, M Quintana, Marie Tolkiehn, Mario Giulianelli, Martha Lewis, Martin Potthast, Matthew Leavitt, Matthias Hagen, M’aty’as Schubert, Medina Baitemirova, Melissa Arnaud, Melvin Andrew McElrath, Michael A. Yee, Michael Cohen, Mi Gu, Michael I. Ivanitskiy, Michael Starritt, Michael Strube, Michal Swkedrowski, Michele Bevilacqua, Michihiro Yasunaga, Mihir Kale, Mike Cain, Mimee Xu, Mirac Suzgun, Monica Tiwari, Mohit Bansal, Moin Aminnaseri, Mor Geva, Mozhdeh Gheini, T MukundVarma, Nanyun Peng, Nathan Chi, Nayeon Lee, Neta Gur-Ari Krakover, Nicholas Cameron, Nicholas S. Roberts, Nicholas Doiron, [ILLEGIBLE]
\newblock 2022.
\newblock Beyond the imitation game: Quantifying and extrapolating the capabilities of language models.
\newblock \emph{arXiv}, abs/2206.04615.

\bibitem{Strong2023}
Eric Strong, Alicia DiGiammarino, Yingjie Weng, Andre Kumar, Poonam Hosamani, Jason Hom, and Jonathan H Chen.
\newblock 2023.
\newblock Chatbot vs medical student performance on free-response clinical reasoning examinations.
\newblock \emph{JAMA Internal Medicine}, 183(9):1028--1030.

\bibitem{Suzgun2023}
Mirac Suzgun, Nathan Scales, Nathanael Sch{"a}rli, Sebastian Gehrmann, Yi Tay, Hyung Won Chung, Aakanksha Chowdhery, Quoc Le, Ed Chi, Denny Zhou, et al.
\newblock 2023.
\newblock Challenging big-bench tasks and whether chain-of-thought can solve them.
\newblock In \emph{Findings of the Association for Computational Linguistics: ACL 2023}, pages 13003--13051.

\bibitem{Taori2023}
Rohan Taori, Ishaan Gulrajani, Tianyi Zhang, Yann Dubois, Xuechen Li, Carlos Guestrin, Percy Liang, and Tatsunori B. Hashimoto.
\newblock 2023.
\newblock Stanford alpaca: An instruction-following llama model.
\newblock \url{[https://github.com/tatsu-lab/stanford_alpaca}](https://github.com/tatsu-lab/stanford_alpaca}).

\bibitem{Taylor2022}
Ross Taylor, Marcin Kardas, Guillem Cucurull, Thomas Scialom, Anthony Hartshorn, Elvis Saravia, Andrew Poulton, Viktor Kerkez, and Robert Stojnic.
\newblock 2022.
\newblock Galactica: A large language model for science.
\newblock \emph{CoRR}, abs/2211.09085.

\bibitem{Touvron2023a}
Hugo Touvron, Thibaut Lavril, Gautier Izacard, Xavier Martinet, Marie-Anne Lachaux, Timoth{'e}e Lacroix, Baptiste Rozi{`e}re, Naman Goyal, Eric Hambro, Faisal Azhar, Aur{'e}lien Rodriguez, Armand Joulin, Edouard Grave, and Guillaume Lample.
\newblock 2023a.
\newblock Llama: Open and efficient foundation language models.
\newblock \emph{CoRR}, abs/2302.13971.

\bibitem{Touvron2023b}
Hugo Touvron, Louis Martin, Kevin Stone, Peter Albert, Amjad Almahairi, Yasmine Babaei, Nikolay Bashlykov, Soumya Batra, Prajjwal Bhargava, Shruti Bhosale, et al.
\newblock 2023b.
\newblock Llama 2: Open foundation and fine-tuned chat models.
\newblock \emph{arXiv preprint arXiv:2307.09288}.

\bibitem{TricotSweller2014}
Andr{'e} Tricot and John Sweller.
\newblock 2014.
\newblock Domain-specific knowledge and why teaching generic skills does not work.
\newblock \emph{Educational Psychology Review}, 26:265--283.

\bibitem{Tunyasuvunakool2021}
Kathryn Tunyasuvunakool, Jonas Adler, Zachary Wu, Tim Green, Michal Zielinski, Augustin {\v Z}{'\i}dek, Alex Bridgland, Andrew Cowie, Clemens Meyer, Agata Laydon, et al.
\newblock 2021.
\newblock Highly accurate protein structure prediction for the human proteome.
\newblock \emph{Nature}, 596(7873):590--596.

\bibitem{Vaswani2017}
Ashish Vaswani, Noam Shazeer, Niki Parmar, Jakob Uszkoreit, Llion Jones, Aidan N. Gomez, Lukasz Kaiser, and Illia Polosukhin.
\newblock 2017.
\newblock Attention is all you need.
\newblock In \emph{Advances in Neural Information Processing Systems}, pages 5998--6008.

\bibitem{Vig2021}
Jesse Vig, Ali Madani, Lav R. Varshney, Caiming Xiong, richard socher, and Nazneen Rajani.
\newblock 2021.
\newblock BERTology meets biology: Interpreting attention in protein language models.
\newblock In \emph{International Conference on Learning Representations}.

\bibitem{Wang2023a}
Peiyi Wang, Lei Li, Liang Chen, Dawei Zhu, Binghuai Lin, Yunbo Cao, Qi Liu, Tianyu Liu, and Zhifang Sui.
\newblock 2023a.
\newblock Large language models are not fair evaluators.
\newblock \emph{arXiv preprint arXiv:2305.17926}.

\bibitem{Wang2023b}
Yizhong Wang, Hamish Ivison, Pradeep Dasigi, Jack Hessel, Tushar Khot, Khyathi Raghavi Chandu, David Wadden, Kelsey MacMillan, Noah A Smith, Iz Beltagy, et al.
\newblock 2023b.
\newblock How far can camels go? Exploring the state of instruction tuning on open resources.
\newblock \emph{arXiv preprint arXiv:2306.04751}.

\bibitem{Wei2022}
Jason Wei, Maarten Bosma, Vincent Zhao, Kelvin Guu, Adams Wei Yu, Brian Lester, Nan Du, Andrew M. Dai, and Quoc V Le.
\newblock 2022.
\newblock Finetuned language models are zero-shot learners.
\newblock In \emph{International Conference on Learning Representations}.

\bibitem{Winata2023}
Genta Indra Winata, Alham Fikri Aji, Samuel Cahyawijaya, Rahmad Mahendra, Fajri Koto, Ade Romadhony, Kemal Kurniawan, David Moeljadi, Radityo Eko Prasojo, Pascale Fung, et al.
\newblock 2023.
\newblock NusaX: Multilingual parallel sentiment dataset for 10 Indonesian local languages.
\newblock In \emph{Proceedings of the 17th Conference of the European Chapter of the Association for Computational Linguistics}, pages 815--834.

\bibitem{Winata2024}
Genta Indra Winata, Hanyang Zhao, Anirban Das, Wenpin Tang, David D Yao, Shi-Xiong Zhang, and Sambit Sahu.
\newblock 2024.
\newblock Preference tuning with human feedback on language, speech, and vision tasks: A survey.
\newblock \emph{arXiv preprint arXiv:2409.11564}.

\bibitem{Workshop2022}
BigScience Workshop, Teven Le Scao, Angela Fan, Christopher Akiki, Ellie Pavlick, Suzana Ilic, Daniel Hesslow, Roman Castagn{'e}, Alexandra Sasha Luccioni, Fran{\c c}ois Yvon, et al.
\newblock 2022.
\newblock Bloom: A 176b-parameter open-access multilingual language model.
\newblock \emph{arXiv preprint arXiv:2211.05100}.

\bibitem{Wu2023}
Shijie Wu, Ozan Irsoy, Steven Lu, Vadim Dabravolski, Mark Dredze, Sebastian Gehrmann, Prabhanjan Kambadur, David Rosenberg, and Gideon Mann.
\newblock 2023.
\newblock BloombergGPT: A large language model for finance.
\newblock \emph{arXiv preprint arXiv:2303.17564}.

\bibitem{Xue2021}
Linting Xue, Noah Constant, Adam Roberts, Mihir Kale, Rami Al-Rfou, Aditya Siddhant, Aditya Barua, and Colin Raffel.
\newblock 2021.
\newblock mt5: A massively multilingual pre-trained text-to-text transformer.
\newblock In \emph{Proceedings of the 2021 Conference of the North American Chapter of the Association for Computational Linguistics: Human Language Technologies}, pages 483--498.

\bibitem{Zack2024}
Travis Zack, Eric Lehman, Mirac Suzgun, Jorge A Rodr{'\i}guez, Leo Anthony Celi, Judy Gichoya, Dan Jurafsky, Peter Szolovits, David W Bates, Raja-Elie E Abdulnour, et al.
\newblock 2024.
\newblock Assessing the potential of GPT-4 to perpetuate racial and gender biases in health care: A model evaluation study.
\newblock \emph{The Lancet Digital Health}, 6(1):e12--e22.

\bibitem{DanielZhang2021}
Daniel Zhang, Saurabh Mishra, Erik Brynjolfsson, John Etchemendy, Deep Ganguli, Barbara Grosz, Terah Lyons, James Manyika, Juan Carlos Niebles, Michael Sellitto, et al.
\newblock 2021.
\newblock The AI index 2021 annual report.
\newblock \emph{arXiv preprint arXiv:2103.06312}.

\bibitem{SusanZhang2022}
Susan Zhang, Stephen Roller, Naman Goyal, Mikel Artetxe, Moya Chen, Shuohui Chen, Christopher Dewan, Mona Diab, Xian Li, Xi Victoria Lin, et al.
\newblock 2022.
\newblock Opt: Open pre-trained transformer language models.
\newblock \emph{arXiv preprint arXiv:2205.01068}.

\bibitem{TianyiZhang2023}
Tianyi Zhang, Faisal Ladhak, Esin Durmus, Percy Liang, Kathleen R. McKeown, and Tatsunori B. Hashimoto.
\newblock 2023.
\newblock Benchmarking large language models for news summarization.
\newblock \emph{CoRR}, abs/2301.13848.

\bibitem{Zhou2023}
Chunting Zhou, Pengfei Liu, Puxin Xu, Srini Iyer, Jiao Sun, Yuning Mao, Xuezhe Ma, Avia Efrat, Ping Yu, Lili Yu, Susan Zhang, Gargi Ghosh, Mike Lewis, Luke Zettlemoyer, and Omer Levy.
\newblock 2023.
\newblock LIMA: Less is more for alignment.
\newblock \emph{CoRR}, abs/2305.11206.

\end{thebibliography}

\end{document}
=====END FILE=====
