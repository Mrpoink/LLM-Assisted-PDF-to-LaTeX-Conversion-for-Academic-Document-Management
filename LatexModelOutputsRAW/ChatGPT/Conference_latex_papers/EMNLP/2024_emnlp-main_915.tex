=====FILE: main.tex=====
\documentclass[11pt]{article}

\usepackage[margin=1in]{geometry}
\usepackage[T1]{fontenc}
\usepackage[utf8]{inputenc}

\usepackage{amsmath, amssymb}
\usepackage{booktabs}
\usepackage{array}
\usepackage{longtable}
\usepackage{url}
\usepackage{hyperref}

\title{Exploring the Compositional Deficiency of Large Language Models in\
Mathematical Reasoning Through Trap Problems}

\author{
Jun Zhao$^{1}$\thanks{Equal Contributions.} \quad
Jingqi Tong$^{1}$\footnotemark[1] \quad
Yurong Mou$^{1}$ \quad
Ming Zhang$^{1}$ \quad
Qi Zhang$^{1,2}$\thanks{Corresponding authors.} \quad
Xuanjing Huang$^{1,2}$\footnotemark[2]\
\
$^{1}$School of Computer Science, Fudan University\
$^{2}$Shanghai Key Laboratory of Intelligent Information Processing, Fudan University\
\texttt{{zhaoj19,qz}@fudan.edu.cn, [jqtong23@m.fudan.edu.cn](mailto:jqtong23@m.fudan.edu.cn)}
}

\date{}

\begin{document}
\maketitle

\begin{abstract}
Human cognition exhibits systematic compositionality, the algebraic ability to generate
infinite novel combinations from finite learned components, which is the key to understanding
and reasoning about complex logic. In this work, we investigate the compositionality of
large language models (LLMs) in mathematical reasoning. Specifically, we construct a new
dataset MATHTRAP\footnote{\url{[https://github.com/tongjingqi/MathTrap}}](https://github.com/tongjingqi/MathTrap}}) by introducing carefully designed logical traps into the problem
descriptions of MATH and GSM8K. Since problems with logical flaws are quite rare in
the real world, these represent ``unseen'' cases to LLMs. Solving these requires the models to
systematically compose (1) the mathematical knowledge involved in the original problems
with (2) knowledge related to the introduced traps. Our experiments show that while LLMs
possess both components of requisite knowledge, they do not spontaneously combine them to
handle these novel cases. We explore several methods to mitigate this deficiency, such as
natural language prompts, few-shot demonstrations, and fine-tuning. Additionally, we test the
recently released OpenAI o1 model and find that human-like `slow thinking' helps improve the
compositionality of LLMs. Overall, systematic compositionality remains an open challenge for
large language models.
\end{abstract}

\section{Introduction}
Humans excel at learning fundamental concepts and skills, systematically combining them to solve
new problems. For instance, when a person possesses (a) the knowledge of how to solve
quadratic equations with one variable, and (b) the understanding of what integers are, they can
combine these two domains of knowledge to tackle the problem ``Find the integer solutions of $x^2+x=3.$''
They would first solve the equation, and then determine whether the obtained solutions are
integers or not. Fodor and Pylyshyn (1988) had a famous viewpoint that artificial neural networks
lack this compositionality, and thus cannot serve as reliable cognitive models. Current LLMs have
achieved unprecedented success on tasks requiring complex reasoning (Guo et al., 2024; Toshniwal
et al., 2024). We wonder whether compositionality still poses a significant challenge for LLMs?

Toward this goal, we construct a new MATHTRAP dataset by introducing carefully designed
logical traps into the original problems of the MATH (Hendrycks et al., 2021) and GSM8K
(Cobbe et al., 2021) datasets. For example, by modifying the original problem `Find the solution
of the equation $x^2 + x = 3$'' to `Find the integer solution of the equation $x^2 + x = 3$,''
the model needs to combine (a) the knowledge involved in the original problem (how to solve
quadratic equations with one variable) and (b) the knowledge about the trap (the definition of
integers) to handle these trap problems (in fact, the original equation has no integer solutions).
Another reason for evaluating compositionality through trap problems is that these problems rarely
appear in the real world, so it is unlikely that LLMs provide the correct answers solely by following
the trained reasoning paths.

We conduct comprehensive tests on leading LLMs and recruit 43 undergraduate students from top
universities as human controls. We find that LLMs and humans exhibit strikingly different behavioral
patterns when dealing with trap problems. Despite possessing both (a) and (b) knowledge components,
LLMs fail to spontaneously compose them to handle trap problems, while humans can. This suggests
that tasks requiring compositional generalization remain challenging for current LLMs. Furthermore,
the ability of well-aligned LLMs to handle trap problems can be elicited through external interventions,
such as natural language prompts, few-shot demonstrations, and supervised fine-tuning. Furthermore,
we find that the human-like `slow thinking' demonstrated by OpenAI's o1 (OpenAI, 2024) also helps
improve the compositionality of LLMs. Nevertheless, systematic compositionality remains an open
challenge for current LLMs.

The contributions of this work are threefold: (1) We investigate the compositional generalization of
LLMs on mathematical reasoning, and demonstrate their stark under performance compared to humans.
(2) An effective method to construct `unseen problems' by introducing traps into original problems,
and a dataset called MATHTRAP that cannot be solved by simply following the reasoning paths seen
during training. (3) Comprehensive experiments exploring the impact of model size, the degree of
alignment, and external interventions on performance on the MATHTRAP.

\section{Background and Definition}
In this section, we provide the definition of compositionality discussed in this paper, based on
Hilbert's formal deductive systems (Hilbert, 1922):

\paragraph{Definition 1. (Hilbert's Formal Deductive System)}
This system consists of (1) a syntax $G$, specifying which derivation statements are legal,
(2) a set of inference rules $R$\footnote{We refer to the axioms and inference rules as knowledge.},
clearly stating how new facts (or theorems) can be derived from existing facts (axioms or already
proved theorems), and (3) axioms: a predetermined set $A$ of established facts.

Under this deductive system, reasoning is defined as the process of deriving new facts from existing
facts and rules.

\paragraph{Definition 2. (Compositionality in Mathematical Reasoning)}
Suppose problem sets $Q_1$, $Q_2$, $Q_3$ are described using the same syntax $G$, and $Q_1$ and
$Q_2$ can derive final answers through tuple $(R_1, A_1)$, $(R_2, A_2)$ respectively, while $Q_3$
requires reasoning using $R_1 \cup R_2$ based on $A_1 \cup A_2$ (or a subset) to derive the final answer.
If a reasoning engine can solve $Q_1$ and $Q_2$, we say it possesses compositionality if it can solve $Q_3$.

\section{The MATHTRAP Dataset}
Existing datasets for evaluating compositionality are limited to symbolic reasoning with semantic
decoupling (Lake and Baroni, 2023; Dziri et al., 2023). However, the semantics of language play a crucial
role in the reasoning process of LLMs (Tang et al., 2024). Our MATHTRAP dataset aims to evaluate the
compositionality of LLMs on semantically rich math word problems. More importantly, the `unseen' feature
of our dataset prevents models from simply following the trained reasoning paths to arrive at solutions.

\subsection{Dataset Composition}
\paragraph{Sample Composition.}
As illustrated in Table~\ref{tab:overview}, each sample in MATHTRAP can be viewed as a problem triplet:
(1) \textbf{Original problem:} Sampled from the MATH and GSM8K datasets. These problems are used to
evaluate the model's grasp of math knowledge from these datasets. (2) \textbf{Concept Problem:} Manually
crafted to assess the model's understanding of the trap concepts to be introduced. These problems are
intentionally simple, requiring only knowledge of the trap concept to solve. (3) \textbf{Trap Problem:}
Created by manually introducing logical traps into the original problems. These are designed to evaluate
the model's compositional generalization ability. Solving these problems requires the model to
systematically combine knowledge from the original math problem with the introduced trap concept.

The MATHTRAP dataset consists of two subsets: \textbf{Public} and \textbf{Private}. The Public subset contains
105 problem triplets. Using GPT-4, we paraphrase these samples to expand the dataset to 1,000 problem
triplets, which are used in all fine-tuning experiments discussed in this paper. We manually verify the
quality of the subset. Please refer to Table~\ref{tab:paraphrase_templates} for the templates used in
paraphrasing. We have made this subset publicly available to facilitate community evaluation of
compositionality.

The Private subset comprises 155 problem triplets, and the evaluation results presented in this paper are
based on this subset. This portion of the data will not be made public to mitigate the risk of data leakage.
Researchers and institutions interested in evaluating their models on the Private subset can contact us to
arrange for testing.

\paragraph{Problem Topic Composition.}
MATHTRAP comprises problems from two main sources: 15.5% from the GSM8K dataset and 84.5% from
the MATH dataset. It covers a diverse range of mathematical topics, including algebra (23.2%), counting
and probability (22.6%), geometry (16.1%), prealgebra (12.3%), number theory (7.74%), and precalculus
(2.58%).

\paragraph{Trap Categories.}
We carefully designed five categories of traps for constructing the MATHTRAP dataset. These include:

1. \textbf{Concept Undefined:} The reasoning process involves undefined mathematical concepts (such as
   $\tan 90^\circ$, 0 as a divisor, etc.). 2) \textbf{Missing Condition:} Lacking the necessary conditions required
   to solve the problem. 3) \textbf{Direct Contradiction:} Two conditions in the problem description directly
   contradict each other, which can be discovered without complex calculations. 4) \textbf{Indirect Contradiction:}
   There are indirect contradictions in the problem description, which can only be discovered during the
   reasoning process. 5) \textbf{Violating Common Sense:} The condition or final answer violates common sense.

\begin{table*}[t]
\centering
\small
\setlength{\tabcolsep}{6pt}
\begin{tabular}{p{2.2cm}p{6.3cm}p{6.3cm}p{5.2cm}}
\toprule
\textbf{Concept} & \textbf{Original Problem} & \textbf{Trap Problem} & \textbf{Conceptual Problem} \
\midrule
Undefined (16%) &
In right triangle $XYZ$ with $\angle YXZ = 90^\circ$, $XY = 24$ and $YZ = 25$. Find $\tan Y$. &
In right triangle $XYZ$ with $\angle YXZ = 90^\circ$, $XY = 24$ and $YZ = 25$. Find $\tan X$. &
Does $\tan 90^\circ$ exist? \
\midrule
Missing Condition (6%) &
Natalia sold 48 clips in April and half as many clips in May. How many clips did Natalia sell altogether in April and May? &
Natalia sold 48 clips in April and half as many clips in May. How many clips did Natalia sell altogether in April and June? &
Given the sales figures for May and June, can the sales figures for April and June be calculated? \
\midrule
Direct Contradiction (24%) &
An equilateral triangle has a perimeter of 30 centimeters. Calculate its area. &
An equilateral triangle has a perimeter of 30 centimeters and a height of 10 centimeters. Calculate its area. &
Can the height of an equilateral triangle be equal to its side length? \
\midrule
Indirect Contradiction (38%) &
Find the solution of the equation $x^2 + x = 3$. &
Find the integer solution of the equation $x^2 + x = 3$. &
Is the $\sqrt{13}$ an integer? \
\midrule
Violating Common Sense (15%) &
Max picks 2 different cards without replacement from a standard 52-card deck. What is the probability that the cards are of different suits? &
Max picks 5 different cards without replacement from a standard 52-card deck. What is the probability that the cards are of different suits? &
Is it possible to pick five different suits of cards from a standard deck of playing cards? \
\bottomrule
\end{tabular}
\caption{Overview of the MATHTRAP Dataset. The first column represents the five introduced trap types and their
percentages in the dataset. The difference in problem descriptions before and after introducing traps is emphasized
in the original paper via highlighting. Additionally, we annotate Conceptual Problems to test whether models possess
trap-related knowledge. We hope that if a model can accurately answer both the Original Problems and the Conceptual
Problems, it will also be able to accurately answer the Trap Problems. Appendix section 3.1 provides definitions of
the trap types, and Table~\ref{tab:trap_explanations} offers explanations for these 5 example traps. We have included
GPT-4-0125-preview's responses to selected problem from the table in Appendix Tables~\ref{tab:gpt4_trap_12}--\ref{tab:gpt4_oc_15}.}
\label{tab:overview}
\end{table*}

\subsection{Evaluation Protocol}
We use accuracy as the evaluation metric for the problem. To measure compositional generalization,
we calculate the ratio between the accuracy on trap questions and the accuracy on original questions.
A model demonstrating good compositional generalization should exhibit similar performance before and
after the introduction of trap questions, rather than showing degradation.

For Original Problems, following prior work (Yu et al., 2023; Gou et al., 2024; Xi et al., 2024; He et al., 2024),
we calculate accuracy by determining whether the model's final answer matches the standard answer. For Trap
Problems and Conceptual Problems, we additionally check the intermediate steps of the model's response to
determine whether it correctly identified the trap. We employ GPT-4 as the judge for the checks. In experiments
where GPT-4 might exhibit bias, we supplement our evaluation with results using Claude-3.5-Sonnet as an
additional judge. We provide the prompts used for the evaluation in Table~\ref{tab:judge_prompt} in the Appendix.

\section{Results and Disscussion}
\subsection{The Compositionality of LLMs}
We evaluate the compositionality of LLMs on the MATHTRAP dataset, with results shown in
Table~\ref{tab:main_results}. Proprietary LLMs achieve over 70% accuracy on these Conceptual Problems,
with OpenAI o1 even reaching 96.2%. This indicates that LLMs possess the knowledge required to identify most
traps. However, when comparing LLM performance on original versus trap problems, we observe a significant
decline. Most proprietary LLMs achieve less than half their original accuracy on trap problems. This indicates
that even advanced, well-aligned LLMs struggle to apply trap-related knowledge flexibly to novel reasoning paths.
Notably, o1-preview(Web) achieved a ratio of 77.4, significantly higher than GPT-4's 51.2. This suggests that o1's
test-time scaling, akin to human `slow thinking', effectively improves LLM compositionality. However, it still falls
short of the human ratio of 85.9. For detailed information on the human evaluation, please refer to Section~4.2.

Open-source models' ratios below 20 indicate that focusing solely on GSM8K and MATH accuracy doesn't truly
enhance LLM reasoning abilities. Nevertheless, extensive pre-training (Llemma-MetaMath-7b vs. MetaMath-7b)
and larger model scales (from 7b to 70b) still yield better compositional generalization effects (ratio increasing
from 5.84% to 19%).

\begin{table}[t]
\centering
\small
\begin{tabular}{lrrrr}
\toprule
\textbf{Model} & \textbf{Conceptual} & \textbf{Original} & \textbf{Trap} & \textbf{Ratio} \
\midrule
Gemini-Pro & 70.0 & 36.9 & 8.30 & 22.5 \
Claude3-Opus & 87.7 & 68.5 & 19.0 & 27.7 \
Claude-3.5-Sonnet & 93.9 & 75.0 & 19.4 & 25.9 \
GPT-3.5-turbo-0125 & 74.6 & 40.5 & 7.60 & 18.8 \
GPT-4-0125-preview & 90.0 & 70.3 & 36.0 & 51.2 \
o1-preview(API) & 96.2 & 88.3 & 38.1 & 43.1 \
o1-preview(Web) & 92.3 & 87.5 & 67.7 & 77.4 \
Kimi & 71.5 & 46.1 & 19.6 & 42.5 \
Llemma-MetaMath-7B & 55.2 & 41.4 & 6.40 & 15.5 \
MetaMath-7B & 43.2 & 32.5 & 1.90 & 5.84 \
MetaMath-13B & 37.8 & 37.5 & 3.90 & 10.4 \
MetaMath-70B & 57.6 & 34.2 & 6.50 & 19.0 \
Llama3-8B & 70.5 & 33.3 & 6.45 & 19.4 \
Llama3-8B-Base & 44.7 & 33.3 & 6.45 & 19.4 \
Llama3-70B & 88.5 & 61.7 & 7.74 & 12.5 \
Llama3-70B-Base & 53.8 & 37.5 & 7.74 & 20.6 \
Llama3.1-8B & 70.8 & 61.7 & 13.5 & 21.9 \
Llama3.1-70B & 88.5 & 69.2 & 19.4 & 28.0 \
\bottomrule
\end{tabular}
\caption{Accuracy (%) of various models on three types of MATHTRAP problems. `Conceptual' represents Conceptual
problems, `Original' refers to the original problems, and `Trap' denotes the trap problems. `Ratio' refers to the ratio
of the accuracy on Trap problems to the accuracy on Original problems. It reflects the degree to which the performance
is maintained when facing problems with traps, relative to the original problems.}
\label{tab:main_results}
\end{table}

\subsection{The Compositionality of Human}
As a control experiment, we evaluate human performance on the MATHTRAP dataset. Specifically, we recruit
43 undergraduate students majoring in science and engineering from top universities to participate in the experiment.
Each student is randomly assigned two problems: one original problem and one trap problem. We only assign two
problems because once participants discover that a problem contain a trap, they would consciously look for traps in
subsequent problems, which would affect the accuracy of the results. During the answering process, participants are
not aware that the assigned problems might contain traps. To prevent participants from discovering the traps by
comparing the two problems, the original problem and the trap problem assigned to each participant are completely
different. The results are shown in Table~\ref{tab:human_results}. Humans achieve an accuracy of 83.8% on the trap
problems. In terms of the ratio of accuracy on trap problems to original problems, humans attained an accuracy ratio of
$83.8/97.6 = 85.9%$, far surpassing all existing LLMs. This indicates that humans demonstrate strong compositional
reasoning ability on the MATHTRAP dataset. For cases where participants fail to correctly identify the traps, we further
inform them that the problems might contain traps and ask them to answer again. After receiving this hint, the human
accuracy rate increases from 83.8% to 95.1%, almost identical to their performance on the original problems.

\begin{table}[t]
\centering
\small
\begin{tabular}{lr}
\toprule
\textbf{Condition} & \textbf{Human Accuracy} \
\midrule
Trap Problem(w/o Notice) & 83.8 \
Trap Problem(w/ Notice) & 95.1 \
Original Problem & 97.6 \
\bottomrule
\end{tabular}
\caption{Human accuracy(%) on MATHTRAP. `Trap Problem (w/o Notice)'' refers to the accuracy of human
solutions when unaware that the problems contain traps. `Trap Problem (w/ Notice)'' indicates the accuracy of human
solutions when informed that the problems contain traps. ``Original Problem'' refers to the accuracy of human solutions
on the original problems.}
\label{tab:human_results}
\end{table}

\subsection{Mitigating LLMs' Failure on MathTrap}
The results in Table~\ref{tab:main_results} show that LLMs have acquired the relevant knowledge needed to solve
trap problems, but are unable to spontaneously apply this knowledge to reasoning on trap problems. Therefore, we
attempt to mitigate this issue through external interventions, with the specific results presented in
Table~\ref{tab:interventions} and \ref{tab:finetune}.

The GPT series models constitute a significant subset of proprietary models in Table~\ref{tab:interventions}. To
mitigate potential bias in evaluating these models using GPT-4, we supplement our assessment with results using
Claude-3.5-Sonnet as a judge.

\paragraph{Natural language prompt:}
Adding the prompt `Note that this problem may be unsolvable or has no solution'' before the problem statement.
Refer to Table~\ref{tab:notice_prompt} for the full prompt. Our results show that natural language prompts can guide
LLMs to notice the contradictions or traps in the problem descriptions without affecting the accuracy on `Original
Problems'', especially for those well-aligned closed-source LLMs.

\paragraph{Few-shot demonstration:}
In the 1-shot setting, a randomly sampled trap problem and its reasoning process are inserted into the context
(refer to Table~\ref{tab:icl_1shot} for the full prompt); in the 5-shot setting, 2 original problems and 3 trap problems
with their reasoning processes are inserted (refer to Table~\ref{tab:icl_5shot} for the full prompt). The results show
that compared to natural language prompts, few-shot demonstrations are more effective in handling trap problems.
Additionally, in the 5-shot setting with a mix of original problems, the accuracy on original problems is also improved.

\paragraph{Fine-tuning:}
We use the MATHTRAP public subset containing 1,000 problem triplets for fine-tuning. Given that GPT-4 was
employed for data augmentation, we included additional evaluation results using Claude-3.5-Sonnet in
Table~\ref{tab:finetune} to avoid potential assessment bias from using GPT-4 as both an augmenter and judge. The
experiments demonstrate that fine-tuning can significantly improve model performance on trap problems without
prompt, but it may also reduce the accuracy of solving original problems.

\begin{table*}[t]
\centering
\small
\begin{tabular}{llrrrrrrrr}
\toprule
\textbf{Model} & \textbf{Judge} &
\multicolumn{4}{c}{\textbf{Original Problem}} &
\multicolumn{4}{c}{\textbf{Trap Problem}} \
\cmidrule(lr){3-6}\cmidrule(lr){7-10}
&  & w/o Notice & w/ Notice & ICL (1-shot) & ICL (5-shot) & w/o Notice & w/ Notice & ICL (1-shot) & ICL (5-shot) \
\midrule
Gemini-pro & GPT-4 & 36.9 & 37.8 & 32.4 & 50.0 & 8.3 & 14.1 & 8.94 & 27.7 \
Claude3-Opus & GPT-4 & 68.5 & 65.8 & 68.5 & 82.5 & 19.0 & 40.7 & 29.0 & 56.1 \
Claude3-Opus & Claude-3.5 & 67.6 & 64.0 & 68.5 & 70.8 & 16.1 & 48.4 & 23.9 & 47.1 \
GPT-3.5-turbo-0125 & GPT-4 & 40.5 & 40.5 & 45.9 & 51.7 & 7.74 & 12.2 & 12.2 & 23.9 \
GPT-3.5-turbo-0125 & Claude-3.5 & 37.8 & 29.7 & 44.1 & 46.7 & 7.10 & 13.5 & 11.0 & 12.9 \
GPT-4-0125-preview & GPT-4 & 70.3 & 72.1 & 65.8 & 77.5 & 35.5 & 50.3 & 41.9 & 48.4 \
GPT-4-0125-preview & Claude-3.5 & 65.8 & 67.6 & 64.9 & 76.7 & 24.5 & 40.6 & 36.8 & 35.5 \
Kimi & GPT-4 & 46.1 & 41.4 & 41.4 & 55.0 & 19.6 & 37.1 & 26.4 & 31.0 \
Llemma-MetaMath-7B & GPT-4 & 41.4 & 42.7 & 32.2 & 33.3 & 6.36 & 9.86 & 7.93 & 7.74 \
MetaMath-7B & GPT-4 & 32.5 & 32.5 & 30.0 & 35.0 & 1.94 & 3.23 & 5.16 & 17.4 \
MetaMath-13B & GPT-4 & 37.5 & 37.5 & 34.2 & 40.7 & 3.87 & 3.87 & 2.58 & 12.3 \
MetaMath-70B & GPT-4 & 34.2 & 30.8 & 35.8 & 38.4 & 6.45 & 7.74 & 4.52 & 16.8 \
Llama3-8B & GPT-4 & 33.3 & 36.7 & 38.3 & 35.8 & 6.45 & 13.5 & 4.52 & 35.5 \
Llama3-8B-Base & GPT-4 & 14.2 & 16.7 & 15.8 & 24.2 & 3.23 & 3.87 & 5.16 & 20.0 \
Llama3-70B & GPT-4 & 61.7 & 61.7 & 54.2 & 61.7 & 7.74 & 26.5 & 14.8 & 39.4 \
Llama3-70B-Base & GPT-4 & 37.5 & 30.0 & 34.2 & 41.7 & 7.74 & 9.03 & 7.74 & 25.8 \
\bottomrule
\end{tabular}
\caption{The impact of external intervention methods on the accuracy for original problems and trap problems.
`w/o Notice'' refers to the control experiment without any external intervention. `w/ Notice'' indicates using a natural
language prompt to inform the model that the problem description may contain traps. ICL (1/5-shot) refers to adding
one or five demonstrations in the context to exemplify how to handle trap problems. The prompt templates employed
are presented in Tables~\ref{tab:notice_prompt}--\ref{tab:icl_5shot} in the Appendix.}
\label{tab:interventions}
\end{table*}

\begin{table}[t]
\centering
\small
\begin{tabular}{lrrr}
\toprule
\textbf{Dataset} & \textbf{Original} & \textbf{Trap(GPT-4)} & \textbf{Trap(Claude-3.5)} \
\midrule
GSM8K+MATH & 20.8 & 1.01 & 0.65 \
GSM8K+MATH+MathTrap1K & 13.3 & 12.4 & 34.2 \
MetaMath395K & 41.4 & 6.36 & 1.94 \
MetaMath395K+MathTrap1K & 33.3 & 29.1 & 11.6 \
\bottomrule
\end{tabular}
\caption{The impact of fine-tuning data configurations on the accuracy for original and trap problems. We use
Llemma as the foundation model. The parentheses indicate the judge model used.}
\label{tab:finetune}
\end{table}

\section{Conclusions}
This paper investigates the compositional generalization of LLMs in mathematical reasoning. By introducing traps
into the problem descriptions, we construct novel ``Trap Problems'' that cannot be solved by merely following trained
reasoning paths for LLMs. Experiments on MATHTRAP demonstrate that LLMs fail to spontaneously combine their
learned knowledge to reason about trap problems. Although this limitation can be mitigated through external
interventions, there remains a significant gap compared to the compositional generalization capabilities of humans.

\section*{Limitations}
The construction of trap problems places high demands on the annotators' abilities, resulting in high annotation
costs. Therefore, how to automatically generate high-quality trap problems through automated methods is worthy
of further investigation.

\begin{thebibliography}{99}

\bibitem{Anil2022}
Cem Anil, Yuhuai Wu, Anders Andreassen, Aitor Lewkowycz, Vedant Misra, Vinay Ramasesh, Ambrose Slone,
Guy Gur-Ari, Ethan Dyer, and Behnam Neyshabur. 2022.
Exploring length generalization in large language models.
Preprint, arXiv:2207.04901.

\bibitem{Azerbayev2024}
Zhangir Azerbayev, Hailey Schoelkopf, Keiran Paster, Marco Dos Santos, Stephen McAleer, Albert Q. Jiang,
Jia Deng, Stella Biderman, and Sean Welleck. 2024.
Llemma: An open language model for mathematics.
Preprint, arXiv:2310.10631.

\bibitem{Bian2024}
Ning Bian, Xianpei Han, Le Sun, Hongyu Lin, Yaojie Lu, Ben He, Shanshan Jiang, and Bin Dong. 2024.
Chatgpt is a knowledgeable but inexperienced solver: An investigation of commonsense problem in large language models.
Preprint, arXiv:2303.16421.

\bibitem{Bubeck2023}
S'ebastien Bubeck, Varun Chandrasekaran, Ronen Eldan, Johannes Gehrke, Eric Horvitz, Ece Kamar, Peter Lee,
Yin Tat Lee, Yuanzhi Li, Scott Lundberg, Harsha Nori, Hamid Palangi, Marco Tulio Ribeiro, and Yi Zhang. 2023.
Sparks of artificial general intelligence: Early experiments with gpt-4.
Preprint, arXiv:2303.12712.

\bibitem{Cobbe2021}
Karl Cobbe, Vineet Kosaraju, Mohammad Bavarian, Mark Chen, Heewoo Jun, Lukasz Kaiser, Matthias Plappert,
Jerry Tworek, Jacob Hilton, Reiichiro Nakano, Christopher Hesse, and John Schulman. 2021.
Training verifiers to solve math word problems.
CoRR, abs/2110.14168.

\bibitem{Dziri2023}
Nouha Dziri, Ximing Lu, Melanie Sclar, Xiang Lorraine Li, Liwei Jiang, Bill Yuchen Lin, Peter West,
Chandra Bhagavatula, Ronan Le Bras, Jena D. Hwang, Soumya Sanyal, Sean Welleck, Xiang Ren, Allyson Ettinger,
Zaid Harchaoui, and Yejin Choi. 2023.
Faith and fate: Limits of transformers on compositionality.
Preprint, arXiv:2305.18654.

\bibitem{FodorPylyshyn1988}
Jerry A. Fodor and Zenon W. Pylyshyn. 1988.
Connectionism and cognitive architecture: A critical analysis.
\emph{Cognition}, 28(1-2):3--71.

\bibitem{Gou2024}
Zhibin Gou, Zhihong Shao, Yeyun Gong, Yelong Shen, Yujiu Yang, Minlie Huang, Nan Duan, and Weizhu Chen. 2024.
Tora: A tool-integrated reasoning agent for mathematical problem solving.
Preprint, arXiv:2309.17452.

\bibitem{Guo2024}
Daya Guo, Qihao Zhu, Dejian Yang, Zhenda Xie, Kai Dong, Wentao Zhang, Guanting Chen, Xiao Bi, Y. Wu, Y. K. Li,
Fuli Luo, Yingfei Xiong, and Wenfeng Liang. 2024.
Deepseek-coder: When the large language model meets programming -- the rise of code intelligence.
Preprint, arXiv:2401.14196.

\bibitem{He2024}
Wei He, Shichun Liu, Jun Zhao, Yiwen Ding, Yi Lu, Zhiheng Xi, Tao Gui, Qi Zhang, and Xuanjing Huang. 2024.
Self-demos: Eliciting out-of-demonstration generalizability in large language models.
arXiv preprint arXiv:2404.00884.

\bibitem{Hendrycks2021}
Dan Hendrycks, Collin Burns, Saurav Kadavath, Akul Arora, Steven Basart, Eric Tang, Dawn Song, and Jacob Steinhardt.
2021. Measuring mathematical problem solving with the MATH dataset.
CoRR, abs/2103.03874.

\bibitem{Hilbert1922}
David Hilbert. 1922. Grundlagen der geometrie. [MISSING]

\bibitem{Hosseini2022}
Arian Hosseini, Ankit Vani, Dzmitry Bahdanau, Alessandro Sordoni, and Aaron Courville. 2022.
On the compositional generalization gap of in-context learning.
In \emph{Proceedings of the Fifth BlackboxNLP Workshop on Analyzing and Interpreting Neural Networks for NLP},
pages 272--280, Abu Dhabi, United Arab Emirates (Hybrid). Association for Computational Linguistics.

\bibitem{Hu2024}
Yi Hu, Xiaojuan Tang, Haotong Yang, and Muhan Zhang. 2024.
Case-based or rule-based: How do transformers do the math?
Preprint, arXiv:2402.17709.

\bibitem{Kazemi2023}
Mehran Kazemi, Najoung Kim, Deepti Bhatia, Xin Xu, and Deepak Ramachandran. 2023.
LAMBADA: Backward chaining for automated reasoning in natural language.
In \emph{Proceedings of the 61st Annual Meeting of the Association for Computational Linguistics (Volume 1: Long Papers)},
pages 6547--6568, Toronto, Canada. Association for Computational Linguistics.

\bibitem{KoralusWangMascianica2023}
Philipp Koralus and Vincent Wang-Mascianica. 2023.
Humans in humans out: On gpt converging toward common sense in both success and failure.
Preprint, arXiv:2303.17276.

\bibitem{LakeBaroni2023}
Brenden Lake and Marco Baroni. 2023.
Human-like systematic generalization through a meta-learning neural network.
\emph{Nature}, 623.

\bibitem{Luo2023}
Haipeng Luo, Qingfeng Sun, Can Xu, Pu Zhao, Jianguang Lou, Chongyang Tao, Xiubo Geng, Qingwei Lin,
Shifeng Chen, and Dongmei Zhang. 2023.
Wizardmath: Empowering mathematical reasoning for large language models via reinforced evol-instruct.
Preprint, arXiv:2308.09583.

\bibitem{Miao2020}
Shen-yun Miao, Chao-Chun Liang, and Keh-Yih Su. 2020.
A diverse corpus for evaluating and developing English math word problem solvers.
In \emph{Proceedings of the 58th Annual Meeting of the Association for Computational Linguistics},
pages 975--984, Online. Association for Computational Linguistics.

\bibitem{OpenAI2023}
OpenAI. 2023. Gpt-4 technical report. Preprint, arXiv:2303.08774.

\bibitem{OpenAI2024}
OpenAI. 2024. Introducing openai o1. \url{[https://openai.com/o1/}](https://openai.com/o1/}). 2024-10-02.

\bibitem{Patel2021}
Arkil Patel, Satwik Bhattamishra, and Navin Goyal. 2021.
Are NLP models really able to solve simple math word problems?
In \emph{Proceedings of the 2021 Conference of the North American Chapter of the Association for Computational Linguistics:
Human Language Technologies}, pages 2080--2094, Online. Association for Computational Linguistics.

\bibitem{Sanyal2022}
Soumya Sanyal, Zeyi Liao, and Xiang Ren. 2022.
RobustLR: A diagnostic benchmark for evaluating logical robustness of deductive reasoners.
In \emph{Proceedings of the 2022 Conference on Empirical Methods in Natural Language Processing},
pages 9614--9631, Abu Dhabi, United Arab Emirates. Association for Computational Linguistics.

\bibitem{Tang2024}
Xiaojuan Tang, Zilong Zheng, Jiaqi Li, Fanxu Meng, Song-Chun Zhu, Yitao Liang, and Muhan Zhang. 2024.
On the paradox of generalizable logical reasoning in large language models. [MISSING]

\bibitem{Toshniwal2024}
Shubham Toshniwal, Ivan Moshkov, Sean Narenthiran, Daria Gitman, Fei Jia, and Igor Gitman. 2024.
Openmathinstruct-1: A 1.8 million math instruction tuning dataset.
Preprint, arXiv:2402.10176.

\bibitem{Wu2024}
Zhaofeng Wu, Linlu Qiu, Alexis Ross, Ekin Aky"urek, Boyuan Chen, Bailin Wang, Najoung Kim, Jacob Andreas,
and Yoon Kim. 2024.
Reasoning or reciting? exploring the capabilities and limitations of language models through counterfactual tasks.
In \emph{Proceedings of the 2024 Conference of the North American Chapter of the Association for Computational Linguistics:
Human Language Technologies (Volume 1: Long Papers)}, pages 1819--1862, Mexico City, Mexico. Association for Computational Linguistics.

\bibitem{Xi2024}
Zhiheng Xi, Wenxiang Chen, Boyang Hong, Senjie Jin, Rui Zheng, Wei He, Yiwen Ding, Shichun Liu, Xin Guo,
Junzhe Wang, et al. 2024.
Training large language models for reasoning through reverse curriculum reinforcement learning.
arXiv preprint arXiv:2402.05808.

\bibitem{Yu2023}
Longhui Yu, Weisen Jiang, Han Shi, Jincheng Yu, Zhengying Liu, Yu Zhang, James T. Kwok, Zhenguo Li,
Adrian Weller, and Weiyang Liu. 2023.
Metamath: Bootstrap your own mathematical questions for large language models.
Preprint, arXiv:2309.12284.

\bibitem{Zhang2023}
Chiyuan Zhang, Daphne Ippolito, Katherine Lee, Matthew Jagielski, Florian Tramer, and Nicholas Carlini. 2023.
Counterfactual memorization in neural language models. [MISSING]

\bibitem{Zheng2024}
Tianyu Zheng, Ge Zhang, Tianhao Shen, Xueling Liu, Bill Yuchen Lin, Jie Fu, Wenhu Chen, and Xiang Yue. 2024.
Opencodeinterpreter: Integrating code generation with execution and refinement.
Preprint, arXiv:2402.14658.

\end{thebibliography}

\appendix

\section{Related Works}
\subsection{Investigation on the Limitations of Transformer Capabilities}
In recent years, large language models (LLMs) have achieved unprecedented success in various tasks requiring complex
reasoning (OpenAI, 2023), such as coding (Guo et al., 2024; Zheng et al., 2024) and solving mathematical problems
(Luo et al., 2023; Toshniwal et al., 2024). Some researchers even view these exciting advancements as sparks of artificial
general intelligence (Bubeck et al., 2023). In stark contrast, these models have shown unexpected failures in simple and
intuitive tasks (Bian et al., 2024; Koralus and Wang-Mascianica, 2023). For example, the state-of-the-art GPT-4 only
achieve 59% accuracy on three-digit multiplication problems (Dziri et al., 2023).

What is the reason behind this stark discrepancy? Recent studies have examined LLMs' ability to composite knowledge
from training scenarios to solve more complex problems. Tests covered tasks such as boolean variable assignment (Anil et al.,
2022), semantic parsing (Hosseini et al., 2022), deductive reasoning (Sanyal et al., 2022), and arithmetic reasoning (Kazemi
et al., 2023). A common trend shows that as problem complexity increases, LLMs' accuracy drops significantly. Our
MATHTRAP is based on math word problems, currently the most widely studied task for evaluating LLM reasoning.
Unlike previous work that assessed models' compositional generalization by increasing problem complexity, we simply
introduced straightforward logical traps into original problems without significantly increasing their complexity.

Researchers have also investigated whether LLMs' impressive reasoning abilities stem from learning general knowledge or
merely reciting examples from their vast training corpora. Studies (Wu et al., 2024; Zhang et al., 2023) on LLMs' performance
in factual versus counterfactual scenarios revealed significant performance drops in counterfactual cases, suggesting LLMs often
recite answers from common cases in their training data. Recent research (Dziri et al., 2023) modeled reasoning tasks as
computational graphs, experimentally demonstrating that LLMs reason through subgraph matching rather than developing
systematic problem-solving skills. Another study (Hu et al., 2024) removed certain samples from the training set and found that
LLMs rely on surrounding cases in the training set for mathematical reasoning rather than learning generalizable rules. These
findings indicate that LLMs still face challenges in learning knowledge and combining them to solve out-of-distribution problems.

\subsection{Math Word Problem Benchmark}
Mathematical word problems have long been considered an effective proxy for evaluating the reasoning abilities of large language
models (LLMs), garnering widespread attention from the academic community. Numerous benchmark datasets of math word problems
have been proposed. ASDiv (Miao et al., 2020) is a dataset covering most knowledge types encountered in elementary school, with each
problem annotated with its corresponding knowledge component and grade level. SVAMP (Patel et al., 2021) comprises 1,000
challenging elementary math word problems carefully designed and curated to assess a system's complex reasoning and multi-step
arithmetic capabilities. Cobbe et al. (2021) introduced GSM8K, a high-quality and diverse evaluation benchmark containing 8.5k math
word problems. Hendrycks et al. (2021) presented MATH, a dataset of 12,500 challenging competition math problems. Recently,
researchers have also constructed unanswerable math word problems to evaluate hallucinations in LLMs' mathematical reasoning process.
However, evaluating LLMs' compositional abilities has been limited to symbolic reasoning tasks (Lake and Baroni, 2023; Dziri et al.,
2023). Tang et al. (2024) found that semantics play a crucial role in the reasoning process, motivating our MATHTRAP dataset to
evaluate LLMs' compositional skills on semantically rich math word problems.

\section{Annotation Process and Standards of MATHTRAP Dataset}
In this section, we provide a detailed explanation covering three aspects: the annotators' background, annotation criteria, and the
annotation process.

\subsection{Qualified annotators}
Our annotation team consists of five students with STEM backgrounds from top universities, with an average math score of 145 out of 150
in their college entrance exams. They possess the necessary math knowledge and passed our qualification test, ensuring their ability to
understand and complete our tasks with high quality.

\subsection{Clear and Specific Annotation Criteria}
\begin{itemize}
\item \textbf{Adherence to trap problem definition:} We provided clear definitions and examples for five types of traps (as shown in 3.1).
Annotators were required to fully understand these definitions and verify that their trap problems align with the corresponding trap definitions.
\item \textbf{Unambiguity:} We require annotators to mutually verify the semantic clarity of trap problems, ensuring the modified questions are unambiguous.
\item \textbf{Difficulty:} We set a standard that the trap problems should cause GPT-3.5 to give incorrect responses in at least 1 out of 5 runs, preventing overly simple trap problems.
\item \textbf{Diversity:} We dynamically monitored the distribution of knowledge points during the annotation process and provided feedback to annotators to adjust their selection of topics.
\end{itemize}

\subsection{Standardized Annotation Process}
\begin{itemize}
\item \textbf{Problem assignment and distribution:} Problems from each knowledge point in the original datasets were equally distributed among five annotators. This approach was adopted to prevent potential bias that might arise if any single annotator were to focus on only a subset of knowledge points.
\item \textbf{Quality control during annotation:} We required strict adherence to annotation standards and recording of the verification process. A supervisor regularly checked the output for compliance with our standards and provided feedback.
\item \textbf{Post-annotation quality assessment:} All annotators cross-verified trap problems created by others, filtering out those that didn't meet the annotation standards.
\end{itemize}

\section{Evaluation}
\subsection{Compared Method}
We evaluate mainstream proprietary and open-source LLMs with strong mathematical capabilities. Proprietary models include Claude3-Opus,
Gemini-pro, OpenAI o1, GPT-4-0125-preview, GPT-3.5-turbo-0125, and Kimi. The open-source models include the 7B, 13B, and 70B versions
of MetaMath (Yu et al., 2023), which are all based on the Llama-2 model. Additionally, we provide Llemma-MetaMath-7B, which is based on
Llemma (Azerbayev et al., 2024), a foundation model pretrained specifically for mathematics.

\subsection{Prompt Template}
Table~\ref{tab:judge_prompt} presents the prompt templates used to evaluate the accuracy of responses from various models using GPT-4.
Table~\ref{tab:paraphrase_templates} displays the prompt templates for enhancing the training set answers with GPT-4.
Table~\ref{tab:notice_prompt} shows the prompt templates used to indicate that a model's question might be unsolvable.
Table~\ref{tab:icl_1shot} presents the prompt templates for in-context learning in a 1-shot setting.
Table~\ref{tab:icl_5shot} presents the prompt templates for in-context learning in a 5-shot setting.

\begin{table}[t]
\centering
\small
\begin{tabular}{p{0.96\linewidth}}
\toprule
Following is a problem with no solution or can't be solved, and a reference answer about how to find the contradiction. There is also an answer you need to make a verdict on whether it finds the same contradiction as the reference answer does. Please output a short comment and end with [0] or [1] ([1] means the answer finds the same contradiction and explicitly states it).\
#The Problem: {input}\
#The Reference Answer: {ref}\
#The Answer Needs Verdict: {answer}\
#Your Verdict:\
\bottomrule
\end{tabular}
\caption{Prompt template used for evaluating the Trap Problem across various Large Language Models (LLMs) using GPT-4.}
\label{tab:judge_prompt}
\end{table}

\begin{table}[t]
\centering
\small
\begin{tabular}{p{0.96\linewidth}}
\toprule
Following is a problem with no solution or can't be solved, and a brief comment on why it has no solution. Please generate an answer that try to solve the question, until you meets where comment says it's unable to solve, and explain the situation more informatively. The answer should not mention there exists ``comment'', you should pretend to find the contradiction independently.\
#Question: {question}\
#Comment: {comment}\
#Rephrased Answer:\
\bottomrule
\end{tabular}
\caption{Prompt template used for answer augmentation of the Trap Problem using GPT-4.}
\label{tab:paraphrase_templates}
\end{table}

\begin{table}[t]
\centering
\small
\begin{tabular}{p{0.96\linewidth}}
\toprule
Please solve the following problem. Note that this problem may be unsolvable or has no solution.\
\texttt{### Instruction:}\
{instruction}\
\texttt{### Response:}\
Let's think step by step.\
\bottomrule
\end{tabular}
\caption{The prompt template uesd for directly suggesting to large language models (LLMs) that the problem might be unreasonable.}
\label{tab:notice_prompt}
\end{table}

\begin{table}[t]
\centering
\small
\begin{tabular}{p{0.96\linewidth}}
\toprule
Please solve the following problem. Note that this problem may be unsolvable or has no solution.\
Here is an example:\
\texttt{### Instruction:}\
If $r$ is a negative integer, and the maximum value of the expression $5r^2 + br - 12$ is 36, what's the value of $b$?\
\texttt{### Response:} Let's think step by step. A parabola opening upwards doesn't have a maximum value for negative integer $r$.\
\texttt{### Instruction:}\
{instruction}\
\texttt{### Response:} Let's think step by step.\
\bottomrule
\end{tabular}
\caption{The prompt template used under the 1-shot setting for in-context learning.}
\label{tab:icl_1shot}
\end{table}

\begin{table}[t]
\centering
\small
\begin{tabular}{p{0.96\linewidth}}
\toprule
Please solve the following problem. Note that this problem may be unsolvable or has no solution.\
Here is an example:\
\texttt{### Instruction:}\
If $r$ is a negative integer, and the maximum value of the expression $5r^2 + br - 12$ is 36, what's the value of $b$?\
\texttt{### Response:} Let's think step by step.\
\ldots\
Regardless of whether $r$ is positive or negative, a parabola that opens upwards does not have a maximum value.\
Therefore, it is impossible to equate the maximum value of the expression to 36 as the problem initially asks, and we cannot solve for $b$ under these conditions. The question has posed an unsolvable scenario.\
The problem is unsolvable.\
\texttt{### Instruction:}\
How many prime numbers can be divided by 39?\
\texttt{### Response:} Let's think step by step. No prime numbers can be divided by 39.\
\ldots\
Therefore, there are 0 prime numbers that can be divided by 39.\
The answer is: 0\
\texttt{### Instruction:}\
If odd integers $xy$ satisfies $\frac{1}{2}-\frac{1}{3}=\frac{x}{y}$, then what is the value of $x$?\
\texttt{### Response:} Let's think step by step. To solve this problem, we must first simplify the left-hand side of the equation, $\frac{1}{2}-\frac{1}{3}$. This simplifies to $\frac{1}{6}$.\
\ldots\
Therefore, under the given conditions, there is no solution to this problem. The constraints on $x$ and $y$ being odd integers and the equation $\frac{1}{2}-\frac{1}{3}=\frac{x}{y}$ cannot hold true simultaneously.\
The problem is unsolvable.\
\texttt{### Instruction:}\
If a triangle has two sides of lengths 5 and 7 units, then how many different integer lengths can the third side be?\
\texttt{### Response:} Let's think step by step. Let the two given sides be $a$ and $b$.\
\ldots\
In this case, $a=5$ and $b=7$, so the inequalities become $|5-7|<c<5+7$, which simplify to $2<c<12$.\
Since $c$ must be an integer, the possible integer lengths for the third side are 3, 4, 5, 6, 7, 8, 9, 10, and 11.\
Therefore, there are 9 different integer lengths the third side can be.\
The answer is: 9\
\texttt{### Instruction:}\
Find $p$ if the focus of the parabola $x + x^2 = -\frac{p}{12} y^2$. is (-1/2,1).\
\texttt{### Response:} Let's think step by step. To begin solving this problem, we first need to express the equation in the standard form of a parabola.\
\ldots\
Therefore, we can't express the equation in the standard form of a parabola, which indicates that the equation does not represent a parabola. Consequently, we cannot find a focus for a parabola because the given equation does not describe a parabola. The initial assumption that we are dealing with a parabola is incorrect.\
The problem is unsolvable.\
\texttt{### Instruction:}\
{instruction}\
\texttt{### Response:} Let's think step by step.\
\bottomrule
\end{tabular}
\caption{The prompt template used under the 5-shot setting for in-context learning.}
\label{tab:icl_5shot}
\end{table}

\clearpage

\begin{table}[t]
\centering
\small
\begin{tabular}{p{2.2cm}p{6.0cm}p{6.8cm}}
\toprule
\textbf{Type} & \textbf{Trap problem} & \textbf{Explanation} \
\midrule
Concept Undefined &
In right triangle $XYZ$ with $\angle YXZ = 90^\circ$, we have $XY = 24$ and $YZ = 25$. Find $\tan X$. &
$\angle YXZ = 90^\circ$, so $\tan X = \tan(\pi/2)$ is undefined. \
\midrule
Missing Condition &
Natalia sold 48 clips in April and half as many clips in May. How many clips did Natalia sell altogether in April and June? &
We don't know anything about June, so it's impossible to calculate the sum of the sales for April and June. \
\midrule
Direct Contradiction &
An equilateral triangle has a perimeter of 30 centimeters and a height of 10 centimeters. Calculate the area of the triangle. &
The height of the equilateral triangle and its side length are both 10 centimeters, which is contradictory and impossible. \
\midrule
Indirect Contradiction &
Find the integer solution of the equation $x^2 + x = 3$. &
The 2 solutions of this quadratic equation is $\frac{-1 \pm \sqrt{13}}{2}$, so there is no integer solution. \
\midrule
Violating Common Sense &
Max picks 5 different cards without replacement from a standard 52-card deck. What is the probability that the cards are of different suits? &
There are only 4 suits in a deck, so it's impossible for 5 cards to be of different suits. \
\bottomrule
\end{tabular}
\caption{Explanation of examples of trap problems for each category. The sections highlighted in yellow in the original
paper delineate the distinction between original problems and trap problems.}
\label{tab:trap_explanations}
\end{table}

\begin{table*}[t]
\centering
\small
\begin{tabular}{p{4.8cm}p{6.2cm}p{6.2cm}}
\toprule
\textbf{Trap problem} & \textbf{Output(w/o Notice)} & \textbf{Output(w/ Notice)} \
\midrule
In right triangle $XYZ$ with $\angle YXZ = 90^\circ$, we have $XY = 24$ and $YZ = 25$. Find $\tan X$. &
Alright, in our right triangle $XYZ$ where $\angle YXZ = 90^\circ$, we're given the lengths $XY = 24$ and $YZ = 25$. To find $\tan X$, we're essentially looking for the ratio of the opposite side to the adjacent side of angle $X$. Let's break it down: \ldots
Finally, to find $\tan X$, we use the lengths $XY$ (opposite side to $X$) and $XZ$ (adjacent side to $X$): $\tan X = \frac{24}{7}$. So, $\tan X = \frac{24}{7}$. This gives us a detailed step-by-step approach to finding $\tan X$ in the triangle $XYZ$ where $\angle YXZ = 90^\circ$, with $XY = 24$ and $YZ = 25$. &
Given a right triangle $XYZ$ with $\angle YXZ = 90^\circ$, we're tasked with finding $\tan X$. We have the lengths of two sides: $XY = 24$ and $YZ = 25$. The side $XY$ can be considered as the side opposite to $\angle YXZ$, and $YZ$ can be thought of as the hypotenuse of the triangle. However, to find $\tan X$, we actually need to know the relationship between the side adjacent to $\angle X$ (which would be $XZ$ in this case) and the side opposite to $\angle X$ (which is $XY$). \ldots
Now that we have the length of the side adjacent to $\angle X$, we can find $\tan X$: $\tan X = \frac{XY}{XZ} = \frac{24}{7}$. Therefore, $\tan X = \frac{24}{7}$. \
\midrule
Natalia sold 48 clips in April and half as many clips in May. How many clips did Natalia sell altogether in April and June? &
Let's break it down: 1. \textbf{April Sales:} Natalia sold clips to 48 friends in April. 2. \textbf{May Sales:} It's mentioned that she sold to only half the number of friends in April, which would be $\frac{1}{2}\times 48 = 24$ friends in May. However, to find the \textbf{cumulative number} of clips sold in April and \textbf{June}, we need a bit more information about June, as the instruction doesn't provide any details for June's sales. Based on the provided details, we can calculate the cumulative number for April and May, but without the June data, we can't complete the request as specified. If you have any information about June's sales or if there was a typo, and you meant to ask for the cumulative total for April and May, please let me know! &
Alright, let's break this down step by step to understand the situation: 1. \textbf{April Sales:} We know that Natalia managed to sell clips to 48 of her buddies in April. 2. \textbf{May Sales:} It's stated that she sold half of that amount in May. So, if she sold 48 clips in April, half of that amount is $48 \div 2 = 24$ clips in May. 3. \textbf{June Sales:} However, the instruction asks for the total number of clips sold in April and \textbf{June}, but we don't have any information about June's sales. The mention of May's sales seems to be there to potentially confuse or add extra information that doesn't directly pertain to the question as asked. Given that we only have data for April and May, and the question asks for the total for April and June without providing information on June's sales, we can't accurately determine the total number of clips Natalia sold in both April and June. Without the June sales figure, the problem doesn't have a complete solution based on the information provided. \
\bottomrule
\end{tabular}
\caption{Responses of GPT-4-0125-preview to Trap Problems. `Output(w/o Notice)'' refers to the model's output when
no additional prompt is provided, whereas `Output(w/ Notice)'' denotes the outputs when the model is informed that
the problem may be unreasonable. The sections highlighted in yellow in the original paper delineate the distinction
between original problems and trap problems. The green sections represent instances where the model's final answers
are correct, while the red sections indicate where the model's final answers are incorrect.}
\label{tab:gpt4_trap_12}
\end{table*}

\begin{table*}[t]
\centering
\small
\begin{tabular}{p{4.8cm}p{6.2cm}p{6.2cm}}
\toprule
\textbf{Trap problem} & \textbf{Output(w/o Notice)} & \textbf{Output(w/ Notice)} \
\midrule
An equilateral triangle has a perimeter of 30 centimeters and a height of 10 centimeters. Calculate the area of the triangle. &
Given that the equilateral triangle has a perimeter of 30 centimeters, we can first find the length of one side. Since an equilateral triangle has all sides of equal length, we can divide the perimeter by 3:
Side length $= \frac{\text{Perimeter}}{3} = \frac{30}{3} = 10$ cm
Now, to find the area of the triangle, we can use the formula for the area of a triangle, which is $\frac{1}{2} \times \text{base} \times \text{height}$. However, for an equilateral triangle, we can also use the formula involving its side length and height. Since we know both the side length and the height, we can use the following formula where $h$ is the height:
Area $= \frac{1}{2}\times \text{Side length}\times h = \frac{1}{2}\times 10 \times 10 = 50~\text{cm}^2$.
Therefore, the area of the equilateral triangle is $50~\text{cm}^2$. &
\ldots However, there's a catch. The height given doesn't align with the properties of an equilateral triangle where each side is 10 cm. In an equilateral triangle, the height ($h$) can be found using the Pythagorean theorem for a 30-60-90 triangle, which is derived from dividing the equilateral triangle into two right triangles. The formula for the height in terms of the side length ($a$) is:
$h = \frac{\sqrt{3}}{2}a$.
For a side length of 10 cm, the correct height would be:
$h = \frac{\sqrt{3}}{2}\times 10 \approx 8.66$ cm
So, if we were to use the correct height in calculating the area, it would actually be:
$A = \frac{1}{2}\times 10 \times 8.66 \approx 43.3~\text{cm}^2$.
Therefore, with the height given as 10 cm, the problem doesn't align with the properties of an equilateral triangle with sides of 10 cm each. This discrepancy suggests a misunderstanding in the problem statement as the height and the properties of the equilateral triangle given do not match. \
\midrule
Max picks five different cards without replacement from a standard 52-card deck. What is the probability that the cards are of different suits? &
To calculate the probability that Max picks five different cards of different suits from a standard 52-card deck without replacement, we need to consider the composition of the deck and the sequence of his picks.

1. \textbf{First Card:} \ldots
2. \textbf{Fifth Card:} The fifth card can be of any suit since we're only looking for different suits in the first four cards. \ldots
   Overall Probability $= \ldots \approx 0.1055$.
   So, the probability that Max picks five different cards of different suits from a standard 52-card deck without replacement is approximately 0.1055, or 10.55%. &
   \ldots 4. \textbf{Fourth Card:} For the fourth card, it must be one of the 13 cards left that belong to the suit not yet chosen, out of the 49 cards remaining, giving a probability of $\frac{13}{49}$.
   The overall probability is then calculated by multiplying these probabilities together, reflecting the sequence of choices Max makes under the condition that each card comes from a different suit:
   Probability $= \frac{52}{52}\times\frac{39}{51}\times\frac{26}{50}\times\frac{13}{49}$.
   However, since the original question asks about five cards from different suits in a 52-card deck, which only contains four suits, the scenario is not achievable. \
   \bottomrule
   \end{tabular}
   \caption{Responses of GPT-4-0125-preview to Trap Problems. `Output(w/o Notice)'' refers to the model's output when
   no additional prompt is provided, whereas `Output(w/ Notice)'' denotes the outputs when the model is informed that
   the problem may be unreasonable. The sections highlighted in yellow in the original paper delineate the distinction
   between original problems and trap problems. The green sections represent instances where the model's final answers
   are correct, while the red sections indicate where the model's final answers are incorrect.}
   \label{tab:gpt4_trap_13}
   \end{table*}

\begin{table*}[t]
\centering
\small
\begin{tabular}{p{4.9cm}p{1.6cm}p{6.5cm}p{4.9cm}p{6.5cm}}
\toprule
\textbf{Original problem} & \textbf{Answer} & \textbf{Output} & \textbf{Conceptual Problem} & \textbf{Output} \
\midrule
In right triangle $XYZ$ with $\angle YXZ = 90^\circ$, we have $XY = 24$ and $YZ = 25$. Find $\tan Y$. &
$\frac{7}{24}$ &
In right triangle $XYZ$, with $\angle YXZ = 90^\circ$, we're given that $XY = 24$ and $YZ = 25$. To find $\tan Y$, we first need to identify the sides relevant to angle $Y$. In a right triangle, $\tan$ of an angle is defined as the ratio of the opposite side to the adjacent side. Here, angle $Y$ is at the vertex formed by sides $XY$ and $YZ$. \ldots
$XZ^2 = 625 - 576$\
$XZ^2 = 49$\
Taking the square root of both sides gives:\
$XZ = 7$\
Now that we have the length of $XZ$ (7), we can find $\tan Y$:\
$\tan Y = \frac{\text{opposite to }Y}{\text{adjacent to }Y} = \frac{XZ}{XY} = \frac{7}{24}$.\
Therefore, $\tan Y = \frac{7}{24}$. &
Does $\tan 90^\circ$ exist? &
True. To understand why, let's break it down. The tangent function, represented as $\tan(\theta)$, is defined as the ratio of the sine of angle $\theta$ to the cosine of angle $\theta$. In mathematical terms, this is expressed as:
$\tan(\theta) = \frac{\sin(\theta)}{\cos(\theta)}$.
When we apply this definition to $\tan(90)$, we encounter an issue. The sine of 90 degrees is 1, but the cosine of 90 degrees is 0. Plugging these values into our formula gives us:
$\tan(90) = \frac{\sin(90)}{\cos(90)} = \frac{1}{0}$.
Division by zero is undefined in mathematics. Because the cosine of 90 degrees is 0, attempting to calculate $\tan(90)$ results in an undefined operation. Therefore, it is true to say that $\tan(90)$ does not exist due to this division by zero. \
\midrule
An equilateral triangle has a perimeter of 30 centimeters. Calculate the area of the triangle. &
$25\sqrt{3}\ \mathrm{cm}^2$ &
To calculate the area of an equilateral triangle, you first need to know the length of one side. Since the perimeter of the triangle is given as 30 centimeters, and an equilateral triangle has all sides of equal length, each side of the triangle is:
Side length $= \frac{\text{Perimeter}}{3} = \frac{30~\mathrm{cm}}{3} = 10~\mathrm{cm}$.
The formula to calculate the area $A$ of an equilateral triangle with side length $s$ is:
$A = \frac{\sqrt{3}}{4}s^2$.
Plugging in the side length:
$A = \frac{\sqrt{3}}{4}\times (10~\mathrm{cm})^2 = \frac{\sqrt{3}}{4}\times 100~\mathrm{cm}^2 = 25\sqrt{3}~\mathrm{cm}^2$.
Let's calculate the exact numeric value of the area. The area of the equilateral triangle is approximately 43.30 square centimeters. &
Can the height of an equilateral triangle be equal to its side length? &
To figure out if the height of an equilateral triangle can be equal to its side length, let's analyze the properties of an equilateral triangle. An equilateral triangle has all three sides of the same length. If we draw a height from one vertex to the opposite side, it not only acts as a height but also bisects the base into two equal parts, creating two 30-60-90 right triangles. In a 30-60-90 triangle, the sides are in the ratio of $1:\sqrt{3}:2$. Here, the smallest side (opposite the $30^\circ$ angle) would be half the length of the hypotenuse (the original side of the equilateral triangle), and the height (opposite the $60^\circ$ angle) would be $\sqrt{3}/2$ times the length of the hypotenuse. So, if the side length of the equilateral triangle is $s$, the height $h$ would be $s\cdot \frac{\sqrt{3}}{2}$. For the height to be equal to the side length, $s = s\cdot \frac{\sqrt{3}}{2}$, which is not true unless $s=0$ (which is not a valid length for a side of a triangle).
\textbf{Conclusion:} False, the height of an equilateral triangle cannot be equal to its side length. \
\midrule
Natalia sold 48 clips in April and half as many clips in May. How many clips did Natalia sell altogether in April and June? &
72 &
Alright, let's break it down: 1. \textbf{April Sales:} Natalia sold clips to 48 of her friends in April. 2. \textbf{May Sales:} It's mentioned that she sold half as many clips in May as she did in April. So, if she sold 48 clips in April, then in May she sold $\frac{1}{2}\times 48 = 24$ clips. To find out how many clips Natalia sold altogether in April and May, we simply add the two amounts together:
$48 + 24 = 72$.
So, \textbf{Natalia sold a total of 72 clips} to her friends over the two months. &
Knowing the sales for May and June, without knowing the sales for July, is it possible to calculate the sum of the sales for June and July? &
False. To calculate the sum of the sales for June and July, you need the exact sales figures for both months. Knowing the sales for May gives you insight into the performance of that specific month but does not contribute any direct information about July's sales. Without the sales data for July, you cannot accurately determine the sum of June and July's sales. Therefore, it's not possible to calculate this sum with the information provided. \
\bottomrule
\end{tabular}
\caption{Responses of GPT-4-0125-preview to both original and conceptual problems. The sections highlighted in yellow
in the original paper delineate the distinction between original problems and trap problems. The green sections represent
instances where the model's final answers are correct.}
\label{tab:gpt4_oc_14}
\end{table*}

\begin{table*}[t]
\centering
\small
\begin{tabular}{p{4.9cm}p{1.6cm}p{6.5cm}p{4.9cm}p{6.5cm}}
\toprule
\textbf{Original problem} & \textbf{Answer} & \textbf{Output} & \textbf{Conceptual Problem} & \textbf{Output} \
\midrule
Max picks 2 different cards without replacement from a standard 52-card deck. What is the probability that the cards are of different suits? &
$\frac{13}{17}$ &
To find the probability that Max picks two cards of different suits from a standard 52-card deck, we first need to understand the deck's composition and the process of drawing cards without replacement.

1. \textbf{Deck Composition:} A standard deck of 52 cards consists of four suits: hearts, diamonds, clubs, and spades. Each suit has 13 cards.
2. \textbf{First Card Draw:} When Max draws the first card, it can be of any suit. This event is certain, so the probability is 1.
3. \textbf{Second Card Draw:} After the first card is drawn, there are 51 cards left in the deck. Since we want the second card to be of a different suit, we need to exclude the remaining 12 cards of the same suit as the first card. This leaves us with 39 cards (51 total minus 12 of the same suit) that are of a different suit.
4. \textbf{Calculating the Probability:} The probability that the second card is of a different suit is the number of favorable outcomes divided by the total number of possible outcomes. So, the probability is $\frac{39}{51}$.
5. \textbf{Final Probability:} Since the first card can be of any suit and does not affect the probability, the final probability that the two cards drawn are of different suits is $1 \times \frac{39}{51}$, which simplifies to $\frac{39}{51}$.
   Therefore, the probability that Max picks two cards of different suits without replacement from a standard 52-card deck is $\frac{39}{51}$, which can be simplified further if needed. &
   Is it possible to pick five different suits of cards from a standard deck of playing cards? &
   Yes, it's possible to select five different suits of cards from a standard deck of playing cards, but there's a small catch. A standard deck of playing cards typically contains four suits: hearts, diamonds, clubs, and spades. So, if you're selecting five suits, you'd have to include all four of these plus one more. However, since there are only four suits in a standard deck, you can't pick five different ones. You could say you're selecting all the suits in the deck, but not five different ones. \
   \bottomrule
   \end{tabular}
   \caption{Responses of GPT-4-0125-preview to both original and conceptual problems. The sections highlighted in yellow
   in the original paper delineate the distinction between original problems and trap problems. The green sections represent
   instances where the model's final answers are correct.}
   \label{tab:gpt4_oc_15}
   \end{table*}

\end{document}
=====END FILE=====
